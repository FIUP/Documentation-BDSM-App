% =================================================================================================
% File:			capitolo_2.tex
% Description:	Definisce il capitolo che descrive generalmente il prodotto per il commitente
% Created:		2014/12/10
% Author:		Roetta Marco
% Email:		roetta.marco@mashup-unipd.it
% =================================================================================================
% Modification History:
% Version		Modifier Date		Change											Author
% 0.0.1 		2014/12/30 			aggiunta sezione e iniziata stesura				Roetta Marco
% =================================================================================================
% Version		Modifier Date		Change											Author
% 0.0.2 		2015/01/12 			Aggiunto contenuto								Roetta Marco
% =================================================================================================
% Version		Modifier Date		Change											Author
% 0.1.0			2015/01/15 			Completata sezione						 		Roetta Marco
% =================================================================================================
% Version		Modifier Date		Change											Author
% 0.1.1			2015/01/15 			Correzioni ortografiche					 		Roetta Marco
% =================================================================================================
% Version		Modifier Date		Change											Author
% 1.0.0			2015/01/19 			Verifica documento					 			Nicola Faccin
% =================================================================================================
%

% CONTENUTO DEL CAPITOLO

\section{Descrizione generale}

\subsection{Contesto d'uso}
L'obbiettivo primario che il prodotto si pone è la creazione di una infrastruttura che
possa gestire diverse funzioni di monitoraggio, aggregazione e analisi statistica di parole chiave utilizzate nei social network precedentemente citati. Ogni utente che usa la piattaforma può creare, modificare ed eliminare View\gloss{} relative a Tag suddivisi in Recipe\gloss{} predefinite e fornite dal sistema.
La gestione delle Recipe\gloss{} e degli utenti è disponibile solo per un alcuni utenti identificati come amministratori.

\subsection{Funzioni del prodotto}
Il prodotto sarà diviso in due macro parti distinte:

\begin{itemize}
\item Un applicativo web per consultazione e personalizzazione delle View\gloss{} e dei dati associati;
\item Un applicativo server-side che interroga i social network in base ai parametri predefiniti rappresentati dalle Recipe\gloss{}. I risultati di tali elaborazioni verranno utilizzati per realizzare le View\gloss{} visualizzate dagli utenti tramite l'interfaccia grafica web.

L'applicativo server-side sarà inoltre suddiviso in diverse parti funzionali distinte:
\begin{itemize}
\item \textbf{Miner\gloss{}:} processo che interroga i social network tramite API\gloss{} dedicate;
\item \textbf{Cron\gloss{}:} processo che genera eventi di sincronizzazione dei dati per il Processor\gloss{};
\item \textbf{Processor\gloss{}:} processo principale di elaborazione delle richieste utente, comunicazione tra basi di dati e interfaccia grafica e gestione del funzionamento del Miner\gloss{}.
\end{itemize}
\end{itemize}

\subsubsection{Convenzione sui nomi delle componenti}
Le parti sopracitate utilizzeranno due formati di dati ben definiti: Recipe\gloss{} e View\gloss{}.
Le Recipe\gloss{} sono collezioni di informazioni ( ad es. i tag dei social network ) periodicamente aggiornate dall'applicativo server-side e utilizzate dagli utenti per generare i grafici desiderati.
Le View\gloss{} rappresentano i grafici e i dati ad essi associati che vengono creati dagli utenti autenticati.

\subsection{Caratteristiche degli utenti}
Il prodotto è rivolto a due classi distinte di persone:

\begin{itemize}
\item Persone esterne all'azienda che desiderano utilizzare i servizi offerti, quali gestione di View\gloss{} aggiornate automaticamente e periodicamente dal sistema;
\item Dipendenti di Zing s.r.l. che assumono il ruolo di amministratori e quindi in grado, oltre ad utilizzare le funzionalità offerte agli utenti normali, anche di gestire gli utenti registrati e le Recipe\gloss{} salvate nel sistema.
A questa categoria di persone sono richieste conoscenze di tipo sistemistico per poter comprendere tutte le funzionalità offerte.
\end{itemize}


\subsection{Vincoli generali}
Per usufruire delle funzionalità offerte dal prodotto si richiede all'utente di disporre
di una connessione internet e delle credenziali di accesso fornite durante la registrazione al servizio nell'apposita pagina.

\subsubsection{Piattaforme di esecuzione}
Il prodotto finale è fruibile da qualsiasi piattaforma che disponga di un browser per la navigazione web, tuttavia sarà garantito il funzionamento corretto solo con alcuni browser: Chrome v. 39.0 e successive, Firefox v. 35.0 e successive, Safari v. 8.0 e successive.
Il back-end del sistema verrà invece eseguito e gestito sulla piattaforma Google App Engine\gloss{}.
