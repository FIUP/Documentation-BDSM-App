% =================================================================================================
% File:			vocaboli_a_e.tex
% Description:	Defiinisce la sezione relativa ai vocaboli che vanno dalla lettera A alla E
% Created:		2014/12/29
% Author:		Ceccon Lorenzo
% Email:		ceccon.lorenzo@mashup-unipd.it
% =================================================================================================
% Modification History:
% Version		Modifier Date		Change											Author
% 0.0.1 		2014/12/29 			iniziata stesura 								Ceccon Lorenzo
% =================================================================================================
%

% CONTENUTO DEL CAPITOLO

\section*{A} % (fold)
\label{sec:a}
	\begin{itemize}
		\item \textbf{Android:} sistema operativo per sistemi mobile sviluppato da Google e basato su kernel Linux. Sviluppato principalmente per dispositivi touchscreen quali smartphone e tablet, lo si può trovare con interfaccie utenti specializzate per televisori (Android TV), automobili (Android Auto), orologi da polso (Android Wear) e occhiali (Google Glass);
		\item \textbf{AngularJS:} conosciuto semplicemente come Angular, è un framework per applicazioni web open-source gestito da Google, da una comunità di singoli sviluppatori e dalle aziende per affrontare molte delle sfide incontrate nello sviluppo di applicazioni in una singola pagina. Il suo scopo è quello di semplificare lo sviluppo e l'analisi di queste applicazioni fornendo un framework con architettura Model-View-Controller per il lato client insieme ai componenti comunemente usati nelle applicazioni web;
		\item \textbf{API:} acronimo di Application Programming Interface, è un insieme di routine, protocolli e strumenti che consentono ai software di interagire tra di loro. Funziona come un'interfaccia tra software differenti facilitandone la loro interazione, allo stesso modo in cui un'interfaccia utente facilita l'interazione tra uomo e computer;
		\item \textbf{Asana:} applicazione web e mobile progettata per permettere il lavoro di gruppo senza il bisogno di utilizzare l'email. Si tratta di un software as a service (SaaS). Ogni team possiede un'area di lavoro la quale contiene dei progetti che a loro volta contengono dei task. Nei task si possono aggiungere note, commenti, allegati e tag. Asana fornisce anche delle API per gli sviluppatori e si integra con molti servizi di produttività come GitHub;
		\item \textbf{Astah:} precedentemente noto come JUDE, è uno strumento di modellazione UML che offre funzionalità di creazione di diagrammi di:
		\begin{itemize}
			\item di classe;
			\item d'uso;
			\item di sequenza;
			\item di collaborazione;
			\item di stato;
			\item d'attività;
			\item di sviluppo;
			\item di componenti;
		\end{itemize}
		Nella versione Professional fornisce ulteriori strumenti tra cui la possibilità di aggiungere le descrizioni dei casi d'uso anche tramite l'utilizzo di template e la creazione di diagrammi di processi;
	\end{itemize}
\pagebreak
% section a (end)

\section*{B} % (fold)
\label{sec:b}
	\begin{itemize}
		\item \textbf{Big data:} termine usato per descrivere una collezione dati così grande e complessa che diventa complicato gestirla e processarla con i tradizionali strumenti informatici;
		\item \textbf{Big query:} servizio web RESTful che consente l'analisi interattiva di collezioni di dati molto grandi in collaborazione con Google Storage. Si tratta di un Infrastructure as a Service (IaaS) e può essere utilizzato in modo complementare con MapReduce;
		\item \textbf{BOM:} acronimo di Byte Order Mark, è un carattere Unicode utilizzato per indicare l'ordine dei byte id un file di testo o di un stream. L'utilizzo di BOM è opzionale, ma se usato, deve apparire all'inizio del flusso di dati di testo. Oltre al suo utilizzo come indicatore di ordine di byte, il carattere BOM può indicare anche con quale delle diverse rappresentazioni Unicode il testo è codificato;
		\item \textbf{Bug:} errore, difetto, guasto o malfunzionamento di un software che porta a produrre un risultato errato o inatteso, o a comportarsi in modi non previsti. La maggior parte dei bug derivano da errori umani nel codice sorgente o nella progettazione di un software, in alcuni casi sono causati dai compilatori che producono codice errato;
	\end{itemize}
\pagebreak
% section b (end)

\section*{C} % (fold)
\label{sec:c}
	\begin{itemize}
		\item \textbf{C++:} linguaggio di programmazione orientato agli oggetti nato nel 1983 come miglioramento del linguaggio C. Tra i miglioramenti più importanti troviamo l'introduzione del paradigma di programmazione ad oggetti, l'overloading degli operatori, l'eredità multipla, i template, le funzioni virtuali e la gestione delle eccezioni;
		\item \textbf{Ciclo di Deming:} conosciuto anche come PDCA (plan-do-check-act) è un modello creato per il miglioramento continuo della qualità. Promuove una cultura della qualità tesa al miglioramento continuo dei processi e all'utilizzo ottimale delle risorse. Specifica che per il raggiungimento del massimo della qualità è necessaria una costante interazione tra ricerca, progettazione, test e produzione. É suddiviso nelle seguenti quattro fasi:
			\begin{itemize}
				\item Plan. Pianificazione;
				\item Do. Esecuzione del programma, dapprima in contesti circoscritti;
				\item Check. Test e controllo, studio e raccolta dei risultati e dei riscontri;
				\item Act. Azione per rendere definitivo e/o migliorare il processo (estendere quanto testato dapprima in contesti circoscritti all'intera organizzazione).
			\end{itemize}
		\item \textbf{Cloud Database:} database che tipicamente viene eseguito su una piattaforma di cloud computing. Ci sono due modelli di distribuzione comune: gli utenti possono eseguire i database nel cloud in modo indipendente usando un'immagine di macchina virtuale, oppure possono acquistare l'accesso ad un servizio di database gestito da un provider di cloud database. I database nel cloud possono essere SQL-based oppure NoSQL;
		\item \textbf{Commit:} comando utilizzato nei sistemi di controllo di versione per codice sorgente come Git. Viene utilizzato per descrivere gli ultimi cambiamenti effettuati nella propria copia locale. Tale descrizione sarà inviata al repository insieme ai file che sono stati modificati;
		\item \textbf{Committente:} figura professionale che commissiona l'esecuzione di un lavoro ad altri. Può essere una persona fisica nel caso di un lavoro privato, una persona giuridica nel caso di un lavoro per un'azienda, un ministero nel caso di un lavoro pubblico;
		\item \textbf{Cron:} utility software, per sistemi Unix-like, che consiste in un job scheduler basato sul tempo. Le persone che impostano e mantengono ambienti software utilizzano cron per pianificare i processi in modo da eseguirli periodicamente ad orari, date o intervalli prestabiliti. Viene tipicamente utilizzato per l'automazione e l'amministrazione del sistema, ma risulta utile anche per altri scopi per esempio per la connessione ad internet e il download di email a intervalli regolari;
		\item \textbf{CSS3:} acronimo di Cascading Style Sheets, è un linguaggio usato per la formattazione di documenti HTML, XHTML e XML. É stato ideato per separare il contenuto dalla formattazione e per permetter una programmazione più chiara e facile. CSS3 implementa diverse migliorie rispetto a CSS2 come una migliore gestione dei sfondi, una soluzione per realizzare i bordi arrotondati e le soluzioni per la correzione di alcuni bug di interpretazione di Internet Explorer;
	\end{itemize}
\pagebreak
% section c (end)

\section*{D} % (fold)
\label{sec:d}
	\begin{itemize}
		\item \textbf{Database:} collezione di dati organizzata. Le informazioni contenute in esso sono strutturate e collegate tra loro mediante un particolare modello logico in modo tale da consentire l'organizzazione efficiente dei dati stessi e l'interfacciamento con le richieste dell'utente attraverso le query di interrogazione;
		\item \textbf{Design pattern:} soluzione generale riutilizzabile ad un problema comune che si verifica all'interno di un determinato contesto nella progettazione del software. Si tratta di una descrizione o modello logico da applicare per la risoluzione di un problema che può presentarsi in situazioni differenti durante le fasi di progettazione e sviluppo software;
		\item \textbf{Design responsive:} o RWD indica una tecnica per la realizzazione di siti in grado di adattarsi graficamente in modo automatico a seconda del dispositivo utilizzato. Riducendo la necessità di ridimensionamento e scorrimento dei contenuti è un importante elemento di accessibilità;
		\item \textbf{Diagramma di Gantt:} strumento che permette di modellare la pianificazione dei compiti necessari alla realizzazione di un progetto. I compiti sono rappresentati nell'asse verticale, mentre nell'asse orizzontale viene rappresentato il tempo. Il tempo stimato viene rappresentato tramite una barra orizzontale in cui la parte iniziale indica la data di inizio prevista mentre la parte finale indica la data di fine prevista;
		\item \textbf{Diagramma di PERT:} acronimo di Project Evaluation and Review Technique, è una tecnica per analizzare le attività coinvolte nel completamento di un determinato progetto, in particolare il tempo necessario per completare ogni attività, e per identificare il tempo minimo necessario per completare l'intero progetto. É stato sviluppato per semplificare la pianificazione e la programmazione di progetti grandi e complessi e viene utilizzato specialmente nei progetti in cui il tempo è il fattore più importante piuttosto che i costi;
		\item \textbf{Driver:} porzioni di codice utilizzate nei test di integrazione con filosofia bottom-up. Si occupa di chiamare i metodi di un livello più alto di quello corrente con l'unica finalità di verificarli tramite l'utilizzo di test;
	\end{itemize}
\pagebreak
% section d (end)

\section*{E} % (fold)
\label{sec:e}
	\begin{itemize}
		\item \textbf{Express.js:} framework per applicazioni web utile per aiutare ad organizzare un applicazione web in un'architettura model-view-controller sul lato server. É disegnato per lo sviluppo di applicazioni web a pagina singola, multipla e ibrida;
	\end{itemize}
\pagebreak
% section e (end)