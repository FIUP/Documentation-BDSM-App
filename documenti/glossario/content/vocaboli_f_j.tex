% =================================================================================================
% File:			vocaboli_f_j.tex
% Description:	Defiinisce la sezione relativa ai vocaboli che vanno dalla lettera F alla J
% Created:		2014/12/29
% Author:		Ceccon Lorenzo
% Email:		ceccon.lorenzo@mashup-unipd.it
% =================================================================================================
% Modification History:
% Version		Modifier Date		Change											Author
% 0.0.1 		2014/12/29 			iniziata stesura 								Ceccon Lorenzo
% =================================================================================================
%

% CONTENUTO DEL CAPITOLO

\section*{F} % (fold)
\label{sec:f}
	\begin{itemize}
		\item \textbf{Framework:} ambiente software universale, riutilizzabile che fornisce funzionalità particolari come parte di una piattaforma software più grande per facilitare lo sviluppo di applicazioni software. Può includere programmi di supporto, compilatori, librerie, strumenti e API che mettono insieme tutte le diverse componenti per consentire lo sviluppo di un progetto;
	\end{itemize}
\pagebreak
% section f (end)

\section*{G} % (fold)
\label{sec:g}
	\begin{itemize}
		\item \textbf{Google App Engine:} conosciuto semplicemente come App Engine, è una platform as a service (PaaS) di cloud computing per lo sviluppo e l'hosting di applicazioni web nei data center gestiti da Google. App Engine offre un sistema di ridimensionamento automatico per le applicazioni web in modo, cioè se il numero di richieste da parte di un'applicazione aumentano, automaticamente App Engine allocherà maggiori risorse a tale applicazione per gestire la richiesta aggiuntiva;
		\item \textbf{Google Cloud Engine:} piattaforma di cloud computing gestita da Google che offre un servizio di hosting sulla stessa infrastruttura di supporto che Google utilizza internamente per i prodotti end-user come Google Search e YouTube. Fornisce prodotti per sviluppatori per creare una vasta serie di programmi che vanno dai semplici siti web fino ad applicazioni complesse;
		\item \textbf{Google Cloud SQL:} servizio completo per la gestione di database MySQL ospitati dalla piattaforma cloud Google Cloud Platform. Offre un collegamento rapido ai database per le applicazioni in esecuzione su Google App Engine;
	\end{itemize}
\pagebreak
% section g (end)

\section*{H} % (fold)
\label{sec:h}
	\begin{itemize}
		\item \textbf{HTML5} acronimo di HyperText Markup Language, è il linguaggio di markup standard per la creazione di pagine web. La versione HTML5 introduce diverse novità rispetto alle versione precedente:
		 \begin{itemize}
		 	\item Disaccoppiamento fra struttura, caratteristiche di resa e contenuti;
		 	\item Supporto per la memorizzazione locale di grandi quantità di dati scaricati dal browser, in modo da consentire l'utilizzo di applicazioni web anche in assenza di connessione;
		 	\item Semplificato il doctype;
		 	\item Introduzione del Web Storage come sistema alternativo ai cookie;
		 	\item Introduzione della geolocalizzazione;
			\item Introduzione di elementi per il controllo di contenuti audio e video;
			\item Introduzione di elementi di controllo per i menu di navigazione;
			\item Introduzione di regole più stringenti per la strutturazione del testo;
			\item Introduzione di attributi legati all'accessibilità per tutti i tag;
			\item Eliminazione di elementi di scarsa utilità;
			\item Supporto ai canvas per creare animazioni e grafica bitmap tramite JavaScript;
		 \end{itemize}
	\end{itemize}
\pagebreak
% section h (end)

\section*{I} % (fold)
\label{sec:i}
	\begin{itemize}
		\item TO DO;
	\end{itemize}
\pagebreak
% section i (end)

\section*{J} % (fold)
\label{sec:j}
	\begin{itemize}
		\item TO DO;
	\end{itemize}
\pagebreak
% section j (end)