% =================================================================================================
% File:			vocaboli_f_j.tex
% Description:	Defiinisce la sezione relativa ai vocaboli che vanno dalla lettera F alla J
% Created:		2014/12/29
% Author:		Ceccon Lorenzo
% Email:		ceccon.lorenzo@mashup-unipd.it
% =================================================================================================
% Modification History:
% Version		Modifier Date		Change											Author
% 0.0.1 		2014/12/29 			iniziata stesura 								Ceccon Lorenzo
% =================================================================================================
%

% CONTENUTO DEL CAPITOLO

\section*{F} % (fold)
\label{sec:f}
	\begin{itemize}
		\item \textbf{Framework:} ambiente software universale, riutilizzabile che fornisce funzionalità particolari come parte di una piattaforma software più grande per facilitare lo sviluppo di applicazioni software. Può includere programmi di supporto, compilatori, librerie, strumenti e API che mettono insieme tutte le diverse componenti per consentire lo sviluppo di un progetto;
	\end{itemize}
\pagebreak
% section f (end)

\section*{G} % (fold)
\label{sec:g}
	\begin{itemize}
		\item \textbf{Google App Engine:} conosciuto semplicemente come App Engine, è una platform as a service (PaaS) di cloud computing per lo sviluppo e l'hosting di applicazioni web nei data center gestiti da Google. App Engine offre un sistema di ridimensionamento automatico per le applicazioni web in modo, cioè se il numero di richieste da parte di un'applicazione aumentano, automaticamente App Engine allocherà maggiori risorse a tale applicazione per gestire la richiesta aggiuntiva;
		\item \textbf{Google Cloud Engine:} piattaforma di cloud computing gestita da Google che offre un servizio di hosting sulla stessa infrastruttura di supporto che Google utilizza internamente per i prodotti end-user come Google Search e YouTube. Fornisce prodotti per sviluppatori per creare una vasta serie di programmi che vanno dai semplici siti web fino ad applicazioni complesse;
		\item \textbf{Google Cloud SQL:} servizio completo per la gestione di database MySQL ospitati dalla piattaforma cloud Google Cloud Platform. Offre un collegamento rapido ai database per le applicazioni in esecuzione su Google App Engine;
	\end{itemize}
\pagebreak
% section g (end)

\section*{H} % (fold)
\label{sec:h}
	\begin{itemize}
		\item \textbf{HTML5:} acronimo di HyperText Markup Language, è il linguaggio di markup standard per la creazione di pagine web. La versione HTML5 introduce diverse novità rispetto alle versione precedente:
		 \begin{itemize}
		 	\item Disaccoppiamento fra struttura, caratteristiche di resa e contenuti;
		 	\item Supporto per la memorizzazione locale di grandi quantità di dati scaricati dal browser, in modo da consentire l'utilizzo di applicazioni web anche in assenza di connessione;
		 	\item Semplificato il doctype;
		 	\item Introduzione del Web Storage come sistema alternativo ai cookie;
		 	\item Introduzione della geolocalizzazione;
			\item Introduzione di elementi per il controllo di contenuti audio e video;
			\item Introduzione di elementi di controllo per i menu di navigazione;
			\item Introduzione di regole più stringenti per la strutturazione del testo;
			\item Introduzione di attributi legati all'accessibilità per tutti i tag;
			\item Eliminazione di elementi di scarsa utilità;
			\item Supporto ai canvas per creare animazioni e grafica bitmap tramite JavaScript;
		 \end{itemize}
	\end{itemize}
\pagebreak
% section h (end)

\section*{I} % (fold)
\label{sec:i}
	\begin{itemize}
		\item \textbf{IDE:} acronimo di integrated development environment, è una applicazione che fornisce servizi per facilitare i programmatori nella fase di sviluppo software. Un IDE è tipicamente costituito da un editor di codice sorgente, strumenti per automatizzare la fase di building e un debugger. La maggior parte degli IDE moderni hanno funzionalità di intelligente del codice;
		\item \textbf{IEC:} acronimo di International Electrotechnical Commission, è un'organizzazione internazionale non governativa e non-profit per la definizione di standard in materia di elettricità, elettronica e tecnologie correlate. Gli standard IEC coprono una vasta gamma di tecnologie di produzione e distribuzione di elettrodomestici, semiconduttori, batterie, nanotecnologie e molti altri ancora. L'IEC gestisce anche tre sistemi di valutazione della conformità globali che certificano se le apparecchiature, i sistemi o i componenti sono conformi ai suoi standard internazionali. Molti dei suoi standard sono definiti in collaborazione con l'ISO;
		\item \textbf{Impress.js:} framework di presentazione JavaScript ispirato dall'idea che sta sotto a Prezi. Utilizza le trasformazioni e transazioni di CSS3 per fornire un'esperienza di presentazione che va al di là del classico scorrimento delle slide che caratterizza la maggior parte dei software di presentazione. Basato sulle tecnologie web standard (HTML, CSS e JavaScript) permette agli utenti di creare presentazioni senza doversi affidare a particolari applicazioni o servizi web;
		\item \textbf{Indice Gulpease:} indice di leggibilità di un testo creato per la lingua italiana e sviluppato presso l'Università degli studi di Roma "La Sapienza". Diversamente da altri ha il vantaggio di utilizzare la lunghezza delle parole in lettere anziché in sillabe, semplificandone il calcolo automatico;
		\item \textbf{IntelliJ IDEA:} IDE per lo sviluppo di software in linguaggio Java. É sviluppato da JetBrains ed è disponibile in versione community edition rilasciata sotto licenza Apache 2 e in versione commerciale rilasciata sotto licenza proprietaria;
		\item \textbf{ISO:} acronimo di International Organization for Standardization, è un'organizzazione internazionale per la definizione di standard composto da rappresentanti di vari organismi nazionali di standardizzazione. Fondata il 23 febbraio 1947, l'organizzazione promuove in tutto il mondo standard proprietari, industriali e commerciali. L'ISO coopera strettamente con l'IEC, responsabile per la standardizzazione degli equipaggiamenti elettrici;
	\end{itemize}
\pagebreak
% section i (end)

\section*{J} % (fold)
\label{sec:j}
	\begin{itemize}
		\item \textbf{Java:} linguaggio di programmazione ad oggetti specificamente progettato per avere il minor numero di dipendenze possibile. Il codice che viene eseguito su una piattaforma non ha bisogno di essere ricompilato per funzionare su un'altra piattaforma. Le applicazioni Java sono in genere compilate in bytecode, il quale può essere eseguito su qualsiasi macchina virtuale Java (JVM) indipendentemente dall'architettura del computer. É uno dei linguaggi di programmazione più popolari, specialmente per quanto riguarda le applicazioni web. La sintassi del linguaggio deriva molto dal C e dal C++, ma offre meno servizi a basso livello rispetto a quest'ultimi;
		\item \textbf{JavaScript:} linguaggio di programmazione orientato agli oggetti e agli eventi comunemente usato nello sviluppo web. Sviluppato originalmente da Netscape per aggiungere elementi dinamici e interattivi ai siti web. É classificato come un linguaggio di scripting lato client, ciò significa che il codice sorgente è processato direttamente dal browser web del client piuttosto che dal server web. Ciò significa che una funzione JavaScript può essere eseguita senza comunicare con il server. Il codice può essere inserito ovunque all'interno di una pagina HTML. Nonostante qualche denominazione e le somiglianze tra le sintassi e tra le librerie standard, JavaScript e Java sono comunque indipendenti tra loro e hanno semantiche molto differenti; la sintassi di JavaScript è derivata da C;
		\item \textbf{JSON:} acronimo di JavaScript Object Notation, è un formato standard che di interscambio di dati testuali progettato per la trasmissione di dati strutturati. Basato sul linguaggio JavaScript ma totalmente indipendente da questo ultimo, viene usato specialmente per il trasferimento di dati tra le applicazioni web e i server. JSON è spesso visto come un'alternativa a XML e usato come alternativa a questo ultimo in AJAX. Nella maggior parte dei casi, la rappresentazione di un oggetto in JSON è molto più compatta della rappresentazione in XML poichè non richiede l'uso di tag per ogni elemento;
	\end{itemize}
\pagebreak
% section j (end)