% =================================================================================================
% File:			vocaboli_k_o.tex
% Description:	Defiinisce la sezione relativa ai vocaboli che vanno dalla lettera K alla O
% Created:		2014/12/29
% Author:		Faccin Nicola
% Email:		faccin.nicola@mashup-unipd.it
% =================================================================================================
% Modification History:
% Version		Modifier Date		Change											Author
% 0.0.1 		2014/12/29 			iniziata stesura 								Faccin Nicola
% =================================================================================================
%

% CONTENUTO DEL CAPITOLO

\section*{L} % (fold)
\label{sec:l}
	\begin{itemize}
		\item \textbf{\LaTeX:} linguaggio di markup usato per la preparazione di testi basato sul programma di composizione tipografica TEX. É largamente utilizzato per la pubblicazione di documenti scientifici in molti settori, specialmente in quelli scientifici;
	\end{itemize}
\pagebreak
% section l (end)

\section*{M} % (fold)
\label{sec:m}
	\begin{itemize}
		\item \textbf{Milestone:} termine inglese che significa pietra miliare. Viene utilizzato nella pianificazione e gestione di progetti per indicare il raggiungimento di obiettivi stabiliti in fase di definizione del progetto. Le milestones indicano quindi importanti traguardi intermedi nello svolgimento del progetto;
		\item \textbf{Miner:} processo utilizzato per interrogare i social network tramite l'utilizzo delle API fornite da questi ultimi.;	
		\item \textbf{MySql:} sistema di gestione per basi di dati relazionali, composto da un client a riga di comando e un server. Entrambi i software sono disponibili sia per sistemi Unix e Unix-like che per Windows;
	\end{itemize}
\pagebreak
% section m (end)

\section*{N} % (fold)
\label{sec:n}
	\begin{itemize}
		\item \textbf{Netsons:} sito web che fornisce servizi di hosting, server virtuali, server dedicati e soluzioni IT;
		\item \textbf{Node.js:} piattaforma software open-source utilizzata per creare applicazioni distribuite facilmente scalabili. Node.js utilizza JavaScript come linguaggio di scripting e gestisce le attese I/O in modo asincrono;
	\end{itemize}
\pagebreak
% section n (end)
