% =================================================================================================
% File:			vocaboli_k_o.tex
% Description:	Defiinisce la sezione relativa ai vocaboli che vanno dalla lettera U alla Z
% Created:		2014/12/29
% Author:		Faccin Nicola
% Email:		faccin.nicola@mashup-unipd.it
% =================================================================================================
% Modification History:
% Version		Modifier Date		Change											Author
% 0.0.1 		2014/12/29 			iniziata stesura 								Faccin Nicola 
% =================================================================================================
%

% CONTENUTO DEL CAPITOLO

% NOTA BENE: se non ci sono vocaboli da inserire dentro una sezione alla fine va eliminata

\section*{U} % (fold)
\label{sec:u}
	\begin{itemize}
		\item \textbf{UML:} acronimo di Unified Modeling Language, è un linguaggio di modellazione e specifica basato sul paradigma object-oriented. È utilizzato per descrivere soluzioni analitiche e progettuali in modo sintetico e comprensibile a un vasto pubblico, per questo ha anche il ruolo importante di lingua franca. Ad oggi si è giunti alla versione 2.0.;
	\end{itemize}
\pagebreak
% section u (end)

\section*{V} % (fold)
\label{sec:v}
	\begin{itemize}
		\item \textbf{Validazione:} Controllo effettuato sul software, per controllare se tutti i requisiti previsti sono stati coperti;
	\end{itemize}
\pagebreak
% section v (end)

\section*{W} % (fold)
\label{sec:w}
	\begin{itemize}
		\item \textbf{WebSocket:} tecnologia Web che fornisce canali di comunicazione full-duplex attraverso una singola connessione TCP. Permette maggiore interazione tra browser e server facilitando la realizzazione di applicazioni che forniscono contenuti realtime;
	\end{itemize}
\pagebreak
% section w (end)

\section*{X} % (fold)
\label{sec:x}
	\begin{itemize}
		\item TO DO;
	\end{itemize}
\pagebreak
% section x (end)

\section*{Y} % (fold)
\label{sec:y}
	\begin{itemize}
		\item TO DO;
	\end{itemize}
\pagebreak
% section y (end)

\section*{Z} % (fold)
\label{sec:z}
	\begin{itemize}
		\item TO DO;
	\end{itemize}
\pagebreak
% section z (end)