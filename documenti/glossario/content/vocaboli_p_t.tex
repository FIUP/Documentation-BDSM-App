% =================================================================================================
% File:			vocaboli_p_t.tex
% Description:	Defiinisce la sezione relativa ai vocaboli che vanno dalla lettera P alla T
% Created:		2014/12/29
% Author:		Faccin Nicola
% Email:		faccin.nicola@mashup-unipd.it
% =================================================================================================
% Modification History:
% Version		Modifier Date		Change											Author
% 0.0.1 		2014/12/29 			iniziata stesura 								Faccin Nicola
% =================================================================================================
%

% CONTENUTO DEL CAPITOLO

\section*{\Huge P} % (fold)
\label{sec:p}
	\begin{itemize}
		\item \textbf{Package:} in alcuni linguaggi orientati agli oggetti permette di organizzare un insieme di classi tra loro correlate che concorrono allo stesso fine;
		\item \textbf{PDCA:} vedi Ciclo di Deming;		
		\item \textbf{PERT:} vedi Diagramma di PERT;		
		\item \textbf{PowerPoint:} programma di presentazione prodotto da Microsoft, disponibile per i sistemi operativi Windows e Mac OS. Viene utilizzato principalmente per proiettare e quindi comunicare su schermo, progetti, idee, e contenuti potendo incorporare testo, immagini, grafici, filmati, audio e potendo presentare tutto questo tramite l'utilizzo di semplici animazioni;
		\item \textbf{Processor:} processo che esegue le operazioni richieste dall'utente, effettua la comunicazione tra le basi di dati e l'interfaccia grafica e gestisce il processo Miner;
		\item \textbf{ProjectLibre:} sistema software che permette di aiutare a pianificare, organizzare e sviluppare risorse (project management software system). Lavora sulla piattaforma Java ed è rilasciato sotto licenza di software libero. Può essere utilizzato su Linux, Mac OS e Windows;		
		\item \textbf{Proponente:} colui che propone un lavoro ad una persona, un gruppo o un'azienda, in questo caso il capitolato presentato da \proposerName;
	\end{itemize}
\pagebreak
% section p (end)

\section*{\Huge Q} % (fold)
\label{sec:q}
	\begin{itemize}
		\item \textbf{Qt:} framework multipiattaforma, ampiamente utilizzato per lo sviluppo di applicazioni software con interfaccia grafica tramite uso di Widget(elementi grafici);
		\item \textbf{Qualità}: insieme di caratteristiche di un'entità che ne determinano la capacità di soddisfare esigenze esplicite o implicite;	
	\end{itemize}
\pagebreak
% section q (end)

\section*{\Huge R} % (fold)
\label{sec:r}
	\begin{itemize}
		\item \textbf{Repository:} ambiente di un sistema informativo, in cui vengono gestiti i metadati attraverso tabelle relazionali. Offre un sistema di versionamento in grado di tener traccia delle modifiche effettuate al suo interno. Generalmente condiviso da più utenti, ognuno in grado di accedervi autonomamente per apportare modifiche, è implicitamente un servizio di condivisione dati. Nel nostro caso il sistema informativo è gestito con GitHub;	
		\item \textbf{Recipe:} collezioni di informazioni raccolte dai social network periodicamente e generate dall'applicativo server-side. Vengono utilizzate per generare e aggiornare i dati e i grafici dagli utenti;
		\item \textbf{REST:} acronimo di REpresentational State Transfer, è un tipo di architettura software per il Word Wide Web molto usato nell'HTTP. REST infatti si riferisce ad un insieme di principi di architetture di rete, i quali delineano come le risorse sono definite e indirizzate;
		\item \textbf{Reveal.js:} è un framework per creare presentazioni in HTML. Queste ultime si possono creare o con visual editor come Slides oppure direttamente scrivendo codice;
		\end{itemize}
\pagebreak
% section r (end)


\section*{\Huge S} % (fold)
\label{sec:s}
	\begin{itemize}
		\item \textbf{SDK:} acronimo di Software Development Kit, insieme di strumenti per lo sviluppo e la documentazione di software;	
		\item \textbf{Socket.io:} è una libreria JavaScript per le applicazioni realtime. Permette le comunicazioni realtime tra server e web client. Divisa in due parti, per il server la prima (una libreria per node.js), e per web client la seconda (una libreria che è usata dal browser), viene usata principalmente per il protocollo WebSocket;
		\item \textbf{SPICE:} acronimo di Software Process Improvement and Capability Determination, è un insieme di documenti di standard tecnici che forniscono informazioni generali su concetti di valutazione dei processi, e dei suoi usi nei due contesti: miglioramento dei processi e valutazione della maturità dei processi. I documenti riguardano processi di sviluppo di software e le relative funzioni gestionali di azienda;
		\item \textbf{Stub:} porzione di codice che, dati certi input, simula il comportamento del codice esistente al fine di verificare se la funzione chiamante fornisce i risultati attesi dal test;	
		\item \textbf{SVG:} acronimo di Scalable Vector Graphics, è una tecnologia in grado di visualizzare oggetti di grafica vettoriale e, pertanto, di gestire immagini scalabili dimensionalmente;
	\end{itemize}
\pagebreak
% section s (end)

\section*{\Huge T} % (fold)
\label{sec:t}
	\begin{itemize}
		\item \textbf{Task:} è un'attività che bisogna completare in un certo periodo di tempo in modo da poter rispettare le deadline concordate;
		\item \textbf{Template:} indica un modello che delinea una struttura generica che verrà completata successivamente, è utile ad esempio nel layout delle pagine per far si che rimanga invariato per tutte quelle di un documento;
		\item \textbf{Ticket:} sinonimo di Task;
		\item \textbf{Ticketing:} insieme di operazioni atte a definire i singoli Task ed assegnarli ai vari componenti del gruppo; 		
		\item \textbf{Twitter Bootstrap:} è un insieme di tools per creare siti e applicazioni Web. Contiene HTML CSS sia per la tipografia che per le varie componenti dell'interfaccia, opzionalmente anche JavaScript;
		
	\end{itemize}
\pagebreak
% section t (end)
