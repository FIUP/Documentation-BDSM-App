% =================================================================================================
% File:			vocaboli_p_t.tex
% Description:	Defiinisce la sezione relativa ai vocaboli che vanno dalla lettera P alla T
% Created:		2014/12/29
% Author:		Faccin Nicola
% Email:		faccin.nicola@mashup-unipd.it
% =================================================================================================
% Modification History:
% Version		Modifier Date		Change											Author
% 0.0.1 		2014/12/29 			iniziata stesura 								Faccin Nicola
% =================================================================================================
%

% CONTENUTO DEL CAPITOLO

\section*{P} % (fold)
\label{sec:p}
	\begin{itemize}
		\item \textbf{PowerPoint} :è il programma di presentazione prodotto da Microsoft, disponibile per i sistemi operativi Windows e Macintosh. Viene utilizzato principalmente per proiettare e quindi comunicare su schermo, progetti, idee, e contenuti potendo incorporare testo, immagini, grafici, filmati, audio e potendo presentare tutto questo con animazioni;
		\item \textbf{Proponente} :colui che presenta una proposta su cui lavorare, in questo caso il capitolato presentato da \proposerName;
		\item \textbf{PDCA} :PDCA (plan-do-check-act) è un metodo iterativo a quattro stadi usato per il controllo e il continuo miglioramento dei processi e dei prodotti;
		\item \textbf{Package} :In alcuni linguaggi orientati agli oggetti permette di organizzare un insieme di classi tra loro correlate che concorrono allo stesso fine;
		\item \textbf{PERT} :vedi Diagramma di PERT;
		\item \textbf{•}
	\end{itemize}
% section p (end)

\section*{Q} % (fold)
\label{sec:q}
	\begin{itemize}
		\item \textbf{Qt} :framework multipiattaforma, ampiamente utilizzato per lo sviluppo di applicazioni software con interfaccia grafica tramite uso di Widget(elementi grafici);
		\item \textbf{Qualità} :Insieme di caratteristiche di un'entità che ne determinano la capacità di soddisfare esigenze espresse o implicite.	
	\end{itemize}
% section q (end)

\section*{R} % (fold)
\label{sec:r}
	\begin{itemize}
		\item TO DO;
	\end{itemize}
% section r (end)

\section*{S} % (fold)
\label{sec:s}
	\begin{itemize}
		\item TO DO;
	\end{itemize}
% section s (end)

\section*{T} % (fold)
\label{sec:t}
	\begin{itemize}
		\item TO DO;
	\end{itemize}
% section t (end)
