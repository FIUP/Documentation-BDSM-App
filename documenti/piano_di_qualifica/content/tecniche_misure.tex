% =================================================================================================
% File:			tecniche_misure.tex
% Description:	In questa sezione si definiscono le metriche e le misure che si intendono adottare
% Created:		2014/12/15
% Author:		Faccin Nicola
% Email:		faccin.nicola@mashup-unipd.it
% =================================================================================================
% Modification History:
% Version		Modifier Date		Change											Author
% 0.0.1 		2014/12/15			iniziata stesura documento						Nicola F.
% =================================================================================================
%

% CONTENUTO DEL CAPITOLO

\subsection{Misure e metriche}
Descrizione delle metriche e delle misure per rendere quantificabili e conseguentemente qualificabili i processi, i documenti e il software prodotto.
\subsubsection{Metriche per i processi}
Le metriche qui utilizzate sono volte a rendere determinati tempi e costi, due delle variabili critiche principali dell'analisi.
\subsubsection{Metriche per i documenti}
\textbf{Indice Gulpease} \textnormal{Questo indice, tarato specificatamente per la lingua italiana, ha anche il vantaggio di utilizzare la lunghezza delle parole in lettere e non delle sillabe, semplificandone il calcolo. }
\\
%89+{300*(Numero delle frasi)-10*(Numero delle lettere) \over(Numero delle parole)}
\\
100 indica la leggibilità più alta mentre 0 quella più bassa, sono presenti dei range così da poter quantificare meglio la complessità del documento in analisi:
\begin{itemize}
\item inferiori a 80 sono difficili da leggere per chi ha la licenza elementare;
\item inferiori a 60 sono difficili da leggere per che ha la licenza media;
\item inferiori a 40 difficili da leggere per chi possiede un diploma superiore;
\end{itemize}
\\
Range-ottimale[50-100], range-accettazione [40-100].

\subsubsection{Metriche per il software}
\begin{itemize}
\item \textbf{Complessità Ciclomatica} \textnormal{: è utilizzata per misurare la complessità di un metodo, attraverso il grafo di controllo di flusso che misura direttamente il numero di cammini linearmente indipendenti. I nodi di questo grafo rappresentano gruppi indivisibili di istruzioni e gli archi connettono due nodi solamente se le istruzioni di un nodo possono essere eseguite immediatamente dopo le istruzioni dell'altro nodo.
In questo progetto si cercherà di rispettare la raccomandazione di \textit{McCabe}, che sviluppò tale teoria, ossia quella di non superare una complessità di 10. Rispettando questo vincolo si aumentano le possibilità di riuso del codice, manutenibilità, coesione e correttezza di quest'ultimo. Il vincolo presentato sarà di tipo lasco, ossia potrà essere portato a valori maggiori nell'eventualità porti a notevoli benefici in termini di velocità di esecuzione.
\\
	\\Valore-ottimale <10, valore-accettazione <15.}
\item \textbf{Numero di metodi} \textnormal{: metrica utilizzata per calcolare una media delle occorrenze dei metodi per package; valori alti potrebbero indicare la necessità di scomporlo.
\\
	\\Range-ottimale [3-8], range-accettazione [3-10].}
\item \textbf{Numero di parametri} \textnormal{: metrica utilizzata per calcolare il numero di parametri formali di un metodo. Un valore basso e indice di maggior manutenibilità e astrazione del codice.
\\	
	\\Range-ottimale [0-4], range-accettazione [0-8].}
\item \textbf{Linee di codice per linee di commento} \textnormal{: metrica atta a migliorare la manutenibilità del codice attraverso il monitoraggio del rapporto tra questi valori.
\\
	\\Valore-ottimale <0.20, valore-accettazione <0.35}
\item \textbf{Bugs for lines of code} \textnormal{: metrica per la misura dei bug trovati per un certo quantitativo di linee di codice. Questa metrica è utile in quanto all'aumentare dell'ampiezza del codice si aumenta la probabilità di nascondere degli errori. Presupponendo che nessuno del gruppo avrà conoscenze sufficienti dello stack tecnologico che si andrà ad utilizzare si partirà con un valore di accettazione alto per poi cercare di ridurlo in modo incrementale. L'obbiettivo fissato è quello di raggiungere valori compresi tra 0 e 20. Difficoltà particolari verranno gestite dal responsabile di progetto.}
\item \textbf{Numero di livelli di annidamento} \textnormal{: metrica per misurare il livello di annidamento dei metodi. Un numero elevato comporta eccessiva complessità del codice e ne riduce il livello di astrazione.
\\	
	\\Range-ottimale [1-4], range-accettazione [1-6].}
\end{itemize}