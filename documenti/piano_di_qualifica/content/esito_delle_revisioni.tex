% =================================================================================================
% File:			esito_delle_revisioni.tex
% Description:	Definisce gli esiti delle revisioni
% Created:		2014/02/28
% Author:		Ceccon Lorenzo
% Email:		ceccon.lorenzo@mashup-unipd.it
% =================================================================================================
% Modification History:
% Version		Modifier Date		Change											Author
% 0.0.1 		2014/01/11 			iniziata stesura appendice					Lorenzo C.
% =================================================================================================
% Version		Modifier Date		Change											Author
% 1.0.1 		2015/03/17 			iniziata stesura appendice B				Nicola F.
% =================================================================================================
% Version		Modifier Date		Change											Author
% 2.0.1 		2015/05/21 			iniziata stesura Revisione di progettazione Nicola F.
% =================================================================================================
%


% CONTENUTO DEL CAPITOLO

\section{Esito delle revisioni}
\label{sub:esito_delle_revisioni}
In questa sezione verranno riportate un elenco delle modifiche per ogni documento apportate dal gruppo a seguito delle valutazioni fatte dal committente alle revisioni.\\
Viene qui sotto riportata una lista con tutte le modifiche effettuate divise per ogni revisione a cui il gruppo ha preso parte.

	\subsection{Revisione dei Requisiti}
	\begin{itemize}
		\item \textbf{\docNameAdR:} a seguito delle segnalazioni, è stato rivisto completamente il documento. Sono stati rivisti completamente i casi d'uso e i requisiti.\\
		Il caso d'uso principale si è deciso di dividerlo in due grandi casi d'uso in modo da dividere i casi d'uso relativi ai servizi REST dal resto del sistema;
		\item \textbf{\docNameNdP:} in base alle segnalazioni ricevute sono stati aggiunti diagrammi nella sezione relativi ai processi primari e sono state fornite le regole e le procedure di rotazione dei ruoli;
		\item \textbf{\docNamePdP:} in base alle segnalazioni ricevute si sono rivisti completamente i consuntivi e preventivi, specialmente è stato rimossa la consuntivazione relativa alla prima parte. É stato corretto l'utilizzo del termine "fase" laddove utilizzato come sinonimo di "attività";
		\item \textbf{\docNamePdQ:} a seguito delle segnalazioni è stato rivisto tutto il documento. Sono state rielaborate le strategie di perseguimento di qualità e migliorate le appendici relative agli standard di qualità. Infine, si è migliorata l'integrazione dei contenuti del documento con quelli delle \docNameNdP.
	\end{itemize}
	
	\subsection{Revisione di Progettazione}
	%Come da segnalazione il registro delle modifiche dei documenti è stato rivisto incrementandolo e modificandone l'ordinamento dal tipo di modifica al numero di versione del documento;
	\begin{itemize}
		\item \textbf{\docNameAdR:} a seguito delle segnalazioni, sono stati rivisti i diagrammi di attività presenti nel documento e eliminato un requisito già specificato precedentemente;
		\item \textbf{\docNameST:} in base alle segnalazioni ricevute sono stati rivisti alcuni design pattern, si è descritto il modo in cui i framework e le librerie utilizzate si integrano nell'architettura del prodotto e sono stati corretti alcuni errori lessicali e di significato.\\
		Nella parte del backend sono state riviste le associazioni di composizione, è stato inoltre definito il tipo \textit{Dictionary} usato nelle classi \textit{Command};
		\item \textbf{\docNamePdP:} come da indicazione, sono state correlate l'analisi dei rischi e la pianificazione, è stata, inoltre, rivista in parte l'organizzazione del documento. Incrementato il contenuto del preventivo a finire e corretto il contenuto del consuntivo relativo alla revisione;
		\item \textbf{\docNamePdQ:} in base alle indicazioni ricevute, è stata arricchita la sezione 3 e corretta la presentazione del PDCA. 
	\end{itemize}
	
	\subsection{Revisione di Qualifica}
	%Come da segnalazione il registro delle modifiche dei documenti è stato rivisto incrementandolo e modificandone l'ordinamento dal tipo di modifica al numero di versione del documento;
	\begin{itemize}
		\item \textbf{\docNameAdR:} a seguito delle segnalazioni sono stati corretti i diagrammi di attività che presentavano errori;
		\item \textbf{\docNameST:} in base alle segnalazioni è stato corretto l'utilizzo errato del termine design pattern per il MVW;
		\item \textbf{\docNameDdP:} in base alle segnalazioni sono state corretti vari paragrafi che risultavano poco chiari; è stato quindi descritto in che modo viene generato il token d'accesso e come viene valorizzato il dizionario che associa le chiamate ai relativi comandi;
		\item \textbf{Manuali:} come da indicazione, sono stati sostituiti gli screenshot con altri di nuovi in cui vengono sottolineati meglio i passaggi fondamentali;
		\item \textbf{\docNamePdP:} come da indicazione, è stata spostata la descrizione del ciclo di vita all'interno della pianificazione e sono stati corretti dei termini utilizzati erroneamente;
		\item \textbf{\docNamePdQ:} in base alle indicazioni ricevute, è stata corretta la presentazione del ciclo PDCA. 
	\end{itemize}

\pagebreak