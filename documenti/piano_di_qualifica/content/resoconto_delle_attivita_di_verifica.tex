% =================================================================================================
% File:			resoconto_delle_attivita_di_verifica.tex
% Description:	Definisce il resoconto sulle attività di verifica
% Created:		2014/02/20
% Author:		Ceccon Lorenzo
% Email:		ceccon.lorenzo@mashup-unipd.it
% =================================================================================================
% Modification History:
% Version		Modifier Date		Change											Author
% 0.0.1 		2014/01/11 			iniziata stesura appendice					Lorenzo C.
% =================================================================================================
%% Version		Modifier Date		Change											Author
% 1.0.1 		2015/03/17 			iniziata stesura appendice B				Nicola F.
% =================================================================================================
%


% CONTENUTO DEL CAPITOLO

\section{Resoconto delle attività di verifica}
	\subsection{Revisione dei Requisiti}
	Nel periodo precedente alla consegna della \RR{} sono state effettuate delle attività di verifica sia per i documenti sia per i processi.\\
	Per quanto riguarda la verifica dei documenti si è effettuata attività di analisi statica descritta nelle \docNameVersionNdP. Inizialmente si è fatto uso della tecnica 	walkthrough per individuare gli errori. Dopodiché sono state avviate le procedure per la segnalazione e la gestione degli errori rilevati descritte nelle \docNameVersionNdP.
	In seguito si è proceduto a:
	\begin{itemize}
		\item correggere gli errori rilevati;
		\item compilare la lista di controllo utilizzando l'apposita sezione \emph{Bug Tracking Document} di Asana.
	\end{itemize}
	Una volta compilata la lista di controllo si è cominciato ad utilizzare la tecnica inspection.\\
	Si è quindi applicato il ciclo PDCA per migliorare i processi che hanno generato gli errori. Infine si sono applicate le metriche per i documenti descritte nella sezione 2.8.2 riportando i risultati nella sezione D dell'appendice di questo documento.
	Per quanto riguarda la verifica dei processi si sono svolte le attività descritte nelle \docNameVersionNdP{}. Si sono quindi applicate le metriche per i processi descritte nella sezione 2.8.1 di questo documento riportando i risultati nella sezione C dell'appendice di questo documento.
	
	\subsection{Revisione di Progettazione}
	Nel periodo precedente alla consegna della \emph{Revisione di Progettazione} sono state effettuate delle attività di verifica sia per i documenti sia per i processi.\\
	Per quanto riguarda la verifica dei documenti si è effettuata attività di analisi statica descritta nelle \docNameVersionNdP. Inizialmente si è fatto uso della tecnica 	walkthrough per individuare gli errori. Dopodiché sono state avviate le procedure per la segnalazione e la gestione degli errori rilevati descritte nelle \docNameVersionNdP.
	In seguito si è proceduto a:
	\begin{itemize}
		\item correggere gli errori rilevati;
		\item compilare la lista di controllo utilizzando l'apposita sezione \emph{Bug Tracking Document} di Asana.
	\end{itemize}
	Una volta compilata la lista di controllo si è cominciato ad utilizzare la tecnica inspection ponendo attenzione ai grafici riportati nella \docNameST.\\
	Si è quindi applicato il ciclo PDCA per migliorare i processi che hanno generato gli errori. Infine si sono applicate le metriche per i documenti descritte nella sezione 2.8.2 riportando i risultati nella sezione D dell'appendice di questo documento.
	Per quanto riguarda la verifica dei processi si sono svolte le attività descritte nelle \docNameVersionNdP{}. Si sono quindi applicate le metriche per i processi descritte nella sezione 2.8.1 di questo documento riportando i risultati nella sezione C dell'appendice di questo documento.
	
\pagebreak