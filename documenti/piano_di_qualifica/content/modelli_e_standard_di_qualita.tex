% =================================================================================================
% File:			modelli_e_standard_di_qualita.tex
% Description:	Definisce gli standard di qualità e i modelli utilizzati
% Created:		2014/01/11
% Author:		Ceccon Lorenzo
% Email:		ceccon.lorenzo@mashup-unipd.it
% =================================================================================================
% Modification History:
% Version		Modifier Date		Change											Author
% 0.0.1 		2014/01/11 			iniziata stesura appendice					Lorenzo C.
% =================================================================================================
%% Version		Modifier Date		Change											Author
% 1.0.1 		2015/03/17 			iniziata stesura appendice B				Nicola F.
% =================================================================================================
%

% CONTENUTO DEL CAPITOLO


\section{Modelli e standard di qualità}
  	\subsection{Standard ISO/IEC 9126}
  	Lo standard ISO/IEC 9126 è uno standard creato per delineare delle normative utili a descrivere un modello di qualità del software. Lo standard propone un approccio in cui viene posta attenzione al miglioramento dell'organizzazione e dei processi di una società di software, in modo da migliorare di conseguenza la qualità del prodotto software.\\
  	\begin{figure}[h]
		\centering
		\includegraphics[width=150mm]{images/iso_9126.png}
		\caption{Schema del modello di qualità ISO/IEC 9126}
		\label{fig:iso9126}
	\end{figure}
  	Lo standard ISO/IEC 9126 è suddiviso in quattro parti:
		\begin{itemize}
			\item \textbf{Modello di qualità: } la prima parte dello standard classifica il modello di qualità in sei caratteristiche generali e in varie sotto caratteristiche misurabili tramite l'utilizzo di metriche.\\
			Le sei caratteristiche generali e le relative sotto caratteristiche sono:
				\begin{itemize}
					\item \textbf{Funzionalità:} capacità di un prodotto software di fornire funzioni che soddisfano esigenze stabilite
						\begin{itemize}
							\item \textbf{Appropriatezza:} capacità del prodotto software di fornire un appropriato insieme di funzioni per i specifici compiti ed obiettivi prefissati all'utente;
							\item \textbf{Accuratezza:} capacità del prodotto software di fornire i risultati richiesti;
							\item \textbf{Interoperabilità:} capacità del prodotto software di interagire con i diversi sistemi specificati;
							\item \textbf{Conformità:} capacità del prodotto software di aderire agli standard e alle convenzioni appartenenti al settore in cui vengono applicati;
							\item \textbf{Sicurezza:} capacità del prodotto software di consentire l'accesso a dati e informazioni solamente alle persone autorizzate.
						\end{itemize}
					\item \textbf{Affidabilità:} capacità del prodotto software di mantenere uno specificato livello di prestazioni
						\begin{itemize}
							\item \textbf{Maturità:} capacità di un software di evitare che si verificano errori, malfunzionamenti o siano prodotti risultati non corretti;
							\item \textbf{Tolleranza agli errori:} capacità del software di mantenere un adeguato livello di prestazioni in presenza di malfunzionamenti;
							\item \textbf{Recuperabilità:} capacità di un prodotto di ripristinare il livello appropriato di prestazioni in seguito a un malfunzionamento;
							\item \textbf{Aderenza:} capacità di aderire a standard, regole e convenzioni inerenti all'affidabilità.
						\end{itemize}
					\item \textbf{Usabilità:} capacità del software di essere capito, appreso e usato dall'utente
						\begin{itemize}
							\item \textbf{Comprensibilità:} esprime la facilità di comprensione delle funzionalità del prodotto;
							\item \textbf{Apprendibilità:} capacità del software di essere appreso in tempo brevi;
							\item \textbf{Operabilità:} capacità di permettere agli utenti di utilizzare al software al fine di raggiungere i propri scopi;
							\item \textbf{Attrattiva:} capacità del prodotto di risultare interessante all'utente;
							\item \textbf{Conformità:} capacità del software di aderire a standard, regole e convenzioni relativi all'usabilità.
						\end{itemize}
					\item \textbf{Efficienza:} capacità di fornire prestazioni relativamente alla quantità di risorse usate
						\begin{itemize}
							\item \textbf{Comportamento rispetto al tempo:} capacità di fornire tempi di risposta, elaborazione e velocità di attraversamento ottimali in relazione alla funzione utilizzata;
							\item \textbf{Utilizzo delle risorse:} capacità del software di utilizzare adeguate quantità di risorse;
							\item \textbf{Conformità:} capacità del software di aderire a standard, regole e convenzioni relativi all'efficienza.
						\end{itemize}
					\item \textbf{Manutenibilità:} capacità del software di essere modificato apportando correzioni, miglioramenti o adattamenti
						\begin{itemize}
							\item \textbf{Analizzabilità:} esprime la facilità nell'analizzare il codice sorgente per ricercare errori; 
							\item \textbf{Modificabilità:} capacità del software di permettere l'implementazione di nuove modifiche;
							\item \textbf{Stabilità:} capacità del software di evitare effetti indesiderati a seguito di modifiche errate;
							\item \textbf{Testabilità:} capacità del software di eseguire facilmente la validazione delle modifiche apportate al software.
						\end{itemize}
					\item \textbf{Portabilità:} capacità del software di lavorare in diversi ambienti di lavoro
						\begin{itemize}
							\item \textbf{Adattabilità:} capacità del software di essere adattato a diversi ambienti senza dover applicare modifiche diverse da quelle fornite;
							\item \textbf{Installabilità:} capacità del software di essere installato in uno specificato ambiente;
							\item \textbf{Conformità:} capacità del prodotto software di aderire a standard, regole e convenzioni relativi alla portabilità;
							\item \textbf{Sostituibilità:} capacità del software di sostituire un altro software analogo per svolgere certi compiti.
						\end{itemize}
				\end{itemize}
			\item \textbf{Qualità esterne: } le metriche esterne applicabili al software, e quindi rilevabili tramite l'analisi dinamica, misurano i comportamento del prodotto sulla base dei test, dall'operatività e dall'osservazione durante la sua esecuzione;
			\item \textbf{Qualità interne: } le metriche interne, misurabili attraverso l'analisi statica, sono utili per prevedere il livello della qualità esterna ed in uso, poiché i suoi attributi interni influiscono su quelli esterni ed in uso. Permettono così di individuare anomalie prima che queste ultime possano influenzare la qualità del prodotto finale;
			\item \textbf{Qualità in uso: } la qualità in uso, raggiungibile solo dopo aver ottenuto la qualità interna ed esterna, fornisce metriche per misurare il grado di utilizzabilità del prodotto da parte dell'utente finale.
		\end{itemize}
		
	\subsection{Standard ISO/IEC 15504}
	Lo standard ISO/IEC 15504, conosciuto anche come SPICE, è un insieme di documenti tecnici per lo sviluppo di processi software, utili a valutare la dimensione dei processi tramite l'utilizzo di specifiche metriche. É derivato dallo standard ISO/IEC 12207 e da modelli di maturità quali Bootstrap, Trillium e il CMM.\\
	Lo standard definisce la dimensione del processo e la suddivide nelle seguenti cinque categorie:
		\begin{itemize}
			\item \textbf{Custormer/Supplier;}
			\item \textbf{Engineering;}
			\item \textbf{Support;}
			\item \textbf{Management;}
			\item \textbf{Organization.}
		\end{itemize}
	Per ogni processo, viene definito un livello di capacità dei processi definito da una scala di sei livelli e da nove attributi suddivisi nei vari livelli:
		\begin{itemize}
			\item \textbf{Level 5. Optimizing process:} il processo è predicibile ed in grado di adattarsi per raggiungere
obiettivi specifici
				\begin{itemize}
					\item \textbf{Process Innovation:} le modifiche ad un processo sono identificate ed implementate al fine di ottenere il miglioramento continuo nel raggiungimento degli obiettivi;
					\item \textbf{Process Optimization:} le modifiche alla definizione, gestione, attuazione di un processo sono controllate.
				\end{itemize}
			\item \textbf{Level 4. Predictable process:} il processo è stabilizzato ed è attuato all'interno di definiti limiti di controllo
				\begin{itemize}
					\item \textbf{Process Measurement:} i risultati raggiunti e le misure rilevate durante l'attuazione di un processo sono utilizzati per garantire il raggiungimento di specifici obiettivi;
					\item \textbf{Process Control:} un processo è controllato attraverso le misure di prodotto e di processo rilevate, al fine di migliorare le modalità di attuazione del processo stesso.
				\end{itemize}
			\item \textbf{Level 3. Established process:} il processo è attuato, pianificato e controllato sulla base di procedure standard basate sui principi dell'ingegneria del software
				\begin{itemize}
					\item \textbf{Process Definition:} l'attuazione di un processo, per raggiungere gli obiettivi, si basa sull'adozione di approcci standard;
					\item \textbf{Process Deployment:} l'attuazione di un processo, per raggiungere gli obiettivi, fa uso di risorse umane e tecniche appropriate.
				\end{itemize}
			\item \textbf{Level 2. Managed process:} il processo è attuato ma anche pianificato, tracciato, verificato ed aggiustato se necessario, sulla base di obiettivi ben definiti
				\begin{itemize}
					\item \textbf{Performance Management:} l'implementazione di un processo è pianificato e controllato al fine di produrre risultati coerenti agli obiettivi attesi;
					\item \textbf{Work Product Management:} l'implementazione di un processo è pianificato e controllato al fine di produrre risultati documentati, controllati e verificati in modo appropriato.
				\end{itemize}			
			\item \textbf{Level 1. Performed process:} il processo viene messo in atto e raggiunge i suoi obiettivi. Il risultato potrebbe non essere stato pianificato e tracciato rigorosamente
				\begin{itemize}
					\item \textbf{Process Performance:} capacità di un processo di raggiungere i suoi obiettivi trasformando input identificabili in output identificabili.
				\end{itemize}
			\item \textbf{Level 0. Incomplete process:} il processo non è stato implementato oppure non raggiunge gli obiettivi.
		\end{itemize}
	Ogni attributo è misurabile tramite l'utilizzo di una scala di valutazione divisa in quattro punti:
		\begin{itemize}
			\item \textbf{Not achieved (0-15\%);}
			\item \textbf{Partially achieved (15-50\%);}
			\item \textbf{Largely achieved (50-85\%);}
			\item \textbf{Fully achieved (85-100\%).}
		\end{itemize}
	Lo standard fornisce una guida per l'effettuazione di una valutazione formata da:
		\begin{itemize}
			\item Processo di valutazione;
			\item Modello per la valutazione;
			\item Strumenti per la valutazione.
		\end{itemize}
	Lo standard infine, stabilisce che per una corretta valutazione i verificatori debbano avere un buon livello di competenza e di esperienza.
	
	\subsection{Ciclo PDCA}
	Il ciclo PDCA, noto anche come ciclo di Deming, è un metodo di gestione iterativo a quattro fasi per il controllo e il miglioramento continuo dei processi.\\
	Le quattro fasi che lo compongono sono:
			\begin{itemize}
				\item \textbf{Plan:} stabilisce gli obiettivi e i processi necessari per ottenere risultati uguali a quelli attesi;
				\item \textbf{Do:} fase composta dall'attuazione del piano, dall'esecuzione del processo e dalla creazione del prodotto. Termina con una raccolta dei dati e creazione di grafici sul risultato di quanto ottenuto;
				\item \textbf{Check:} si confrontano i risultati ottenuti dalla fase precedente con i risultati stabiliti durante la pianificazione per verificare la presenza di differenze;
				\item \textbf{Act:} si effettuano correzioni laddove sono presenti differenze tra i risultati ottenuti e quelli previsti. Si determinano le cause delle discrepanze e dove c'è bisogno di applicare delle modifiche per ottenere un miglioramento del processo e di conseguenza del prodotto.
			\end{itemize}
			\begin{figure}[h]
				\centering
				\includegraphics[width=120mm]{images/pdca.png}
				\caption{Ciclo di miglioramento della qualità PDCA}
				\label{fig:pdca}
			\end{figure}
		
\pagebreak