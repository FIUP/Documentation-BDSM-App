% =================================================================================================
% File:			introduzione.tex
% Description:	Defiinisce la sezione relativa all'introduzione
% Created:		2014/12/05
% Author:		Ceccon Lorenzo
% Email:		ceccon.lorenzo@mashup-unipd.it
% =================================================================================================
% Modification History:
% Version		Modifier Date		Change											Author
% 0.0.1 		2014/12/05 			iniziata stesura documento di prova				Ceccon Lorenzo
% =================================================================================================
%

% CONTENUTO DEL CAPITOLO

\section{Introduzione}
	\subsection{Scopo del documento}
	Questo documento ha lo scopo di definire la strategia e descrivere le modalità di verifica e validazione che il gruppo MashUp intende adottare per lo sviluppo del progetto al fine di raggiungere gli obbiettivi qualitativi prefissati. Per perseguire questi obbiettivi è necessaria una costante attività di verifica in modo da permettere rilevare e risolvere eventuali anomalie.

	\subsection{Scopo del prodotto}
		\productScope

	\subsection{Glossario}
		\glossarioDesc

	\subsection{Riferimenti}
		\subsubsection{Normativi}
			\begin{itemize}
  				\item \textbf{Norme di progetto}: \docNameVersionNdP;
  				\item \textbf{Capitolato d'appalto C1:} \textit{BDSMApp: Big Data Social Monitoring App} \url{http://www.math.unipd.it/~tullio/IS-1/2014/Progetto/C1.pdf};
  				\item \textbf{Standard ISO/IEC 9126:} \url{http://en.wikipedia.org/wiki/ISO/IEC_9126};
  				\item \textbf{Standard ISO/IEC 15504:} \url{http://en.wikipedia.org/wiki/ISO/IEC_15504}.
			\end{itemize}

		\subsubsection{Informativi}
			\begin{itemize}
  				\item \textbf{Piano di progetto}: \docNameVersionPdP;
  				\item \textbf{Slides di Ingegneria del Software modulo A:} \url{http://www.math.unipd.it/~tullio/IS-1/2014/};
  				\item \textbf{SWEBOK 2004:} \textit{Chapter 11 - Software Quality} \url{http://www.computer.org/portal/web/swebok/html/ch11};
  				\item \textbf{\sommerville}: \textit{Chapters 8, 24, 26}.
			\end{itemize}
			\pagebreak