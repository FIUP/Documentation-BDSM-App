% =================================================================================================
% File:			pianificazione_dei_test.tex
% Description:	Definisce la pianificazione dei test da effettuare
% Created:		2014/03/05
% Author:		Ceccon Lorenzo
% Email:		ceccon.lorenzo@mashup-unipd.it
% =================================================================================================
% Modification History:
% Version		Modifier Date		Change											Author
% 0.0.1 		2014/01/11 			iniziata stesura appendice					Lorenzo C.
% =================================================================================================
%% Version		Modifier Date		Change											Author
% 1.0.1 		2015/03/17 			iniziata stesura appendice B				Nicola F.
% =================================================================================================
%


% CONTENUTO DEL CAPITOLO

\section{Pianificazione dei test}
	\subsection{Descrizione test}
		In seguito sono descritti tutti i test di validazione, sistema e di integrazione pianificati. I test di unità invece verranno inseriti successivamente.
Lo stato \textbf{N.I.} presente nelle tabelle sottostanti è da intendersi come non applicato, tali test verranno infatti svolti successivamente come descritto in \docNameVersionPdP.\\
Per una descrizione più dettagliata sui test si rimanda alla sezione dei test del documento \docNameVersionNdP.

	\subsection{Test di sistema}
		I test di sistema servono per verificare che il sistema software completamente integrato soddisfi tutti i requisiti software individuati e descritti nel documento \docNameVersionAdR.
		\subsubsection{Descrizione dei test di sistema}
			\begin{center}

			\def\arraystretch{1.5}
			\bgroup
			\begin{longtable}{| p{2cm} | p{7cm} | p{1.5cm} | p{2cm} |}
					\hline
					\textbf{Test} & \textbf{Descrizione} & \textbf{Stato} & \textbf{Requisito}\\
					\hline						
					TSF1 & Viene verificato che il sistema permetta, all'utente non autenticato, la registrazione al servizio & N.I. & ROF1\\
					\hline
					TSF2 & Viene verificato che il sistema permetta all'utente di autenticarsi & N.I. & ROF2\\
					\hline
					TSF3 & Viene verificato che il sistema permetta, all'utente autenticato, di accedere al menù informazioni personali & N.I. & ROF3\\
					\hline
					TSF4 & Viene verificato che il sistema permetta, all'utente autenticato, di effettuare la deautenticazione & N.I. & ROF4\\
					\hline
					TSF5 & Viene verificato che il sistema permetta, all'utente autenticato, di visualizzare tutte le Recipe presenti nel sistema & N.I. & ROF5\\
					\hline
					TSF6 & Viene verificato che il sistema permetta all'utente di gestire le proprie View & N.I. & ROF6\\
					\hline
					TSF7 & Viene verificato che il sistema permetta di richiedere l'inserimento di una nuova Recipe & N.I. & ROF7\\
					\hline
					TSF8 & Viene verificato che il sistema permetta, all'utente amministratore autenticato, di accedere all'area riservata del sistema & N.I. & ROF8\\
					\hline
					TSF10 & Viene verificato che il sistema permetta, all'utente amministratore, di gestire la richiesta di nuove Recipe & N.I. & ROF10\\
					\hline
					TSF11 & Viene verificato che il sistema fornisca una serie di servizi REST & N.I. & ROF11\\
					\hline
					TSP1 & Viene verificato che l'utente visualizzi le proprie View entro 10 secondi & N.I. & RDP1\\
					\hline
					TSP2 & Viene verificato che l'interfaccia web utilizzi un design di tipo responsive & N.I. & RFP2\\
					\hline
					TSQ1 & Viene verificato che sia disponibile un manuale per l'utente & N.I. & ROQ1\\
					\hline
					TSQ2 & Viene verificato che tutto il codice rispetti le norme e le metriche descritte nel \docNameVersionPdQ{} e \docNameVersionNdP & N.I. & ROQ2\\
					\hline
					TSQ3 & Viene verificato che sia disponibile un manuale per l'uso dei servizi REST & N.I. & ROQ3\\
					\hline
					TSV1 & Viene verificato che il sistema utilizzi gli strumenti offerti da Google Cloud Platform & N.I. & RDV1\\
					\hline
					TSV2 & Viene verificato che il linguaggio di programmazione utilizzato per il server sia Python & N.I. & RDV2\\
					\hline
					TSV3 & Viene verificato che il codice soggetto sia soggetto a versionamento tramite il modello di branching descritto nelle \docNameVersionNdP & N.I. & RDV3\\
					\hline
					TSV4 & Viene verificato che l'interfaccia web sia di tipo single-page & N.I. & RDV4\\
					\hline
					TSV5 & Viene verificato che l'interfaccia web funzioni con i principali browser attualmente sul mercato& N.I. & ROV5\\
					\hline
					TSV6 & Viene verificato che il sistema utilizzi librerie esterne per effettuare le chiamate alle API dei diversi social network & N.I. & ROV6\\
					\hline
					TSV7 & Viene verificato che il sistema utilizzi librerie esterne per generare i grafici necessari alle View & N.I. & ROV7\\
					\hline
			\caption{Tracciamento test di sistema - requisiti}
			\end{longtable}
				\egroup
\end{center}
		
	\subsection{Test di integrazione}
		I test di integrazione vengono utilizzati per verificare che tutti i componenti del sistema siano integrati correttamente tra di loro e che il flusso dati all'interno del sistema sia corretto.\\
		Si è deciso di fare affidamento ad una strategia di integrazione di tipo incrementale in modo da permetterci di sviluppare e verificare più componenti in parallelo.\\
		La strategia incrementale ci permette anche di effettuare la ricerca dei difetti in maniera più precisa, infatti nel caso si presenti un errore, questo sarà probabilmente causato dall'ultima componente inserita e permettendoci di ritornare ad uno stato del sistema corretto.\\
		Viene utilizzato il metodo bottom-up, cioè vengono prima integrate le componenti con minor dipendenze funzionali e con maggiori funzionalità, che corrispondono, quindi, ai requisiti obbligatori. In questo modo è possibile ottenere un prodotto funzionante, che soddisfa tutti i requisiti obbligatori, il prima possibile. Sarà, quindi, necessario testare più volte le componenti per assicurarci che il prodotto software finale non contenga difetti.\\ Successivamente si procederà ad aggiungere le componenti che corrispondono ai requisiti desiderabili e opzionali.
		 
		\subsubsection{Descrizione dei test di integrazione}
			\begin{center}

			\def\arraystretch{1.5}
			\bgroup
			\begin{longtable}{| p{2.5cm} | p{5cm} | p{3.5cm} | p{1.5cm} |}
					\hline
					\textbf{Test} & \textbf{Descrizione} & \textbf{Componente} & \textbf{Stato}\\
					\hline						
					TO DO & TO DO & TO DO & TO DO\\
					\hline
			\caption{Tabella test di integrazione}
			\end{longtable}
				\egroup
\end{center}

	\subsection{Test di validazione}
		I test di validazione vengono utilizzati per accertarsi che il prodotto finale sviluppato sia conforme alle attese.\\
		Per ogni test viene riportata una descrizione contenente i passi che l'utente deve seguire per verificare che i requisiti siano soddisfatti. Il tracciamento tra test di validazione e i requisiti correlati viene riportati nel documento \docNameVersionAdR.
		
		\subsubsection{TVF1}
			L'utente vuole verificare che si possa registrare al sistema. All'utente è richiesto di:
			\begin{itemize}
				\item inserire un username;
				\item inserire una email;
				\item inserire una password;
				\item inserire una password di conferma;
				\item verificare che il sistema avverta l'utente in fase di compilazione nel caso in cui i dati inseriti non siano conformi alle norme del sistema;
				\item confermare la registrazione;
				\item verificare che il sistema avverta l'utente nel caso in cui i dati inseriti non siano conformi alle norme del sistema;
				\item verificare che, nel caso la registrazione abbia avuto esito positivo, venga inviata un email di conferma.
			\end{itemize}
			
		\subsubsection{TVF1.1}
			L'utente vuole verificare che l'inserimento di un username avvenga correttamente. All'utente è richiesto di:
			\begin{itemize}
				\item inserire un username;
				\item verificare che, se lo username è già presente nel database, venga visualizzato un errore;
				\item verificare che, nel caso lo username non sia conforme alle norme del sistema, venga visualizzato un errore;
				\item verificare che, nel caso in cui lo username sia conforme con le norme di sistema, non vengano visualizzati errori.
			\end{itemize}
			
		\subsubsection{TVF1.2}
			L'utente vuole verificare che l'inserimento di una email avvenga correttamente. All'utente è richiesto di:
			\begin{itemize}
				\item inserire una email;
				\item verificare che, se la email è già presente nel database, venga visualizzato un errore;
				\item verificare che, nel caso la email non sia conforme alle norme del sistema, venga visualizzato un errore;
				\item verificare che, nel caso la email sia conforme con le norme di sistema, non vengano visualizzati errori.
			\end{itemize}
			
		\subsubsection{TVF1.3}
			L'utente vuole verificare che l'inserimento di una password avvenga correttamente. All'utente è richiesto di:
			\begin{itemize}
				\item inserire una password;
				\item verificare che, nel caso la password contenga lo username al suo interno, venga visualizzato un errore;
				\item verificare che, nel caso la password non sia conforme alle norme del sistema, venga visualizzato un errore;
				\item verificare che, nel caso la password sia conforme alle norme del sistema, non vengano visualizzati errori.
			\end{itemize}
			
		\subsubsection{TVF1.4}
			L'utente vuole verificare che l'inserimento della password di conferma avvenga correttamente. All'utente è richiesto di:
			\begin{itemize}
				\item inserire una password di conferma;
				\item verificare che, nel caso in cui la password di conferma non sia uguale alla password precedentemente inserita, venga visualizzato un errore;
				\item verificare che, nel caso in cui la password di conferma coincida con la password precedentemente inserita, non vengano visualizzati errori.
			\end{itemize}
			
		\subsubsection{TVF1.8}
			L'utente vuole verificare che, nel caso la registrazione abbia avuto esito positivo, venga inviata un email di conferma. All'utente è richiesto di:
			\begin{itemize}
				\item completare la registrazione;
				\item accedere alla propria casella email;
				\item verificare che sia arrivata l'email di conferma dal sistema.
			\end{itemize}
			
		\subsubsection{TVF2}
			L'utente vuole verificare che si possa autenticarsi al sistema. All'utente è richiesto di:
			\begin{itemize}
				\item inserire un username;
				\item inserire una password;
				\item confermare l'autenticazione;
				\item verificare che il sistema avverta l'utente nel caso in cui i dati inserirti dall'utente risultino errati;
				\item verificare che il sistema, nel caso in cui la form sia compilata correttamente, non mostri errori;
				\item verificare che il sistema, in caso di autenticazione avvenuta con successo, aggiorni la home screen dell'utente autenticato.
			\end{itemize}
			
		\subsubsection{TVF2.4}
			L'utente vuole verificare che il sistema avverta l'utente nel caso in cui i dati inseriri dall'utente risultino errati. All'utente è richiesto di:
			\begin{itemize}
				\item confermare l'autenticazione;
				\item verificare che il sistema avverta l'utente nel caso in cui il username inserito non sia presente nel sistema;
				\item verificare che il sistema avverta l'utente nel caso in cui la password inserita non corrisponda a quella relativa allo username inserito.
			\end{itemize}erificare che, nel caso la nuova password sia conforme alle norme del sistema, non venga visualizzati errori.
			
		\subsubsection{TVF3}
			L'utente vuole verificare che si possa visualizzare le proprie informazioni personali. All'utente è richiesto di:
			\begin{itemize}
				\item effettuare l'autenticazione;
				\item aprire il menù delle informazioni personali;
				\item verificare che sia possibile visualizzare i propri dati;
				\item verificare che sia possibile visualizzare le proprie statistiche;
				\item verificare che sia possibile modificare i propri dettagli personali.
			\end{itemize}
			
		\subsubsection{TVF3.1}
			L'utente vuole verificare che si possa visualizzare i propri dati. All'utente è richiesto di:
			\begin{itemize}
				\item effettuare l'autenticazione;
				\item aprire il menù delle informazioni personali;
				\item verificare che l'utente non visualizzi i propri dati personali prima di aver premuto l'apposito pulsante; 
				\item premere il pulsante di visualizzazione dei dati personali;
				\item verificare che il sistema reperisca i dati personali dell'utente;
				\item verificare che l'utente visualizzi il proprio username;
				\item verificare che l'utente visualizzi la propria email;
				\item verificare che l'utente visualizzi l'ultimo accesso effettuato.
			\end{itemize}
			
		\subsubsection{TVF3.2}
			L'utente vuole verificare che si possa visualizzare le proprie statistiche. All'utente è richiesto di:
			\begin{itemize}
				\item effettuare l'autenticazione;
				\item aprire il menù delle informazioni personali;
				\item verificare che l'utente non visualizzi le proprie statistiche prima di aver premuto l'apposito pulsante; 
				\item premere il pulsante di visualizzazione delle statistiche;
				\item verificare che il sistema reperisca le informazioni dell'utente;
				\item verificare che si visualizzi il numero di View attive;
				\item verificare che si visualizzi il numero di Recipe disponibili.
			\end{itemize}
			
		\subsubsection{TVF3.3}
			L'utente vuole verificare che si possa modificare i propri dettagli personali. All'utente è richiesto di:
			\begin{itemize}
				\item effettuare l'autenticazione;
				\item aprire il menù delle informazioni personali;
				\item selezionare il pulsante di modifica dei dati personali;
				\item verificare che si apra una finestra contenente la form per modificare i propri dati personali;
				\item verificare che l'utente possa inserire un nuovo username;
				\item verificare che l'utente possa inserire un nuovo indirizzo email;
				\item verificare che l'utente possa cambiare la propria password;
				\item verificare che il sistema avverta l'utente in fase di compilazione nel caso in cui i dati inseriti non siano conformi alle norme del sistema;
				\item premere il pulsante di conferma delle modifiche;
				\item verificare che, alla pressione del pulsante di conferma delle modifiche, il sistema analizzi le conformità dei dati inseriti;
				\item verificare che l'utente possa eliminare il proprio account.
			\end{itemize}
		
		\subsubsection{TVF3.3.2}
			L'utente vuole verificare che si possa modificare il proprio username. All'utente è richiesto di:
			\begin{itemize}
				\item effettuare l'autenticazione;
				\item aprire il menù delle informazioni personali;
				\item selezionare il pulsante di modifica dei dati personali;
				\item inserire un nuovo username nell'apposito campo;
				\item verificare che, nel caso il nuovo username coincidi con quello vecchio, venga visualizzato un errore;
				\item verificare che, se lo username è già presente nel database, venga visualizzato un errore;
				\item verificare che, nel caso lo username non sia conforme alle norme del sistema, venga visualizzato un errore;
				\item verificare che, nel caso in cui lo username sia conforme con le norme di sistema, non vengano visualizzati errori.
			\end{itemize}
			
		\subsubsection{TVF3.3.3}
			L'utente vuole verificare che si possa modificare la propria email. All'utente è richiesto di:
			\begin{itemize}
				\item effettuare l'autenticazione;
				\item aprire il menù delle informazioni personali;
				\item selezionare il pulsante di modifica dei dati personali;
				\item inserire una nuova email nell'apposito campo;
				\item verificare che, nel caso la nuova email coincidi con quella vecchia, venga visualizzato un errore;
				\item verificare che, se la email è già presente nel database, venga visualizzato un errore;
				\item verificare che, nel caso la email non sia conforme alle norme del sistema, venga visualizzato un errore;
				\item verificare che, nel caso la email sia conforme con le norme di sistema, non vengano visualizzati errori.
			\end{itemize}
			
		\subsubsection{TVF3.3.4}
			L'utente vuole verificare che si possa modificare la propria password. All'utente è richiesto di:
			\begin{itemize}
				\item effettuare l'autenticazione;
				\item aprire il menù delle informazioni personali;
				\item selezionare il pulsante di modifica dei dati personali;
				\item inserire la password corrente nell'apposito campo;
				\item inserire la nuova password nell'apposito campo;
				\item inserire la password di conferma nell'apposito campo;
				\item verificare che, nel caso la nuova password contenga lo username al suo interno, venga visualizzato un errore;
				\item verificare che, nel caso la nuova password non sia conforme alle norme del sistema, venga visualizzato un errore;
				\item verificare che, nel caso la nuova password non coincida con quella di conferma, venga visualizzato un errore;
				\item verificare che, nel caso la nuova password sia conforme alle norme del sistema e la vecchia password sia stata inserita correttamente, non vengano visualizzati errori.
			\end{itemize}
			
		\subsubsection{TVF3.3.7}
			L'utente vuole verificare che l'utente possa eliminare il proprio account. All'utente è richiesto di:
			\begin{itemize}
				\item effettuare l'autenticazione;
				\item aprire il menù delle informazioni personali;
				\item premere il pulsante di eliminazione dell'account;
				\item premere il pulsante di conferma dell'eliminazione dell'account;
				\item verificare che, alla conferma dell'eliminazione dell'account, il sistema rimuova tutti i dati associati a quell'utente;
				\item verificare che il sistema, una volta confermata l'eliminazione dell'account, riporti l'utente alla home page di default del sistema per gli utenti non registrati.
			\end{itemize}
			
		\subsubsection{TVF4}
			L'utente vuole verificare che si possa effettuare la deautenticazione dal sistema. All'utente è richiesto di:
			\begin{itemize}
				\item effettuare l'autenticazione nel caso non si fosse attualmente autenticati;
				\item accedere alla home screen;
				\item premere il pulsante di deautenticazione;
				\item verificare che, alla pressione del pulsante di deautenticazione, la sessione termini.
			\end{itemize}
			
		\subsubsection{TVF4.1}
			L'utente vuole verificare che, alla pressione del pulsante di deautenticazione, la sessione termini. All'utente è richiesto di:
			\begin{itemize}
				\item effettuare l'autenticazione nel caso non si fosse attualmente autenticati;
				\item accedere alla home screen;
				\item premere il pulsante di deautenticazione;
				\item verificare che il sistema chieda la conferma di logout se la sessione è valida;
				\item verificare che, se non si conferma il logout, il sistema non effettui la deautenticazione;
				\item verificare che, se si conferma il logout, il sistema stampi un messaggio d'errore se la sessione è scaduta;
				\item verificare che il sistema reindirizzi l'utente alla schermata di login.
			\end{itemize}
			
		\subsubsection{TVF5}
			L'utente vuole verificare che nella home screen dell'utente autenticato siano mostrate tutte le Recipe presenti nel sistema. All'utente è richiesto di:
			\begin{itemize}
				\item effettuare l'autenticazione;
				\item accedere alla home screen;
				\item verificare che il sistema recuperi l'elenco delle Recipe dal database;
				\item verificare che sia possibile visualizzare tutte le metriche di una Recipe.
			\end{itemize}
			
		\subsubsection{TVF5.1}
			L'utente vuole verificare che il sistema recuperi l'elenco delle Recipe dal database. All'utente è richiesto di:
			\begin{itemize}
				\item effettuare l'autenticazione;
				\item accedere alla home screen;
				\item verificare che il sistema fornisca per ogni Recipe il titolo e, se presente, la descrizione;
				\item verificare che il sistema fornisca per ogni Recipe due pulsanti per scegliere se visualizzare le metriche della Recipe o effettuare un confronto tra le metriche.
			\end{itemize}
			
		\subsubsection{TVF5.2}
			L'utente vuole verificare che sia possibile visualizzare tutte le metriche di una Recipe. All'utente è richiesto di:
			\begin{itemize}
				\item effettuare l'autenticazione;
				\item accedere alla home screen;
				\item verificare che la visualizzazione delle metriche sia suddivisa per categoria;
				\item verificare che il sistema, per ogni metrica, fornisca il nome, la descrizione se presente e la tipologia della metrica;
				\item verificare che il sistema, per ogni metrica, fornisca il pulsante per accedere alle visualizzazione delle View presenti per quella metrica.				
			\end{itemize}
			
		\subsubsection{TVF5.3}
			L'utente vuole verificare che il sistema fornisca tutte le View, di una metrica della categoria Facebook, predisposte dal sistema. All'utente è richiesto di:
			\begin{itemize}
				\item effettuare l'autenticazione;
				\item accedere alla home screen;
				\item premere il pulsante per la visualizzazione delle View di una metrica della categoria Facebook;
				\item verificare che siano mostrati i grafici relativi a tale metrica.
			\end{itemize}
			
		\subsubsection{TVF5.3.1}
			L'utente vuole verificare che vengano mostrati i grafici relativi ad una metrica della categoria Facebook. All'utente è richiesto di:
			\begin{itemize}
				\item effettuare l'autenticazione;
				\item accedere alla home screen;
				\item premere il pulsante per la visualizzazione delle View di una metrica della categoria Facebook;
				\item verificare che sia visualizzato un Line Chart contenente l'andamento dei "likes" e dei "talking about" di una pagina;
				\item verificare che sia visualizzato un Pie Chart che per tutti gli eventi di una pagina mostri le percentuali di persone che hanno messo partecipa, forse o rifiuta;
				\item verificare che sia visualizzato un Pie Chart che mostri la percentuale di commenti di terzi a tutti i post di una pagina rispetto la percentuale di commenti della pagina stessa;
				\item verificare che sia visualizzato un Pie Chart che mostri la percentuale di commenti della pagina e di terzi rispetto a tutti i post di terzi sulla pagina in questione;
				\item verificare che sia visualizzato un Map Chart che illustri le zone nella quale si sono creati degli eventi di una pagina e mostra dei cerchi di diverse dimensioni a seconda della media di partecipanti agli eventi di quella zona;
				\item verificare che sia visualizzato un Bar Chart che mostri la media di commenti delle pagina per ogni post effettuato dalla stessa;
				\item verificare che sia visualizzato un Line Chart che mostri l'andamento del numero di post giornalieri di una pagina.
			\end{itemize}
			
		\subsubsection{TVF5.4}
			L'utente vuole verificare che il sistema fornisca tutte le View, di una metrica della categoria Twitter, predisposte dal sistema. All'utente è richiesto di:
			\begin{itemize}
				\item effettuare l'autenticazione;
				\item accedere alla home screen;
				\item premere il pulsante per la visualizzazione delle View di una metrica della categoria Twitter;
				\item verificare che siano mostrati i grafici relativi a tale metrica.
			\end{itemize}
			
		\subsubsection{TVF5.4.1}
			L'utente vuole verificare che vengano mostrati i grafici relativi ad una metrica della categoria Twitter. All'utente è richiesto di:
			\begin{itemize}
				\item effettuare l'autenticazione;
				\item accedere alla home screen;
				\item premere il pulsante per la visualizzazione delle View di una metrica della categoria Twitter;
				\item verificare che sia visualizzato un Bar Chart orizzontale che mostri un riepilogo istantaneo dei Tweet e dei Follower di un singolo utente;
				\item verificare che sia visualizzato un Line Chart che indichi il numero di Tweet e il numero di Follower dell’utente;
				\item verificare che sia visualizzato un Map Chart che illustra l'area geografica di appartenenza dei Tweet con un particolare hashtag;
				\item verificare che sia visualizzato un Line Chart che mostri il numero di Tweet con un determinato hashtag;
				\item verificare che sia visualizzato un Pie Chart che illustri quanti Tweet, quanti Preferiti e quanti Retweet ha un utente;
				\item verificare che sia visualizzato un Radar Chart che illustri in che momento della giornata vengono fatti i Tweet;
				\item verificare che sia visualizzato un Pie Chart che indichi su che tipo di piattaforma gli utenti hanno twittato;
			\end{itemize}
			
		\subsubsection{TVF5.5}
			L'utente vuole verificare che il sistema fornisca tutte le View, di una metrica della categoria Instagram, predisposte dal sistema. All'utente è richiesto di:
			\begin{itemize}
				\item effettuare l'autenticazione;
				\item accedere alla home screen;
				\item premere il pulsante per la visualizzazione delle View di una metrica della categoria Instagram;
				\item verificare che siano mostrati i grafici relativi a tale metrica.
			\end{itemize}
			
		\subsubsection{TVF5.5.1}
			L'utente vuole verificare che vengano mostrati i grafici relativi ad una metrica della categoria Instagram. All'utente è richiesto di:
			\begin{itemize}
				\item effettuare l'autenticazione;
				\item accedere alla home screen;
				\item premere il pulsante per la visualizzazione delle View di una metrica della categoria Instagram;
				\item verificare che sia visualizzato un Bar Chart che mostri il numero di like e il numero di commenti per ogni post di un utente;
				\item verificare che sia visualizzato un Line Chart che mostri i nuovi follower di un utente per giorno e mostri i momenti in cui l'utente ha pubblicato un post tramite una linea verticale;
				\item verificare che sia visualizzato un Line Chart che mostri il numero di like e i nuovi commenti ricevuti da ogni utente e mostri i momenti in cui l'utente ha pubblicato un post tramite una linea verticale;
				\item verificare che sia visualizzato un Line Chart che illustri l'andamento dei like e quello dei commenti ricevuti da un utente fratto il numero di follower che possiede;
				\item verificare che sia visualizzato un Line Chart che mostri il numero di like ricevuti al giorno da un utente fratto il numero di post giornalieri dello stesso utente;
				\item verificare che sia visualizzato un Line Chart che mostri il numero di post e il numero di utenti che hanno utilizzato un determinato hashtag;
				\item verificare che sia visualizzato un Line Chart che mostri il numero di like e il numero di commenti ricevuto dai post che contengono un determinato hashtag;
				\item verificare che sia visualizzato un Map Chart che mostri le zone geografiche dove sono stati fatti i post contenenti un determinato hashtag;
				\item ROF5.5.1.9 ??;
				\item verificare che sia visualizzato un Radar Chart che illustri, per i vari momenti della giornata, la frequenza con cui un utente posta;
			\end{itemize}
			
		\subsubsection{TVF5.6}
			L'utente vuole verificare che il sistema permetta all'utente di effettuare il confronto tra le metriche. All'utente è richiesto di:
			\begin{itemize}
				\item effettuare l'autenticazione;
				\item accedere alla home screen;
				\item selezionare una categoria tra Facebook, Twitter e Instagram;
				\item selezionare un tipo di metrica tra quelle presenti nella categoria selezionata;
				\item verificare che vengano visualizzate tutte le metriche del tipo e della categoria selezionati;
				\item selezionare delle metriche tra quelle visualizzate;
				\item verificare che sia possibile generare un confronto tra le metriche selezionate;
			\end{itemize}
			
		\subsubsection{TVF5.6.4}
			L'utente vuole verificare che il sistema generi il confronto tra le metriche selezionate. All'utente è richiesto di:
			\begin{itemize}
				\item effettuare l'autenticazione;
				\item accedere alla home screen;
				\item selezionare una categoria tra Facebook, Twitter e Instagram;
				\item verificare che sia possibile cliccare il pulsante per effettuare il confronto;
				\item verificare che venga visualizzata la View di confronto;
				\item selezionare delle metriche tra quelle visualizzate;
				\item verificare che, nel caso non siano presenti dati sufficienti per generare la View di confronto, venga visualizzato un messaggio di avvertimento e si ritorni alla pagina di selezione delle metriche;
			\end{itemize}
			
		\subsubsection{TVF6}
			L'utente vuole verificare che il sistema permetta all'utente di gestire le View che più preferisce. All'utente è richiesto di:
			\begin{itemize}
				\item effettuare l'autenticazione;
				\item accedere alla home screen;
				\item verificare che il sistema permetta all'utente di visualizzare tutte le View che sono state marcate da lui come preferite;
				\item verificare che il sistema permetta all'utente di aggiungere qualsiasi View a quelle preferite cliccando il pulsante vicino alla View che vuole aggiungere;
				\item verificare che il sistema permetta all'utente di rimuovere le View che ha aggiunto tra le preferite cliccando il pulsante vicino alla View che vuole rimuovere;
				\item verificare che l'utente possa inserire come preferite un numero illimitato di View.
			\end{itemize}
			
		\subsubsection{TVF6.1}
			L'utente vuole verificare che il sistema permetta all'utente di visualizzare tutte le View che sono state marcate da lui come preferite. All'utente è richiesto di:
			\begin{itemize}
				\item effettuare l'autenticazione;
				\item accedere alla home screen;
				\item accedere al menù delle View preferite;
				\item verificare che il sistema avvisi l'utente nel caso non ci fossero View salvate come preferite;
				\item verificare che il sistema mostri tutte le View preferite dell'utente nel caso ce ne sia almeno una.
			\end{itemize}
			
		\subsubsection{TVF6.2}
			L'utente vuole verificare che il sistema permetta all'utente di aggiungere qualsiasi View a quelle preferite cliccando il pulsante vicino alla View che vuole aggiungere. All'utente è richiesto di:
			\begin{itemize}
				\item effettuare l'autenticazione;
				\item accedere alla home screen;
				\item premere il pulsante di aggiunta della View tra le preferite;
				\item verificare che il sistema mostri un errore nel caso non sia stata aggiunta correttamente una View tra quelle preferite;
				\item verificare che il sistema abbia aggiunto correttamente la View tra i preferiti.
			\end{itemize}
			
		\subsubsection{TVF6.3}
			L'utente vuole verificare che il sistema permetta all'utente di rimuovere le View che  ha aggiunto tra le preferite cliccando il pulsante vicino alla View che vuole rimuovere. All'utente è richiesto di:
			\begin{itemize}
				\item effettuare l'autenticazione;
				\item accedere alla home screen;
				\item accedere al menù delle View preferite;
				\item premere il pulsante di rimozione della View dalle preferite;
				\item verificare che il sistema mostri un errore nel caso non sia stata rimossa correttamente una View tra quelle preferite;
				\item verificare che il sistema mostri abbia rimosso correttamente una View da quelle preferite.
			\end{itemize}
			
		\subsubsection{TVF7}
			L'utente vuole verificare che il sistema permetta all'utente di richiedere l'inserimento di una nuova Recipe. All'utente è richiesto di:
			\begin{itemize}
				\item effettuare l'autenticazione;
				\item accedere alla sezione per la richiesta di una nuova Recipe;
				\item verificare che l'utente possa generare la richiesta dall'apposito form;
				\item verificare che il sistema invii una notifica agli amministratori che è stata inserita  una nuova richiesta di Recipe.
			\end{itemize}
			
		\subsubsection{TVF7.1}
			L'utente vuole verificare che il sistema permetta all'utente di generare la richiesta dall'apposito form. All'utente è richiesto di:
			\begin{itemize}
				\item effettuare l'autenticazione;
				\item accedere alla sezione per la richiesta di una nuova Recipe;
				\item verificare che il sistema controlli che l'utente inserisca tutti i parametri richiesti dalla form correttamente;
				\item premere il pulsante di invio della richiesta;
				\item verificare che la richiesta fallisca se i dati inseriti sono errati o si sono lasciati campi vuoti.
			\end{itemize}
			
		\subsubsection{TVF7.1.1}
			L'utente vuole verificare che il sistema controlli che l'utente inserisca tutti i parametri richiesti dalla form correttamente. All'utente è richiesto di:
			\begin{itemize}
				\item effettuare l'autenticazione;
				\item accedere alla sezione per la richiesta di una nuova Recipe;
				\item inserire il titolo della Recipe;
				\item inserire almeno una metrica nella richiesta della Recipe;
				\item selezionare la categoria della metrica tra Facebook, Twitter e Instagram;
				\item selezionare la tipologia di metrica in base alla categoria scelta;
				\item compilare i campi in base alla tipologia e alla categoria di metrica scelta;
				\item verificare che il sistema avvisi l'utente in modo istantaneo se un campo è stato compilato correttamente;
				\item verificare che il sistema avvisi l'utente in modo istantaneo se un campo obbligatorio è stato lasciato vuoto.
			\end{itemize}
			
		\subsubsection{TVF8}
			L'utente vuole verificare che il sistema permetta ad un utente con privilegi di amministratore di usufruire di tutte le funzioni offerte agli amministratori. All'utente è richiesto di:
			\begin{itemize}
				\item effettuare l'autenticazione;
				\item accedere all'area riservata;
				\item verificare che un amministratore può accedere al pannello per inserire una nuova Recipe;
				\item verificare che il sistema richieda la conferma dei dati inseriti prima di creare la nuova Recipe;
				\item verificare che l'amministratore possa eliminare una Recipe memorizzata nel sistema dalla schermata di elenco delle Recipe.
			\end{itemize}
			
		\subsubsection{TVF8.1}
			L'utente vuole verificare che il sistema permetta ad un utente con privilegi di amministratore di accedere al pannello per inserire una nuova Recipe. All'utente è richiesto di:
			\begin{itemize}
				\item effettuare l'autenticazione;
				\item accedere all'area riservata;
				\item verificare che sia richiesto l'inserimento di determinati parametri per individuare il contesto dei dati richiesti dalla Recipe.
			\end{itemize}
			
		\subsubsection{TVF8.1.1}
			L'utente vuole verificare che il sistema richieda l'inserimento di determinati parametri per individuare il contesto dei dati richiesti dalla Recipe. All'utente è richiesto di:
			\begin{itemize}
				\item effettuare l'autenticazione;
				\item accedere all'area riservata;
				\item inserire il titolo della Recipe;
				\item inserire almeno una metrica nella Recipe;
				\item selezionare la categoria della metrica tra Facebook, Twitter e Instagram;
				\item selezionare la tipologia di metrica in base alla categoria scelta;
				\item compilare i campi in base alla tipologia e alla categoria di metrica scelta. 
			\end{itemize}
			
		\subsubsection{TVF8.2}
			L'utente vuole verificare che il sistema richieda la conferma dei dati inseriti prima di creare la nuova Recipe. All'utente è richiesto di:
			\begin{itemize}
				\item effettuare l'autenticazione;
				\item accedere all'area riservata;
				\item inserire i dati della nuova Recipe nel form;
				\item verificare che sia visualizzato un messaggio d'errore nel caso in cui venga confermata una Recipe che non contiene almeno una metrica per categoria;
				\item verificare che sia visualizzato un messaggio d'errore nel caso in cui venga confermata una Recipe in cui l'utente non ha inserito dei parametri specificati nel requisito ROF8.1.1.4;
				\item verificare che nel caso venga confermata una nuova Recipe, il sistema salvi i suoi parametri nel database e reindirizzi l'amministratore all'elenco delle Recipe.
			\end{itemize}
			
		\subsubsection{TVF8.3}
			L'utente vuole verificare che l'amministratore possa eliminare una Recipe memorizzata nel sistema dalla schermata di elenco delle Recipe. All'utente è richiesto di:
			\begin{itemize}
				\item effettuare l'autenticazione;
				\item accedere all'area riservata;
				\item premere il pulsante di eliminazione posto di fianco alla Recipe che si desidera eliminare;
				\item verificare che il sistema non provvedi all'eliminazione di tutte le informazioni legate alla Recipe senza aver confermato prima l'eliminazione;
				\item verificare che il sistema provvedi all'eliminazione di tutte le informazioni legate alla Recipe che si desidera eliminare alla pressione del tasto di conferma eliminazione;
				\item verificare che il sistema provvedi ad eliminare tutti i dati grezzi dal database legati alla Recipe.
			\end{itemize}
			
		\subsubsection{TVF9}
			L'utente vuole verificare che l'amministratore possa gestire gli utenti. All'utente è richiesto di:
			\begin{itemize}
				\item effettuare l'autenticazione;
				\item accedere alla home screen;
				\item accedere all'elenco degli utenti;
				\item verificare che il sistema permetta all'utente amministratore di modificare i permessi di un utente;
				\item verificare che il sistema permetta all'utente amministratore di eliminare un utente dal sistema;
				\item verificare che il sistema permetta all'utente amministratore di confermare le modifiche applicate.
			\end{itemize}
			
		\subsubsection{TVF9.1}
			L'utente vuole verificare che il sistema permetta all'utente amministratore di modificare i permessi di un utente. All'utente è richiesto di:
			\begin{itemize}
				\item effettuare l'autenticazione;
				\item accedere alla home screen;
				\item accedere all'elenco degli utenti;
				\item verificare che, alla pressione del tasto di modifica dei permessi, le modifiche vengano applicate solo localmente in attesa che vengano confermate.
			\end{itemize}
			
		\subsubsection{TVF9.2}
			L'utente vuole verificare che il sistema permetta all'utente amministratore di eliminare un utente dal sistema. All'utente è richiesto di:
			\begin{itemize}
				\item effettuare l'autenticazione;
				\item accedere alla home screen;
				\item accedere all'elenco degli utenti;
				\item verificare che, alla pressione del tasto di modifica dei permessi, le modifiche vengano applicate solo localmente in attesa che vengano confermate.
			\end{itemize}
			
		\subsubsection{TVF9.3}
			L'utente vuole verificare che il sistema permetta all'utente amministratore di confermare le modifiche applicate. All'utente è richiesto di:
			\begin{itemize}
				\item effettuare l'autenticazione;
				\item accedere alla home screen;
				\item accedere all'elenco degli utenti;
				\item verificare che il sistema mostri un errore e le modifiche vengano annullate nel caso in cui un utente cambi i permessi a se stesso;
				\item verificare che il sistema mostri un errore e le modifiche vengano annullate nel caso in cui si provi ad eliminare un utente attualmente autenticato;
				\item verificare che il sistema mostri un messaggio di conferma nel caso le modifiche abbiano avuto esito positivo.
			\end{itemize}
			
		\subsubsection{TVF10}
			L'utente vuole verificare che l'amministratore possa gestire le richieste di nuove Recipe ricevute dagli utenti. All'utente è richiesto di:
			\begin{itemize}
				\item effettuare l'autenticazione;
				\item accedere alla home screen;
				\item accedere all'elenco delle richieste di nuove Recipe;
				\item verificare che l'utente amministratore possa vedere l'elenco delle nuove Recipe;
				\item verificare che il sistema mostri un messaggio nel caso non siano presenti richieste di nuove Recipe;
				\item verificare che, cliccando sulla visualizzazione dei dettagli di una Recipe, si acceda ad un pannello contenente il form per la creazione di una nuova Recipe precompilata con i dati richiesti;
				\item verificare che, cliccando sul pulsante di accettazione, venga inserita la Recipe richiesta nel sistema;
				\item verificare che, cliccando sul pulsante di rifiuto, venga rifiutato l'inserimento della Recipe richiesta.
			\end{itemize}
			
		\subsubsection{TVF10.1}
			L'utente vuole verificare che l'utente amministratore possa vedere l'elenco delle nuove Recipe. All'utente è richiesto di:
			\begin{itemize}
				\item effettuare l'autenticazione;
				\item accedere alla home screen;
				\item accedere all'elenco degli utenti;
				\item verificare che, per ogni voce dell'elenco, sia fornito il titolo della Recipe richiesta, lo username dell'utente che l'ha richiesta e la descrizione nel caso sia presente;
				\item verificare che, per ogni voce dell'elenco, sia presente un pulsante per accedere alla visualizzazione dei dettagli della richiesta.
			\end{itemize}
			
		\subsubsection{TVF10.3}
			L'utente vuole verificare che, cliccando sulla visualizzazione dei dettagli di una Recipe, si acceda ad un pannello contenente il form per la creazione di una nuova Recipe precompilata con i dati richiesti. All'utente è richiesto di:
			\begin{itemize}
				\item effettuare l'autenticazione;
				\item accedere alla home screen;
				\item accedere all'elenco degli utenti;
				\item verificare che il sistema permetta all'utente amministratore di kodificare il titolo e la descrizione della Recipe richiesta;
				\item verificare che il sistema non permetta di modificare altri campi della richiesta della Recipe al di fuori del titolo e della descrizione.
			\end{itemize}
			
		\subsubsection{TVF11}
			L'utente vuole verificare che il sistema fornisca una serie di servizi REST. All'utente è richiesto di:
			\begin{itemize}
				\item verificare che il sistema fornisca la possibilità di richiedere un token di accesso per utilizzare i servizi REST;
				\item verificare che il sistema permetta all'utente di richiedere l'annullamento di un token;
				\item verificare che il sistema gestisca la richiesta dei dati di una View contenente un token di accesso.
			\end{itemize}
			
		\subsubsection{TVF11.1}
			L'utente vuole verificare che il sistema fornisca la possibilità di richiedere un token di accesso per utilizzare i servizi REST. All'utente è richiesto di:
			\begin{itemize}
				\item verificare che il sistema restituisca un errore se il sistema ha avuto problemi nel generare il token;
				\item verificare che il sistemi generi e restituisca un token di accesso se l'utente è autenticato.
			\end{itemize}
			
		\subsubsection{TVF11.2}
			L'utente vuole verificare che il sistema permetta all'utente di richiedere l'annullamento di un token. All'utente è richiesto di:
			\begin{itemize}
				\item verificare che il sistema gestisca una richiesta di chiusura sessione contenente un token di accesso valido rendendolo non valido;
				\item verificare che il sistema restituisca un errore se non si è riuscito ad invalidare il token;
				\item verificare che il sistema restituisca una conferma dell'avvenuta invalidazione del token.
			\end{itemize}
			
		\subsubsection{TVF11.3}
			L'utente vuole verificare che il sistema gestisca la richiesta dei dati di una View contenente un token di accesso. All'utente è richiesto di:
			\begin{itemize}
				\item verificare che il sistema restituisca i dati richiesti se il token utilizzato è valido;
				\item verificare che il sistema restituisca un errore se il token utilizzato non è valido.
			\end{itemize}
			
		\subsubsection{TVQ1}
			L'utente vuole verificare che venga fornito un manuale per l'utente. All'utente è richiesto di:
			\begin{itemize}
				\item accedere al sito;
				\item verificare che il sito fornisca il manuale utente attraverso un link;
				\item verificare che, cliccando il link, si apra il manuale utente correttamente.
			\end{itemize}
			
		\subsubsection{TVQ1.1}
			L'utente vuole verificare che venga fornito un manuale per l'utente in lingua inglese. All'utente è richiesto di:
			\begin{itemize}
				\item accedere al sito;
				\item verificare che il sito fornisca il manuale utente in lingua inglese attraverso un link;
				\item verificare che, cliccando il link, si apra il manuale utente in lingua inglese correttamente.
			\end{itemize}
			
		\subsubsection{TVQ3}
			L'utente vuole verificare che venga fornito un manuale per l'uso dei servizi REST. All'utente è richiesto di:
			\begin{itemize}
				\item accedere al sito;
				\item verificare che il sito fornisca il manuale d'uso dei servizi REST attraverso un link;
				\item verificare che, cliccando il link, si apra il manuale d'uso dei servizi REST correttamente.
			\end{itemize}
			
		\subsubsection{TVQ3.1}
			L'utente vuole verificare che venga fornito un manuale per l'uso dei servizi REST in lingua inglese. All'utente è richiesto di:
			\begin{itemize}
				\item accedere al sito;
				\item verificare che il sito fornisca il manuale d'uso dei servizi REST per i in lingua inglese attraverso un link;
				\item verificare che, cliccando il link, si apra il manuale d'uso dei servizi REST in lingua inglese correttamente.
			\end{itemize}
			
		\subsubsection{TVV1}
			L'utente vuole verificare che il sistema sfrutti gli strumenti offerti da Google Cloud Platform. All'utente è richiesto di:
			\begin{itemize}
				\item effettuare l'autenticazione su Google Cloud Platform;
				\item verificare che il sistema sfrutti le Google App Engine;
				\item verificare che il sistema sfrutti il Google Cloud Datastore;
				\item verificare che il sistema sfrutti gli strumenti offerti da Google Cloud Platform.
			\end{itemize}
			
		\subsubsection{TVV1.1}
			L'utente vuole verificare che il sistema sfrutti le Google App Engine. All'utente è richiesto di:
			\begin{itemize}
				\item effettuare l'autenticazione su Google Cloud Platform;
				\item accedere alla console della Google Cloud Platform;
				\item verificare che sia presente il progetto;
			\end{itemize}
			
		\subsubsection{TVV1.2}
			L'utente vuole verificare che il sistema sfrutti il Google Cloud Datastore. All'utente è richiesto di:
			\begin{itemize}
				\item effettuare l'autenticazione su Google Cloud Platform;
				\item accedere alla console della Google Cloud Platform;
				\item verificare che nella sezione Cloud Datastore sia presente il database.
			\end{itemize}
			
		\subsubsection{TVV1.3}
			L'utente vuole verificare che il sistema sfrutti gli strumenti offerti da Google Cloud Platform. All'utente è richiesto di:
			\begin{itemize}
				\item effettuare l'autenticazione su Google Cloud Platform;
				\item accedere alla console della Google Cloud Platform;
				\item TO DO.
			\end{itemize}
			
		\subsubsection{TVV2}
			L'utente vuole verificare che il linguaggio di programmazione principalmente usato per il back-end sia Pyhton. All'utente è richiesto di:
			\begin{itemize}
				\item avere accesso al codice sorgente;
				\item verificare che il codice sorgente per il back-end sia scritto in Python.
			\end{itemize}
			
		\subsubsection{TVV3}
			L'utente vuole verificare che il codice sorgente sia soggetto a versionamento tramite il modello di branching descritto nelle \docNameVersionNdP. All'utente è richiesto di:
			\begin{itemize}
				\item accedere al repository;
				\item entrare all'interno del progetto contenente il codice sorgente;
				\item verificare che il modello di branching utilizzato sia lo stesso di quello descritto nelle \docNameVersionNdP.
			\end{itemize}
			
		\subsubsection{TVV4}
			L'utente vuole verificare che l'interfaccia web sia di tipo single-page. All'utente è richiesto di:
			\begin{itemize}
				\item accedere al sito;
				\item effettuare l'autenticazione;
				\item verificare, navigando nel sito, che l'interfaccia sia di tipo single-page.
			\end{itemize}
			
		\subsubsection{TVV5}
			L'utente vuole verificare che l'interfaccia web funzioni con i principali browser attualmente sul mercato. All'utente è richiesto di:
			\begin{itemize}
				\item verificare il corretto funzionamento dell'interfaccia web con il browser Google Chrome v. 39.0 e successive;
				\item verificare il corretto funzionamento dell'interfaccia web con il browser Firefox v. 35.0 e successive;
				\item verificare il corretto funzionamento dell'interfaccia web con il browser Safari v. 8.0 e successive.
			\end{itemize}
			
		\subsubsection{TVV5.1}
			L'utente vuole verificare il corretto funzionamento dell'interfaccia web con il browser Google Chrome v. 39.0 e successive. All'utente è richiesto di:
			\begin{itemize}
				\item aprire il browser Google Chrome v. 39.0;
				\item accedere al sito web;
				\item verificare il corretto funzionamento dell'interfaccia web.
			\end{itemize}
			
		\subsubsection{TVV5.2}
			L'utente vuole verificare il corretto funzionamento dell'interfaccia web con il browser Firefox v. 35.0 e successive. All'utente è richiesto di:
			\begin{itemize}
				\item aprire il browser Firefox v. 35.0;
				\item accedere al sito web;
				\item verificare il corretto funzionamento dell'interfaccia web.
			\end{itemize}
			
		\subsubsection{TVV5.3}
			L'utente vuole verificare il corretto funzionamento dell'interfaccia web con il browser Safari v. 8.0 e successive. All'utente è richiesto di:
			\begin{itemize}
				\item aprire il browser Safari v. 8.0;
				\item accedere al sito web;
				\item verificare il corretto funzionamento dell'interfaccia web.
			\end{itemize}
			
		\subsubsection{TVV6}
			L'utente vuole verificare che il sistema utilizzi le librerie esterne per effettuare le chiamate alle API dei diversi social network. All'utente è richiesto di:
			\begin{itemize}
				\item avere accesso al codice sorgente;
				\item verificare che sia utilizzata la libreria facebook-sdk per le chiamate alle API di Facebook;
				\item verificare che sia utilizzata la libreria tweepy per le chiamate alle API di Twitter;
				\item verificare che sia utilizzata la libreria python-instagram per le chiamate alle API di Instagram.
			\end{itemize}
			
		\subsubsection{TVV6.1}
			L'utente vuole verificare che sia utilizzata la libreria facebook-sdk per le chiamate alle API di Facebook. All'utente è richiesto di:
			\begin{itemize}
				\item avere accesso al codice sorgente;
				\item verificare che sia importata la libreria facebook-sdk;
				\item verificare che siano effettuate le chiamate tramite la libreria facebook-sdk.
			\end{itemize}
			
		\subsubsection{TVV6.2}
			L'utente vuole verificare che sia utilizzata la libreria tweepy per le chiamate alle API di Twitter. All'utente è richiesto di:
			\begin{itemize}
				\item avere accesso al codice sorgente;
				\item verificare che sia importata la libreria tweepy;
				\item verificare che siano effettuate le chiamate tramite la libreria tweepy.
			\end{itemize}
			
		\subsubsection{TVV6.3}
			L'utente vuole verificare che sia utilizzata la libreria python-instagram per le chiamate alle API di Instagram. All'utente è richiesto di:
			\begin{itemize}
				\item avere accesso al codice sorgente;
				\item verificare che sia importata la libreria python-instagram;
				\item verificare che siano effettuate le chiamate tramite la libreria python-instagram.
			\end{itemize}
			
		\subsubsection{TVV7}
			L'utente vuole verificare che siano utilizzate librerie esterne per generare i grafici necessari alle View. All'utente è richiesto di:
			\begin{itemize}
				\item avere accesso al codice sorgente;
				\item verificare che sia utilizzata la libreria Chart.js per generare i Line Chart, Bar Chart, Pie Chart e Radar Chart;
				\item verificare che sia utilizzata la libreria TO DO per generare i Map Chart.
			\end{itemize}
			
	\subsubsection{TVV7.1}
			L'utente vuole verificare che sia utilizzata la libreria Chart.js per generare i Line Chart, Bar Chart, Pie Chart e Radar Chart. All'utente è richiesto di:
			\begin{itemize}
				\item avere accesso al codice sorgente;
				\item verificare che sia importata la libreria Chart.js;
				\item verificare che siano utilizzata la libreria Chart.js per la generazione dei Line Chart;
				\item verificare che siano utilizzata la libreria Chart.js per la generazione dei Bar Chart;
				\item verificare che siano utilizzata la libreria Chart.js per la generazione dei Pie Chart;
				\item verificare che siano utilizzata la libreria Chart.js per la generazione dei Radar Chart.
			\end{itemize}
			
		\subsubsection{TVV7.2}
			L'utente vuole verificare che sia utilizzata la libreria TO DO per generare i Map Chart. All'utente è richiesto di:
			\begin{itemize}
				\item avere accesso al codice sorgente;
				\item verificare che sia importata la libreria TO DO;
				\item verificare che siano utilizzata la libreria TO DO per la generazione dei Map Chart.
			\end{itemize}
			
		\subsubsection{TVV8}
			L'utente vuole verificare che il linguaggio di programmazione principalmente usato per il front-end sia Javascript. All'utente è richiesto di:
			\begin{itemize}
				\item avere accesso al codice sorgente;
				\item verificare che il codice sorgente per il front-end sia scritto in Javascript.
			\end{itemize}
			
		\subsubsection{TVV8.1}
			L'utente vuole verificare che il venga utilizzato AngularJS come framework per il front-end. All'utente è richiesto di:
			\begin{itemize}
				\item avere accesso al codice sorgente;
				\item verificare che sia importato il framework AngularJS per il front-end;
				\item verificare che il codice sorgente per il front-end utilizzi AngularJS.
			\end{itemize}
			
		\subsubsection{TVV9}
			L'utente vuole verificare che il token di accesso sia nel formato: "id utente" + "data creazione token" + "valore random". All'utente è richiesto di:
			\begin{itemize}
				\item avere accesso al codice sorgente;
				\item verificare nel codice sorgente che la creazione del token di accesso sia programmato in modo da rispettare la combinazione seguente: "id utente" + "data creazione token" + "valore random".d
			\end{itemize}
			
\pagebreak