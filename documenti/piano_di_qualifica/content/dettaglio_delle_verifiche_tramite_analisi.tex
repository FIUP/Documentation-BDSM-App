% =================================================================================================
% File:			dettaglio_delle_verifiche_tramite_analisi.tex
% Description:	Definisce le verifiche effettuate tramite analisi
% Created:		2014/03/05
% Author:		Faccin Nicola
% Email:		faccin.nicola@mashup-unipd.it
% =================================================================================================
% Modification History:
% Version		Modifier Date		Change											Author
% 0.0.1 		2014/01/11 			iniziata stesura appendice					Lorenzo C.
% =================================================================================================
%% Version		Modifier Date		Change											Author
% 1.0.1 		2015/03/17 			iniziata stesura appendice B				Nicola F.
% =================================================================================================
%


% CONTENUTO DEL CAPITOLO

\section{Dettaglio delle verifiche tramite analisi}
	\subsection{Ricerca ed implementazione degli strumenti}
	Essendo in questa fase non presente ancora alcun documento, non è stato possibile verificarli e quindi avere dei valori da esporre. Per correttezza è stato comunque inserito il nome della fase.

	\subsection{Analisi dei requisiti}
			\subsubsection{Processi}
		Vengono qui riportati i valori degli indici di budget e schedule variance per le attività della fase \textbf{Analisi dei requisiti}.
			\begin{table}[!ht]
			\begin{center}
				\begin{tabularx}{0.9\textwidth}{|l|l|X|}
					\hline
					\textbf{Attività} & \textbf{Schedule variance} & \textbf{Budget variance}\\
					\hline						
					\docNameVersionAdR & ?? & ??\\
					\hline
					\docNameVersionGlo & ?? & ??\\
					\hline					
					\docNameVersionNdP & ?? & ??\\
					\hline					
					\docNameVersionPdP & ?? & ??\\
					\hline					
					\docNameVersionPdQ & ?? & ??\\
					\hline					
					\docNameVersionSdF & ?? & ??\\
					\hline				
				\end{tabularx}
			\end{center}
		\caption{Esiti verifica sui processi fase - Analisi dei requisiti}
	\end{table}
	Complessivamente si registrano:
	\begin{itemize}
	\item \textbf{Schedule variance:} ;
	\item \textbf{Budget variance:} .
	\end{itemize}
	Dai valori presentati si può dedurre che i periodi di slack introdotti aiutano ad avere una schedule variance positiva. L'inesperienza del gruppo invece ha fatto si che la budget variance sia negativa ma comunque non compromettente in quanto al di sopra del minimo accettabile di...
	 	\subsubsection{Documenti}
		\begin{table}[!ht]
			\begin{center}
				\begin{tabularx}{0.9\textwidth}{|l|l|X|}
					\hline
					\textbf{Nome documento} & \textbf{Valore indice} & \textbf{Esito}\\
					\hline						
					\docNameVersionAdR & 56 & \textcolor{green}{Superato}\\
					\hline
					\docNameVersionGlo & 50 & \textcolor{green}{Superato}\\
					\hline					
					\docNameVersionNdP & 54 & \textcolor{green}{Superato}\\
					\hline					
					\docNameVersionPdP & 55 & \textcolor{green}{Superato}\\
					\hline					
					\docNameVersionPdQ & 55 & \textcolor{green}{Superato}\\
					\hline					
					\docNameVersionSdF & 48 & \textcolor{green}{Superato}\\
					\hline				
				\end{tabularx}
			\end{center}
			\caption{Risultati indice Gulpease}
		\end{table}
		\subsection{Analisi di dettaglio}
		\subsubsection{Processi}
		Vengono qui riportati i valori degli indici di budget e schedule variance per le attività della fase \textbf{Analisi di dettaglio}.
			\begin{table}[!ht]
			\begin{center}
				\begin{tabularx}{0.9\textwidth}{|l|l|X|}
					\hline
					\textbf{Attività} & \textbf{Schedule variance} & \textbf{Budget variance}\\
					\hline						
					\docNameVersionAdR & ?? & ??\\
					\hline
					\docNameVersionGlo & ?? & ??\\
					\hline					
					\docNameVersionNdP & ?? & ??\\
					\hline					
					\docNameVersionPdP & ?? & ??\\
					\hline					
					\docNameVersionPdQ & ?? & ??\\
					\hline					
					\docNameVersionSdF & ?? & ??\\
					\hline				
				\end{tabularx}
			\end{center}
		\caption{Esiti verifica sui processi fase - Analisi di dettaglio}
	\end{table}
	Complessivamente si registrano:
	\begin{itemize}
	\item \textbf{Schedule variance:} ;
	\item \textbf{Budget variance:} .
	\end{itemize}
	 	\subsubsection{Documenti}
		\begin{table}[!ht]
			\begin{center}
				\begin{tabularx}{0.9\textwidth}{|l|l|X|}
					\hline
					\textbf{Nome documento} & \textbf{Valore indice} & \textbf{Esito}\\
					\hline						
					\docNameVersionAdR & ?? & \textcolor{green}{Superato}\\
					\hline
					\docNameVersionGlo & ?? & \textcolor{green}{Superato}\\
					\hline					
					\docNameVersionNdP & ?? & \textcolor{green}{Superato}\\
					\hline					
					\docNameVersionPdP & ?? & \textcolor{green}{Superato}\\
					\hline					
					\docNameVersionPdQ & ?? & \textcolor{green}{Superato}\\
					\hline					
					\docNameVersionSdF & ?? & \textcolor{green}{Superato}\\
					\hline				
				\end{tabularx}
			\end{center}
			\caption{Risultati indice Gulpease}
		\end{table}
		\subsection{Progettazione architetturale}
		\subsubsection{Processi}
		Vengono qui riportati i valori degli indici di budget e schedule variance per le attività della fase \textbf{Progettazione architetturale}.
			\begin{table}[!ht]
			\begin{center}
				\begin{tabularx}{0.9\textwidth}{|l|l|X|}
					\hline
					\textbf{Attività} & \textbf{Schedule variance} & \textbf{Budget variance}\\
					\hline						
					\docNameVersionAdR & ?? & ??\\
					\hline
					\docNameVersionGlo & ?? & ??\\
					\hline					
					\docNameVersionNdP & ?? & ??\\
					\hline					
					\docNameVersionPdP & ?? & ??\\
					\hline					
					\docNameVersionPdQ & ?? & ??\\
					\hline					
					\docNameVersionSdF & ?? & ??\\
					\hline		
					\docNameVersionST & ?? & ??\\
					\hline		
				\end{tabularx}
			\end{center}
		\caption{Esiti verifica sui processi fase - Progettazione architetturale}
	\end{table}
	Complessivamente si registrano:
	\begin{itemize}
	\item \textbf{Schedule variance:} ;
	\item \textbf{Budget variance:} .
	\end{itemize}
	 	\subsubsection{Documenti}
		\begin{table}[!ht]
			\begin{center}
				\begin{tabularx}{0.9\textwidth}{|l|l|X|}
					\hline
					\textbf{Nome documento} & \textbf{Valore indice} & \textbf{Esito}\\
					\hline						
					\docNameVersionAdR & ?? & \textcolor{green}{Superato}\\
					\hline
					\docNameVersionGlo & ?? & \textcolor{green}{Superato}\\
					\hline					
					\docNameVersionNdP & ?? & \textcolor{green}{Superato}\\
					\hline					
					\docNameVersionPdP & ?? & \textcolor{green}{Superato}\\
					\hline					
					\docNameVersionPdQ & ?? & \textcolor{green}{Superato}\\
					\hline					
					\docNameVersionSdF & ?? & \textcolor{green}{Superato}\\
					\hline	
					\docNameVersionST & ?? & \textcolor{green}{Superato}\\
					\hline			
				\end{tabularx}
			\end{center}
			\caption{Risultati indice Gulpease}
		\end{table}
		\subsection{Progettazione di dettaglio e codifica dei requisiti obbligatori}
		\subsubsection{Processi}
		Vengono qui riportati i valori degli indici di budget e schedule variance per le attività della fase \textbf{Progettazione di dettaglio e codifica dei requisiti obbligatori}.
			\begin{table}[!ht]
			\begin{center}
				\begin{tabularx}{0.9\textwidth}{|l|l|X|}
					\hline
					\textbf{Attività} & \textbf{Schedule variance} & \textbf{Budget variance}\\
					\hline						
					\docNameVersionAdR & ?? & ??\\
					\hline
					\docNameVersionGlo & ?? & ??\\
					\hline					
					\docNameVersionNdP & ?? & ??\\
					\hline					
					\docNameVersionPdP & ?? & ??\\
					\hline					
					\docNameVersionPdQ & ?? & ??\\
					\hline					
					\docNameVersionSdF & ?? & ??\\
					\hline				
				\end{tabularx}
			\end{center}
		\caption{Esiti verifica sui processi fase - Progettazione di dettaglio e codifica dei requisiti obbligatori}
	\end{table}
	Complessivamente si registrano:
	\begin{itemize}
	\item \textbf{Schedule variance:} ;
	\item \textbf{Budget variance:} .
	\end{itemize}
	 	\subsubsection{Documenti}
		\begin{table}[!ht]
			\begin{center}
				\begin{tabularx}{0.9\textwidth}{|l|l|X|}
					\hline
					\textbf{Nome documento} & \textbf{Valore indice} & \textbf{Esito}\\
					\hline						
					\docNameVersionAdR & ?? & \textcolor{green}{Superato}\\
					\hline
					\docNameVersionGlo & ?? & \textcolor{green}{Superato}\\
					\hline					
					\docNameVersionNdP & ?? & \textcolor{green}{Superato}\\
					\hline					
					\docNameVersionPdP & ?? & \textcolor{green}{Superato}\\
					\hline					
					\docNameVersionPdQ & ?? & \textcolor{green}{Superato}\\
					\hline					
					\docNameVersionSdF & ?? & \textcolor{green}{Superato}\\
					\hline				
				\end{tabularx}
			\end{center}
			\caption{Risultati indice Gulpease}
		\end{table}
	
\pagebreak