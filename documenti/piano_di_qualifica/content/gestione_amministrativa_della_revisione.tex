% =================================================================================================
% File:			gestione_amministrativa_della_revisione.tex
% Description:	Definisce la sezione relativa alla gestione amminitrativa della revisione
% Created:		2014/12/16
% Author:		Ceccon Lorenzo
% Email:		ceccon.lorenzo@mashup-unipd.it
% =================================================================================================
% Modification History:
% Version		Modifier Date		Change											Author
% 0.0.1 		2014/12/16 			iniziata stesura sezione gestione revisione	Lorenzo C.
% =================================================================================================
%

% CONTENUTO DEL CAPITOLO

\section{Gestione amministrativa della revisione}

	\subsection{Gestione delle anomalie e delle discrepanze}
	La fase di verifica porta alla ricerca di eventuali difetti, i quali possono essere errori logici oppure anomalie presenti nel codice. Il \roleVerifier{} ha il compito di esaminare scrupolosamente il codice del prodotto e sottolineare le eventuali anomalie e problemi per essere risolti successivamente.
	
	\subsection{Procedure di controllo di qualità di processo}
	Il PDCA, noto anche come ciclo di Deming, è un metodo di gestione iterativo a quattro fasi per il controllo e il miglioramento continuo dei processi.\\
	Le quattro fasi che lo compongono sono:
		\begin{itemize}
			\item \textbf{Plan:} stabilisce gli obiettivi e i processi necessari per ottenere risultati uguali a quelli attesi;
			\item \textbf{Do:} fase composta dall'attuazione del piano, dall'esecuzione del processo e dalla creazione del prodotto. Termina con una raccolta dei dati e creazione di grafici sul risultato di quanto ottenuto;
			\item \textbf{Check:} si confrontano i risultati ottenuti dalla fase precedente con i risultati attesi per verificare la presenza di differenze;
			\item \textbf{Act:} si effettuano correzioni laddove sono presenti differenze tra i risultati ottenuti e previsti. Si determinano le cause delle discrepanze e dove c'è bisogno di applicare le modifiche per ottenere un miglioramento del processo e del prodotto.
		\end{itemize}
		\begin{figure}[h]
			\centering
			\includegraphics[width=90mm]{images/pdca.png}
			\caption{Ciclo di miglioramento della qualità PDCA}
		\end{figure}
	
	\subsection{Procedure di controllo di qualità di prodotto}
	Per garantire la qualità del prodotto software si fa affidamento a due modalità di controllo:
		\begin{itemize}
			\item \textbf{Software Quality Assurance (SQA):} è un insieme di attività che assicurano che il software sviluppato sia conforme alle specifiche di qualità standard o definite dal gruppo. L'SQA è un processo appartenente al ciclo di vita del software che controlla regolarmente e preventivamente il software sviluppato per assicurare il rispetto degli obbiettivi di qualità prefissati;
			\item \textbf{Verifica e Validazione (V\&V):} è un processo che controlla che il sistema software soddisfi le specifiche e che raggiunga appieno il suo scopo.\\
			Per verifica si intende il processo attuo a valutare che il software in una determinata fase di sviluppo soddisfi le condizioni imposte all'inizio di tale fase. Per validazione si intende il processo attuo a valutare se al termine del processo di sviluppo questo soddisfi i requisiti specificati. In altre parole, la verifica garantisce che il software è stato creato correttamente, mentre la validazione assicura che si è creato il giusto prodotto.
		\end{itemize}
	
	\pagebreak