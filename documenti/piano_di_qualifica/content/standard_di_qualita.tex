% =================================================================================================
% File:			appendici.tex
% Description:	Definisce le appendici riguardanti gli standard di qualità
% Created:		2014/01/11
% Author:		Ceccon Lorenzo
% Email:		ceccon.lorenzo@mashup-unipd.it
% =================================================================================================
% Modification History:
% Version		Modifier Date		Change											Author
% 0.0.1 		2014/01/11 			iniziata stesura appendice					Lorenzo C.
% =================================================================================================
%

% CONTENUTO DEL CAPITOLO
\section{Standard ISO/IEC di qualità}
  	\subsection{Standard ISO/IEC 9126}
  	Lo standard ISO/IEC 9126 è uno standard creato per delineare delle normative utili a descrivere un modello di qualità del software. Lo standard propone un approccio in cui viene posta attenzione al miglioramento dell'organizzazione e dei processi di una società di software, in modo da migliorare di conseguenza la qualità del prodotto software.\\
  	Lo standard ISO/IEC 9126 è suddiviso in quattro parti:
		\begin{itemize}
			\item \textbf{Modello di qualità: } \textnormal{La prima parte dello standard classifica il modello di qualità in sei caratteristiche generali e in varie sottocaratteristiche misurabili tramite l'utilizzo di metriche.\\
			Le sei caratteristiche generali e le relative sottocaratteristiche sono:}
				\begin{itemize}
					\item \textbf{Funzionalità:}
						\begin{itemize}
							\item \textbf{Appropriatezza;}
							\item \textbf{Accuratezza;}
							\item \textbf{Interoperabilità;}
							\item \textbf{Conformità;}
							\item \textbf{Sicurezza.}
						\end{itemize}
					\item \textbf{Affidabilità:}
						\begin{itemize}
							\item \textbf{Maturità;}
							\item \textbf{Tolleranza agli errori;}
							\item \textbf{Recuperabilità;}
							\item \textbf{Aderenza.}
						\end{itemize}
					\item \textbf{Usabilità:}
						\begin{itemize}
							\item \textbf{Comprensibilità;}
							\item \textbf{Apprendibilità;}
							\item \textbf{Operabilità;}
							\item \textbf{Attrattiva;}
							\item \textbf{Conformità.}
						\end{itemize}
					\item \textbf{Efficienza:}
						\begin{itemize}
							\item \textbf{Comportamento rispetto al tempo;}
							\item \textbf{Utilizzo delle risorse;}
							\item \textbf{Conformità.}
						\end{itemize}
					\item \textbf{Manutenibilità:}
						\begin{itemize}
							\item \textbf{Analizzabilità;}
							\item \textbf{Modificabilità;}
							\item \textbf{Stabilità;}
							\item \textbf{Testabilità.}
						\end{itemize}
					\item \textbf{Portabilità:}
						\begin{itemize}
							\item \textbf{Adattabilità;}
							\item \textbf{Installabilità;}
							\item \textbf{Conformità;}
							\item \textbf{Sostituibilità.}
						\end{itemize}
				\end{itemize}
			\item \textbf{Qualità esterne: } \textnormal{Le metriche esterne applicabili al software, e quindi rilevabili tramite l'analisi dinamica, misurano i comportamento del prodotto sulla base dei test, dall'operatività e dall'osservazione durante la sua esecuzione;}
			\item \textbf{Qualità interne: } \textnormal{Le metriche interne, misurabili attraverso l'analisi statica, sono utili per prevedere il livello della qualità esterna ed in uso, poiché i suoi attributi interni influiscono su quelli esterni ed in uso. Permettono così di individuare anomalie prima che queste ultime possano influenzare la qualità del prodotto finale;}
			\item \textbf{Qualità in uso: } \textnormal{La qualità in uso, raggiungibile solo dopo aver ottenuto la qualità interna ed esterna, fornisce metriche per misurare il grado di utilizzabilità del prodotto da parte dell'utente finale.}
		\end{itemize}
		
	\subsection{Standard ISO/IEC 15504}
	Lo standard ISO/IEC 15504, conosciuto anche come SPICE, è un insieme di documenti tecnici per lo sviluppo di processi software, utili a valutare la dimensione dei processi tramite l'utilizzo di specifiche metriche. É derivato dallo standard ISO/IEC 12207 e da modelli di maturità quali Bootstrap, Trillium e il CMM.\\
	Lo standard definisce la dimensione del processo e la suddivide nelle seguenti cinque categorie:
		\begin{itemize}
			\item \textbf{Custormer/Supplier;}
			\item \textbf{Engineering;}
			\item \textbf{Support;}
			\item \textbf{Management;}
			\item \textbf{Organization.}
		\end{itemize}
	Per ogni processo, viene definito un livello di capacità dei processi definito da una scala di sei livelli e da nove attributi suddivisi nei vari livelli:
		\begin{itemize}
			\item \textbf{Level 5. Optimizing process;}
				\begin{itemize}
					\item \textbf{Process Innovation;}
					\item \textbf{Process Optimization.}
				\end{itemize}
			\item \textbf{Level 4. Predictable process;}
				\begin{itemize}
					\item \textbf{Process Measurement;}
					\item \textbf{Process Control.}
				\end{itemize}
			\item \textbf{Level 3. Established process;}
				\begin{itemize}
					\item \textbf{Process Definition;}
					\item \textbf{Process Deployment.}
				\end{itemize}
			\item \textbf{Level 2. Managed process;}
				\begin{itemize}
					\item \textbf{Performance Management;}
					\item \textbf{Work Product Management.}
				\end{itemize}			
			\item \textbf{Level 1. Performed process;}
				\begin{itemize}
					\item \textbf{Process Performance.}
				\end{itemize}
			\item \textbf{Level 0. Incomplete process.}
		\end{itemize}
	Ogni attributo è misurabile tramite l'utilizzo di una scala di valutazione divisa in quattro punti:
		\begin{itemize}
			\item \textbf{Not achieved (0-15\%);}
			\item \textbf{Partially achieved (15-50\%);}
			\item \textbf{Largely achieved (50-85\%);}
			\item \textbf{Fully achieved (85-100\%).}
		\end{itemize}
	Lo standard fornisce una guida per l'effettuazione di una valutazione formata da:
		\begin{itemize}
			\item \textbf{Processo di valutazione;}
			\item \textbf{Modello per la valutazione;}
			\item \textbf{Strumenti per la valutazione.}
		\end{itemize}
	Lo standard infine, stabilisce che per una corretta valutazione i verificatori debbano avere un buon livello di competenza e di esperienza.