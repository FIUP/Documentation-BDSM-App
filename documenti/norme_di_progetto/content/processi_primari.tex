% =================================================================================================
% File:			processi_primari.tex
% Description:	Definisce il capitolo dei processi primari
% Created:		2014/12/11
% Author:		Tesser paolo
% Email:		tesser.paolo@mashup-unipd.it
% =================================================================================================
% Modification History:
% Version		Modifier Date		Change													Author
% 0.0.1 		2014/12/11 			Inizializzazione del file e inizio stesura				Tesser Paolo
% =================================================================================================
% 0.0.2			2014/12/12			continuazione stesura sottosezione fornitura			Tesser Paolo
% =================================================================================================
% 0.0.3			2014/12/12			completamento sottosezione stesura e inizio sviluppo	Tesser Paolo
% =================================================================================================
%


% CONTENUTO DEL CAPITOLO

\section{Processi Primari}
Nella sezione dei Processi Primari sono descritti i processi e le attività che servono alle parti prime durante il ciclo di vita del software. \\
Queste parti sono l'acquirente, il fornitore, lo sviluppatore, l'operatore e il manutentore del prodotto software. \\
Il gruppo \groupName{} si occuperà solo dei Processi di Fornitura e di Sviluppo. Il Processo di Acquisizione spetterà al proponente e al committente del capitolato scelto, mentre il Processo di Manutenzione non potrà essere attuato per vincoli temporali. \\
Nonostante questo l'applicativo \projectName{} e la documentazione fornita con esso dovranno garantire la possibilità futura di procedere con le attività presenti in questo processo a discrezione del proponente.

	\subsection{Processo di Fornitura}
		\subsubsection{Attività}
		
			\paragraph{Accettazione}
				\subparagraph{Discussione e scelta del capitolato}
Il \roleProjectManager{} ha il compito di organizzare	gli incontri iniziali per permettere ai componenti del gruppo di discutere e esaminare i capitolati selezionati dal committente. \\
Dalle riunioni emergerà la scelta effettuata dal team. \\
Le valutazioni che hanno portato ha prendere questa decisione verranno documentate in dei verbali.
				\subparagraph{Studio di Fattibilità}
Agli \emph{Analisti} spetta il compito di redigere lo \docNameVersionSdF{}, basandosi sui verbali citati in precedenza.
Per realizzare il documento dovranno essere presi in considerazione i seguenti punti per ogni capitolato, con maggiore dettaglio per quello preso in carico.
					\begin{itemize}
						\item \textbf{dominio tecnologico e applicativo:} si valutano le conoscenze attuali del team riguardo le tecnologie richieste;
						\item \textbf{interesse strategico:} si valuta quanto interesse abbia il team ad apprendere gli strumenti che serviranno a realizzare il capitolato in questione;
						\item \textbf{rischi possibili:} si analizzano i possibili punti critici in merito anche alle valutazione effettuate nei punti precedenti.
					\end{itemize}
			\paragraph{Preparazione della risposta}
				\subparagraph{Definizione e preparazione della proposta}
I membri del gruppo \groupName{} dovranno redigere i seguenti documenti: \\
					\begin{itemize}
						\item \docNameVersionSdF
						\item \docNameVersionAdR
						\item \docNameVersionPdP
						\item \docNameVersionPdQ
						\item \docNameVersionNdP
					\end{itemize}
In allegato a essi si dovrà consegnare una \emph{lettera di presentazione}. \\
Starà quindi al committente e al proponete valutare l'offerta ricevuta e decidere se accettare o meno la proposta fatta.
					
			\paragraph{Pianificazione}
				\subparagraph{Scelta del modello di ciclo di vita}
Il \roleProjectManager{} avrà il compito di scegliere il modello di ciclo di vita adatto per lo sviluppo del prodotto richiesto.
				\subparagraph{Sviluppo e documentazione del Piano di Progetto}
Il \roleProjectManager{} dovrà eseguire una serie di compiti per delineare i lavori che i membri del team dovranno andare ad eseguire, calcolandone i costi e le tempistiche. Il tutto verrà documento nel \docNameVersionPdP{}.
		
	\subsection{Processo di Sviluppo}
		\subsubsection{Attività}
			\paragraph{Analisi dei requisiti}
Dopo avere redatto lo \docNameSdF{}, agli \emph{Analisti} spetterà il compito di scrivere il documento \docNameVersionAdR{}. \\
L'obiettivo sarà quello di cercare e produrre dei requisiti a partire dalle informazioni a disposizione del team. \\
Le risorse che potranno essere utilizzate per questo scopo provengono dal capitolato d'appalto e dagli incontri con il proponente e con il committente. \\

		\subsubsection{Norme}
			\paragraph{Classificazione requisiti}
TO DO (da completare una volta deciso come classificare i requisiti)			
			\paragraph{Classificazione casi d'uso}
I casi d'uso devono essere divisi in ordine gerarchico secondo il seguente schema: \\
				\begin{center}
					UC[Codice]
				\end{center}
TO DO (prendere STANDARD per la classificazione)

		\subsubsection{Strumenti}
			\paragraph{requirementsTool}
TO DO (da completare una volta generato l'applicativo)