% =================================================================================================
% File:			processi_primari.tex
% Description:	Definisce il capitolo dei processi primari
% Created:		2014/12/11
% Author:		Tesser paolo
% Email:		tesser.paolo@mashup-unipd.it
% =================================================================================================
% Modification History:
% Version		Modifier Date		Change											Author
% 0.0.1 		2014/12/11 			Inizializzazione del file e inizio stesura		Tesser Paolo
% =================================================================================================
% 0.0.2			2014/12/12			continuazione stesura sottosezione fornitura	Tesser Paolo
% =================================================================================================
%
%


% CONTENUTO DEL CAPITOLO

\section{Processi Primari}
Nella sezione dei Processi Primari sono descritti i processi e le attività che servono alle parti prime durante il ciclo di vita del software. \\
Queste parti sono l'acquirente, il fornitore, lo sviluppatore, l'operatore e il manutentore del prodotto software. \\
Il gruppo \groupName{} si occuperà solo della fornitura e dello sviluppo. Il Processo di Acquisizione spetterà al proponente e al committente del capitolato scelto, mentre il Processo di Manutenzione non potrà essere attuato per vincoli temporali. \\
Nonostante questo l'applicativo \projectName{} e la documentazione fornita con esso dovranno garantire la possibilità futura di procedere con le attività presenti in questo processo a discrezione del proponente.

	\subsection{Processo di Fornitura}
		\subsubsection{Attività}
		
			\paragraph{Accettazione}
				\subparagraph{Discussione e scelta del capitolato}
Il Project Manager ha il compito di organizzare	gli incontri iniziali per permettere ai componenti del gruppo di discutere e esaminare i capitolati selezionati dal committente. \\
Dalle riunioni emergerà la scelta effettuata dal team. \\
Le valutazioni che hanno portato ha prendere questa decisione verranno documentate in dei verbali.
				\subparagraph{Studio di Fattibilità}
Agli analisti spetta il compito di redigere lo TO DO, basandosi sui verbali citati in precedenza.
Per realizzare il documento dovranno essere presi in considerazione i seguenti punti per ogni capitolato, con maggiore dettaglio per quello preso in carico.
					\begin{itemize}
						\item \textbf{dominio tecnologico e applicativo:}
						\item \textbf{interesse strategico:}
						\item \textbf{rischi possibili:}
					\end{itemize}
			\paragraph{Preparazione della risposta}
				\subparagraph{Definizione e preparazione della proposta}
TO DO
			
			\paragraph{Pianificazione}
				\subparagraph{Scelta del modello di ciclo di vita}
TO DO
				\subparagraph{Sviluppo e documentazione del Piano di Progetto}
TO DO

				
	\subsection{Processo di Sviluppo}
		\subsubsection{Attività}
			\paragraph{Analisi dei requisiti}
			
		\subsubsection{Norme}
			\paragraph{Classificazione requisiti}
TO DO			
			\paragraph{Classificazione casi d'uso}
TO DO

		\subsubsection{Strumenti}
