% =================================================================================================
% File:			processi_primari.tex
% Description:	Definisce il capitolo dei processi primari
% Created:		2014-12-11
% Author:		Tesser paolo
% Email:		tesser.paolo@mashup-unipd.it
% =================================================================================================
% Modification History:
% Version		Modifier Date		Change													Author
% 0.0.1 		2014-12-11 			Inizializzazione del file e inizio stesura				Tesser Paolo
% =================================================================================================
% 0.0.2			2014-12-12			continuazione stesura sottosezione fornitura			Tesser Paolo
% =================================================================================================
% 0.0.3			2014-12-12			completamento sottosezione stesura e inizio sviluppo	Tesser Paolo
% =================================================================================================
% 0.0.4			2015-01-08			stesura classificazione requisiti						Tesser Paolo
% =================================================================================================
%


% CONTENUTO DEL CAPITOLO

\section{Processi Primari}
Nella sezione dei Processi Primari sono descritti i processi e le attività che servono alle parti prime durante il ciclo di vita del software. \\
Queste parti sono l'acquirente, il fornitore, lo sviluppatore, l'operatore e il manutentore del prodotto software. \\
Il gruppo \groupName{} si occuperà solo dei Processi di Fornitura e di Sviluppo. Il Processo di Acquisizione spetterà al proponente e al committente del capitolato scelto, mentre il Processo di Manutenzione non potrà essere attuato per vincoli temporali. \\
Nonostante questo l'applicativo \projectName{} e la documentazione fornita con esso dovranno garantire la possibilità futura di procedere con le attività presenti in questo processo a discrezione del proponente.

	\subsection{Processo di fornitura}
		\subsubsection{Attività}
		
			\paragraph{Accettazione}
				\subparagraph{Discussione e scelta del capitolato}
Il \roleProjectManager{} ha il compito di organizzare	gli incontri iniziali per permettere ai componenti del gruppo di discutere e esaminare i capitolati selezionati dal committente. \\
Dalle riunioni emergerà la scelta effettuata dal team. \\
Le valutazioni che hanno portato ha prendere questa decisione verranno documentate in dei verbali.
				\subparagraph{Studio di Fattibilità}
Agli \emph{Analisti} spetta il compito di redigere lo \docNameVersionSdF{}, basandosi sui verbali citati in precedenza.
Per realizzare il documento dovranno essere presi in considerazione i seguenti punti per ogni capitolato, con maggiore dettaglio per quello preso in carico.
					\begin{itemize}
						\item \textbf{dominio tecnologico e applicativo:} si valutano le conoscenze attuali del team riguardo le tecnologie richieste;
						\item \textbf{interesse strategico:} si valuta quanto interesse abbia il team ad apprendere gli strumenti che serviranno a realizzare il capitolato in questione;
						\item \textbf{rischi possibili:} si analizzano i possibili punti critici in merito anche alle valutazione effettuate nei punti precedenti.
					\end{itemize}
			\paragraph{Preparazione della risposta}
				\subparagraph{Definizione e preparazione della proposta}
I membri del gruppo \groupName{} dovranno redigere i seguenti documenti: \\
					\begin{itemize}
						\item \docNameVersionSdF
						\item \docNameVersionAdR
						\item \docNameVersionPdP
						\item \docNameVersionPdQ
						\item \docNameVersionNdP
					\end{itemize}
In allegato a essi si dovrà consegnare una \emph{lettera di presentazione}. \\
Starà quindi al committente e al proponete valutare l'offerta ricevuta e decidere se accettare o meno la proposta fatta.
					
			\paragraph{Pianificazione}
				\subparagraph{Scelta del modello di ciclo di vita}
Il \roleProjectManager{} avrà il compito di scegliere, se non indicato dal proponente, il modello di ciclo di vita adatto per lo sviluppo del prodotto richiesto.
				\subparagraph{Sviluppo e documentazione del Piano di Progetto}
Il \roleProjectManager{} dovrà eseguire una serie di compiti per delineare i lavori che i membri del team dovranno andare ad eseguire, calcolandone i costi e le tempistiche. Il tutto verrà documento nel \docNameVersionPdP{}.
		
	\subsection{Processo di sviluppo}
		\subsubsection{Attività}
			\paragraph{Analisi dei requisiti}
Dopo avere redatto lo \docNameSdF{}, agli \emph{Analisti} spetterà il compito di scrivere il documento \docNameVersionAdR{}. \\
L'obiettivo sarà quello di cercare e produrre dei requisiti a partire dalle informazioni a disposizione del team. \\
Le risorse che potranno essere utilizzate per questo scopo provengono dal capitolato d'appalto e dagli incontri con il proponente e con il committente. \\

		\subsubsection{Norme}
			\paragraph{Classificazione requisiti}
I requisiti prodotti dovranno essere classificati a seconda del tipo e dell'importanza, seguendo la seguente notazione:
				\begin{center}
					R[Importanza][Tipo][Codice]
				\end{center}
			\noindent
				\begin{itemize}
					\item \textbf{Importanza}: i valori che può assumere sono:
						\begin{itemize}
							\item 0: requisito obbligatorio;
							\item 1: requisito desiderabile;
							\item 2: requisito opzionale.
						\end{itemize}
					\item \textbf{Tipo}: i valori che può assumere sono:
						\begin{itemize}
							\item F: requisito funzionale;
							\item P: requisito prestazionale;
							\item Q: requisito di qualità;
							\item V: requisito di vincolo.
						\end{itemize}
					\item \textbf{Codice}: è il codice gerarchico e univoco del vincolo espresso nella forma X.X.X dove X è un valore numerico.
				\end{itemize}
			\noindent
			Ogni requisito inoltre dovrà contenere le seguente informazioni:
				\begin{itemize}
					\item \textbf{Descrizione}: descrizione del requisito il meno possibile ambigua;
					\item \textbf{Fonte}: la scelta può ricadere tra:
						\begin{itemize}
							\item \textbf{capitolato}: è il requisito che viene trovato dalle specifiche del capitolato di appalto;
							\item \textbf{interno}: è il requisito che viene trovato dagli \emph{Analisti} durante un'analisi più approfondita;
							\item \textbf{caso d'uso}: è il requisito che viene trovato a partire da uno o più casi d'uso. In tal caso andrà specificato il codice del caso d'uso a cui si riferisce.
						\end{itemize}
				\end{itemize}
			
			\paragraph{Classificazione casi d'uso}
I casi d'uso devono essere divisi in ordine gerarchico secondo il seguente schema: \\
				\begin{center}
					UC[Codice]
				\end{center}
				\begin{itemize}
					\item \textbf{Codice}: è il codice gerarchico e univoco per identificare ogni caso d'uso.
				\end{itemize}
			\noindent Per ogni caso d'uso dovranno essere presenti anche le seguenti informazioni:
				\begin{itemize}
					\item \textbf{descrizione}: fornire una breve descrizione del caso d'uso che sia il meno ambigua possibile;
					\item \textbf{attori principali}: elenco degli attori principali coinvolti nel caso d'uso;
					\item \textbf{attori secondari}: elenco degli attori secondari coinvolti nel caso d'uso;
					\item \textbf{scenari principali}: descrizione dei possibili scenari principali;
					\item \textbf{scenari alternativi}: descrizione dei possibili scenari secondari;
					\item \textbf{pre-condizioni}: condizione di partenza sempre vera all'inizio del caso d’uso;
					\item \textbf{flusso degli eventi}: ordine di esecuzione dei casi d'uso figli;
					\item \textbf{inclusioni}: spiegazione di tutte le inclusioni se presenti;
					\item \textbf{estensioni}: spiegazione di tutte le estensioni se presenti;
					\item \textbf{generalizzazioni}: spiegazione di tutte le generalizzazioni, se presenti;;
					\item \textbf{post-condizioni}: condizione finale sempre vera alla fine dell'esecuzione del caso d'uso.
				\end{itemize}
			
			\paragraph{Codifica file} % (fold)
			\label{par:codifica}
			Tutti i file che vengono creati, contenenti sia codice sia documentazione, devono essere codificati tramite UTF-8 senza BOM. \\
			Se verrà a verificarsi la necessità di un cambiamento il \roleProjectManager{} ne dovrà approvare le modifiche.
			% paragraph codifica (end)
			
			\paragraph{Nomi e norme di codifica} % (fold)
			\label{par:nomi_e_norme_di_codifica}
			Questa sezione verrà redatta nel dettaglio durante le fasi successive. \\
			Vengono però introdotte alcune norme che andranno applicate, indifferentemente dal linguaggio di programmazione che si adotterà per lo sviluppo del codice. \\
				\begin{itemize}
					\item ci dovrà essere un header su ciascun file che conterrà le seguente informazioni:
						\begin{itemize}
							\item percorso del file e nome del file;
							\item cognome e nome dell'autore;
							\item data di creazione;
							\item email dell'autore;
							\item dovranno inoltre essere inserite per ogni modifica che verrà effettuata: la successiva versione generata dall'avanzamento, la data, una breve descrizione e l'autore.
						\end{itemize}
					\item i nomi delle variabili dovranno essere in inglese;
					\item i commenti dovranno essere in italiano
				\end{itemize}
			% paragraph nomi_e_norme_di_codifica (end)
			
			\paragraph{Ricorsione} % (fold)
			\label{par:ricorsione}
			La ricorsione va evitata quando possibile. Per ogni funzione ricorsiva è necessario fornire una prova di terminazione. È inoltre necessario valutare il costo in termini di occupazione della memoria. Nel caso in cui l’utilizzo di memoria risulti troppo elevato, la ricorsione verrà rimossa.
			% paragraph ricorsione (end)
		
		\subsubsection{Strumenti}
		
			\paragraph{RequirementsTool}
			\label{par:requirements_tool}
			Il gruppo ha creato un applicativo web che gestisce l'immagazzinamento e il tracciamento dei requisiti e delle fonti attraverso l'inserimento in appositi form. \\
			Il servizio viene offerto nel servizio di hosting del team e ci si può accedere attraverso le API Key di Asana. \\
			L'applicativo è anche in grado di tracciare tutti i task che sono assegnati al membro loggatto e consentire la gestione del tempo di esecuzione di ciascuno di essi. \\
			Lo strumento è scritto in PHP.
			
			\paragraph{PhpStorm} % (fold)
			\label{par:php_storm}
			L'IDE JetBrain PhpStorm edizione professional, disponibile tramite licenza studentesca, viene utilizzato per scrivere il codice PHP relativo all'applicatvio descritto nella sezione \ref{par:requirements_tool}
			% paragraph php_storm (end)
			 