% =================================================================================================
% File:			introduzione.tex
% Description:	Definisce il capitolo d'introduzione
% Created:		2014/12/11
% Author:		Santacatterina Luca
% Email:		santacatterina.luca@mashup-unipd.it
% =================================================================================================
% Modification History:
% Version		Modifier Date		Change											Author
% 0.0.1 		2014/12/11 			Inizializzazione del file						Santacatterina Luca
% =================================================================================================
%

% CONTENUTO DEL CAPITOLO

\section{Introduzione}


\subsection{Scopo del documento}
Questo documento vuole definire tutte le norme che i membri del gruppo \goupName{} dovranno rispettare durante lo svolgimento del progetto \projectName.\\
Tutti i membri sono tenuti a leggere il documento e a seguire rigorosamente le norme descritte.\\
Ogni membro del gruppo dovrà segnalare all'\roleAdministrator{} qualsiasi richiesta per la modifica o l'introduzione di nuove norme.\\
Successivamente l'\roleAdministrator{} coinvolgerà i membri del gruppo per discuterne e avrà a disposizione la possibilità di approvare o negare la richiesta.
Tutte le norme raccolte dal seguente documento si riferiscono al progetto \projectName.\\
Le seguenti norme permetteranno di ottenere una maggiore uniformità dei documenti prodotti, porteranno ad un incremento dell'efficienza e della efficacia e ridurranno la possibilità di commettere errori.\\
In particolare si vuole definire:
\begin{itemize}
\item strumenti e metodi per la comunicazione interna ed esterna al gruppo;
\item strumenti e metodi per la stesura di tutti i documenti;
\item tempi e modalità di lavoro.
\end{itemize}





\subsection{Glossario}
Al fine di evitare ogni ambiguità relativa al linguaggio usato nei documenti viene allegato il \docNameVersionGlo.
Esso ha lo scopo di definire ed analizzare tutti i termini tecnici, di progetto, ambiguità e di fornirne una ben precisa descrizione.\\
Tutte le occorrenze presenti nei documenti verranno contrassegnate con una "G" in formato pedice.\\


\subsection{Riferimenti}

\subsubsection{Informativi}
TO DO
\subsubsection{Normativi}
\begin{itemize}
\item ISO 8601 \url{http://it.wikipedia.org/wiki/ISO_8601}
\end{itemize}
