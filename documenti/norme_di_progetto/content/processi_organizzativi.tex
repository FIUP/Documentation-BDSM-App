% =================================================================================================
% File:			processi_organizzativi.tex
% Description:	Definisce il capitolo deei processi organizzativi
% Created:		2014/12/11
% Author:		Tesser paolo
% Email:		tesser.paolo@mashup-unipd.it
% =================================================================================================
% Modification History:
% Version		Modifier Date		Change												Author
% 0.0.1 		2014/12/11 			Inizializzazione del file							Tesser Paolo
% =================================================================================================
% 0.0.2			2014/12/12			Inizio stesura della struttura dei sottocapitoli	Tesser Paolo
% =================================================================================================
% 0.0.3			2014/12/14			Inizio stesura sottosezione processi di gestione	Tesser Paolo
% =================================================================================================
% 0.0.4			2014/12/16			Miglioramento sottosezioni e continuata stesura		Tesser Paolo
% =================================================================================================
%


% CONTENUTO DEL CAPITOLO

\section{Processi Organizzativi}

	\subsection{Gestione dei Processi}
		\subsubsection{Attività}
			\paragraph{Gestione delle Comunicazioni}
					\subparagraph{Mail}
Ogni membro del gruppo avrà una mail personale creata grazie all'acquisizione di un dominio web sul servizio di hosting NetSons. \\
Il formato dell'indirizzo dovrà essere del tipo esposto di seguito e servirà per registrarsi ad ogni servizio che il team andrà ad utilizzare:
						\begin{center}
							cognome.nome@mashup-unipd.it
						\end{center}
												
				\subparagraph{Comunicazioni interne}
Le comunicazione interne e prettamente informali verranno gestite tramite un gruppo su WhatsApp denominato MashUp. \\
Quelle formali avverranno attraverso il sistema di ticketing Asana che consente una chat di gruppo al suo interno e che notifica tramite mail se qualcuno scrive qualcosa in essa. \\
Le norme e le procedure relative a questo servizio verranno trattate in dettaglio nella sezione \ref{sec:gestione_dei_ticket} e in quella \ref{sec:Asana}. \\
Se si necessita di un interazione vocale con gli altri membri, qualora non fossero presenti nello stesso luogo, verrà utilizzata l'applicazione Skype.
				\subparagraph{Comunicazioni esterne}
Le comunicazione esterne vengono effettuate dal \roleProjectManager{} in quanto rappresenta il gruppo \groupName. \\
Egli, attraverso la seguente mail, manterrà i contatti con il proponente e con il committente. In caso lo ritenga necessario, girerà tali TO DO agli altri membri del team.
					\begin{center}
						info@mashup-unipd.it
					\end{center}
					
			\paragraph{Gestione delle Riunioni}
				\subparagraph{Riunioni interne}
TO DO
				\subparagraph{Riunioni esterne}
TO DO			

			\paragraph{Gestione dei Ticket} \label{sec:gestione_dei_ticket}
			
			
		\subsubsection{Procedure}
TO DO
		\subsubsection{Norme}

			\paragraph{Ruoli di Progetto}
TO DO
	
	
	
		\subsubsection{Strumenti}
			\paragraph{Asana} \label{sec:Asana}
TO DO
			\paragraph{NetSons}
TO DO
			
			\paragraph{}
	\subsection{Gestione delle Infrastrutture}
		\subsubsection{Attività}
			\paragraph{Gestione del Repository}
TO DO			
		\subsubsection{Norme}
			\paragraph{Repository}
				\subparagraph{Convenzioni sui nomi dei file}
TO DO				
				\subparagraph{Messaggio di commit}
TO DO		
			
		\subsection{Strumenti}
			\paragraph{Git}
TO DO			
			\paragraph{GitHub}
TO DO			
			\paragraph{Git Hooks}
TO DO
			\paragraph{Google Drive}
TO DO
			\paragraph{Sistema Operativo}
TO DO





	\subsection{Formazione dei membri del team}
I membri del gruppo, per soddisfare le richieste assegnate dal \roleProjectManager{} al quale non sanno fare fronte con le conoscenze attuali in loro possesso, dovranno documentarsi adeguatamente durante ore esterne a quelle di lavoro, non imputabili perciò come costi al proponente.