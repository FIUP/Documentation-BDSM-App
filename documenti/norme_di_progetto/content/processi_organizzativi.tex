% =================================================================================================
% File:			processi_organizzativi.tex
% Description:	Definisce il capitolo deei processi organizzativi
% Created:		2014/12/11
% Author:		Tesser paolo
% Email:		tesser.paolo@mashup-unipd.it
% =================================================================================================
% Modification History:
% Version		Modifier Date		Change												Author
% 0.0.1 		2014/12/11 			Inizializzazione del file							Tesser Paolo
% =================================================================================================
% 0.0.2			2014/12/12			Inizio stesura della struttura dei sottocapitoli	Tesser Paolo
% =================================================================================================
% 0.0.3			2014/12/14			Inizio stesura sottosezione processi di gestione	Tesser Paolo
% =================================================================================================
% 0.0.4			2014/12/16			Miglioramento sottosezioni e continuata stesura		Tesser Paolo
% =================================================================================================
%


% CONTENUTO DEL CAPITOLO

\section{Processi Organizzativi}

	\subsection{Gestione dei Processi}
		\subsubsection{Attività}
			\paragraph{Gestione delle Comunicazioni}
					\subparagraph{Mail} \label{sec:mail}
Ogni membro del gruppo avrà una mail personale creata grazie all'acquisizione di un dominio web sul servizio di hosting NetSons. \\
Il formato dell'indirizzo dovrà essere del tipo esposto di seguito e servirà per registrarsi ad ogni servizio che il team andrà ad utilizzare:
						\begin{center}
							cognome.nome@mashup-unipd.it
						\end{center}
												
				\subparagraph{Comunicazioni interne} \label{sec:comunicazioni_interne}
Le comunicazione interne e prettamente informali verranno gestite tramite un gruppo su WhatsApp denominato MashUp. \\
Quelle formali avverranno attraverso il sistema di ticketing Asana che consente una chat di gruppo al suo interno e che notifica a tutti gli altri membri tramite mail quando qualcuno scrive qualcosa in essa. \\
Le norme e le procedure relative a questo servizio verranno trattate in dettaglio nella sezione \ref{sec:gestione_dei_ticket} e in quella \ref{sec:Asana}. \\
Se si necessita di un interazione vocale con gli altri membri, qualora non fossero presenti nello stesso luogo, verrà utilizzata l'applicazione Skype.
				\subparagraph{Comunicazioni esterne} \label{sec:comunicazioni_esterne}
Le comunicazione esterne vengono effettuate dal \roleProjectManager{} in quanto rappresenta il gruppo \groupName. \\
Egli, attraverso la seguente mail, manterrà i contatti con il proponente e con il committente. In caso lo ritenga necessario, girerà tali TO DO agli altri membri del team.
					\begin{center}
						info@mashup-unipd.it
					\end{center}
\'E stato inoltre stabilito, insieme al proponente, che l'interazione con lui, qualora non potesse avvenire tramite un incontro esterno, specificato nella sezione \ref{sec:riunioni_esterne}, possa avvenire tramite videochiamata in Skype	e condividendo parte dei documenti del progetto.	
			\paragraph{Gestione delle Riunioni}
				\subparagraph{Riunioni interne}
Il \roleProjectManager{} ha il compito di convocare le riunioni interne al team. Dovrà quindi informare i componenti tramite le metodologie viste nella sezione \ref{sec:comunicazioni_interne}.\\
Per ogni nuovo incontro dovranno essere specificati la data, l’ora, il luogo, il proponente e la motivazione che lo hanno reso necessario. \\
Ad ogni membro del gruppo è consentito chiedere la convocazione di una riunione interna. Il \roleProjectManager{}, una volta valutati i motivi e la necessità di tale incontro, provvederà ad organizzarlo secondo le norme viste in precedenza.
				\subparagraph{Riunioni esterne} \label{sec:riunioni_esterne}
Il \roleProjectManager{} ha il compito di concordare la data, l’ora e il luogo dell'incontro con il proponente o con il committente attraverso il meccanismo visto nella sezione \ref{sec:comunicazioni_esterne}. \\
Una volta trovato l'accordo dovrà notificarlo agli altri membri secondo i metodi presenti nella sezione \ref{sec:comunicazioni_interne}. \\
Ad ogni membro del gruppo è consentito chiedere la convocazione di una riunione esterna. Il \roleProjectManager, oltre ad accertarsi dei motivi e delle necessità di tale incontro, dovrà garantire che siano presenti almeno due componenti del team. Sarà compito di uno dei presenti, delegato di volta in volta, redigere il verbale dell’incontro avvenuto.
			\paragraph{Gestione dei Ticket} \label{sec:gestione_dei_ticket}
			
			\paragraph{Gestione delle Change Request}
			
		\subsubsection{Procedure}
			\paragraph{Procedura d'assegnazione dei ticket}
TO DO
			\paragraph{Procedura di ricezione dei ticket}
TO DO
			\paragraph{Procedura di generazione di una change request}
TO DO
			\paragraph{Procedura di rilevazione dei rischi}
TO DO
		\subsubsection{Norme}
			\paragraph{Ruoli di Progetto}
TO DO
				\subparagraph{Responsabile di Progetto}
TO DO
				\subparagraph{Amministratore}
TO DO
				\subparagraph{Analista}
TO DO
				\subparagraph{Progettista}
TO DO
				\subparagraph{Programmatore}
TO DO
				\subparagraph{Verificatore}
TO DO
			
			\paragraph{Formato dei ticket}
TO DO			
			\paragraph{Formato delle change request}
TO DO	
	
		\subsubsection{Strumenti}
			\paragraph{Asana} \label{sec:Asana}
TO DO
			\paragraph{Astah}
TO DO
			\paragraph{ProjectLibre}
TO DO
			\paragraph{NetSons}
NetSons è il servizio di hosting che il team ha deciso di adottare a scopo principalmente formativo. \\
Il dominio creato è:
				\begin{center}
					\url{http://www.mashup-unipd.it}
				\end{center}
Il servizio fornisce anche una serie di email personali che il gruppo ha deciso di utilizzare come spiegato nelle sezioni \ref{sec:mail} e \ref{sec:comunicazioni_esterne}.
			\paragraph{Skype}
Skype è l'applicativo scelto per effettuare videochiamate o chiamate VoIP tra i componenti del gruppo quando c'è la necessità di consultarsi o risolvere problemi e non è possibile essere presenti fisicamente nello stesso ambiente.		
			\paragraph{WhatsApp}
WhatsApp è l'applicativo di messaggistica scelto per le comunicazioni interne e informali al gruppo.
			\paragraph{TO DO - Strumento per la Presentazione}
TO DO

	\subsection{Gestione delle Infrastrutture}
		\subsubsection{Attività}
			\paragraph{Gestione del Repository}
TO DO		
			\paragraph{Gestione dei Git Hooks}
TO DO
		\subsubsection{Norme}
			\paragraph{Repository}
				\subparagraph{Convenzioni sui nomi dei file}
TO DO		
				\subparagraph{Messaggio di commit}
TO DO		
			
		\subsection{Strumenti}
			\paragraph{Git}
TO DO	
			\paragraph{GitHub}
TO DO			
			\paragraph{Git Hooks}
TO DO
			\paragraph{Google Drive}
Google Drive è lo strumento che si è scelto di utilizzare per gestire file che non necessitano di essere sottoposti a controllo di versione. \\
In particolare verrà impiegato per condividere manuali di utilità alla formazione dei membri del gruppo o per la stesura di idee veloci che andranno poi riviste e documentate ufficialmente nell'apposito repository.
			\paragraph{Sistema Operativo}
TO DO



	\subsection{Formazione dei membri del team}
I membri del gruppo, per soddisfare le richieste assegnate dal \roleProjectManager{} al quale non sanno fare fronte con le conoscenze attuali in loro possesso, dovranno documentarsi adeguatamente durante ore esterne a quelle di lavoro, non imputabili perciò come costi al proponente.