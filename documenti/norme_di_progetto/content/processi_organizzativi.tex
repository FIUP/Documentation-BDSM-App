% =================================================================================================
% File:			processi_organizzativi.tex
% Description:	Definisce il capitolo deei processi organizzativi
% Created:		2014/12/11
% Author:		Tesser paolo
% Email:		tesser.paolo@mashup-unipd.it
% =================================================================================================
% Modification History:
% Version		Modifier Date		Change												Author
% 0.0.1 		2014/12/11 			Inizializzazione del file							Tesser Paolo
% =================================================================================================
% 0.0.2			2014/12/12			Inizio stesura della struttura dei sottocapitoli	Tesser Paolo
% =================================================================================================
% 0.0.3			2014/12/14			Inizio stesura sottosezione processi di gestione	Tesser Paolo
% =================================================================================================
%


% CONTENUTO DEL CAPITOLO

\section{Processi Organizzativi}
	\subsection{Gestione dei Processi}
		\subsubsection{Attività}
			\paragraph{Gestione delle Comunicazioni}
				\subparagraph{Comunicazioni interne}
TO DO				
				\subparagraph{Comunicazioni esterne}
TO DO				
			\paragraph{Gestione delle Riunioni}
				\subparagraph{Riunioni interne}
TO DO				
				\subparagraph{Riunioni esterne}
TO DO				

			\paragraph{Gestione dei Ticket}
			
		\subsubsection{Procedure}
TO DO
		\subsubsection{Norme}
TO DO (Inserire Ruoli di progetto)	
		\subsubsection{Strumenti}
TO DO
		
	\subsection{Gestione delle Infrastrutture}
TO DO


	\subsection{Formazione dei membri del team}
I membri del gruppo, per soddisfare le richieste assegnate dal \roleProjectManager{} al quale non sanno fare fronte con le conoscenze attuali in loro possesso, dovranno documentarsi adeguatamente durante ore esterne a quelle di lavoro, non imputabili perciò come costi al proponente.