% =================================================================================================
% File:			processi_organizzativi.tex
% Description:	Definisce il capitolo deei processi organizzativi
% Created:		2014/12/11
% Author:		Tesser paolo
% Email:		tesser.paolo@mashup-unipd.it
% =================================================================================================
% Modification History:
% Version		Modifier Date		Change												Author
% 0.0.1 		2014/12/11 			Inizializzazione del file							Tesser Paolo
% =================================================================================================
% 0.0.2			2014/12/12			Inizio stesura della struttura dei sottocapitoli	Tesser Paolo
% =================================================================================================
% 0.0.3			2014/12/14			Inizio stesura sottosezione processi di gestione	Tesser Paolo
% =================================================================================================
% 0.0.4			2014/12/16			Miglioramento sottosezioni e continuata stesura		Tesser Paolo
% =================================================================================================
% 0.0.5			2014/12/20			Inseriti ruoli di progetto e alcuni strumenti		Tesser Paolo
% =================================================================================================
%
%


% CONTENUTO DEL CAPITOLO

\section{Processi Organizzativi}

	\subsection{Gestione dei Processi}
		\subsubsection{Attività}
			\paragraph{Gestione delle Comunicazioni}
					\subparagraph{Mail} \label{sec:mail}
					Ogni membro del gruppo avrà una mail personale creata grazie all'acquisizione di un dominio web sul servizio di hosting NetSons. \\
					Il formato dell'indirizzo dovrà essere del tipo esposto di seguito e servirà per registrarsi ad ogni servizio che il team andrà ad utilizzare:
						\begin{center}
							cognome.nome@mashup-unipd.it
						\end{center}
												
				\subparagraph{Comunicazioni interne} \label{sec:comunicazioni_interne}
				Le comunicazione interne e prettamente informali verranno gestite tramite un gruppo su WhatsApp denominato MashUp. \\
				Quelle formali avverranno attraverso il sistema di ticketing Asana che consente una chat di gruppo al suo interno e che notifica a tutti gli altri membri tramite mail quando qualcuno scrive qualcosa in essa. \\
				Le norme e le procedure relative a questo servizio verranno trattate in dettaglio nella sezione \ref{sec:gestione_dei_ticket} e in quella \ref{sec:Asana}. \\
				Se si necessita di un interazione vocale con gli altri membri, qualora non fossero presenti nello stesso luogo, verrà utilizzata l'applicazione Skype.
				\subparagraph{Comunicazioni esterne} \label{sec:comunicazioni_esterne}
				Le comunicazione esterne vengono effettuate dal \roleProjectManager{} in quanto rappresenta il gruppo \groupName. \\
				Egli, attraverso la seguente mail, manterrà i contatti con il proponente e con il committente. In caso lo ritenga necessario, girerà tali messaggi agli altri membri del team.
					\begin{center}
						info@mashup-unipd.it
					\end{center}
					\'E stato inoltre stabilito, insieme al proponente, che l'interazione con lui, qualora non potesse avvenire tramite un incontro esterno, specificato nella sezione \ref{sec:riunioni_esterne}, possa avvenire tramite videochiamata in Skype	e condividendo parte dei documenti del progetto.	
			\paragraph{Gestione delle Riunioni}
				\subparagraph{Riunioni interne}
				Il \roleProjectManager{} ha il compito di convocare le riunioni interne al team. Dovrà quindi informare i componenti tramite le metodologie viste nella sezione \ref{sec:comunicazioni_interne}.\\
				Per ogni nuovo incontro dovranno essere specificati la data, l’ora, il luogo, il proponente e la motivazione che lo hanno reso necessario. \\
				Ad ogni membro del gruppo è consentito chiedere la convocazione di una riunione interna. Il \roleProjectManager{}, una volta valutati i motivi e la necessità di tale incontro, provvederà ad organizzarlo secondo le norme viste in precedenza.
				\subparagraph{Riunioni esterne} \label{sec:riunioni_esterne}
				Il \roleProjectManager{} ha il compito di concordare la data, l’ora e il luogo dell'incontro con il proponente o con il committente attraverso 	il meccanismo visto nella sezione \ref{sec:comunicazioni_esterne}. \\
				Una volta trovato l'accordo dovrà notificarlo agli altri membri secondo i metodi presenti nella sezione \ref{sec:comunicazioni_interne}. \\
				Ad ogni membro del gruppo è consentito chiedere la convocazione di una riunione esterna. Il \roleProjectManager, oltre ad accertarsi dei motivi e delle necessità di tale incontro, dovrà garantire che siano presenti almeno due componenti del team. Sarà compito di uno dei presenti, delegato di volta in volta, redigere il verbale dell’incontro avvenuto.
				
			\paragraph{Gestione dei ticket} \label{sec:gestione_dei_ticket}
			TO DO			
			\paragraph{Gestione delle change request}
			TO DO
			
					
		\subsubsection{Procedure}
			\paragraph{Procedura d'assegnazione dei ticket}
			TO DO
			\paragraph{Procedura di ricezione dei ticket}
			TO DO
			\paragraph{Procedura di generazione di una change request}
			TO DO
			\paragraph{Procedura di rilevazione dei rischi}
			TO DO


		\subsubsection{Norme}
			\paragraph{Ruoli di Progetto}
			Lo sviluppo di un progetto necessita della collaborazione di diverse persone che andranno a ricoprire incarichi che rappresentano delle figure aziendali. \\
			Viene garantito che ogni componente del gruppo \groupName{} dovrà ricoprire, almeno una volta, tutti i ruoli. \\
			Questa rotazione potrebbe generare dei problemi dovuti a conflitti d'interesse in quanto lo stesso soggetto che svolge un compito potrebbe ritrovarsi a verificare l'operato svolto in precedenza. \\
			Un'attenta pianificazione da parte del \roleProjectManager{} dovrà preoccuparsi di non fare avvenire queste situazioni. \\
			Il compito di garantire che non siano stati fatti errori spetta al \roleVerifier, il quale, una volta riscontrata una incongruenza, dovrà notificarla al \roleProjectManager{} che avrà il compito di risolverla. \\ 
			Vengono ora presentati i diversi ruoli, delineandone le mansioni e le responsabilità.
				\subparagraph{Responsabile di Progetto}
				Il \roleProjectManager{} rappresenta il team e il progetto verso il committente e il proponente TO DO\\
				Ha quindi le seguenti responsabilità e compiti:
					\begin{itemize}
						\item pianificazione e coordinamento delle attività;
						\item gestione e controllo delle risorse;
						\item analisi e gestione dei rischi;
						\item approvazione dell'offerta economica;
						\item assicurarsi che tutte le attività svolte siano conformi alle \docNameVersionNdP{} e rispettino la pianificazione effettuata nel \docNameVersionPdP;
						\item garantire che non ci siano conflitti di interesse. A tal proposito, se il \roleProjectManager{} dovesse prendere parte alla stesura di qualche documento, dovrà nominare un \roleProjectManager delegato che avrà il compito di sostituirlo nell'approvazione di quei documenti;
						\item redigere il \docNameVersionPdP.
					\end{itemize}
					
				\subparagraph{Amministratore}
				L'\roleAdministrator{} è il responsabile dell'ambiente di lavoro.
				Ha quindi il compito di:
					\begin{itemize}
						\item ricercare e implementare strumenti che automatizzino il maggior numero di operazioni;
						\item gestire il versionamento del codice e della documentazione di progetto;
						\item fornire procedure 
						\item redigere le \docNameVersionNdP.
					\end{itemize}
				\subparagraph{Analista}
				L'\roleAnalyst{} TO DO \\
				Ha quindi il compito di:
					\begin{itemize}
						\item TO DO
						\item TO DO
						\item redigere lo \docNameVersionSdF;
						\item redigere l'\docNameVersionAdR.
					\end{itemize}
				\subparagraph{Progettista}
				Il \roleDesigner{} TO DO \\
				Ha quindi il compito di:
					\begin{itemize}
						\item TO DO
						\item TO DO
						\item redigere il \docNameVersionPdQ.
					\end{itemize}
				\subparagraph{Programmatore}
				Il \roleProgrammer{} TO DO \\
				Ha quindi il compito di:
					\begin{itemize}
						\item TO DO
						\item TO DO
					\end{itemize}
				\subparagraph{Verificatore}
				Ha quindi il compito di:
				Il \roleVerifier{} TO DO
					\begin{itemize}
						\item TO DO
						\item TO DO
						\item redige il \docNameVersionPdQ{} per la parte che illustra l'esito e la completezza delle verifiche effettuate.
					\end{itemize}
			
			
			\paragraph{Formato dei ticket}
			TO DO		
			\paragraph{Formato delle change request}
			TO DO	
	
		\subsubsection{Strumenti}
			\paragraph{Asana} \label{sec:Asana}
			Asana è l'applicazione web scelta per la gestione dei task.\\
			Registrandosi con una mail di dominio personale come quella riportata nella sezione \ref{sec:mail} è possibile usufruire di maggiori servizi che consentono una più facile gestione del team da parte del \roleProjectManager.
			\paragraph{Astah}
			Astah è l'applicativo scelto per la creazione di grafici UML.
			La versione adottata è quella Professional, resa disponibile con una licenza gratuita per gli studenti dei corsi universitari registrandosi tramite l'indirizzo:
				\begin{center}
					nome.cognome.X@studenti.unipd.it
				\end{center}

			\paragraph{ProjectLibre}
			ProjectLibre è il prodotto scelto per la realizzazione dei diagrammi di Gaant e quelli di PERT.
			\paragraph{NetSons}
			NetSons è il servizio di hosting che il team ha deciso di adottare a scopo principalmente formativo. \\
			Il dominio creato è:
				\begin{center}
					\url{http://www.mashup-unipd.it}
				\end{center}
			Il servizio fornisce anche una serie di email personali che il gruppo ha deciso di utilizzare come spiegato nelle sezioni \ref{sec:mail} e \ref{sec:comunicazioni_esterne}.
			\paragraph{Skype}
			Skype è l'applicativo scelto per effettuare videochiamate o chiamate VoIP tra i componenti del gruppo quando c'è la necessità di consultarsi o risolvere problemi e non è possibile essere presenti fisicamente nello stesso ambiente.		
			\paragraph{WhatsApp}
			WhatsApp è l'applicativo di messaggistica scelto per le comunicazioni interne e informali al gruppo.
			\paragraph{TO DO - Strumento per la Presentazione}
			TO DO


	\subsection{Gestione delle Infrastrutture}
		\subsubsection{Attività}
			\paragraph{Gestione del Repository}
			Il gruppo ha deciso di utilizzare due repository che servono a svolgere funzioni diverse, ma necessarie, allo sviluppo del sistema finale. \\
			Il servizio di hosting scelto consente, tramite licenza ``educational'', di impostare la visibilità degli ambienti come privata. \\
			Una volta iscritti i membri dovranno comunicare all'\roleAdministrator{} il loro nome utente che provvederà ad autorizzarne l'accesso.
				\begin{itemize}
					\item \textbf{doc\_BDSM\_App}: gestione della documentazione;
					\item \textbf{src\_BDSM\_App}: gestione del codice;
				\end{itemize}
			\paragraph{Gestione dei Git Hooks}
			TO DO
			
			
		\subsubsection{Procedure}
			\paragraph{Installazione Git Hooks}
			TO DO
			\paragraph{Installazione messaggio di commit}
			TO DO
			
			
		\subsubsection{Norme}
			\paragraph{Repository}
				\subparagraph{Nomi dei file in doc\_BDSM\_App}
				I file e le cartelle presenti nel repository devono essere conformi al seguente formalismo tratto dallo Standard ISO 9660:1999 (Level 2):
					\begin{itemize}
						\item i caratteri usati sono solo quelli minuscoli a-z, 0-9, l'underscore (\_) e il punto (.) (esempio: nome\_del\_documento.tex);
						\item non sono ammessi caratteri accentati;
						\item i nomi non possono includere spazi o finire con un punto (.);
						\item i nomi non devono contenere più di un punto (.) ad eccezione di quelli che fanno riferimento ad una specifica versione (esempio: studio\_di\_fattibilita\_v1.0.0.pdf);
						\item i nomi non devono essere più lunghi di 21 caratteri esclusi i 3 destinati all'estensione.
					\end{itemize}

				\subparagraph{Struttura di doc\_BDSM\_App}
				Le cartelle nel repository verranno organizzate nel seguente modo a partire dalla root:
					\begin{itemize}
						\item \textbf{consegne}: contenente i documenti finali con anche la versione da consegnare alle diverse revisioni. Sono presenti le cartelle:
							\begin{itemize}
								\item \textbf{revisione\_dei\_requisiti}: contenente i documenti necessari alla revisione dei requisiti;
								\item \textbf{revisione\_di\_progettazione}: contenente i documenti necessari alla revisione di progettazione;
								\item \textbf{revisione\_di\_qualifica}: contenente i documenti necessari alla revisione di qualifica;
								\item \textbf{revisione\_di\_accettazione}: contenente i documenti necessari alla revisione di accettazione.
							\end{itemize}
						\noindent
						All'interno di ciascuna di esse ci sarà un'ulteriore suddivisione in due categorie:
							\begin{itemize}
								\item \textbf{interni}: contenente i documenti necessari al gruppo per la sua organizzazione;
								\item \textbf{esterni}: contenente i documenti necessari alla pianificazione e all'avanzamento dello sviluppo.
							\end{itemize}
							
						\item \textbf{documenti}:
							\begin{itemize}
								\item \textbf{template}: contenente i file che servono per gestire in maniera univoca la redazione di un documento;
								\item \textbf{template\_document}: contenente i file di esempio che dovranno essere utilizzati per ogni documento reale;
								\item una cartella per ogni documento che avrà come nome quello del documento in questione (esempio: norme\_di\_progetto).
							\end{itemize}
							
						\item \textbf{script}: contenente tutti gli script necessari ad automatizzare il lavoro e il controllo della documentazione.
					\end{itemize}
					
				\subparagraph{Nomi dei file in src\_BDSM\_App}
				La presente sezione verrà redatta in futuro, presumibilmente nella prima fase di progettazione di dettaglio e codifica dei requisiti come esposto nel documento \docNameVersionPdP.
				\subparagraph{Struttura di src\_BDSM\_App}
				La presente sezione verrà redatta in futuro, presumibilmente nella prima fase di progettazione di dettaglio e codifica dei requisiti come esposto nel documento \docNameVersionPdP.

				\subparagraph{Modello di sviluppo}
				Per lo sviluppo della documentazione e del codice necessari al progetto si è scelto di adottare il modello proposto dal proponente, spiegato nel dettaglio al seguente link:
					\begin{center}
						\url{http://nvie.com/posts/a-successful-git-branching-model/}
					\end{center}
					Ogni membro del gruppo dovrà leggere l'articolo e applicarlo secondo le norme di denominazione dei branch presenti in esso.

				\subparagraph{Messaggio di commit} \label{sec:messaggio_di_commit}
				Il messaggio di commit dovrà essere conforme alla seguente notazione:
				\begin{verbatim}
					Title:
					Desc:
					Task: Reported in #id_task
					Option: [option][option]
					END
				\end{verbatim}
				% \lstnewenvironment{testo}[1][]
				% {
				%	\lstset{
				%		basicstyle=\small\ttfamily, 
				%		columns=fullflexible,
				%		keywordstyle=\color{red}\bfseries, 
				%		commentstyle=\color{blue},
				%		language=C++, % non importa il linguaggio in questo caso perchè è solo testo
				%		basicstyle=\small,
				%		numbers=left,
				%		numberstyle=\tiny,
				%		stepnumber=1,
				%		numbersep=5pt,
				%		frame=shadowbox,
				%		#1
				%	}
				% }{}
				
				% \begin{testo}[caption={Template messaggio di commit}]
				%	Title:
				%	Desc:
				%	Task: Reported in #id_task
				%	Option: [option][option]
				%	END
				% \end{testo}
				
					\begin{itemize}
						\item \textbf{Title}: inserire una breve descrizione come titolo di quello che è stato fatto;
						\item \textbf{Desc}: inserire una descrizione più esaustiva dell'attività svolta qualora non fosse sufficiente quella data nel titolo;
						\item \textbf{Task}: al posto della dicitura ``id\_task'' inserire l'id del task effettivo reperibile su Asana. TO DO
						\item \textbf{Option}: Le opzioni applicabili sono:
							\begin{itemize}
								\item \textbf{complete}: per chiudere il task qualora fosse l'ultimo necessario allo svolgimento del compito indicato;
								\item \textbf{delay}: per aggiungere il tag DELAY al task se si è in ritardo con lo svolgimento del compito assegnato.
							\end{itemize}
					\end{itemize}
									
		\subsubsection{Strumenti}
			\paragraph{Git}
			Git è il sistema di controllo di versione utilizzato per entrambi i repository del team.	
			\paragraph{GitHub}
			GitHub è il servizio web di hosting adottato per tenere una copia del repository del progetto.
			\paragraph{Git Hooks}
			I Git Hooks sono degli script personalizzabili che vengono eseguiti in corrispondenza di un determinato evento avvenuto nel repository. \\
			Vengono utilizzati per controllare il rispetto delle norme in maniera automatizzata non permettendo a tutto ciò che non è conforme di entrare nel sistema. \\
			Sono anche impiegati per automatizzare la gestione dei task da parte dei componenti del gruppo secondo le norme presenti nella sezione \ref{sec:messaggio_di_commit}. \\
			Risiedono nella seguente cartella a partire dalla root principale del repository:
				\begin{center}
					.git/hooks
				\end{center}
			\paragraph{Google Drive} \label{sec:google_drive}
			Google Drive è lo strumento che si è scelto di utilizzare per gestire file che non necessitano di essere sottoposti a controllo di versione. \\
			In particolare verrà impiegato per condividere manuali di utilità alla formazione dei membri del gruppo o per la stesura di idee veloci che andranno poi riviste e documentate ufficialmente nell'apposito repository.
			\paragraph{Sistema Operativo}
			I membri del gruppo operano su due sistemi operativi diversi. \\
			Una parte utilizza Linux con distribuzione Ubuntu 14.04 e l'altra utilizza Mac OS X con versione 10.10 Yosemite.

	\subsection{Formazione dei membri del team}
	I membri del gruppo, per soddisfare le richieste assegnate dal \roleProjectManager{} al quale non sanno fare fronte con le conoscenze attuali in loro possesso, dovranno documentarsi adeguatamente durante ore esterne a quelle di lavoro, non imputabili perciò come costi al proponente.\\
	I membri possono trovare materiale utile a questo scopo nel luogo dove risiedono i file che non necessitano di versionamento come specificato nella sezione \ref{sec:google_drive}