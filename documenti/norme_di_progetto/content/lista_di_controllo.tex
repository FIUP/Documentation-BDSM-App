% =================================================================================================
% File:			lista_di_controllo.tex
% Description:	Definisce il capitolo di appendice contenente la lista di controllo per i verificatori
% Created:		2015-01-14
% Author:		Tesser paolo
% Email:		tesser.paolo@mashup-unipd.it
% =================================================================================================
% Modification History:
% Version		Modifier Date		Change												Author
% 0.0.1 		2015-01-15 			Inizializzazione del file							Tesser Paolo
% =================================================================================================
% 0.0.2			2015-01-20			stesura lista per errori, italiano, LaTex			Luca S.
% =================================================================================================
% 3.0.1

% CONTENUTO DEL CAPITOLO

\appendix

\section{Lista di controllo} % (fold)
\label{sec:lista_di_controllo}
Di seguito viene presentata la lista di controllo con gli errori più comuni da controllare durante una inspection:
	\begin{itemize}
		\item \textbf{Errori ortografici}:
			\begin{itemize}
				\item negli elenchi non numerati non è rispettato l'uso dei due punti nel primo termine in grassetto;
				\item negli elenchi non numerati la punteggiatura alla fine non è corretta;
				\item nell'uso dei comandi rapidi non viene inserita la parentesi graffa alla fine per generare la spaziatura.
			\end{itemize}

		\item \textbf{Italiano}:
			\begin{itemize}
				\item i caratteri accentati all'inizio della frase vengono erroneamente sostituti con l'apostrofo;
				\item proponente e committente: viene confuso il loro significato.
			\end{itemize}

		\item \textbf{\LaTeX}:
			\begin{itemize}
				\item non vengono usati i comandi rapidi definiti nel file \emph{commands.tex};
				\item non vengono usate inserite le caption nelle figure e nelle tabelle.
			\end{itemize}

		\item \textbf{UML}:
			\begin{itemize}
				\item non viene utilizzato il carattere corsivo per identificare le classi astratte;
				\item non viene utilizzata la notazione di UML 2.0 per la rappresentazione delle interfacce;
				\item non si utilizza la notazione UML 2.0 per la rappresentazione delle associazioni tra classi;
				\item i nomi delle variabili e dei metodi non rispettano le norme definite.
			\end{itemize}
	\end{itemize}
% section lista_di_controllo (end)
