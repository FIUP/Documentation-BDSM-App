% =================================================================================================
% File:			conf_angular_file.tex
% Description:	Definisce il capitolo di appendice contenente il modo con cui andranno codificati determinati file di AngularJS
% Created:		2015-04-24
% Author:		Tesser paolo
% Email:		tesser.paolo@mashup-unipd.it
% =================================================================================================
% Modification History:
% Version		Modifier Date		Change												Author
% 0.0.1 		2015-04-24 			Inizializzazione del file							Tesser Paolo
% =================================================================================================
%

% CONTENUTO DEL CAPITOLO


\section{Configurazione File AngularJS} % (fold)
\label{sec:configurazione_file_angularjs}
Di seguito vengono presentati i template che ogni \roleProgrammer{} dovrà utilizzare quando andrà ad effettuare la codifica del lato client. Andranno inseriti seguendo la procedura descritta in \ref{par:aggiunta_template_di_struttura_file_in_webstorm}. Il linguaggio utilizzato per il sistema di templating è quello offerto dagli applicativi JetBrains e cioè Apache Velocity. \newline

	\subsection{View} % (fold)
	\label{sub:view}
	Le View sono template HTML che verranno richiamati opportunamente dal sistema di routing e di state offerto dal modulo \textbf{ui.router} disponibile in AngularJS. \newline
	Per rendere validabili queste pagine, secondo le regole imposte dal W3C, ogni direttiva dovrà essere preceduta dal prefisso \textbf{data-}. \newline
	Esempi:
		\begin{itemize}
			\item data-ng-app
			\item data-ng-controller
			\item data-ng-repeat
		\end{itemize}

	% subsection view (end)

	\subsection{Page Controller} % (fold)
	\label{sub:page_controller}
	\begin{verbatim}
	 (function(){
	    'use strict';

	   /**
	   * Name: nome.js
	   * Author: Cognome Nome
	   * Mail. cognome.nome@mashup-unipd.it
	   *
	   * Modify
	   * Version  Date        Author          Desc
	   * ==========================================================
	   * 0.0.1    aaaa-mm-dd  Cognome Nome    description
	   * -----------------------------------------------------------
	   *
	   */

	   /**
	   * @ngdoc function
	   * @name nameModule.controller:NameCtrl
	   * @description
	   * # NameCtrl
	   * Controller of the clientApp
	   */

	  function $NameCtrl(){

	    var vm = this;

	  };

	  $NameCtrl.${DS}inject = [];

	  angular
		.module('$nameModule')
		.controller('$NameCtrl', $NameCtrl);

	})();
	\end{verbatim}
	% subsubsection controller (end)

	\subsection{Route Controller} % (fold)
	\label{sub:route_controller}
	\begin{verbatim}
	(function(){
	  'use strict';

	  /**
	   * Name: $nameConfigRoute.js
	   * Author: Cognome Nome
	   * Mail. cognome.nome@mashup-unipd.it
	   *
	   * Modify
	   * Version  Date        Author          Desc
	   * ==========================================================
	   * 0.0.1    aaaa-mm-dd  Cognome Nome    description
	   * -----------------------------------------------------------
	   *
	   */

	  function $NameConfigRoutes(${DS}stateProvider) {

	    ${DS}stateProvider
	      .state('state',{
	        url: '/url',
	        views: {
	          '':{
	            templateUrl: 'url',
	            controller: '$NameCtrl'
	          }
	        }
	      })
	  };

	  angular
		.module('$nameRouteApp', ['ui.router'])
	    .config(['${DS}stateProvider', '${DS}urlRouterProvider', $NameConfigRoutes]);

	})();
	\end{verbatim}
	% subsubsection route_controller (end)


	\subsection{Services Model} % (fold)
	\label{sub:services_model}
	\begin{verbatim}
		(function() {

			'use strict';

		  /**
		   * Name: $nameService.js
		   * Author: Cognome Nome
		   * Mail. cognome.nome@mashup-unipd.it
		   *
		   * Modify
		   * Version  Date        Author          Desc
		   * ==========================================================
		   * 0.0.1    aaaa-mm-dd  Cognome Nome    description
		   * -----------------------------------------------------------
		   *
		   */

			function $NameService(){

				var factory = {
					publicVar: '',
					publicFunction: publicFunction
				};

				return factory;

				///////////////


				/**
				 * TODO
				 */
				function publicFunction(){
					// TODO
				}

				////////////////
				
				/**
				 * TODO
				 */
				function privateFunction(){
					// TODO
				}

			}


			$NameService.$inject = [];

			angular
				.module('$nameModule')
				.factory('$NameService', $NameService);


		})();
	\end{verbatim}
	% subsection services_model (end)

% section configurazione_file_angularjs (end)