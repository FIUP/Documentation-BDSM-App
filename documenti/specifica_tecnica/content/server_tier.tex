% =================================================================================================
% File:			server_tier.tex
% Description:	Defiinisce la sezione relativa al back-end dell'applicazione
% Created:		2015-03-27
% Author:		Tesser Paolo
% Email:		tesser.paolo@mashup-unipd.it
% =================================================================================================
% Modification History:
% Version		Modifier Date		Change											Author
% 0.0.1 		2015-03-27 			creato scheletro								Tesser Paolo
% =================================================================================================
%
% =================================================================================================
%

% CONTENUTO DEL CAPITOLO

\subsection{Server (Back-end)} % (fold)
\label{sub:server}
  \subsubsection{bdsm\_app::server::db} % (fold)
  \label{ssub:bdsm_app_server_db}
  [TO DO] (diagramma) \newline \newline

  \begin{itemize}
    \item \textbf{Descrizione}: [TO DO];
    \item \textbf{Package contenuti}:
      \begin{itemize}
        \item bdsm\_app::server::raw\_model;
        \item bdsm\_app::server::app\_model.
      \end{itemize}
    \item \textbf{Interazione con altri componenti}: [TO DO];
  \end{itemize}
  % subsubsection bdsm_app_server_db (end)


  % PACKAGE DB

  \subsubsection{bdsm\_app::server::db::raw\_model} % (fold)
  \label{ssub:bdsm_app_server_raw_model}
  [TO DO] (diagramma) \newline \newline

  \begin{itemize}
    \item \textbf{Descrizione}: Package che definisce il modello del database in cui vengono salvati i dati grezzi ricavati dai vari social network;
    \item \textbf{Padre}: server::db;
    \item \textbf{Interazione con altri componenti}: interagisce con il miner definito nella sezione [TO DO] il quale salva i dati secondo questo modello, interagisce con l'endpoints definito nella sezione [TO DO] il quale interroga la base di dati;
  \end{itemize}
  		
  		\paragraph{Classi} % (fold)
			\subparagraph{bdsm\_app::server::db::raw\_model::AbsRawData} % (fold)
			\label{subp:bdsm_app_server_raw_model_AbsRawData}
				\begin{itemize}
					\item \textbf{Descrizione}: classe astratta che definisce il modello di un dato grezzo;
					\item \textbf{Utilizzo}: [TO DO];
					\item \textbf{Relazioni con altre classi}:
					\begin{itemize}
						\item AbsFbRawData;
						\item RawFbPageTrend;
						\item RawFbEventTrend;
						\item AbsTwRawData;
						\item RawTwUserTrend;
						\item RawTwHashtagTrend;	
						\item AbsIgRawData;
						\item RawIgUserTrend;
						\item RawIgHashtagTrend;				
					\end{itemize}
				\end{itemize}
  % subsubsection bdsm_app_server_raw_model (end)

  \subsubsection{bdsm\_app::server::db::app\_model} % (fold)
  \label{ssub:bdsm_app_server_app_model}
  [TO DO] (diagramma) \newline \newline

  \begin{itemize}
    \item \textbf{Descrizione}: [TO DO];
    \item \textbf{Padre}: [TO DO] (qualora presente);
    \item \textbf{Package contenuti}: [TO DO] (qualora presente);
    \item \textbf{Interazione con altri componenti}: [TO DO];
  \end{itemize}
  % subsubsection bdsm_app_server_app_model (end)
  
	  \subsubsection{bdsm\_app::server::miner} % (fold)
  \label{ssub:bdsm_app_server_miner}
  [TO DO] (diagramma) \newline \newline

  \begin{itemize}
    \item \textbf{Descrizione}: [TO DO];
    \item \textbf{Padre}: [TO DO] (qualora presente);
    \item \textbf{Package contenuti}: [TO DO] (qualora presente);
    \item \textbf{Interazione con altri componenti}: [TO DO];
  \end{itemize}
  % subsubsection 
  
  	  \subsubsection{bdsm\_app::server::processor} % (fold)
  \label{ssub:bdsm_app_server_processor}
  [TO DO] (diagramma) \newline \newline

  \begin{itemize}
    \item \textbf{Descrizione}: [TO DO];
    \item \textbf{Padre}: [TO DO] (qualora presente);
    \item \textbf{Package contenuti}: [TO DO] (qualora presente);
    \item \textbf{Interazione con altri componenti}: [TO DO];
  \end{itemize}
  % subsubsection

	  	  \subsubsection{bdsm\_app::server::endpoints} % (fold)
  \label{ssub:bdsm_app_server_endpoints}
  [TO DO] (diagramma) \newline \newline

  \begin{itemize}
    \item \textbf{Descrizione}: [TO DO];
    \item \textbf{Padre}: [TO DO] (qualora presente);
    \item \textbf{Package contenuti}: [TO DO] (qualora presente);
    \item \textbf{Interazione con altri componenti}: [TO DO];
  \end{itemize}
  % subsubsection
[TO DO]
% subsection server (end)