% =================================================================================================
% File:			dp_creazionali.tex
% Description:	Defiinisce la sezione relativa a ...
% Created:		2015-03-26
% Author:		Tesser Paolo
% Email:		tesser.paolo@mashup-unipd.it
% =================================================================================================
% Modification History:
% Version		Modifier Date		Change											Author
% 0.0.1 		2015-03-26 			creato scheltro sezione							Tesser Paolo
% =================================================================================================
% 0.0.2			2015-04-08			inserito scheletro per DP: Prototype e Module	Tesser Paolo
% =================================================================================================
%

% CONTENUTO DEL CAPITOLO

\subsection{Design pattern creazionali} % (fold)
\label{sub:design_pattern_creazionali}
	\subsubsection{Prototype Pattern} % (fold)
	\label{ssub:prototype_pattern}
	[TO DO] (grafico del pattern applicato al caso di utilizzo nell'applicativo)
		\begin{itemize}
			\item \textbf{Scope dell'utilizzo}: [TO DO];
			\item \textbf{Contesto dell'utilizzo}:
				\begin{itemize}
					\item \textbf{Client}: [TO DO];
				\end{itemize}
		\end{itemize}
	% subsubsection prototype_pattern (end)

	\subsubsection{Module Pattern} % (fold)
	\label{ssub:module_pattern}
	[TO DO] (grafico del pattern applicato al caso di utilizzo nell'applicativo)
		\begin{itemize}
			\item \textbf{Scope dell'utilizzo}: [TO DO];
			\item \textbf{Contesto dell'utilizzo}:
				\begin{itemize}
					\item \textbf{Client}: [TO DO]
				\end{itemize}
		\end{itemize}
	% subsubsection module_pattern (end)

	\subsubsection{Factory Method} % (fold)
	\label{ssub:factory_method}
	[TO DO] (grafico del pattern applicato al caso di utilizzo nell'applicativo)
		\begin{itemize}
			\item \textbf{Scope dell'utilizzo}: [TO DO];
			\item \textbf{Contesto dell'utilizzo}:
				\begin{itemize}
					\item \textbf{Server}: [TO DO]
				\end{itemize}
		\end{itemize}
	% subsubsection factory_method (end)


	\subsubsection{Singleton} % (fold)
	\label{ssub:singleton}
	[TO DO] (grafico del pattern applicato al caso di utilizzo nell'applicativo)
		\begin{itemize}
			\item \textbf{Scope dell'utilizzo}: questo pattern è utilizzato per limitare l'instaziazione di un certo tipo classe ad un solo oggetto in modo tale che esso rimanga unico nel sistema in cui risiede;
			\item \textbf{Contesto dell'utilizzo}:
				\begin{itemize}
					\item \textbf{Client}: viene utilizzato ;
					\item \textbf{Server}: viene utilizzato dalla classe \texttt{server::endpoints::RequestHandler} in quanto quest'ultima, essendo implementata secondo il pattern Front Controller, rappresenta il punto di accesso comune a tutto il sistema in cui confluiscono le richieste provenienti dalle classi appartenenti al package \texttt{server::endpoints} (\ref{ssub:bdsm_app_server_endpoints}).
				\end{itemize}
		\end{itemize}
	% subsubsection singleton (end)
% subsection design_pattern_creazionali (end)
