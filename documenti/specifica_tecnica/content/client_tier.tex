% =================================================================================================
% File:			client_tier.tex
% Description:	Defiinisce la sezione relativa al front-end dell'applicazione
% Created:		2015-03-27
% Author:		Tesser Paolo
% Email:		tesser.paolo@mashup-unipd.it
% =================================================================================================
% Modification History:
% Version		Modifier Date		Change											Author
% 0.0.1 		2015-03-27 			creato scheletro								Tesser Paolo
% =================================================================================================
% 0.0.2			2015-03-28			inserito scheletro del package					Tesser Paolo
% =================================================================================================
%

% CONTENUTO DEL CAPITOLO

\subsection{Client} % (fold)
\label{sub:client}
Nel descrivere le componenti si parlerà di classi, che però saranno da intendersi nella loro accezione logica. Il linguaggio di programmazione impiegato infatti non possiede la struttura di classe, ma solo di oggetto, sarà quindi compito del codificatore riportare le classi di seguito identificate nelle strutture fornite dallo specifico linguaggio.

	\subsubsection{bdsm\_app::client} % (fold)
	\label{ssub:bdsm_app_client}
	[TO DO] (diagramma) \newline \newline
	
	\begin{itemize}
		\item \textbf{Descrizione}: [TO DO];
		\item \textbf{Package contenuti}:
			\begin{itemize}
				\item model;
				\item view;
				\item controller.
			\end{itemize}
		\item \textbf{Interazione con altri componenti}: interagisce con il server definito nella sezione [TO DO] effettuando delle chiamate ai servizi REST offerti;
	\end{itemize}
	% subsubsection bdsm_app_client (end)

	\subsubsection{bdsm\_app::client::model} % (fold)
	\label{ssub:bdsm_app_client_model}
	[TO DO]
	% subsubsection bdsm_app_client_model (end)

	\subsubsection{bdsm\_app::client::view} % (fold)
	\label{ssub:bdsm_app_client_view}
	[TO DO]
	% subsubsection bdsm_app_client_view (end)

	\subsubsection{bdsm\_app::client::controller} % (fold)
	\label{ssub:bdsm_app_client_controller}
	[TO DO]
	% subsubsection bdsm_app_client_controller (end)


	% [TO DO] (subpackage del model)

	\subsubsection{bdsm\_app::client::view::public} % (fold)
	\label{ssub:bdsm_app_client_view_public}
	[TO DO]
	% subsubsection bdsm_app_client_view_public (end)

	\subsubsection{bdsm\_app::client::view::shared} % (fold)
	\label{ssub:bdsm_app_client_view_shared}
	[TO DO]
	% subsubsection bdsm_app_client_view_shared (end)

	\subsubsection{bdsm\_app::client::view::user} % (fold)
	\label{ssub:bdsm_app_client_view_user}
	[TO DO]
	% subsubsection bdsm_appclient__view_user (end)

	\subsubsection{bdsm\_app::client::view::admin} % (fold)
	\label{ssub:bdsm_app_client_view_admin}
	
	% subsubsection bdsm_app_client_view_admin (end)


	% [TO DO] (subpackage del controller)


	\subsubsection{Nome package} % (fold)
	\label{ssub:nome_del_package}
	[TO DO] (diagramma) \newline \newline

	\begin{itemize}
		\item \textbf{Descrizione}: [TO DO];
		\item \textbf{Padre}: [TO DO] (qualora presente);
		\item \textbf{Package contenuti}: [TO DO] (qualora presente);
		\item \textbf{Interazione con altri componenti}: [TO DO];
	\end{itemize}

		\paragraph{Classi} % (fold)
			\subparagraph{Nome package::Nome classe} % (fold)
			\label{subp:subparagraph_name}
				\begin{itemize}
					\item \textbf{Descrizione}: [TO DO];
					\item \textbf{Utilizzo}: [TO DO];
					\item \textbf{Classi ereditate}: [TO DO];
					\item \textbf{Relazioni con altre classi}: [TO DO].
				\end{itemize}
			% subparagraph subparagraph_name (end)

			% subsection nome_classe (end)

		% paragraph classi (end)
	% subsubsection nome_del_package (end)

% subsection client (end)