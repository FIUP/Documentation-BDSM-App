% =================================================================================================
% File:			introduzione.tex
% Description:	Defiinisce la sezione relativa a ...
% Created:		2015-02-23
% Author:		Tesser Paolo
% Email:		tesser.paolo@mashup-unipd.it
% =================================================================================================
% Modification History:
% Version		Modifier Date		Change											Author
% 0.0.1 		2015-02-23 			sistemato header								Tesser Paolo
% =================================================================================================
% 0.0.2			2015-023-05			stesura introduzione							Tesser Paolo
% =================================================================================================
%

% CONTENUTO DEL CAPITOLO

\section{Introduzione} % (fold)
\label{sec:introduzione}

	\subsection{Scopo del documento} % (fold)
	\label{sub:scopo_del_documento}
	Questo documento ha come scopo quello di definire la progettazione ad alto livello per il prodotto \projectName. Verrà presentata l’architettura generale secondo la quale saranno organizzate le varie componenti software e i Design Pattern utilizzati nella creazione del prodotto. Verrà inoltre dettagliato il tracciamento tra le componenti software individuate ed i requisiti.
	% subsection scopo_del_documento (end)

	\subsection{Scopo del prodotto} % (fold)
	\label{sub:scopo_del_prodotto}
	\productScope
	% subsection scopo_del_prodotto (end)

	\subsection{Glossario} % (fold)
	\label{sub:glossario}
	\glossarioDesc
	% subsection glossario (end)

	\subsection{Riferimenti} % (fold)
	\label{sub:riferimenti}
		\subsubsection{Normativi} % (fold)
		\label{ssub:normativi}
			\begin{itemize}
				\item TO DO;
			\end{itemize}
		% subsubsection normativi (end)

		\subsubsection{Informativi} % (fold)
		\label{ssub:informativi}
			\begin{itemize}
				\item \url{https://github.com/mgechev/angularjs-in-patterns#services};
				\item \url{https://developer.mozilla.org/it/docs/Web/JavaScript/Introduzione_al_carattere_Object-Oriented_di_JavaScript};
			\end{itemize}
		% subsubsection informativi (end)
	% subsection riferimenti (end)
% section introduzione (end)