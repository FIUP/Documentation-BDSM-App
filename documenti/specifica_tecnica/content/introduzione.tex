% =================================================================================================
% File:			introduzione.tex
% Description:	Defiinisce la sezione relativa a ...
% Created:		2015-02-23
% Author:		Tesser Paolo
% Email:		tesser.paolo@mashup-unipd.it
% =================================================================================================
% Modification History:
% Version		Modifier Date		Change											Author
% 0.0.1 		2015-02-23 			sistemato header								Tesser Paolo
% =================================================================================================
% 0.0.2			2015-023-05			stesura introduzione							Tesser Paolo
% =================================================================================================
% 0.0.3			2015-04-02			aggiunti riferimenti informativi				Tesser Paolo
% =================================================================================================
%

% CONTENUTO DEL CAPITOLO

\section{Introduzione} % (fold)
\label{sec:introduzione}

	\subsection{Scopo del documento} % (fold)
	\label{sub:scopo_del_documento}
	Questo documento ha come scopo quello di definire la progettazione ad alto livello per il prodotto \projectName. Verrà presentata l’architettura generale secondo la quale saranno organizzate le varie componenti software e i Design Pattern utilizzati nella creazione del prodotto. Verrà inoltre dettagliato il tracciamento tra le componenti software individuate ed i requisiti.
	% subsection scopo_del_documento (end)

	\subsection{Scopo del prodotto} % (fold)
	\label{sub:scopo_del_prodotto}
	\productScope
	% subsection scopo_del_prodotto (end)

	\subsection{Glossario} % (fold)
	\label{sub:glossario}
	\glossarioDesc
	% subsection glossario (end)

	\subsection{Riferimenti} % (fold)
	\label{sub:riferimenti}
		\subsubsection{Normativi} % (fold)
		\label{ssub:normativi}
			\begin{itemize}
				\item \textbf{Norme di Progetto}: \docNameVersionNdP
			\end{itemize}
		% subsubsection normativi (end)

		\subsubsection{Informativi} % (fold)
		\label{ssub:informativi}
			\begin{itemize}
				\item \textbf{Progettazione Software}: \url{http://www.math.unipd.it/~tullio/IS-1/2014/Dispense/P09.pdf}
				\item \textbf{Introduzione ai Design Pattern}: vengono utilizzate tutte le slide fornite dal professore Cardin presenti ai seguenti indirizzi:
					\begin{itemize}
						\item \url{http://www.math.unipd.it/~tullio/IS-1/2014/}
						\item \textbf{Design Pattern MVC, MVP e MVVM}: \url{http://www.math.unipd.it/~rcardin/sweb.html}
					\end{itemize}
				\item \textbf{AngularJS Pattern}: \url{https://github.com/mgechev/angularjs-in-patterns}
				\item \textbf{Javascript OOP}: \url{https://developer.mozilla.org/it/docs/Web/JavaScript/Introduzione_al_carattere_Object-Oriented_di_JavaScript}
				\item \textbf{Template View Pattern}: \url{http://martinfowler.com/eaaCatalog/templateView.html}
			\end{itemize}
		% subsubsection informativi (end)
	% subsection riferimenti (end)
% section introduzione (end)