% =================================================================================================
% File:			server_tier/db.tex
% Description:	Definisce la sezione relativa al back-end dell'applicazione
% Created:		2015-04-07
% Author:		Cusinato Giacomo
% Email:		cusinato.giacomo@mashup-unipd.it
% =================================================================================================
% Modification History:
% Version		Modifier Date		Change											Author
% 0.0.1
% =================================================================================================

% CONTENUTO DEL CAPITOLO

\subsubsection{bdsm\_app::server::db} % (fold)
\label{ssub:bdsm_app_server_db}
[TO DO] (diagramma) \newline \newline

\begin{itemize}
  \item \textbf{Descrizione}: [TO DO];
  \item \textbf{Package contenuti}:
    \begin{itemize}
      \item bdsm\_app::server::raw\_model;
      \item bdsm\_app::server::app\_model.
    \end{itemize}
  \item \textbf{Interazione con altri componenti}: [TO DO];
\end{itemize}
% subsubsection bdsm_app_server_db (end)


  % PACKAGE RAW_MODEL

  \subsubsection{bdsm\_app::server::db::raw\_model} % (fold)
  \label{ssub:bdsm_app_server_raw_model}
  [TO DO] (diagramma) \newline \newline

  \begin{itemize}
  \item \textbf{Descrizione}: Package che definisce il modello dei dati grezzi ricavati dai vari social network;
  \item \textbf{Padre}: server::db;
  \item \textbf{Interazione con altri componenti}: interagisce con il miner definito nella sezione [TO DO] il quale salva i dati secondo questo modello, interagisce con l'endpoints definito nella sezione [TO DO] il quale interroga la base di dati;
  \end{itemize}

    \paragraph{Classi} % (fold)
    \subparagraph{bdsm\_app::server::db::raw\_model::AbsRawData} % (fold)
    \label{subp:bdsm_app_server_raw_model_AbsRawData}
      \begin{itemize}
        \item \textbf{Descrizione}: classe astratta che definisce il modello di un dato grezzo;
        \item \textbf{Utilizzo}: [TO DO];
        \item \textbf{Classi figlie}:
        \begin{itemize}
          \item AbsFbRawData;
          \item RawFbPageTrend;
          \item RawFbEventTrend;
          \item AbsTwRawData;
          \item RawTwUserTrend;
          \item RawTwHashtagTrend;
          \item AbsIgRawData;
          \item RawIgUserTrend;
          \item RawIgHashtagTrend;
        \end{itemize}
      \end{itemize}
    \subparagraph{bdsm\_app::server::db::raw\_model::AbsRawData::AbsFbRawData} % (fold)
    \label{subp:bdsm_app_server_raw_model_AbsRawData_AbsFbRawData}
      \begin{itemize}
        \item \textbf{Descrizione}: classe astratta che definisce il modello dati di Facebook;
        \item \textbf{Utilizzo}: [TO DO];
        \item \textbf{Classi figlie}:
        \begin{itemize}
          \item RawFbPage;
          \item RawFbEvent;
        \end{itemize}
        \item \textbf{Classi ereditate}:
        \begin{itemize}
          \item AbsRawData;
        \end{itemize}
      \end{itemize}
    \subparagraph{bdsm\_app::server::db::raw\_model::AbsRawData::RawFbPageTrend} % (fold)
    \label{subp:bdsm_app_server_raw_model_AbsRawData_RawFbPageTrend}
      \begin{itemize}
        \item \textbf{Descrizione}: classe che memorizza i trend di una pagina Facebook;
        \item \textbf{Utilizzo}: classe utilizzata per memorizzare i like e i talking about di ogni singola pagina;
        \item \textbf{Classi ereditate}:
        \begin{itemize}
          \item AbsRawData;
        \end{itemize}
      \end{itemize}
      \subparagraph{bdsm\_app::server::db::raw\_model::AbsRawData::RawFbEventTrend} % (fold)
    \label{subp:bdsm_app_server_raw_model_AbsRawData_RawFbEventTrend}
      \begin{itemize}
        \item \textbf{Descrizione}: classe che memorizza i trend di un evento Facebook;
        \item \textbf{Utilizzo}: classe utilizzata per memorizzare l'andamento dell'evento in relazione ai partecipanti, agli invitati, alle persone in forse e ai rifiuti;
        \item \textbf{Classi ereditate}:
        \begin{itemize}
          \item AbsRawData;
        \end{itemize}
      \end{itemize}
        \subparagraph{bdsm\_app::server::db::raw\_model::AbsRawData::AbsFbRawData::RawFbPage} % (fold)
    \label{subp:bdsm_app_server_raw_model_AbsRawData_AbsFbRawData_RawFbPage}
      \begin{itemize}
        \item \textbf{Descrizione}: classe che memorizza una pagina Facebook;
        \item \textbf{Utilizzo}: classe utilizzata per memorizzare i dati statici di una pagina Facebook;
        \item \textbf{Relazione con altre classi}:
        \begin{itemize}
          \item RawFbPageTrend;
          \item RawFbEvent;
          \item RawFbPost;
        \end{itemize}
        \item \textbf{Classi ereditate}:
        \begin{itemize}
          \item AbsFbRawData
        \end{itemize}
      \end{itemize}
      \subparagraph{bdsm\_app::server::db::raw\_model::AbsRawData::AbsFbRawData::RawFbEvent} % (fold)
    \label{subp:bdsm_app_server_raw_model_AbsRawData_AbsFbRawData_RawFbEvent}
      \begin{itemize}
        \item \textbf{Descrizione}: classe che memorizza un evento Facebook;
        \item \textbf{Utilizzo}: classe utilizzata per memorizzare i dati statici di un evento Facebook;
        \item \textbf{Relazione con altre classi}:
        \begin{itemize}
          \item RawFbEventTrend;
          \item RawFbPost;
        \end{itemize}
        \item \textbf{Classi ereditate}:
        \begin{itemize}
          \item AbsFbRawData
        \end{itemize}
      \end{itemize}
      \subparagraph{bdsm\_app::server::db::raw\_model::RawFbPost} % (fold)
    \label{subp:bdsm_app_server_raw_model_RawFbPost}
      \begin{itemize}
        \item \textbf{Descrizione}: classe che memorizza tutti i post di una pagina o di un evento Facebook;
        \item \textbf{Utilizzo}: classe utilizzata per memorizzare l'insieme di dati ricavato dai post analizzati;
      \end{itemize}



  %subsubsection di Istagram
      \subparagraph{bdsm\_app::server::db::raw\_model::AbsRawData::AbsIgRawData} % (fold)
    \label{subp:bdsm_app_server_raw_model_AbsRawData_AbsIgRawData}
      \begin{itemize}
        \item \textbf{Descrizione}: classe astratta che definisce il modello dati di Istagram;
        \item \textbf{Utilizzo}: [TO DO];
        \item \textbf{Classi figlie}:
        \begin{itemize}
          \item RawIgUser;
          \item RawIgHashtag;
        \end{itemize}
        \item \textbf{Classi ereditate}:
        \begin{itemize}
          \item AbsRawData;
        \end{itemize}
      \end{itemize}
        \subparagraph{bdsm\_app::server::db::raw\_model::AbsRawData::RawIgUserTrend} % (fold)
    \label{subp:bdsm_app_server_raw_model_AbsRawData_RawIgUserTrend}
      \begin{itemize}
        \item \textbf{Descrizione}: classe che definisce i trend di un utente Instagram;
        \item \textbf{Utilizzo}: classe utilizzata per memorizzare l'andamento del numero di media, di follows e di followed\_by di un utente;
        \item \textbf{Classi ereditate}:
        \begin{itemize}
          \item AbsRawData;
        \end{itemize}
      \end{itemize}
    \subparagraph{bdsm\_app::server::db::raw\_model::AbsRawData::RawIgHashtagTrend} % (fold)
    \label{subp:bdsm_app_server_raw_model_AbsRawData_RawIgHashtagTrend}
      \begin{itemize}
        \item \textbf{Descrizione}: classe che definisce i trend di un hashtag di Instagram;
        \item \textbf{Utilizzo}: classe utilizzata per memorizzare l'andamento del numero di media di un hashtag;
        \item \textbf{Classi ereditate}:
        \begin{itemize}
          \item AbsRawData;
        \end{itemize}
      \end{itemize}


  % subsubsection bdsm_app_server_raw_model (end)

  \subsubsection{bdsm\_app::server::db::app\_model} % (fold)
  \label{ssub:bdsm_app_server_app_model}
  [TO DO] (diagramma) \newline \newline

  \begin{itemize}
  \item \textbf{Descrizione}: [TO DO];
  \item \textbf{Padre}: [TO DO] (qualora presente);
  \item \textbf{Package contenuti}: [TO DO] (qualora presente);
  \item \textbf{Interazione con altri componenti}: [TO DO];
  \end{itemize}
  % subsubsection bdsm_app_server_app_model (end)
