% =================================================================================================
% File:			server_tier/miner.tex
% Description:	Definisce la sezione relativa al back-end dell'applicazione
% Created:		2015-04-07
% Author:		Cusinato Giacomo
% Email:		cusinato.giacomo@mashup-unipd.it
% =================================================================================================
% Modification History:
% Version		Modifier Date		Change											Author
% 0.0.1
% =================================================================================================

% CONTENUTO DEL CAPITOLO


\subsubsection{bdsm\_app::server::miner} % (fold)
\label{ssub:bdsm_app_server_miner}
[TO DO] (diagramma) \newline \newline

\begin{itemize}
  \item \textbf{Descrizione}: è il package che contiene tutte le classi con i metodi per prelevare i dati dai social network e per salvarli nel database;
  \item \textbf{Padre}: server;
  \item \textbf{Package contenuti}:
  	\begin{itemize}
  		\item fb;
  		\item ig;
  		\item tw;
  	\end{itemize}
  \item \textbf{Interazione con altri componenti}:
  	\begin{itemize}
  		\item server::processor
  		\item server::database
  	\end{itemize}
\end{itemize}

\paragraph{Classi} % (fold)
		\subparagraph{server::miner::MinerScheduler} % (fold)
		\label{subp:server_miner_MinerScheduler}
			\begin{itemize}
				\item \textbf{Descrizione}: classe che crea i fetcher per ogni social network;
				\item \textbf{Utilizzo}: classe che contiene un campo che rappresenta una recipe e un metodo che andrà a creare i fetcher;
				\item \textbf{Relazioni con altre classi}:
					\begin{itemize}
						\item server::miner::AbsFetcher;
					\end{itemize}
			\end{itemize}
		% subparagraph server_miner_MinerScheduler [end]
		
		\subparagraph{server::miner::AbsCounter} % (fold)
		\label{subp:server_miner_AbsCounter}
			\begin{itemize}
				\item \textbf{Descrizione}: classe astratta che rappresenta il padre delle classi counter delle varie metriche;
				\item \textbf{Utilizzo}: classe che descrive lo scheletro dell'algoritmo di counting;
				\item \textbf{Relazioni con altre classi}:
					\begin{itemize}
						\item server::miner::fb::AbsFbCounter;
						\item server::miner::tw::AbsTwCounter;
						\item server::miner::ig::AbsIgCounter;
					\end{itemize}
			\end{itemize}
		% subparagraph server_miner_AbsCounter [end]
		
		\subparagraph{server::miner::AbsFetcher} % (fold)
		\label{subp:server_miner_AbsFetcher}
				\begin{itemize}
				\item \textbf{Descrizione}: classe astratta che rappresenta il padre delle classi fetcher dei vari social network;
				\item \textbf{Utilizzo}: classe utile in caso l'applicazione venga estesa con altri social network oltre a quelli presi in considerazione;
				\item \textbf{Relazioni con altre classi}:
					\begin{itemize}
						\item server::miner::fb::AbsFbFetcher;
						\item server::miner::tw::AbsTwFetcher;
						\item server::miner::ig::AbsIgFetcher;
					\end{itemize}
			\end{itemize}
		% subparagraph server_miner_AbsFetcher [end]
		
\subsubsection{bdsm\_app::server::miner::fb} % (fold)
\label{ssub:bdsm_app_server_miner_fb}
[TO DO] (diagramma) \newline \newline

\begin{itemize}
  \item \textbf{Descrizione}: è il package che contiene tutte le classi con i metodi per prelevare i dati da Facebook e per salvarli nel database;
  \item \textbf{Padre}: server::miner;
  \item \textbf{Interazione con altri componenti}:
  	\begin{itemize}
  		\item server::database;
  	\end{itemize}
\end{itemize}	

	\paragraph{Classi} % (fold)
		\subparagraph{server::miner::fb::AbsFbFetcher} % (fold)
		\label{subp:server_miner_fb_AbsFbFetcher}
			\begin{itemize}
				\item \textbf{Descrizione}: classe astratta che ricava i dati da Facebook;
				\item \textbf{Utilizzo}: classe che contiene un metodo che ricava i dati statici e un metodo astratto che verrà definito nelle classi figlie;
				\item \textbf{Classi ereditate}: server::miner::AbsFetcher;				
				\item \textbf{Relazioni con altre classi}:
					\begin{itemize}
						\item server::miner::fb::FbPageFetcher;
						\item server::miner::fb::FbEventFetcher;
					\end{itemize}
			\end{itemize}
	% subparagraph server_miner_fb_AbsFbFetcher [end]
	
		\subparagraph{server::miner::fb::FbPageFetcher} % (fold)
		\label{subp:server_miner_fb_FbPageFetcher}
			\begin{itemize}
				\item \textbf{Descrizione}: classe che ricava i dati dalle pagine Facebook;
				\item \textbf{Utilizzo}: classe che contiene un campo che descrive i dati statici di una pagina, un campo che descrive i dati dinamici e un metodo che li ricava tramite l'invocazione della classe PostCounter;
				\item \textbf{Classi ereditate}: server::miner::fb::AbsFbFetcher;
				\item \textbf{Relazioni con altre classi}:
					\begin{itemize}
						\item server::miner::fb::PostCounter;
					\end{itemize}
			\end{itemize}
	% subparagraph server_miner_fb_FbPageFetcher [end]
	
		\subparagraph{server::miner::fb::FbEventFetcher} % (fold)
		\label{subp:server_miner_fb_FbEventFetcher}
			\begin{itemize}
				\item \textbf{Descrizione}: classe che ricava i dati dagli eventi Facebook
				\item \textbf{Utilizzo}: classe che contiene un campo che descrive i dati statici dell'evento Facebook, un campo che descrive i dati dinamici e un metodi che li ricava tramite l'invocazione della classe PostCounter;
				\item \textbf{Classi ereditate}: server::miner::fb::AbsFbFetcher;
				\item \textbf{Relazioni con altre classi}:
					\begin{itemize}
						\item server::miner::fb::PostCounter;
						\item server::miner::fb::AttendingCounter;
						\item server::miner::fb::MaybeCounter;
						\item server::miner::fb::InvitedCounter;
						\item server::miner::fb::RefusedCounter;
					\end{itemize}
			\end{itemize}
	% subparagraph server_miner_fb_FbEventFetcher [end]
	
		\subparagraph{server::miner::fb::AbsFbCounter} % (fold)
		\label{subp:server_miner_fb_AbsFbCounter}
			\begin{itemize}
				\item \textbf{Descrizione}: classe astratta che rappresenta il padre per tutte le classe counter delle metriche di Facebook;
				\item \textbf{Utilizzo}: classe che contiene l'id delle metriche su cui effettuare il counting, questa classe effettua l'overloading dei metodi della classe padre per specificarne l'utilizzo;
				\item \textbf{Classi ereditate}: server::miner::AbsCounter;
				\item \textbf{Relazioni con altre classi}:
					\begin{itemize}
						\item server::miner::fb::AttendingCounter;
						\item server::miner::fb::MaybeCounter;
						\item server::miner::fb::InvitedCounter;
						\item server::miner::fb::RefusedCounter;
						\item server::miner::fb::PostCounter;
						\item server::miner::fb::CommentCounter;
						\item server::miner::fb::LikeCounter;
					\end{itemize}
			\end{itemize}
	% subparagraph server_miner_fb_AbsFbCounter [end]
	
		\subparagraph{server::miner::fb::AttendingCounter} % (fold)
		\label{subp:server_miner_fb_AttendingCounter}
			\begin{itemize}
				\item \textbf{Descrizione}: classe che descrive l'algoritmo per il counting degli invitati che non hanno ancora risposto in un evento Facebook;
				\item \textbf{Utilizzo}: classe che ricava il numero di persone invitate che non hanno ancora risposto di un evento Facebook;
				\item \textbf{Classi ereditate}: server::miner::fb::AbsFbCounter;
			\end{itemize}
	% subparagraph server_miner_fb_AttendingCounter
	
		\subparagraph{server::miner::fb::MaybeCounter} % (fold)
		\label{subp:server_miner_fb_MaybeCounter}
			\begin{itemize}
				\item \textbf{Descrizione}: classe che descrive l'algoritmo per il counting dei maybe in un evento Facebook;
				\item \textbf{Utilizzo}: classe che ricava il numero di persone in forse per un evento Facebook;
				\item \textbf{Classi ereditate}: server::miner::fb::AbsFbCounter;
			\end{itemize}
	% subparagraph server_miner_fb_MaybeCounter
	
	\subparagraph{server::miner::fb::InvitedCounter} % (fold)
		\label{subp:server_miner_fb_InvitedCunter}
			\begin{itemize}
				\item \textbf{Descrizione}: classe che descrive l'algoritmo per il counting dei invited in un evento Facebook;
				\item \textbf{Utilizzo}: classe che ricava il numero di persone invitate ad un evento Facebook;
				\item \textbf{Classi ereditate}: server::miner::fb::AbsFbCounter;
			\end{itemize}
	% subparagraph server_miner_fb_InvitedCounter

	\subparagraph{server::miner::fb::RefusedCounter} % (fold)
		\label{subp:server_miner_fb_RefusedCounter}
			\begin{itemize}
				\item \textbf{Descrizione}: classe che descrive l'algoritmo per il counting dei refused in un evento Facebook;
				\item \textbf{Utilizzo}: classe che ricava il numero di persone che hanno rifiutato l'invito per un evento Facebook;
				\item \textbf{Classi ereditate}: server::miner::fb::AbsFbCounter;
			\end{itemize}
	% subparagraph server_miner_fb_RefusedCounter
	
	\subparagraph{server::miner::fb::PostCounter} % (fold)
		\label{subp:server_miner_fb_PostCounter}
			\begin{itemize}
				\item \textbf{Descrizione}: classe che descrive l'algoritmo per calcolare il numero dei post per ogni evento o pagina
				\item \textbf{Utilizzo}: classe che contiene un campo per il numero dei like e un campo per il numero dei talking about per ogni post di Facebook;
				\item \textbf{Classe ereditate}: server::miner::fb::AbsFbCounter;				
				\item \textbf{Relazioni con altre classi}:
					\begin{itemize}
						\item server::miner::fb::CommentCounter;
						\item server::miner::fb::LikeCounter;
					\end{itemize}
			\end{itemize}
	% subparagraph server_miner_fb_PostCounter
	
	\subparagraph{server::miner::fb::CommentCounter} % (fold)
		\label{subp:server_miner_fb_CommentCounter}
			\begin{itemize}
				\item \textbf{Descrizione}:classe che descrive l'algoritmo per calcolare il numero di commenti per ogni post
				\item \textbf{Utilizzo}: classe che ricava il numero di commenti per ogni post;
				\item \textbf{Classe ereditate}: server::miner::fb::AbsFbCounter;
			\end{itemize}
	% subparagraph server_miner_fb_CommentCounter
	
	\subparagraph{server::miner::fb::LikeCounter} % (fold)
		\label{subp:server_miner_fb_LikeCounter}
			\begin{itemize}
					\item \textbf{Descrizione}:classe che descrive l'algoritmo per calcolare il numero di like per ogni post
				\item \textbf{Utilizzo}: classe che ricava il numero di like per ogni post;
				\item \textbf{Classe ereditate}: server::miner::fb::AbsFbCounter;
			\end{itemize}
	% subparagraph server_miner_fb_LikeCounter

\subsubsection{bdsm\_app::server::miner::tw} % (fold)
\label{ssub:bdsm_app_server_miner_tw}
[TO DO] (diagramma) \newline \newline

\begin{itemize}
  \item \textbf{Descrizione}:è il package che contiene tutte le classi con i metodi per prelevare
i dati da Twitter e per salvarli nel database;
  \item \textbf{Padre}: server::miner;
  \item \textbf{Interazione con altri componenti}:
  	\begin{itemize}
  		\item server::database; 	
  	\end{itemize}
\end{itemize}	

	\paragraph{Classi} % (fold)
	\subparagraph{server::miner::tw::AbsTwFetcher} % (fold)
		\label{subp:server_miner_tw_AbsTwFetcher}
			\begin{itemize}
				\item \textbf{Descrizione}: classe astratta che ricava i dati da Twitter;
				\item \textbf{Utilizzo}: classe che contiene un metodo che ricava i dati statici e un metodo
astratto che verrà definito nelle classi figlie;
				\item \textbf{Classi ereditate}: server::miner::AbsFetcher;
				\item \textbf{Relazioni con altre classi}:
					\begin{itemize}
						\item server::miner::tw::TwHashtagFetcher;
						\item server::miner::tw::TwUserFetcher;
					\end{itemize}
			\end{itemize}
		% subparagraph server_miner_tw_AbsTwFetcher
		
	\subparagraph{server::miner::tw::TwHashtagFetcher} % (fold)
		\label{subp:server_miner_tw_TwHashtagFetcher}
			\begin{itemize}
				\item \textbf{Descrizione}: classe che ricava i dati degli hashtag di Twitter;
				\item \textbf{Utilizzo}: classe che contiene un campo che descrive i dati statici dell' hashtag Twitter, e un metodo che ricava i dati dinamici da HashtagTweetCounter;
				\item \textbf{Classi ereditate}: server::miner::tw::AbsTwFetcher;
				\item \textbf{Relazioni con altre classi}: 
					\begin{itemize}
						\item server::miner::tw::HashtagTweetCounter;
					\end{itemize}
			\end{itemize}
		% subparagraph server_miner_tw_TwHashtagFetcher
		
	\subparagraph{server::miner::tw::TwUserFetcher} % (fold)
		\label{subp:server_miner_tw_TwUserFetcher}
			\begin{itemize}
				\item \textbf{Descrizione}: classe che ricava i dati degli utenti di Twitter;
				\item \textbf{Utilizzo}: classe che contiene un campo che descrive i dati statici dell'utente, un campo che descrive i dati dinamici [TO DO]
				\item \textbf{Classi ereditate}: server::miner::tw::AbsTwFetcher;				
				\item \textbf{Relazioni con altre classi}:
					\begin{itemize}
						\item server::miner::tw::UserTweetCounter;
					\end{itemize}
			\end{itemize}
		% subparagraph server_miner_tw_TwUserFetcher
		
	\subparagraph{server::miner::tw::AbsTwCounter} % (fold)
		\label{subp:server_miner_tw_AbsTwCounter}
			\begin{itemize}
				\item \textbf{Descrizione}: classe astratta che rappresenta il padre per tutte le classe counter delle metriche di Twitter;
				\item \textbf{Utilizzo}: classe che contiene l’id delle metriche su cui effettuare il counting, questa classe effettua l’overloading dei metodi della classe padre per specificarne l’utilizzo;
				\item \textbf{Classi ereditate}: server::miner::AbsCounter
				\item \textbf{Relazioni con altre classi}:
					\begin{itemize}
						\item server::miner::UserTweetCounter;
					\end{itemize}
			\end{itemize}
		% subparagraph server_miner_tw_AbsTwCounter
		
	\subparagraph{server::miner::tw::UserTweetCounter} % (fold)
		\label{subp:server_miner_tw_UserTweetCounter}
			\begin{itemize}
				\item \textbf{Descrizione}: classe che descrive l'algoritmo per il counting dei campi di un tweet, relativo ad un utente, di cui ci interessa fare un trend;
				\item \textbf{Utilizzo}: classe che ricava il numero dei retweets, il numero dei favoriti, il tipo di device, e il numero di media per ogni tweet;
				\item \textbf{Classi ereditate}: server::miner::tw::AbsTwCounter;
			\end{itemize}
		% subparagraph server_miner_tw_UserTweetCounter
		
		
	\subparagraph{server::miner::tw::HashtagTweetCounter} % (fold)
		\label{subp:server_miner_tw_HashtagTweetCounter}
			\begin{itemize}
				\item \textbf{Descrizione}: classe che descrive l'algoritmo per il counting dei campi di un tweet, relativo ad un hashtag, di cui ci interessa fare un trend;
				\item \textbf{Utilizzo}: classe che ricava anche la time zone di ogni tweet;
				\item \textbf{Classi ereditate}: server::miner::tw::UserTweetCounter;
			\end{itemize}
		% subparagraph server_miner_tw_HashtagTweetCounter

\subsubsection{bdsm\_app::server::miner::ig} % (fold)
\label{ssub:bdsm_app_server_miner_ig}
[TO DO] (diagramma) \newline \newline

\begin{itemize}
  \item \textbf{Descrizione}: è il package che contiene tutte le classi con i metodi per prelevare i dati da Instagram e per salvarli nel database;
  \item \textbf{Padre}: server::miner
   \item \textbf{Interazione con altri componenti}:
  	\begin{itemize}
  		\item server::database;
  	\end{itemize}
\end{itemize}	

	\paragraph{Classi} % (fold)
	\subparagraph{server::miner::ig::AbsIgFetcher} % (fold)
		\label{subp:server_miner_ig_AbsIgFetcher}
			\begin{itemize}
				\item \textbf{Descrizione}: classe astratta che ricava i dati da Instagram
				\item \textbf{Utilizzo}: classe che contiene un metodo che ricava i dati statici e un metodo astratto che verrà definito nelle classi figlie;
				\item \textbf{Classi ereditate}: server::miner::AbsFetcher;				
				\item \textbf{Relazioni con altre classi}:
					\begin{itemize}
						\item server::miner::ig::IgUserFetcher;
						\item server::miner::ig::IgHashtagFetcher;
					\end{itemize}
			\end{itemize}
		% subparagraph server_miner_ig_AbsIgFetcher		

	\subparagraph{server::miner::ig::IgUserFetcher} % (fold)
		\label{subp:server_miner_ig_IgUserFetcher}
			\begin{itemize}
				\item \textbf{Descrizione}: classe che ricava i dati dagli utenti Instagram;
				\item \textbf{Utilizzo}: classe che contiene un campo che descrive i dati statici di un utente, un campo che descrive i dati dinamici [TO DO]
				\item \textbf{Classi ereditate}: server::miner::ig::AbsIgFetcher;				
				\item \textbf{Relazioni con altre classi}:
					\begin{itemize}
						\item server::miner::ig::MediaCounter;
					\end{itemize}
			\end{itemize}
		% subparagraph server_miner_ig_IgUserFetcher		
		
	\subparagraph{server::miner::ig::IgHashtagFetcher} % (fold)
		\label{subp:server_miner_ig_IgHashtagFetcher}
			\begin{itemize}
				\item \textbf{Descrizione}: classe che ricava i dati dagli hashtag di Instagram;
				\item \textbf{Utilizzo}: classe che contiene un campo che descrive i dati statici di un hashtag, un campo che descrive i dati dinamici [TO DO]
				\item \textbf{Classi ereditate}: server::miner::ig::AbsIgFetcher;				
				\item \textbf{Relazioni con altre classi}:
					\begin{itemize}
						\item server::miner::ig::MediaCounter;
					\end{itemize}
			\end{itemize}
		% subparagraph server_miner_ig_IgHashtagFetcher		
		
	\subparagraph{server::miner::ig::AbsIgCounter} % (fold)
		\label{subp:server_miner_ig_AbsIgCounter}
			\begin{itemize}
				\item \textbf{Descrizione}: classe astratta che rappresenta il padre per tutte le classe counter delle metriche di Instagram;
				\item \textbf{Utilizzo}: classe che contiene l’id delle metriche su cui effettuare il counting, questa classe effettua l’overloading dei metodi della classe padre per specificarne l’utilizzo;
				\item \textbf{Classi ereditate}: server::miner::AbsCounter
				\item \textbf{Relazioni con altre classi}:
					\begin{itemize}
						\item server::miner::ig::MediaCounter;
					\end{itemize}
			\end{itemize}
		% subparagraph server_miner_ig_AbsIgCounter		
		
		
	\subparagraph{server::miner::ig::MediaCounter} % (fold)
		\label{subp:server_miner_ig_MediaCounter}
			\begin{itemize}
				\item \textbf{Descrizione}: classe che descrive l’algoritmo per il counting delle metriche che ci interessano per fare un trend.
				\item \textbf{Utilizzo}: classe che ricava il numero di commenti, il numero di like, il numero di foto e il numero di video, di un media;
				\item \textbf{Classi ereditate}: server::miner::ig::AbsIgCounter;
			\end{itemize}
		% subparagraph server_miner_ig_MediaCounter		

		
% subsubsection
