% =================================================================================================
% File:			db_tier.tex
% Description:	Defiinisce la sezione relativa a ...
% Created:		2015-02-23
% Author:		Tesser Paolo
% Email:		tesser.paolo@mashup-unipd.it
% =================================================================================================
% Modification History:
% Version		Modifier Date		Change											Author
% 0.0.1 		2015-02-23 			sistemato header								Tesser Paolo
% =================================================================================================
% 0.0.2			2015-03-05			stesura sezione sul db schema-less				Tesser Paolo
% =================================================================================================
%

% CONTENUTO DEL CAPITOLO

\subsection{Database} % (fold)
\label{sec:database}
Il database utilizzato, sia per quanto riguarda i dati grezzi sia per quanto riguarda i dati aggregati e le varie configurazioni, sarà di tipo schema-less. Questo significa che non c'è un diagramma di come i dati siano in relazione tra loro. \newline
Il modo quindi nel quale saranno salvati nel database, descritto nella sezione \ref{sub:database}, e in che formato viene descritto dal Model dell'applicazione come illustrato alla sezione [TO DO]. \newline

  % TEMPLATE PER IL PACKAGE
  \subsubsection{Nome package} % (fold)
  \label{ssub:nome_del_package}
  [TO DO] (diagramma) \newline \newline

  \begin{itemize}
    \item \textbf{Descrizione}: [TO DO];
    \item \textbf{Padre}: [TO DO] (qualora presente);
    \item \textbf{Package contenuti}: [TO DO] (qualora presente);
    \item \textbf{Interazione con altri componenti}: [TO DO];
  \end{itemize}

    \paragraph{Classi} % (fold)
      \subparagraph{Nome package::Nome classe} % (fold)
      \label{subp:subparagraph_name}
        \begin{itemize}
          \item \textbf{Descrizione}: [TO DO];
          \item \textbf{Utilizzo}: [TO DO];
          \item \textbf{Classi ereditate}: [TO DO];
          \item \textbf{Relazioni con altre classi}: [TO DO].
        \end{itemize}
      % subparagraph subparagraph_name (end)

      % subsection nome_classe (end)

    % paragraph classi (end)
  % subsubsection nome_del_package (end)

  % subsection database (end)
