% =================================================================================================
% File:			design_pattern.tex
% Description:	Defiinisce la sezione relativa a ...
% Created:		2015-02-23
% Author:		Tesser Paolo
% Email:		tesser.paolo@mashup-unipd.it
% =================================================================================================
% Modification History:
% Version		Modifier Date		Change											Author
% 0.0.1 		2015-02-23 			sistemato header								Tesser Paolo
% =================================================================================================
% 0.0.2			2015-03-31			aggiunta introduzione ai DP						Tesser Paolo
% =================================================================================================
%


% CONTENUTO DEL CAPITOLO

\section{Design Pattern} % (fold)
\label{sec:design_pattern}
In questa sezione verranno presentati i diversi design pattern utilizzati per la progettazione architetturale. I design pattern sono soluzioni a problemi ricorrenti. Adottarli porta diversi benefici:
	\begin{itemize}
		\item favorisce il riutilizzo del codice;
		\item semplifica l’attività di progettazione;
		\item rende l’architettura più manutenibile.
	\end{itemize}
	\noindent
I design pattern possono essere suddivisi in:
	\begin{itemize}
		\item \textbf{Architetturali}: definiscono l’architettura dell’applicazione ad un livello elevato;
		\item \textbf{Creazionali}: permettono di nascondere i costruttori delle classi, consentendo la creazione di oggetti senza conoscerne la loro implementazione;
		\item \textbf{Strutturali}: consentono di riutilizzare classi preesistenti, fornendo un’interfaccia più adatta;
		\item \textbf{Comportamentali}: definiscono soluzioni per le interazioni tra oggetti.
	\end{itemize}
	\noindent
	\newline
Per una descrizione più approfondita dei design pattern utilizzati si faccia riferimento all’appendice \ref{sec:descdp}. I vari diagrammi che riprendono l’architettura non espongono tutte le sottoclassi e i metodi delle stesse. Lo scopo dei diagrammi è di mostrare le caratteristiche del design pattern adottato e come le varie classi interagiscono tra di loro. Nella realizzazione del progetto \projectName{} si è deciso di implementare i seguenti design pattern.

	\pagebreak

	% =================================================================================================
% File:			dp_architetturali.tex
% Description:	Defiinisce la sezione relativa a ...
% Created:		2015-03-26
% Author:		Tesser Paolo
% Email:		tesser.paolo@mashup-unipd.it
% =================================================================================================
% Modification History:
% Version		Modifier Date		Change											Author
% 0.0.1 		2015-03-26 			aggiunto sezioni								Tesser Paolo
% =================================================================================================
%
% =================================================================================================
%

% CONTENUTO DEL CAPITOLO

\subsection{Design pattern architetturali} % (fold)
\label{sub:design_pattern_architetturali}
[TO DO]
	\subsubsection{Three-Tier} % (fold)
	\label{ssub:three_tier}
	[TO DO]
	% subsubsection three_tier (end)

	\subsubsection{MVC} % (fold)
	\label{ssub:mvc}
	[TO DO]
	% subsubsection mvc (end)
% subsection design_pattern_architetturali (end) \clearpage \newpage
	% =================================================================================================
% File:			dp_creazionali.tex
% Description:	Defiinisce la sezione relativa a ...
% Created:		2015-03-26
% Author:		Tesser Paolo
% Email:		tesser.paolo@mashup-unipd.it
% =================================================================================================
% Modification History:
% Version		Modifier Date		Change											Author
% 0.0.1 		2015-03-26 			creato scheltro sezione							Tesser Paolo
% =================================================================================================
% 0.0.2			2015-04-08			inserito scheletro per DP: Prototype e Module	Tesser Paolo
% =================================================================================================
% 0.0.3			2015-04-14			descritto Prototype, Module e Constructor		Tesser Paolo
% =================================================================================================
%

% CONTENUTO DEL CAPITOLO

\subsection{Design pattern creazionali} % (fold)
\label{sub:design_pattern_creazionali}

	\subsubsection{Constructor Pattern} % (fold)
	\label{ssub:constructor_pattern}
		\begin{itemize}
			\item \textbf{Scope dell'utilizzo}: questo pattern è utilizzato per emulare il costruttore tipico della programmazione ad oggetti attraverso delle funzioni che lavorano con gli oggetti;
			\item \textbf{Contesto dell'utilizzo}:
				\begin{itemize}
					\item \textbf{Client}: viene utilizzato in tutte le classi del package \texttt{client::model}. \newline
					Non è possibile fornirne una rappresentazione grafica in quanto questo pattern viene realizzato durante la codifica effettiva delle componenti.
				\end{itemize}
		\end{itemize}
	% subsubsection constructor_pattern (end)

	\subsubsection{Prototype Pattern} % (fold)
	\label{ssub:prototype_pattern}
		\begin{itemize}
			\item \textbf{Scope dell'utilizzo}: questo pattern è utilizzato per generare il meccanismo di ereditarietà tra due classi. Viene anche scelto in modo che i metodi di una classe siano condivisi tra i diversi oggetti in quanto altrimenti, in JavaScript, siccome non è presente il concetto di classi si andrebbe a ripetere un metodo ogni volta che si istanzia un nuovo oggetto e questo non è ottimale;
			\item \textbf{Contesto dell'utilizzo}:
				\begin{itemize}
					\item \textbf{Client}: viene utilizzato nelle classi del package \texttt{client::model} ad esempio tra \texttt{UserModel} e \texttt{UserAdminModel}. \newline
					Non è possibile fornirne una rappresentazione grafica in quanto questo pattern viene realizzato durante la codifica effettiva delle componenti.
				\end{itemize}
		\end{itemize}
	% subsubsection prototype_pattern (end)

	\subsubsection{Module Pattern} % (fold)
	\label{ssub:module_pattern}
		\begin{itemize}
			\item \textbf{Scope dell'utilizzo}: questo pattern serve per garantire, in particolare in JavaScript, l'incapsulamento e la privacy. \'E quindi utilizzato principalmente quando si vuole emulare il concetto di classe, definendo dei membri e dei metodi sia privati che pubblici;
			\item \textbf{Contesto dell'utilizzo}:
				\begin{itemize}
					\item \textbf{Client}: viene utilizzato in tutte le classi del package \texttt{client::model::data} per incapsulare al meglio i membri e i metodi che i modelli dei dati usano. \newline
					Non è possibile fornirne una rappresentazione grafica in quanto questo pattern viene realizzato durante la codifica effettiva delle componenti.
				\end{itemize}
		\end{itemize}
	% subsubsection module_pattern (end)

	\subsubsection{Singleton} % (fold)
	\label{ssub:singleton}
		\begin{itemize}
			\item \textbf{Scope dell'utilizzo}: questo pattern è utilizzato per limitare l'instaziazione di un certo tipo classe ad un solo oggetto in modo tale che esso rimanga unico nel sistema in cui risiede;
			\item \textbf{Contesto dell'utilizzo}:
				\begin{itemize}
					\item \textbf{Client}: viene utilizzato direttamente da AngularJS quando si utilizzano i ``service'' o i ``factory'', utilizzati per creare la logica di business del front-end nel package \texttt{client::model}. Il framework gestisce internamente questo pattern attraverso una hash map che risiede nella cache. Essa rappresenta un singleton manager che detiene le dipendenze che vengono istanziate e le restituisce quando richieste senza crearne una nuova se questa esiste già. \newline
					Non ne viene fornita nessuna rappresentazione grafica in quanto non è una cosa che viene progettata dal team, ma usata direttamente attraverso il framework scelto;
					\item \textbf{Server}: viene utilizzato dalla classe \texttt{server::endpoints::RequestHandler} in quanto quest'ultima, essendo implementata secondo il pattern Front Controller, rappresenta il punto di accesso comune a tutto il sistema in cui confluiscono le richieste provenienti dalle classi appartenenti al package \texttt{server::endpoints} (\ref{ssub:bdsm_app_server_endpoints}). \newline
					[TO DO] (grafico del pattern applicato al caso di utilizzo nell'applicativo)
				\end{itemize}
		\end{itemize}
	% subsubsection singleton (end)
% subsection design_pattern_creazionali (end)
 \clearpage \newpage
	% =================================================================================================
% File:			dp_strutturali.tex
% Description:	Defiinisce la sezione relativa a ...
% Created:		2015-03-26
% Author:		Tesser Paolo
% Email:		tesser.paolo@mashup-unipd.it
% =================================================================================================
% Modification History:
% Version		Modifier Date		Change											Author
% 0.0.1 		2015-03-26 			sistemato header								Tesser Paolo
% =================================================================================================
% 0.0.2			2015-04-08			aggiunto scheletro per pattern Facade			Tesser Paolo
% =================================================================================================
% 0.0.3			2015-04-13			descritto pattern Facade						Tesser Paolo
% =================================================================================================
%

% CONTENUTO DEL CAPITOLO

\subsection{Design pattern strutturali} % (fold)
\label{sub:design_pattern_strutturali}
	\subsubsection{Fa\c{c}ade} % (fold)
	\label{ssub:facade}
		\begin{itemize}
			\item \textbf{Scope dell'utilizzo}: questo pattern è utilizzato per fornire un'interfaccia di alto livello unificata di tante interfacce di un sotto sistema più complesso. Questo rende più semplice al client interagire con quel sistema senza preoccuparsi di come le cose vengono implementate da esso;
			\item \textbf{Contesto dell'utilizzo}:
				\begin{itemize}
					\item \textbf{Client}: viene utilizzato direttamente da AngularJS in alcuni servizi come quello \$http o \$resource che permettono all'utilizzatore di non sapere come viene effettivamente implementata la chiamata alle API, effettuandola quindi in maniera più semplice di come è realmente. \newline
					Non ne viene fornita nessuna rappresentazione grafica in quanto non è una cosa che viene progettata dal team, ma usata direttamente attraverso il framework scelto.
				\end{itemize}
		\end{itemize}
	% subsubsection facade (end)


	\subsubsection{Front controller} % (fold)
	\label{ssub:front_controller}
		\begin{itemize}
			\item \textbf{Scope dell'utilizzo}: [TO DO];
			\item \textbf{Contesto dell'utilizzo}:
				\begin{itemize}
					\item \textbf{Server}: [TO DO]. \newline
					[TO DO] (grafico del pattern applicato al caso di utilizzo nell'applicativo)
				\end{itemize}
		\end{itemize}
		% subsubsection front_controller (end)



% subsection design_pattern_strutturali (end) \clearpage \newpage
	% =================================================================================================
% File:			dp_comportamentali.tex
% Description:	Defiinisce la sezione relativa a ...
% Created:		2015-03-26
% Author:		Tesser Paolo
% Email:		tesser.paolo@mashup-unipd.it
% =================================================================================================
% Modification History:
% Version		Modifier Date		Change											Author
% 0.0.1 		2015-03-26 			sistemato header								Tesser Paolo
% =================================================================================================
%
% =================================================================================================
%

% CONTENUTO DEL CAPITOLO

\subsection{Design pattern comportamentali} % (fold)
\label{sub:design_pattern_comportamentali}
[TO DO]
% subsection design_pattern_comportamentali (end) \clearpage \newpage
% section design_pattern (end)
