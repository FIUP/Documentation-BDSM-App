% =================================================================================================
% File:			design_pattern.tex
% Description:	Defiinisce la sezione relativa a ...
% Created:		2015-02-23
% Author:		Tesser Paolo
% Email:		tesser.paolo@mashup-unipd.it
% =================================================================================================
% Modification History:
% Version		Modifier Date		Change											Author
% 0.0.1 		2015-02-23 			sistemato header								Tesser Paolo
% =================================================================================================
% 0.0.2			2015-03-31			aggiunta introduzione ai DP						Tesser Paolo
% =================================================================================================
%


% CONTENUTO DEL CAPITOLO

\section{Design Pattern} % (fold)
\label{sec:design_pattern}
In questa sezione verranno presentati i diversi design pattern utilizzati per la progettazione architetturale. I design pattern sono soluzioni a problemi ricorrenti. Adottarli porta diversi benefici:
	\begin{itemize}
		\item favorisce il riutilizzo del codice;
		\item semplifica l’attività di progettazione;
		\item rende l’architettura più manutenibile.
	\end{itemize}
	\noindent
I design pattern possono essere suddivisi in:
	\begin{itemize}
		\item \textbf{Architetturali}: definiscono l’architettura dell’applicazione ad un livello elevato;
		\item \textbf{Creazionali}: permettono di nascondere i costruttori delle classi, consentendo la creazione di oggetti senza conoscerne la loro implementazione;
		\item \textbf{Strutturali}: consentono di riutilizzare classi preesistenti, fornendo un’interfaccia più adatta;
		\item \textbf{Comportamentali}: definiscono soluzioni per le interazioni tra oggetti.
	\end{itemize}
	\noindent
	\newline
Per una descrizione più approfondita dei design pattern utilizzati si faccia riferimento all’appendice \ref{sec:descdp}. I vari diagrammi che riprendono l’architettura non espongono tutte le sottoclassi e i metodi delle stesse. Lo scopo dei diagrammi è di mostrare le caratteristiche del design pattern adottato e come le varie classi interagiscono tra di loro. Nella realizzazione del progetto \projectName{} si è deciso di implementare i seguenti design pattern.

	\pagebreak

	% =================================================================================================
% File:			dp_architetturali.tex
% Description:	Defiinisce la sezione relativa a ...
% Created:		2015-03-26
% Author:		Tesser Paolo
% Email:		tesser.paolo@mashup-unipd.it
% =================================================================================================
% Modification History:
% Version		Modifier Date		Change											Author
% 0.0.1 		2015-03-26 			aggiunto sezioni								Tesser Paolo
% =================================================================================================
% 0.0.2			2015-04-03			descritto pattern MVC, MVW e DI					Tesser Paolo
% =================================================================================================
% 0.0.3			2015-04-13			descritto Three-Tier							Tesser Paolo
% =================================================================================================
%

% CONTENUTO DEL CAPITOLO

\subsection{Design pattern architetturali} % (fold)
\label{sub:design_pattern_architetturali}
	\subsubsection{Three-Tier} % (fold)
	\label{ssub:three_tier}
		\begin{itemize}
			\item \textbf{Scope dell'utilizzo}: è stato scelto il pattern Three-tier per rendere massima la distribuzione delle componenti principali del sistema: client(front-end), server(back-end) e database (Datastore). La decisione di adottare un'architettura REST-like ha contribuito alla scelta dell'infrastruttura, che ci permette di separare al meglio le diverse parti;
			\item \textbf{Contesto dell'utilizzo}:
				\begin{itemize}
					\item \textbf{Intera applicazione}: la decomposizione del sistema avviene secondo lo schema citato precedentemente. Ogni componente rappresenta un livello del pattern. Il client comunica con il server attraverso i servizi REST esposti da quest'ultimo, mentre il server comunica con il database attraverso le funzionalità offerte dalla Google App Engine. \newline
					[TO DO] (grafico del pattern applicato al caso di utilizzo nell'applicativo)
				\end{itemize}
		\end{itemize}
	% subsubsection three_tier (end)


	\subsubsection{MVW} % (fold)
	\label{ssub:mvw}
		\begin{itemize}
			\item \textbf{Scope dell'utilizzo}: è un Design Pattern simile a MVC, che permette di avere una corrispondenza più diretta e automatica tra la \emph{view} e il \emph{model}. L'acronimo MVW sta infatti per Model-View-Whatever, dove \emph{Whatever} indica \emph{``whatever works for you''};
			\item \textbf{Contesto dell'utilizzo}: 
				\begin{itemize}
					\item \textbf{Client}: viene utilizzato per gestire il lato client dell'applicazione. Questo ci viene fornito direttamente dal framework AngularJS. La parte W (Whatever) assume internamente due diversi aspetti. \newline
					Il diagramma che riporta lo schema del pattern è quello presente alla sezione \ref{sub:client}.
				\end{itemize}
		\end{itemize}

		\paragraph{MVC} % (fold)
		\label{par:mvc}
			\begin{itemize}
				\item \textbf{Scope dell'utilizzo}: questo pattern è utilizzato per separare le responsabilità dell’applicazione a diversi componenti e permettere di fare una chiara divisione tra presentazione, struttura dei dati e operazioni su di essi;

				\item \textbf{Contesto dell'utilizzo}: 
					\begin{itemize}
						\item \textbf{Client}: viene utilizzato nel livello client per separare i componenti a seconda delle loro responsabilità e secondo una connotazione semantica. Il formalismo che viene quindi assunto per dividere i package è proprio quello fornito dal pattern in questione e cioè: \textbf{model, view e controller}. Questi però non interagiscono tra di loro nella maniera canonica che il pattern offre, ma seguono lo stile proposto da MVVM descritto alla sezione \ref{par:mvvm}.
					\end{itemize}
			\end{itemize}
		% paragraph mvc (end)

		\paragraph{MVVM} % (fold)
		\label{par:mvvm}
			\begin{itemize}
				\item \textbf{Scope dell'utilizzo}: questo pattern è utilizzato per gestire il modo con il quale le diverse parti comunicano tra loro per scambiare i dati e per gestire le operazioni che l'utente richiede attraverso l'interazione con la View;

				\item \textbf{Contesto dell'utilizzo}: 
					\begin{itemize}
						\item \textbf{Client}: viene utilizzato nel livello client per gestire lo scambio dei dati e le interazioni che fanno cambiare le View e lo stato del Model, attraverso principalmente un sistema di two-way data binding e di eventi. Il Model però non interagisce direttamente con le View, ma passa attraverso un controller che fa da collante tra la View e il Model (ViewModel) attraverso l'oggetto \$scope.
					\end{itemize}
			\end{itemize}
		% paragraph mvvm (end)
	% subsubsection mvw (end)

	\subsubsection{Dependency Injection} % (fold)
	\label{ssub:dependency_injection}
		\begin{itemize}
			\item \textbf{Scope dell'utilizzo}: è un Design Pattern che viene utilizzato per favorire la separazione delle responsabilità tra i componenti dalla risoluzione delle dipendenze. Questo permette di avere una migliore modularità del codice, di avere un minor accoppiamento tra le diverse parti e garantisce una più facile fase di testing;
			\item \textbf{Contesto dell'utilizzo}: 
				\begin{itemize}
					\item \textbf{Client}: viene utilizzato direttamente da AngularJS per iniettare le diverse dipendenze nei moduli che le richiedono. In particolare vengono iniettati nei controller tutti i servizi e i modelli dei dati necessari all'interazione dell'utente con le viste. \newline
					Non ne viene fornita nessuna rappresentazione grafica in quanto non è una cosa che viene progettata dal team, ma usata direttamente attraverso il framework scelto;
				\end{itemize}
		\end{itemize}
	% subsubsection dependency_injection (end)

% subsection design_pattern_architetturali (end) \clearpage \newpage
	% =================================================================================================
% File:			dp_creazionali.tex
% Description:	Defiinisce la sezione relativa a ...
% Created:		2015-03-26
% Author:		Tesser Paolo
% Email:		tesser.paolo@mashup-unipd.it
% =================================================================================================
% Modification History:
% Version		Modifier Date		Change											Author
% 0.0.1 		2015-03-26 			creato scheltro sezione							Tesser Paolo
% =================================================================================================
% 0.0.2			2015-04-08			inserito scheletro per DP: Prototype e Module	Tesser Paolo
% =================================================================================================
%

% CONTENUTO DEL CAPITOLO

\subsection{Design pattern creazionali} % (fold)
\label{sub:design_pattern_creazionali}
	\subsubsection{Prototype Pattern} % (fold)
	\label{ssub:prototype_pattern}
	[TO DO] (grafico del pattern applicato al caso di utilizzo nell'applicativo)
		\begin{itemize}
			\item \textbf{Scope dell'utilizzo}: [TO DO];
			\item \textbf{Contesto dell'utilizzo}: [TO DO];
		\end{itemize}
	% subsubsection prototype_pattern (end)

	\subsubsection{Module Pattern} % (fold)
	\label{ssub:module_pattern}
	[TO DO] (grafico del pattern applicato al caso di utilizzo nell'applicativo)
		\begin{itemize}
			\item \textbf{Scope dell'utilizzo}: [TO DO];
			\item \textbf{Contesto dell'utilizzo}: [TO DO];
		\end{itemize}
	% subsubsection module_pattern (end)
% subsection design_pattern_creazionali (end) \clearpage \newpage
	% =================================================================================================
% File:			dp_strutturali.tex
% Description:	Defiinisce la sezione relativa a ...
% Created:		2015-03-26
% Author:		Tesser Paolo
% Email:		tesser.paolo@mashup-unipd.it
% =================================================================================================
% Modification History:
% Version		Modifier Date		Change											Author
% 0.0.1 		2015-03-26 			sistemato header								Tesser Paolo
% =================================================================================================
% 0.0.2			2015-04-08			aggiunto scheletro per pattern Facade			Tesser Paolo
% =================================================================================================
%

% CONTENUTO DEL CAPITOLO

\subsection{Design pattern strutturali} % (fold)
\label{sub:design_pattern_strutturali}
	\subsubsection{Fa\c{c}ade} % (fold)
	\label{ssub:facade}
	[TO DO] (grafico del pattern applicato al caso di utilizzo nell'applicativo)
		\begin{itemize}
			\item \textbf{Scope dell'utilizzo}: [TO DO];
			\item \textbf{Contesto dell'utilizzo}: [TO DO];
		\end{itemize}
	% subsubsection facade (end)



% subsection design_pattern_strutturali (end) \clearpage \newpage
	% =================================================================================================
% File:			dp_comportamentali.tex
% Description:	Defiinisce la sezione relativa a ...
% Created:		2015-03-26
% Author:		Tesser Paolo
% Email:		tesser.paolo@mashup-unipd.it
% =================================================================================================
% Modification History:
% Version		Modifier Date		Change											Author
% 0.0.1 		2015-03-26 			sistemato header								Tesser Paolo
% =================================================================================================
% 0.0.2			2015-04-14			descritit DP Page Controller, Template View		Tesser Paolo
% =================================================================================================
% 0.0.3			2015-04-14			descritto DP Template Method					Tesser Paolo
% =================================================================================================
%

% CONTENUTO DEL CAPITOLO

\subsection{Design pattern comportamentali} % (fold)
\label{sub:design_pattern_comportamentali}
	\subsubsection{Page Controller} % (fold)
	\label{ssub:page_controller}
		\begin{itemize}
			\item \textbf{Scope dell'utilizzo}: questo pattern serve [TO DO];
			\item \textbf{Contesto dell'utilizzo}:
				\begin{itemize}
					\item \textbf{Client}: viene utilizzato \newline
					[TO DO] (grafico del pattern applicato al caso di utilizzo nell'applicativo)
				\end{itemize}
		\end{itemize}
	% subsubsection page_controller (end)


	\subsubsection{Template Method} % (fold)
	\label{ssub:template_method}
		\begin{itemize}
			\item \textbf{Scope dell'utilizzo}: questo pattern serve per definire lo scheletro di un algoritmo, lasciando l'implementazione di alcuni passi alle sottoclassi;
			\item \textbf{Contesto dell'utilizzo}:
				\begin{itemize}
					\item \textbf{Client}: viene utilizzato nel package \ref{ssub:bdsm_app_client_model_services}, per permettere di generare tipi di grafici diversi che però hanno in comune la prima parte dell'algoritmo che li genera. In particolare la parte comune si occupa del recupero dei dati a prescindere da quali grafico li userà. [TO DO] \newline
					[TO DO] (grafico del pattern applicato al caso di utilizzo nell'applicativo)

					\item \textbf{Server}: [TO DO]. \newline
					[TO DO] (grafico del pattern applicato al caso di utilizzo nell'applicativo)
				\end{itemize}
		\end{itemize}
	% subsubsection template_method (end)


	\subsubsection{Template View} % (fold)
	\label{ssub:template_view}
		\begin{itemize}
			\item \textbf{Scope dell'utilizzo}: questo pattern serve per interpretare alcune informazioni incorporate nei template HTML. Nei sistemi di template generalmente vengono utilizzati dei segnaposto (markers) di qualche formato che verranno interpretati e sostituiti con il codice HTML adeguato. In AngularJS invece non c'è un formato intermediario perché vengono usate direttamente delle direttive HTML che quando saranno trovate dal compilatore di Angular, verrà invocata la logica ad esse associata;
			\item \textbf{Contesto dell'utilizzo}:
				\begin{itemize}
					\item \textbf{Client}: viene utilizzato in tutti i template HTML presentati nel package \textt{client::view} presente alla sezione \ref{ssub:bdsm_app_client_view}. \newline
					Ne viene qui di seguito illustrata una implementazione relativa al template HTML Settings. \newline
					[TO DO]
				\end{itemize}
		\end{itemize}
	% subsubsection template_view (end)


	\subsubsection{Command} % (fold)
	\label{ssub:command}
	[TO DO] (grafico del pattern applicato al caso di utilizzo nell'applicativo)
		\begin{itemize}
			\item \textbf{Scope dell'utilizzo}: [TO DO];
			\item \textbf{Contesto dell'utilizzo}:
				\begin{itemize}
					\item \textbf{Server}: [TO DO]
				\end{itemize}
		\end{itemize}
	% subsubsection command (end)
% subsection design_pattern_comportamentali (end) \clearpage \newpage
% section design_pattern (end)