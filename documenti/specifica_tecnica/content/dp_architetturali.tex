% =================================================================================================
% File:			dp_architetturali.tex
% Description:	Defiinisce la sezione relativa a ...
% Created:		2015-03-26
% Author:		Tesser Paolo
% Email:		tesser.paolo@mashup-unipd.it
% =================================================================================================
% Modification History:
% Version		Modifier Date		Change											Author
% 0.0.1 		2015-03-26 			aggiunto sezioni								Tesser Paolo
% =================================================================================================
% 0.0.2			2015-04-03			descritto pattern MVC, MVW e DI					Tesser Paolo
% =================================================================================================
%

% CONTENUTO DEL CAPITOLO

\subsection{Design pattern architetturali} % (fold)
\label{sub:design_pattern_architetturali}
	\subsubsection{Three-Tier} % (fold)
	\label{ssub:three_tier}
	[TO DO] (grafico del pattern applicato al caso di utilizzo nell'applicativo)
		\begin{itemize}
			\item \textbf{Scope dell'utilizzo}: [TO DO];
			\item \textbf{Contesto dell'utilizzo}: [TO DO];
		\end{itemize}
	% subsubsection three_tier (end)

	\subsubsection{MVC} % (fold)
	\label{ssub:mvc}
	[TO DO] (grafico del pattern applicato al caso di utilizzo nell'applicativo)
		\begin{itemize}
			\item \textbf{Scope dell'utilizzo}: questo pattern è utilizzato per separare le responsabilità dell’applicazione a diversi componenti e permettere di fare una chiara divisione tra presentazione, struttura dei dati e operazioni su di essi;
			\item \textbf{Contesto dell'utilizzo}: viene utilizzato nel livello client per separare i componenti a seconda delle loro responsabilità e secondo una connotazione semantica. Di fatto però quello che verrà effettivamente impiegato sarà il Design Pattern MVW esposto alla sezione \ref{ssub:mvw}.
		\end{itemize}
	% subsubsection mvc (end)

	\subsubsection{MVW} % (fold)
	\label{ssub:mvw}
	[TO DO] (grafico del pattern applicato al caso di utilizzo nell'applicativo)
		\begin{itemize}
			\item \textbf{Scope dell'utilizzo}: è un Design Pattern simile a MVC, che permette di avere una corrispondenza più diretta e automatica tra la \emph{view}e il \emph{model}. L'acronimo MVW sta infatti per Model-View-Whatever, dove \emph{Whatever} indica \emph{``whatever works for you''};
			\item \textbf{Contesto dell'utilizzo}: viene utilizzato per gestire il lato client dell'applicazione. Questo ci viene fornito direttamente dal framework AngularJS;
		\end{itemize}
	% subsubsection mvw (end)

	\subsubsection{Dependency Injection} % (fold)
	\label{ssub:dependency_injection}
	[TO DO] (grafico del pattern applicato al caso di utilizzo nell'applicativo)
		\begin{itemize}
			\item \textbf{Scope dell'utilizzo}: [TO DO];
			\item \textbf{Contesto dell'utilizzo}: [TO DO];
		\end{itemize}
	% subsubsection dependency_injection (end)

% subsection design_pattern_architetturali (end)