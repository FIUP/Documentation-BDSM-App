% =================================================================================================
% File:			dp_architetturali.tex
% Description:	Defiinisce la sezione relativa a ...
% Created:		2015-03-26
% Author:		Tesser Paolo
% Email:		tesser.paolo@mashup-unipd.it
% =================================================================================================
% Modification History:
% Version		Modifier Date		Change											Author
% 0.0.1 		2015-03-26 			aggiunto sezioni								Tesser Paolo
% =================================================================================================
% 0.0.2			2015-04-03			descritto pattern MVC, MVW e DI					Tesser Paolo
% =================================================================================================
% 0.0.3			2015-04-13			descritto Three-Tier							Tesser Paolo
% =================================================================================================
%

% CONTENUTO DEL CAPITOLO

\subsection{Design pattern architetturali} % (fold)
\label{sub:design_pattern_architetturali}

		\subsubsection{MVC} % (fold)
		\label{par:mvc}
			\begin{itemize}
				\item \textbf{Scope dell'utilizzo}: questo pattern è utilizzato per separare le responsabilità dell’applicazione a diversi componenti e permettere di fare una chiara divisione tra presentazione, struttura dei dati e operazioni su di essi;

				\item \textbf{Contesto dell'utilizzo}: 
					\begin{itemize}
						\item \textbf{Client}: viene utilizzato nel livello client per separare i componenti a seconda delle loro responsabilità e secondo una connotazione semantica. Il formalismo che viene quindi assunto per dividere i package è proprio quello fornito dal pattern in questione e cioè: \textbf{model, view e controller}. Questi però non interagiscono tra di loro nella maniera canonica che il pattern offre, ma seguono lo stile proposto da MVVM descritto alla sezione \ref{par:mvvm}.
					\end{itemize}
			\end{itemize}
		% subsubsection mvc (end)

		\subsubsection{MVVM} % (fold)
		\label{par:mvvm}
			\begin{itemize}
				\item \textbf{Scope dell'utilizzo}: questo pattern è utilizzato per gestire il modo con il quale le diverse parti comunicano tra loro per scambiare i dati e per gestire le operazioni che l'utente richiede attraverso l'interazione con la View;

				\item \textbf{Contesto dell'utilizzo}: 
					\begin{itemize}
						\item \textbf{Client}: viene utilizzato nel livello client per gestire lo scambio dei dati e le interazioni che fanno cambiare le View e lo stato del Model, attraverso principalmente un sistema di two-way data binding e di eventi. Il Model però non interagisce direttamente con le View, ma passa attraverso un controller che fa da collante tra la View e il Model (ViewModel) attraverso l'oggetto \$scope.
					\end{itemize}
			\end{itemize}
		% subsubsection mvvm (end)


	\subsubsection{Dependency Injection} % (fold)
	\label{ssub:dependency_injection}
		\begin{itemize}
			\item \textbf{Scope dell'utilizzo}: è un Design Pattern che viene utilizzato per favorire la separazione delle responsabilità tra i componenti dalla risoluzione delle dipendenze. Questo permette di avere una migliore modularità del codice, di avere un minor accoppiamento tra le diverse parti e garantisce una più facile fase di testing;
			\item \textbf{Contesto dell'utilizzo}: 
				\begin{itemize}
					\item \textbf{Client}: viene utilizzato direttamente da AngularJS per iniettare le diverse dipendenze nei moduli che le richiedono. In particolare vengono iniettati nei controller tutti i servizi e i modelli dei dati necessari all'interazione dell'utente con le viste. \newline
					Non ne viene fornita nessuna rappresentazione grafica in quanto non è una cosa che viene progettata dal team, ma usata direttamente attraverso il framework scelto;
				\end{itemize}
		\end{itemize}
	% subsubsection dependency_injection (end)

% subsection design_pattern_architetturali (end)