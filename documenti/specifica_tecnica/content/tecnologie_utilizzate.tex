% =================================================================================================
% File:			tecnologie_utilizzate.tex
% Description:	Defiinisce la sezione relativa a ...
% Created:		2015-02-23
% Author:		Tesser Paolo
% Email:		tesser.paolo@mashup-unipd.it
% =================================================================================================
% Modification History:
% Version		Modifier Date		Change											Author
% 0.0.1 		2015-02-23 			sistemato header								Tesser Paolo
% =================================================================================================
% 0.0.2			2015-02-05			iniziata impostazione contenuto e stesura		Tesser Paolo
% =================================================================================================
%
% =================================================================================================
%

% CONTENUTO DEL CAPITOLO

\section{Tecnologie utilizzate} % (fold)
\label{sec:tecnologie_utilizzate}
In questa sezione verranno descritte le tecnologie su cui si basa lo sviluppo del progetto. Per ognuna di esse, verrà indicato l’ambito di utilizzo della tecnologia ed i vantaggi/svantaggi che ne derivano.

	\subsection{Linguaggi} % (fold)
	\label{sub:linguaggi}

		\subsubsection{CSS} % (fold)
		\label{ssub:css}
		TO DO \newline
		\textbf{Vantaggi}:
			\begin{itemize}
				\item 
			\end{itemize}
			\noindent
		
		\textbf{Svantaggi}:
			\begin{itemize}
				\item 
			\end{itemize}
			\noindent
		% subsubsection css (end)

		\subsubsection{HTML5} % (fold)
		\label{ssub:html}
		TO DO \newline
		\textbf{Vantaggi}:
			\begin{itemize}
				\item 
			\end{itemize}
			\noindent
		
		\textbf{Svantaggi}:
			\begin{itemize}
				\item 
			\end{itemize}
			\noindent
		% subsubsection html (end)

		\subsubsection{Javascript} % (fold)
		\label{ssub:javascript}
		TO DO \newline
		\textbf{Vantaggi}:
			\begin{itemize}
				\item 
			\end{itemize}
			\noindent
		
		\textbf{Svantaggi}:
			\begin{itemize}
				\item 
			\end{itemize}
			\noindent
		% subsubsection javascript (end)

		\subsubsection{Python} % (fold)
		\label{sub:python}
		TO DO \newline
		\textbf{Vantaggi}:
			\begin{itemize}
				\item 
			\end{itemize}
			\noindent
	
		\textbf{Svantaggi}:
			\begin{itemize}
				\item 
			\end{itemize}
			\noindent
		% subsubsection python (end)

		\subsubsection{YAML} % (fold)
		\label{ssub:yaml}
		TO DO \newline
		\textbf{Vantaggi}:
			\begin{itemize}
				\item 
			\end{itemize}
			\noindent
		
		\textbf{Svantaggi}:
			\begin{itemize}
				\item 
			\end{itemize}
			\noindent
		% subsubsection yaml (end)

	% subsection linguaggi (end)

	\subsection{Framework} % (fold)
	\label{sub:framework}
		\subsubsection{AngularJS} % (fold)
		\label{ssub:angularjs}
		TO DO \newline
		\textbf{Vantaggi}:
			\begin{itemize}
				\item 
			\end{itemize}
			\noindent
		
		\textbf{Svantaggi}:
			\begin{itemize}
				\item 
			\end{itemize}
			\noindent
		% subsubsection angularjs (end)

		\subsubsection{JINJA} % (fold)
		\label{ssub:jinja}
		TO DO (da decidere)
		% subsubsection jinja (end)

	% subsection framework (end)

	\subsection{Librerie} % (fold)
	\label{sub:librerie}
	TO DO
	% subsection librerie (end)

	\subsection{Database} % (fold)
	\label{sub:database}

		\subsubsection{Datastore} % (fold)
		\label{ssub:datastore}
		TO DO \newline
		\textbf{Vantaggi}:
			\begin{itemize}
				\item 
			\end{itemize}
			\noindent
		
		\textbf{Svantaggi}:
			\begin{itemize}
				\item 
			\end{itemize}
			\noindent
		% subsubsection datastore (end)
	% subsection database (end)


% section tecnologie_utilizzate (end)