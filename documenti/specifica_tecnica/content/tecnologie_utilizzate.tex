% =================================================================================================
% File:			tecnologie_utilizzate.tex
% Description:	Defiinisce la sezione relativa a ...
% Created:		2015-02-23
% Author:		Tesser Paolo
% Email:		tesser.paolo@mashup-unipd.it
% =================================================================================================
% Modification History:
% Version		Modifier Date		Change											Author
% 0.0.1 		2015-02-23 			sistemato header								Tesser Paolo
% =================================================================================================
% 0.0.2			2015-03-05			iniziata impostazione contenuto e stesura		Tesser Paolo
% =================================================================================================
% 0.0.3			2015-03-09			aggiunta librerie e sistemata formattazione		Tesser Paolo
% =================================================================================================
% 0.0.4			2015-03-27			rimozione di JINJA e inseriti contro Datastore	Tesser Paolo
% =================================================================================================
% 0.0.5			2015-04-07			aggiunta libreria Satelizzer per gestione auth	Tesser Paolo
% =================================================================================================
% 0.0.6			2015-04-16			aggiunta libreria Google Chart					Tesser Paolo
% =================================================================================================
%

% CONTENUTO DEL CAPITOLO

\section{Tecnologie utilizzate} % (fold)
\label{sec:tecnologie_utilizzate}
In questa sezione verranno descritte le tecnologie su cui si basa lo sviluppo del progetto. Per ognuna di esse, verrà indicato l’ambito di utilizzo della tecnologia ed i vantaggi/svantaggi che ne derivano.

	\subsection{Linguaggi} % (fold)
	\label{sub:linguaggi}
		\subsubsection{CSS3} % (fold)
		\label{ssub:css}
		CSS (Cascading Style Sheets) è il linguaggio che verrà utilizzato per la formattazione delle pagine web offerte dal sistema qualora non bastassero i costrutti forniti dal framework Bootstrap \ref{ssub:twitter_bootstrap}. \newline
		\textbf{Pro}:
			\begin{itemize}
				\item permette una buona separazione dal contenuto della pagina rispetto a come verrà visualizzata. Questo garantisce un maggiore controllo sull'aspetto grafico e una più facile manutenzione.
			\end{itemize}

		% subsubsection css (end)

		\subsubsection{HTML5} % (fold)
		\label{ssub:html}
		HTML5 è il linguaggio di markup che verrà utilizzato per la strutturazione delle pagine web che l'applicazione andrà ad offrire sia per gli utenti che per gli amministratori del sistema. \newline
		\textbf{Pro}:
			\begin{itemize}
				\item permette di definire in maniera semplice la struttura delle pagine web;
				\item permette una maggiore semantica della pagina web, garantendo così una migliore indicizzazione da parte dei motori di ricerca;
				\item presenza di una vasta documentazione a supporto per i membri del team.
			\end{itemize}
			\noindent
		% subsubsection html (end)

		\subsubsection{JavaScript} % (fold)
		\label{ssub:javascript}
		JavaScript è il linguaggio di scripting che verrà utilizzato lato client per la creazione e l'interazione con l'interfaccia utente. \newline
		\textbf{Pro}:
			\begin{itemize}
				\item permette alle pagine di essere dinamiche, reagendo a eventi scaturiti dall'utente;
				\item permette di validare in prima istanza i dati inseriti dagli utenti.
			\end{itemize}
		\noindent
		\textbf{Contro}:
			\begin{itemize}
				\item l'assenza di tipizzazione potrebbe ostacolare la valutazione della correttezza del codice;
				\item assenza dei costrutti tipici della programmazione ad oggetti come ad esempio le classi. Sarà compito del \roleDesigner{} progettare, dove necessario, soluzione implementativi per sopperire a queste mancanze. I \emph{Programmatori} dovranno seguire tali decisioni attraverso l'attuazione di norme di codifica presenti nelle \docNameVersionNdP;
				\item il codice è visibile e può essere letto da chiunque, mettendo quindi a rischio la sicurezza dell'utente nel caso venisse eseguito del codice malevolo.
			\end{itemize}
			\noindent
		% subsubsection javascript (end)

		\subsubsection{JSON} % (fold)
		\label{ssub:json}
		JSON (JavaScript Object Notation) è il formato di dati scelto per lo scambio di informazioni tra il client e il server dell'applicazione. \newline
		\textbf{Pro}:
			\begin{itemize}
				\item utilizzo semplice tramite Javascript;
				\item formato utilizzato dai Google Endpoints che si andranno a sviluppare per i servizi REST resi disponibili dall'applicazione. 
			\end{itemize}
			\noindent
			\textbf{Contro}:
			\begin{itemize}
				\item supporta meno tipi di dati nativi rispetto a JavaScript;
				\item non permette di specificare il tipo dei campi numerici salvati al suo interno;
			\end{itemize}
		\noindent
		% subsubsection json (end)

		\subsubsection{Python} % (fold)
		\label{sub:python}
		Python è il linguaggio utilizzato per gestire il back-end dell'applicazione. Viene scelta la versione 2.7.9 in quanto è quella che la Google App Engine mette a disposizione. \newline
		\textbf{Pro}:
			\begin{itemize}
				\item notevole velocità di apprendimento del linguaggio;
				\item presenza di una code convention molto forte;
				\item permette una potente manipolazione dei dati.
			\end{itemize}
		\noindent
		\textbf{Contro}:
			\begin{itemize}
				\item assenza di costrutti come lo switch;
				\item assenza di tipizzazione statica. Sarà compito del programmatore rendere esplicito il tipo delle variabili nel codice seguendo delle norme stabilite nel \docNameVersionNdP.
			\end{itemize}
		\noindent
		% subsubsection python (end)

		\subsubsection{YAML} % (fold)
		\label{ssub:yaml}
		YAML (YAML Ain't a Markup Language) è il formato per la serializzazione di dati utilizzabile da esseri umani. \newline
		La sua scelta è stata vincolata all'utilizzo della Google App Engine con Python. \newline
		\textbf{Pro}:
			\begin{itemize}
				\item formato di una facile comprensione per un umano.
			\end{itemize}
		\noindent
		% subsubsection yaml (end)

	% subsection linguaggi (end)

	\subsection{Framework} % (fold)
	\label{sub:framework}
		\subsubsection{AngularJS} % (fold)
		\label{ssub:angularjs}
		AngularJS è il framework JavaScript open source utilizzato per gestire il lato client dell'applicazione sia per quanto riguarda la parte HTML sia per la parte algoritmica. Permette di sviluppare più facilmente tutto ciò che riguarda l'applicazione web. \newline
		\textbf{Pro}:
			\begin{itemize}
				\item presenza di una vasta comunità a supporto;
				\item presenza di numerosi moduli per far fronte a diversi compiti non presenti nativamente nel framework o che richiederebbero altrimenti un maggiore lavoro per la loro realizzazione;
				\item design pattern MVC e Dependency Injection già presente all'interno del framework.
			\end{itemize}
		\noindent
		\textbf{Contro}:
			\begin{itemize}
				\item si è molto vincolati all'ambiente, che non permette facili implementazioni con librerie esterne.
			\end{itemize}
		% subsubsection angularjs (end)

		\subsubsection{Twitter Bootstrap} % (fold)
		\label{ssub:twitter_bootstrap}
		Bootstrap è un insieme di elementi grafici, scritti in JavaScript e CSS3, che viene utilizzato per permettere di avere una grafica moderna e responsive dell'applicazione. \newline
		\textbf{Pro}:
			\begin{itemize}
				\item permette all'applicativo di essere responsive applicando facilmente delle regole già pronte;
				\item le diverse componenti vengono già incorporate direttamente nel framework AngularJS;
				\item presenza di una vasta comunità a supporto.
			\end{itemize}
		\noindent
		\textbf{Contro}:
			\begin{itemize}
				\item molto oneroso personalizzare pesantemente gli elementi grafici integrati
			\end{itemize}
		% subsubsection twitter_bootstrap (end)

	% subsection framework (end)

	\subsection{Librerie} % (fold)
	\label{sub:librerie}
		\subsubsection{Chart.js} % (fold)
		\label{ssub:chartsjs}
		Chart.js è la libreria di grafici in JavaScript utilizzata per la generazione della maggior parte dei grafici che l'utente andrà a vedere. La libreria è reperibile al seguente indirizzo: \url{http://www.chartjs.org/}. \newline
		\textbf{Pro}:
			\begin{itemize}
				\item permette ai grafici di essere responsive;
				\item presenza di diversi moduli implementativi per il framework AngularJS.
			\end{itemize}
		\noindent
		\textbf{Contro}:
			\begin{itemize}
				\item assenza di un Map Chart, che andrà cercato in un'altra libreria.
			\end{itemize}
		% subsubsection chartsjs (end)

		\subsubsection{Google Chart} % (fold)
		\label{ssub:google_chart}
		Google Chart è la libreria di grafici in JavaScript utilizzata per la generazione dei Geo Chart (Map Chart) reperibile al seguente indirizzo \url{https://developers.google.com/chart/interactive/docs/gallery/geochart}. \newline
		\textbf{Pro}:
			\begin{itemize}
				\item grosso supporto da parte della comunità;
				\item presenza di diversi moduli implementativi per il framework AngularJS in particolare di \textbf{angular-google-chart}, il più diffuso e supportato dalla comunità.
			\end{itemize}
		\noindent
		\textbf{Contro}:
			\begin{itemize}
				\item i grafici offerti, compresi quindi i Geo Chart utilizzati, non sono responsive. Bisognerà quindi che il team provveda a renderli tali.
			\end{itemize}
		% subsubsection google_chart (end)


		\subsubsection{facebook-sdk} % (fold)
		\label{ssub:facebook_sdk}
		Facebook-sdk è la libreria scelta per effettuare in maniera più semplice e astratta le chiamate alle API di Facebook. La libreria è reperibile al seguente indirizzo: \url{https://facebook-sdk.readthedocs.org/en/latest/}. \newline
		\textbf{Pro}:
			\begin{itemize}
				\item è la libreria maggiormente consigliata dalla comunità di Python.
			\end{itemize}

		% subsubsection facebook_sdk (end)

		\subsubsection{tweepy.api} % (fold)
		\label{ssub:tweetpy}
		Tweepy.api è la libreria scelta per effettuare in maniera più semplice e astratta le chiamate alle API di Twitter. La libreria è reperibile al seguente indirizzo: \url{http://www.tweepy.org/}. \newline
		\textbf{Pro}:
			\begin{itemize}
				\item è la libreria maggiormente consigliata dalla comunità di Python.
			\end{itemize}
		\noindent
		\textbf{Contro}:
			\begin{itemize}
				\item alcune librerie utilizzate dalla suddetta, hanno causato qualche problema. In particolare la libreria \textbf{requests} con la versione 2.5.3 ha avuto dei problemi con la Google Cloud Platform, per questo si è deciso di utilizzare la versione 2.3.0. Anche la libreria \textbf{six} ha riscontrato dei problemi, ma sono stati risolti posizionando il file six.py nella root della libreria tweepy.api.
			\end{itemize}
		\noindent
		% subsubsection tweetpy (end)

		\subsubsection{python-instagram} % (fold)
		\label{ssub:python_instagram}
		Python-instagram è la libreria scelta per effettuare in maniera più semplice e astratta le chiamate alle API di Instagram . La libreria è reperibile al seguente indirizzo: \url{https://github.com/Instagram/python-instagram}. \newline
		\textbf{Pro}:
			\begin{itemize}
				\item è la libreria maggiormente consigliata dalla comunità di Python.
			\end{itemize}
		\noindent
		% subsubsection python_instagram (end)

		\subsubsection{ng-token-auth} % (fold)
		\label{ssub:satellizer}
		ng-token-auth è un modulo di autenticazione basato sui token per AngularJS. Fornisce anche un supporto per la registrazione tramite altri sistemi attraverso il protocollo OAuth2. Il modulo è reperibile al seguente indirizzo: \url{https://github.com/lynndylanhurley/ng-token-auth}. \newline
		\textbf{Pro}:
			\begin{itemize}
				\item permette di effettuare tutte le operazioni relative alla gestione di un account utente andando a configurare in maniera semplice le API che dovranno essere chiamate nei diversi metodi offerti;
				\item fornisce numerosi eventi in risposta alle chiamate precedentemente citate per andare a gestire l'interfaccia utente e le diverse azioni che gli saranno permesse;
				\item la libreria ha una coverage del 96\%. 
			\end{itemize}
		\noindent
		\textbf{Contro}:
			\begin{itemize}
				\item la libreria ha pochi contributori rispetto ad un'altra che si era valutata per eseguire lo stesso compito e cioè Satellizer (\url{https://github.com/sahat/satellizer}), quest'ultima però ha una coverage molto inferiore a quella scelta. Inoltre, per come si è deciso di progettare l'integrazione, non sarà particolarmente problematico decidere di sostituirla in caso non fosse più adeguata alle esigenze dell'applicativo.
			\end{itemize}
		% subsubsection satellizer (end)
	% subsection librerie (end)

	\subsection{Database} % (fold)
	\label{sub:database}

		\subsubsection{Google Cloud Datastore} % (fold)
		\label{ssub:datastore}
		Google Cloud Datastore è il database NoSQL che viene utilizzato per l'immagazzinamento di tutti i dati. È quello fornito dalla Google App Engine. \newline
		\textbf{Pro}:
			\begin{itemize}
				\item permette un forte livello di astrazione che nasconde la complessità dei database schema-less;
				\item presenza di un linguaggio (GQL) utilizzato per l'effettuazione di query con la stessa sintassi di MySQL.
			\end{itemize}
			\noindent
			\textbf{Contro}:
				\begin{itemize}
					\item forte accoppiamento tra le classi che modellano i dati e la comunicazione con il DBMS stesso.
				\end{itemize}
			\noindent
		% subsubsection datastore (end)
	% subsection database (end)


% section tecnologie_utilizzate (end)