% =================================================================================================
% File:			commands.tex
% Description:	Defiinisce i comandi personalizzati utilizzati per redarre i documenti
% Created:		2014-12-05
% Author:		Santacatterina Luca
% Email:		s88.luca@gmail.com
% =================================================================================================
% Modification History:
% Version		Modifier Date		Change												Author
% 0.0.1 		2014-12-05 			Prima emissione 									Luca S.
% =================================================================================================
% 0.0.2 		2014-12-13 			Inseriti comandi nome e la vers. dei file 			Luca S.
% =================================================================================================
% 0.0.3			2014-12-19			inserito comando per richiamare il libro di testo	Tesser Paolo
% =================================================================================================
% 0.0.4			2015-01-02			inserito comando per le revisioni					Tesser Paolo
% =================================================================================================
% 0.0.5			2015-01-07			inseriti comandi per i documenti mancanti			Tesser Paolo
% =================================================================================================
% 0.0.6			2015-01-12			cambiata versione dei documenti a 1.0.0				Tesser Paolo
% =================================================================================================
% 0.0.7			2015-01-14			comandi per la generazione di grafici a torta/barre	Tesser Paolo
% =================================================================================================
% 0.0.8			2015-02-09			cambiata vers in 2.0.0 per doc fase analisi dett.	Tesser Paolo
% =================================================================================================
% 0.0.9			2015-03-20			cambiata vers in 3.0.0	per doc fase prog. archit	Tesser Paolo
% =================================================================================================
% 0.0.10		2015-04-21			cambiata vers in 4.0.0 per doc fase prog. dett. ob.	Tesser Paolo
% =================================================================================================
% 0.0.11		2015-04-21			aggiunto comando per il Manuale Amministratore		Tesser Paolo
% ================================================================================================
%

% DEFINIZIONE COMANDI BASE CHE POI ANDRANNO RIDEFINITI

% ==================================================================================================
% COMANDI DA RIDEFINIRE
% ==================================================================================================

% Nome/Versione/Data del documento
\newcommand{\documentName}{ERRORE}
\newcommand{\documentVersion}{ERRORE}
\newcommand{\documentDate}{ERRORE}

% Editori del documento
\newcommand{\documentEditors}{ERRORE}

% Verificatore del documento
\newcommand{\documentVerifiers}{ERRORE}

% Approvazione del documento
\newcommand{\documentApprovers}{ERRORE}

% Lista di distribuzione del documento
\newcommand{\documentDistributionList}{\groupName\\\commitNameM\\\commitNameS\\\proposerName}

% Uso del documento
\newcommand{\documentUsage}{ERRORE}

% Sommario del documento
\newcommand{\documentSummary}{ERRORE}

% Vesione dei documenti
\newcommand{\docVersionAdR}{\emph{v4.0.0}} % Analisi dei Requisiti
\newcommand{\docVersionGlo}{\emph{v3.0.0}} % Glossario
\newcommand{\docVersionNdP}{\emph{v5.0.0}} % Norme di Progetto
\newcommand{\docVersionPdP}{\emph{v5.0.0}} % Piano di Progetto
\newcommand{\docVersionPdQ}{\emph{v4.0.0}} % Piano di Qualifica
\newcommand{\docVersionSdF}{\emph{v1.0.0}} % Studio di Fattibilità
\newcommand{\docVersionST}{\emph{v2.0.0}} % Specifica Tecnica
\newcommand{\docVersionDdP}{\emph{v2.0.0}} % Definizione di Prodotto
\newcommand{\docVersionMU}{\emph{v1.0.0}} % Manuale Utente
\newcommand{\docVersionMA}{\emph{v1.0.0}} % Manuale Amministratore
% ==================================================================================================
% COMANDI DA NON RIDEFINIRE
% ==================================================================================================

% Nome del progetto
\newcommand{\projectName}{BDSMApp}
\newcommand{\groupEmail}{\textit{\href{mailto:info@mashup-unipd.it}{info@mashup-unipd.it}}}
\newcommand{\groupName}{\emph{MashUp}}

% Ruoli del progetto
\newcommand{\roleProjectManager}{\emph{Responsabile di Progetto}}
\newcommand{\roleAdministrator}{\emph{Amministratore di Progetto}}
\newcommand{\roleAnalyst}{\emph{Analista}}
\newcommand{\roleProgrammer}{\emph{Programmatore}}
\newcommand{\roleDesigner}{\emph{Progettista}}
\newcommand{\roleVerifier}{\emph{Verificatore}}

% Referenti e commitenti (M = master, S = slave)
\newcommand{\proposerName}{\emph{Dott. David Santucci} - Zing Srl}
\newcommand{\commitNameM}{\emph{Prof. Tullio Vardanega}}
\newcommand{\commitNameS}{\emph{Prof. Riccardo Cardin}}

% Abbreviazioni per richiamare il nome esteso dei diversi documenti da redarre
\newcommand{\docGlossary}{\emph{Glossario}} 
\newcommand{\docAdR}{\emph{Analisi dei Requisiti \currentVersion}}
\newcommand{\docNdP}{\emph{Norme di Progetto \currentVersion}}
\newcommand{\docPdP}{\emph{Piano di Progetto \currentVersion}}
\newcommand{\docPdQ}{\emph{Piano di Qualifica \currentVersion}}
\newcommand{\docSdF}{\emph{Studio di Fattibilità \currentVersion}}

% Nome dei documenti
\newcommand{\docNameAdR}{\emph{Analisi dei Requisiti}} % Analisi dei Requisiti
\newcommand{\docNameGlo}{\emph{Glossario}} % Glossario
\newcommand{\docNameNdP}{\emph{Norme di Progetto}} % Norme di Progetto
\newcommand{\docNamePdP}{\emph{Piano di Progetto}} % Piano di Progetto
\newcommand{\docNamePdQ}{\emph{Piano di Qualifica}} % Piano di Qualifica
\newcommand{\docNameSdF}{\emph{Studio di Fattibilità}} % Studio di Fattibilità
\newcommand{\docNameST}{\emph{Specifica Tecnica}} % Specifica Tecnica
\newcommand{\docNameDdP}{\emph{Definizione di Prodotto}} % Definizione di Prodotto
\newcommand{\docNameMU}{\emph{Manuale Utente}} % Manuale Utente
\newcommand{\docNameMA}{\emph{Manuale Amministratore}} % Manuale Amministratore

% Nome e versione dei documenti
\newcommand{\docNameVersionAdR}{\docNameAdR{} \docVersionAdR} % Analisi dei Requisiti
\newcommand{\docNameVersionGlo}{\docNameGlo{} \docVersionGlo} % Glossario
\newcommand{\docNameVersionNdP}{\docNameNdP{} \docVersionNdP} % Norme di Progetto
\newcommand{\docNameVersionPdP}{\docNamePdP{} \docVersionPdP} % Piano di Progetto
\newcommand{\docNameVersionPdQ}{\docNamePdQ{} \docVersionPdQ} % Piano di Qualifica
\newcommand{\docNameVersionSdF}{\docNameSdF{} \docVersionSdF} % Studio di Fattibilità
\newcommand{\docNameVersionST}{\docNameST{} \docVersionST} % Specifica Tecnica
\newcommand{\docNameVersionDdP}{\docNameDdP{} \docVersionDdP} % Definizione di Prodotto
\newcommand{\docNameVersionMU}{\docNameMU{} \docVersionMU} % Manuale Utente
\newcommand{\docNameVersionMA}{\docNameMA{} \docVersionMA} % Manuale Amministratore

% Nome delle revisioni
\newcommand{\RR}{\emph{Revisione dei Requisiti}}
\newcommand{\RPmin}{\emph{Revisione di Progettazione minima}}
\newcommand{\RPmax}{\emph{Revisione di Progettazione massima}}
\newcommand{\RQ}{\emph{Revisione di Qualifica}}
\newcommand{\RA}{\emph{Revisione di Accettazione}}

% Descrizione scopo del glossario
\newcommand{\glossarioDesc}{Al fine di evitare ogni ambiguità relativa al linguaggio usato nei documenti viene allegato il \docNameVersionGlo.
Esso ha lo scopo di definire ed analizzare tutti i termini tecnici del progetto e di fugare eventuali ambiguità fornendo un'accurata descrizione.\\
Tutte le occorrenze di tali termini nei documenti verranno contrassegnate con una ``G'' a pedice.}

% Descrizione dello scopo del prodtto
\newcommand{\productScope}{Lo scopo del prodotto è di creare una nuova infrastruttura che permetta di interrogare Big Data recuperati dai social network, quali: Facebook, Twitter, Instagram.
L'applicazione sarà composta da due parti:
	\begin{itemize}
		\item consultazione e interrogazione con interfaccia web per utente;
		\item servizi web REST interrogabili.
	\end{itemize}
}

\newcommand{\sommerville}{Software Engineering - Ian Sommerville - 9th Edition(2010)}

% Comandi per la generazione di grafici a torta e a barre
\newcommand{\pie}[3][]{
    \begin{scope}[#1]
    \pgfmathsetmacro{\curA}{90}
    \pgfmathsetmacro{\r}{1}
    \def\c{(0,0)}
    \node[pie title] at (90:1.3) {#2};
    \foreach \v\s in{#3}{
        \pgfmathsetmacro{\deltaA}{\v/100*360}
        \pgfmathsetmacro{\nextA}{\curA + \deltaA}
        \pgfmathsetmacro{\midA}{(\curA+\nextA)/2}

        \path[slice,\s] \c
            -- +(\curA:\r)
            arc (\curA:\nextA:\r)
            -- cycle;
        \pgfmathsetmacro{\d}{max((\deltaA * -(.5/50) + 1) , .5)}

        \begin{pgfonlayer}{foreground}
        \path \c -- node[pos=\d,pie values,values of \s]{$\v\%$} +(\midA:\r);
        \end{pgfonlayer}

        \global\let\curA\nextA
    }
    \end{scope}
}

\newcommand{\legend}[2][]{
    \begin{scope}[#1]
    \path
        \foreach \n/\s in {#2}
            {
                  ++(0,-5pt) node[\s,legend box] {} +(5pt,0) node[legend label] {\n} % ++(x,y) -> y: spazio tra i vari campi della legenda, x: l'obliquità
            }
    ;
    \end{scope}
}

% Comandi per il glossario
\newcommand{\gloss}{\ped{\textbf{\tiny G}}}

% Comando per abbassare versione PDF
\pdfminorversion=4