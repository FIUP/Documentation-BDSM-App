% =================================================================================================
% File:			commands.tex
% Description:	Defiinisce i comandi personalizzati utilizzati per redarre i documenti
% Created:		2014/12/05
% Author:		Santacatterina Luca
% Email:		s88.luca@gmail.com
% =================================================================================================
% Modification History:
% Version		Modifier Date		Change											Author
% 0.0.2 		2014/12/13 			Inseriti comandi nome e la vers. dei file 		Luca S.
% 0.0.1 		2014/12/05 			Prima emissione 								Luca S.
% =================================================================================================
%

% DEFINIZIONE COMANDI BASE CHE POI ANDRANNO RIDEFINITI

% ==================================================================================================
% COMANDI DA RIDEFINIRE
% ==================================================================================================

% Nome/Versione/Data del documento
\newcommand{\documentName}{ERRORE}
\newcommand{\documentVersion}{ERRORE}
\newcommand{\documentDate}{ERRORE}

% Editori del documento
\newcommand{\documentEditors}{ERRORE}

% Verificatore del documento
\newcommand{\documentVerifiers}{ERRORE}

% Approvazione del documento
\newcommand{\documentApprovers}{ERRORE}

% Lista di distribuzione del documento
\newcommand{\documentDistributionList}{\groupName\\\commitNameM\\\commitNameS\\\proposerName}

% Uso del documento
\newcommand{\documentUsage}{ERRORE}

% Sommario del documento
\newcommand{\documentSummary}{ERRORE}

% Vesione dei documenti
\newcommand{\docVersionAdR}{\emph{v0.0.1}}
\newcommand{\docVersionGlo}{\emph{v0.0.1}}
\newcommand{\docVersionNdP}{\emph{v0.0.1}}
\newcommand{\docVersionPdP}{\emph{v0.0.1}}
\newcommand{\docVersionPdQ}{\emph{v0.0.1}}
\newcommand{\docVersionSdF}{\emph{v0.0.1}}


% ==================================================================================================
% COMANDI DA NON RIDEFINIRE
% ==================================================================================================

% Nome del progetto
\newcommand{\projectName}{BDSMApp}
\newcommand{\groupEmail}{\textit{\href{mailto:info@mashup-unipd.it}{info@mashup-unipd.it}}}
\newcommand{\groupName}{\emph{MashUp}}

% Ruoli del progetto
\newcommand{\roleProjectManager}{\emph{Responsabile di Progetto}}
\newcommand{\roleAdministrator}{\emph{Amministratore di Progetto}}
\newcommand{\roleAnalyst}{\emph{Analista}}
\newcommand{\roleProgrammer}{\emph{Programmatore}}
\newcommand{\roleDesigner}{\emph{Progettista}}
\newcommand{\roleVerifier}{\emph{Verificatore}}

% Referenti e commitenti (M = master, S = slave)
\newcommand{\proposerName}{\emph{Dott. David Santucci} - Zing}
\newcommand{\commitNameM}{\emph{Prof. Tullio Vardanega}}
\newcommand{\commitNameS}{\emph{Prof. Riccardo Cardin}}

% Abbreviazioni per richiamare il nome esteso dei diversi documenti da redarre
\newcommand{\docGlossary}{\emph{Glossario}}
\newcommand{\docAdR}{\emph{Analisi dei Requisiti \currentVersion}}
\newcommand{\docNdP}{\emph{Norme di Progetto \currentVersion}}
\newcommand{\docPdP}{\emph{Piano di Progetto \currentVersion}}
\newcommand{\docPdQ}{\emph{Piano di Qualifica \currentVersion}}
\newcommand{\docSdF}{\emph{Studio di Fattibilità \currentVersion}}

% Nome dei documenti
\newcommand{\docNameAdR}{\emph{Analisi dei Requisiti}}
\newcommand{\docNameGlo}{\emph{Glossario}}
\newcommand{\docNameNdP}{\emph{Norme di Progetto}}
\newcommand{\docNamePdP}{\emph{Piano di Progetto}}
\newcommand{\docNamePdQ}{\emph{Piano di Qualifica}}
\newcommand{\docNameSdF}{\emph{Studio di Fattibilità}}

% Nome e versione dei documenti
\newcommand{\docNameVersionAdR}{\docNameAdR{} \docVersionAdR}
\newcommand{\docNameVersionGlo}{\docNameGlo{} \docVersionGlo}
\newcommand{\docNameVersionNdP}{\docNameNdP{} \docVersionNdP}
\newcommand{\docNameVersionPdP}{\docNamePdP{} \docVersionPdP}
\newcommand{\docNameVersionPdQ}{\docNamePdQ{} \docVersionPdQ}
\newcommand{\docNameVersionSdF}{\docNameSdF{} \docVersionSdF}

% Descrizione scopo del glossario
\newcommand{\glossarioDesc}{Al fine di evitare ogni ambiguità relativa al linguaggio usato nei documenti viene allegato il \docNameVersionGlo.\\
Esso ha lo scopo di definire ed analizzare tutti i termini tecnici, di progetto e di ambiguità fornendo una ben precisa descrizione.\\
Tutte le occorrenze presenti nei documenti verranno contrassegnate con una ``G'' in formato pedice.}
