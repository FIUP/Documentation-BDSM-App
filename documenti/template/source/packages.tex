% =================================================================================================
% File:			package.tex
% Description:	Defiinisce i pacchetti utilizzati per generare i documenti
% Created:		2014-12-05
% Author:		Santacatterina Luca
% Email:		s88.luca@gmail.com
% =================================================================================================
% Modification History:
% Version		Modifier Date		Change												Author
% 0.0.1 		2014-12-05 			Prima emissione 									Luca S.
% =================================================================================================
% 0.0.2			2014-12-17			cambio colori per i link in black					Tesser Paolo
% =================================================================================================
% 0.0.3			2014-12-23			inserito package caption e settato valore skip a 1	Tesser Paolo
% =================================================================================================
% 0.0.4			2015-01-15			aggiunto package per grafici a barre				Tesser Paolo

\documentclass[11pt,a4paper]{article}
\usepackage[left=3.5cm,right=3.5cm,top=3.0cm,bottom=2.5cm]{geometry}
\usepackage[font=small,labelfont=bf]{caption}
\usepackage[italian]{babel}
\usepackage[utf8]{inputenc}
\usepackage[T1]{fontenc}

\usepackage{color}
\usepackage{enumitem}
\usepackage{eurosym}
\usepackage{fancyhdr}
\usepackage{geometry}
\usepackage{graphicx}
\usepackage{hyperref}
\usepackage{lastpage}
\usepackage{lastpage}
\usepackage{listings}
\usepackage{longtable}
\usepackage{multirow}
\usepackage{pdflscape}
\usepackage{pdfpages}
\usepackage{subcaption}
\usepackage{tabularx}
\usepackage{tikz}
\usepackage{pgfkeys}
\usepackage{titlesec}
\usepackage{verbatim}
\usepackage{xr}
\usepackage{xspace}
\usepackage[skip=1pt]{caption} % package che gestisce lo spazione tra la figura e la sua caption
\usepackage{sectsty}
\usepackage{pgfplots}

\pagestyle{fancy}

\hypersetup{
	colorlinks=true,
	linkcolor=black,
	urlcolor=blue
}

\setcounter{secnumdepth}{5}
\setcounter{tocdepth}{5}
