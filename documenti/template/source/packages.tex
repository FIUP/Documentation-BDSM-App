% DEFINIZIONE DEI PACKAGES DA IMPORTARE NEI DOCUMENTI


\documentclass[11pt,a4paper]{article}					% Inizializzazione documento

\usepackage[left=3.5cm,right=3.5cm,top=3.0cm,bottom=2.5cm]{geometry}
\usepackage[italian]{babel}
\usepackage[utf8]{inputenc}
\usepackage[T1]{fontenc}
\usepackage{enumitem}									% package per la gestione degli elementi lista (rimozione simboli e spazi)
\usepackage{graphicx}									% package per l'inclusione delle immagini
\usepackage{subcaption}									% package per le sottoimmagini
\usepackage{fancyhdr}									% package per l'inserimento di header e footer
\usepackage{longtable}									% package per l'inserimento di tabelle lunghe
\usepackage{color}										% package per i colori
\usepackage{verbatim}									% package per l'inclusione di codice e inserimento commenti
\usepackage{hyperref}
	\hypersetup{
		colorlinks=true,
		linkcolor=black,
		urlcolor=black
	}
\usepackage{eurosym}									% package per il simbolo dell'euro
