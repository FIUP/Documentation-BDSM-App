% =================================================================================================
% File:			riassunto_riunione.tex
% Description:	Defiinisce la sezione relativa alle decisioni emerse durante la riunione
% Created:		2015-01-17
% Author:		Faccin Nicola
% Email:		faccin.nicola@mashup-unipd.it
% =================================================================================================
% Modification History:
% Version		Modifier Date		Change											Author
% 0.0.1 		2015-01-17 			stesura informazioni incontro e ordine			Faccin Nicola
% =================================================================================================
%

% CONTENUTO DEL CAPITOLO

\section{Riassunto della riunione} % (fold)
\label{sec:riassunto_della_riunione}
	\subsection{Risposte all'ordine del giorno} % (fold)
	\label{sub:risposte_all_ordine_del_giorno}
	Le risposte di seguito fornite non sono la trascrizione esatta di quanto detto al momento, ma una elaborazione finale in accordo con il proponente.
		\begin{itemize}
			\item L'applicativo avrà un target molto ampio. Sarà infatti predisposto per qualsiasi persona voglia ricevere delle statistiche su un determinato settore, non verrà messo quindi nessun vincolo durante la registrazione;
			\item Vengono definite come ``ricette'' le tipologie di dato che l'applicativo dovrà andare a recuperare periodicamente 
			\item Non viene consentita all'utente la libertà di creare delle ricette personalizzate. Le ricerche effettuate sulle ricette potranno essere però generiche;
			\item TO DO (dati grezzi o elaborati);
			\item TO DO (tipologia di database);
			\item Per il proponente è indifferente quale linguaggio di programmazione verrà scelto. Nonostante questo consigliano l'uso di Python in quanto la sua organizzazione lavora principalmente con quello. Se la scelta dovesse ricadere proprio su quest'ultimo viene consigliato l'utilizzo del framework Django; 
			\item TO DO;
			\item L'amministratore avrà a disposizione una sezione privata nella quale potrà TO DO. Questo non è per il momento un aspetto fondamentale dell'applicativo.
		\end{itemize}
	% subsection risposte_all_ordine_del_giorno (end)
	
	\subsection{Altre considerazioni} % (fold)
	\label{sub:altre_considerazioni}
	Durante l'incontro si è discusso molto con il proponente anche su altre questioni. \\
	TO DO \\
	Nelle attività di codifica viene consigliato di utilizzare, per i nomi delle varie componenti del sistema, una sola lingua, preferibilmente quella inglese. Per i commenti può andare bene anche l'italiano, l'importante è che siano il più chiari possibili e solo quando il codice non è già auto esplicativo. \\
	
	% subsection altre_considerazioni (end)
% section riassunto_della_riunione (end)