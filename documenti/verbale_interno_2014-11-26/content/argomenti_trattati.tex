% =================================================================================================
% File:			argomenti_trattati.tex
% Description:	Defiinisce la sezione relativa al riassunto dell'incontro
% Created:		2014/12/13
% Author:		Ceccon Lorenzo
% Email:		ceccon.lorenzo@mashup-unipd.it
% =================================================================================================
% Modification History:
% Version		Modifier Date		Change											Author
% 0.0.1 		2014/12/13 			iniziata stesura riassunto dell'incontro		Lorenzo C.
% =================================================================================================
%

% CONTENUTO DEL CAPITOLO
\section{Argomenti trattati}
   Come prima attività si è cercato di trovare un nome per il gruppo, dopo diverse proposte la scelta è ricaduta sul nome pensato da Tesser Paolo: \textbf{\groupName} \textnormal{; è stata inoltre data una prima impostazione grafica al logo da utilizzare.}\\
   \textnormal{Dopo un'attenta analisi dei singoli membri sui capitolati proposti si è discusso  insieme dei pregi e dei difetti di ciascuno. Avendo scartato i capitolati C2, C4 e C5 si è ricorsi ad una votazione tra il capitolato C1 e C3. La maggioranza ha quindi deciso di sviluppare il capitolato \textbf{C1} \textnormal{intitolato:} \textbf{BDSMApp: Big Data Social Monitoring App}\textnormal{.}}\\
  \textnormal{In seguito l'attenzione si è spostata su quali strumenti software utilizzare per lavorare al progetto; per le comunicazioni urgenti si è optato quindi, di creare una chat di gruppo su WhatsApp, mentre per assegnare i lavori da svolgere ai vari membri del team si è scelto di utilizzare un sistema di ticketing denominato Asana.
  È stato scelto anche di registrare un dominio su Netsons e di crearci degli indirizzi email personali.
  Per quanto riguarda il repository si è deciso di utilizzare Git a discapito di SVN, mentre come servizio hosting per il repository si è optato per GitHub. Per la gestione di documenti che non necessitano versionamento si è scelto di creare una cartella condivisa su Google Drive.}\\
  \textnormal{Per finire si è consigliata la lettura di alcuni manuali per l'utilizzo di \LaTeX
   e di Git.}