% =================================================================================================
% File:			dia_seq_server.tex
% Description:	Definisce la sezione relativa al capitolo relativo ai diagrammi di sequenza tra i componenti del server
% Created:		2015-05-22
% Author:		Faccin Nicola
% Email:		faccin.nicola@mashup-unipd.it
% =================================================================================================
% Modification History:
% Version		Modifier Date		Change											Author
% 0.0.1 		2015-05-22 			creato scheletro doc							Nicola Faccin
% =================================================================================================
%

% CONTENUTO DEL CAPITOLO

\subsection{Diagrammi di sequenza del server} % (fold)
\label{ssub:diagrammi_di_sequenza_server}
Quelli che seguono sono i diagrammi di sequenza che rappresentano le più rilaventi interazioni tra i vali moduli presenti nel livello server. Per quanto riguarda il client, fare riferimento alla sezione \ref{ssub:diagrammi_di_sequenza_client}.

    \subsubsection{Cron task} % (fold)
    \label{ssub:cron_task}
    Il seguente diagramma descrive il processo di avvio dell'aggiornamento dei dati grezzi relativi ad ogni Recipe presente nel sistema. Tale sequenza è attivata a cadenza regolare tramite le funzionalità di \textit{task scheduling} offerte da Google App Engine. Per semplicità, è stato descritto in dettaglio esclusivamente l'aggiornamento dei dati relativi a Facebook, indicando solamente i metodi principali per i restanti social network. \newline

    TODO (grafico) \newline
    \noindent


    \begin{itemize}
        \item \textbf{UpdateRecipeTask}: classe implementata secondo gli standard di Google App Engine, definisce il metodo statico \texttt{get()} attivato ad ogni avvio del componente Cron di Google. \texttt{UpdateRecipeTask} istanzia la classe \texttt{MinerManager} invocando il metodo \texttt{schedule\_fetchers(recipe\_id\_list)} per effettuare l'aggiornamento dei dati grezzi;
        \item \textbf{MinerScheduler}: definisce il metodo \texttt{schedule\_fetchers(recipe\_id\_list)} che cicla la lista di id delle Recipe scadute invocando per ognuna il metodo \texttt{update\_fb\_metrics(id)} per ogni metrica Facebook contenuta nella Recipe in questione. Tale metodo avvia un fetcher differente a seconda del tipo di metrica ricavata. Tale procedimento avviene sequenzialmente anche per gli altri social tramite i metodi \texttt{update\_tw\_metrics(id)} e \texttt{update\_ig\_metrics(id)}
        \item \textbf{FbPageFetcher}: classe che viene istanziata ed avviata tramite il metodo \texttt{run()} per avviare l'aggiornamento dei dati relativi ad una pagina Facebook.
        \item \textbf{FbEventFetcher}: classe che viene istanziata ed avviata tramite il metodo \texttt{run()} per avviare l'aggiornamento dei dati relativi ad un evento Facebook.
    \end{itemize}



% section diagrammi_di_sequenza_server (end)