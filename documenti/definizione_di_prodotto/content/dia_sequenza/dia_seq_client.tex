% =================================================================================================
% File:			dia_seq_server.tex
% Description:	Definisce la sezione relativa al capitolo relativo ai diagrammi di sequenza tra i componenti del client
% Created:		2015-05-22
% Author:		Faccin Nicola
% Email:		faccin.nicola@mashup-unipd.it
% =================================================================================================
% Modification History:
% Version		Modifier Date		Change											Author
% 0.0.1 		2015-05-22 			creato scheletro doc							Nicola Faccin
% =================================================================================================
% 0.0.2			2015-05-23			iniziata stesura sezioni						Tesser Paolo
% =================================================================================================
% 0.0.3			2015-05-24			iniziata descrizione oggetti					Tesser Paolo
% =================================================================================================
%

% CONTENUTO DEL CAPITOLO

\subsection{Client} % (fold)
\label{ssub:diagrammi_di_sequenza_client}
Quasi tutte le operazioni offerte dall'applicazione comportano lo stesso tipo di interazione tra le classi presenti nella componente client. Ne sono quindi state descritte alcune ritenute tra le più significative. \newline
Questi diagrammi riguardano solo la parte appunto del client. La comunicazione quindi con il server e le operazioni che esso esegue per ritornare dei valori al client viene mostrata nella sezione \ref{ssub:diagrammi_di_sequenza_server}.

	\subsubsection{Login} % (fold)
	\label{ssub:login}
	Il seguente diagramma di sequenza descrive le collaborazioni tra gli oggetti del front-end dell'applicazione durante l'operazione di autenticazione. \newline

	TODO (grafico) \newline
	\noindent

	Il diagramma descrive le interazioni tra gli oggetti che compongono il front-end quando un utente vuole effettuare il login nell'applicazione.
	\begin{itemize}
		\item \textbf{Utente non autenticato}: l'oggetto Utente non autenticato individua un qualsiasi attore che vuole effettuare l'accesso al sistema. L'utente inserisce i propri dati e preme il bottone ``Login'';
		\item \textbf{PublicRoute}: l'oggetto PublicRoute re-indirizza l'utente alla corretta pagina, dalla quale potrà effettuare l'accesso;
		\item \textbf{Login (login.html)}: l'oggetto Login è la vera pagina che permette l'autenticazione dell'utente. Qui verranno inserite le credenziali per accedere e sarà presente il pulsante che una volta premuto invocherà il metodo \textbf{login()};
		\item \textbf{LoginCtrl}: TODO
		\item \textbf{AuthService}: TODO
		\item \textbf{UserApi}: TODO
		\item \textbf{Recipe (recipe.html)}: TODO
	\end{itemize}
	% subsubsection login (end)

	\subsubsection{Visualizzazione delle Recipe} % (fold)
	\label{ssub:visualizzazione_delle_recipe}
	Il seguente diagramma di sequenza descrive le collaborazioni tra gli oggetti del front-end dell'applicazione durante la visualizzazione delle Recipe. \newline

	TODO (grafico) \newline
	\noindent

	Il diagramma descrive le interazioni tra gli oggetti che compongono il front-end quando un utente vuole visualizzare la lista delle Recipe presenti nel sistema.
	\begin{itemize}
		\item \textbf{Utente autenticato}: TODO
		\item \textbf{RecipeRoute}: TODO
		\item \textbf{Recipe (recipe.html)}: TODO
		\item \textbf{RecipeCtrl}: TODO
		\item \textbf{RecipeService}: TODO
		\item \textbf{DataManagerService}: TODO
	\end{itemize}
	% subsubsection visualizzazione_delle_recipe (end)

	\subsubsection{Visualizzazione dei grafici} % (fold)
	\label{sub:visualizzazione_dei_grafici}
	TODO
	% subsection visualizzazione_dei_grafici (end)

% section diagrammi_di_sequenza_client (end)