% =================================================================================================
% File:			dia_seq_server.tex
% Description:	Definisce la sezione relativa al capitolo relativo ai diagrammi di sequenza tra i componenti del client
% Created:		2015-05-22
% Author:		Faccin Nicola
% Email:		faccin.nicola@mashup-unipd.it
% =================================================================================================
% Modification History:
% Version		Modifier Date		Change											Author
% 0.0.1 		2015-05-22 			creato scheletro doc							Nicola Faccin
% =================================================================================================
% 0.0.2			2015-05-23			iniziata stesura sezioni						Tesser Paolo
% =================================================================================================
%

% CONTENUTO DEL CAPITOLO

\subsection{Client} % (fold)
\label{ssub:diagrammi_di_sequenza_client}
Quasi tutte le operazioni offerte dall'applicazione comportano lo stesso tipo di interazione tra le classi presenti nella componente client. Ne sono quindi state descritte alcune ritenute tra le più significative. \newline
Questi diagrammi riguardano solo la parte appunto del client. La comunicazione quindi con il server e le operazioni che esso esegue per ritornare dei valori al client viene mostrata nella sezione \ref{ssub:diagrammi_di_sequenza_server}.

	\subsubsection{Login} % (fold)
	\label{ssub:login}
	Il seguente diagramma di sequenza descrive le collaborazioni tra le classi del front-end dell'applicazione durante l'operazione di autenticazione. Di come  \newline

	\noindent
	TODO (grafico) \newline

	Il diagramma descrive le interazioni tra gli oggetti che compongono il front-end quando un utente vuole effettuare il login nell'applicazione.
	\begin{itemize}
		\item \textbf{Utente non autenticato}: TODO
		\item \textbf{PublicRoute}: TODO
		\item \textbf{Login (login.html)}: TODO
		\item \textbf{LoginCtrl}: TODO
		\item \textbf{AuthService}: TODO
		\item \textbf{UserApi}: TODO
	\end{itemize}
	% subsubsection login (end)

	\subsubsection{Visualizzazione delle Recipe} % (fold)
	\label{ssub:visualizzazione_delle_recipe}
	TODO
	% subsubsection visualizzazione_delle_recipe (end)

	\subsubsection{Visualizzazione dei grafici} % (fold)
	\label{sub:visualizzazione_dei_grafici}
	TODO
	% subsection visualizzazione_dei_grafici (end)

% section diagrammi_di_sequenza_client (end)