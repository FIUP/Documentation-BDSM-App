% =================================================================================================
% File:			introduzione.tex
% Description:	Defiinisce la sezione relativa al capitolo introduttivo del documento
% Created:		2015-04-21
% Author:		Tesser Paolo
% Email:		tesser.paolo@mashup-unipd.it
% =================================================================================================
% Modification History:
% Version		Modifier Date		Change											Author
% 0.0.1 		2015-04-21 			creato scheletro doc e primo abbozzo			Tesser Paolo
% =================================================================================================
%

% CONTENUTO DEL CAPITOLO

\section{Introduzione} % (fold)
\label{sec:introduzione}
	\subsection{Scopo del documento} % (fold)
	\label{sub:scopo_del_documento}
	Questo documento ha come scopo quello di definire la progettazione di dettaglio per il prodotto \projectName. Verrà presentata la struttura e le relazioni tra le componenti a basso livello. Queste serviranno ai \emph{Programmatori} come guida per le attività di codifica.
	% subsection scopo_del_documento (end)

	\subsection{Scopo del prodotto} % (fold)
	\label{sub:scopo_del_prodotto}
	\productScope
	% subsection scopo_del_prodotto (end)

	\subsection{Glossario} % (fold)
	\label{sub:glossario}
	\glossarioDesc
	% subsection glossario (end)

	\subsection{Riferimenti} % (fold)
	\label{sub:riferimenti}
		\subsubsection{Normativi} % (fold)
		\label{ssub:normativi}
			\begin{itemize}
				\item \textbf{Analisi dei Requisiti}: \docNameVersionAdR
				\item \textbf{Norme di Progetto}: \docNameVersionNdP
				\item \textbf{Specifica Tecnica}: \docNameVersionST
			\end{itemize}
		% subsubsection normativi (end)

		\subsubsection{Informativi} % (fold)
		\label{ssub:informativi}
			\begin{itemize}
				\item \textbf{ControllerAs syntax}: \url{http://www.johnpapa.net/angularjss-controller-as-and-the-vm-variable/};
				\item \textbf{Promise AngularJS}: \url{http://www.webdeveasy.com/javascript-promises-and-angularjs-q-service/};
				\item \textbf{John Papa style guide}: \url{https://github.com/johnpapa/angular-styleguide};
				\item \textbf{ng-auth doc}: \url{https://github.com/lynndylanhurley/ng-token-auth};
				\item
			\end{itemize}
		% subsubsection informativi (end)
	% subsection riferimenti (end)
% section introduzione (end)
% section introduzione (end)