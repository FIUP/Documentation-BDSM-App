% =================================================================================================
% File:			introduzione.tex
% Description:	Defiinisce la sezione relativa al capitolo introduttivo del documento
% Created:		2015-04-21
% Author:		Tesser Paolo
% Email:		tesser.paolo@mashup-unipd.it
% =================================================================================================
% Modification History:
% Version		Modifier Date		Change											Author
% 0.0.1 		2015-04-21 			creato scheletro doc e primo abbozzo			Tesser Paolo
% =================================================================================================
%

% CONTENUTO DEL CAPITOLO

\section{Introduzione} % (fold)
\label{sec:introduzione}
	\subsection{Scopo del documento} % (fold)
	\label{sub:scopo_del_documento}
	Questo documento ha come scopo quello di definire la progettazione di dettaglio per il prodotto \projectName. Verrà presentata la struttura e le relazioni tra le componenti a basso livello. Queste serviranno ai \emph{Programmatori} come guida per le attività di codifica.
	% subsection scopo_del_documento (end)

	\subsection{Scopo del prodotto} % (fold)
	\label{sub:scopo_del_prodotto}
	\productScope
	% subsection scopo_del_prodotto (end)

	\subsection{Glossario} % (fold)
	\label{sub:glossario}
	\glossarioDesc
	% subsection glossario (end)

	\subsection{Riferimenti} % (fold)
	\label{sub:riferimenti}
		\subsubsection{Normativi} % (fold)
		\label{ssub:normativi}
			\begin{itemize}
				\item \textbf{Analisi dei Requisiti}: \docNameVersionAdR
				\item \textbf{Norme di Progetto}: \docNameVersionNdP
				\item \textbf{Specifica Tecnica}: \docNameVersionST
			\end{itemize}
		% subsubsection normativi (end)

		\subsubsection{Informativi} % (fold)
		\label{ssub:informativi}
			\begin{itemize}
				\item \textbf{ControllerAs syntax}: \newline	\url{http://www.johnpapa.net/angularjss-controller-as-and-the-vm-variable/}
				\item \textbf{Promise AngularJS}: \newline
\url{http://www.webdeveasy.com/javascript-promises-and-angularjs-q-service/}
				\item \textbf{John Papa style guide}:\newline \url{https://github.com/johnpapa/angular-styleguide}
				\item \textbf{ng-auth doc}:\newline \url{https://github.com/lynndylanhurley/ng-token-auth}
				\item \textbf{HTTPS su GitHub pages}:\newline
					\url{https://konklone.com/post/github-pages-now-sorta-supports-https-so-use-it}
				\item \textbf{Google Datastore}:\newline \url{https://cloud.google.com/appengine/docs/python/ndb/}
				\item \textbf{Google Cloud Endpoints}:\newline \url{https://cloud.google.com/appengine/docs/python/endpoints/}
				\item \textbf{Google Cron}:\newline \url{https://cloud.google.com/appengine/docs/python/config/cron}
				\item \textbf{Api facebook (facebook-sdk)}:\newline \url{https://github.com/pythonforfacebook/facebook-sdk}
				\item \textbf{Api twitter (tweepy)}:\newline \url{http://tweepy.readthedocs.org/en/v3.2.0/}
				\item \textbf{Api instagram (python-instagram)}:\newline \url{https://github.com/Instagram/python-instagram}
			\end{itemize}
		% subsubsection informativi (end)
	% subsection riferimenti (end)
% section introduzione (end)
% section introduzione (end)