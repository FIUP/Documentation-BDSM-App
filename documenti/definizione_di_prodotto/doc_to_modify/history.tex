% =================================================================================================
% File:			history.tex
% Description:	Defiinisce la history delle modifiche al documento
% Created:		2015-04-21
% Author:		Tesser Paolo
% Email:		tesser.paolo@mashup-unipd.it
% =================================================================================================
% Modification History:
% Version		Modifier Date		Change											Author
% 0.0.1 		2015-04-21 			creazione struttura								Tesser Paolo
% =================================================================================================
%

\begin{center}
\begin{small}
	\textbf{\huge Diario Revisioni}
	\vspace{0.5cm}
	\begin{longtable}{p{6cm}|c|c|c}
		\label{tab:history}
		\textbf{Modifica} & \textbf{Autore \& Ruolo} & \textbf{Data} & \textbf{Versione} \\
		\hline





		\emph{Aggiornati contenuti classi sez. Client - Attributi e Metodi} & 
			\begin{tabular}[c]{c c}
				Roetta Marco \\
				\emph{Progettista} \\
		\end{tabular} & 2015-04-14 & v0.0.5 \\
		\hline

		\emph{Aggiornati contenuti classi sez. Server - Attributi e Metodi} & 
			\begin{tabular}[c]{c c}
				Roetta Marco \\
				\emph{Progettista} \\
		\end{tabular} & 2015-04-13 & v0.0.4 \\
		\hline

		\emph{Stesura sezione \ref{ssub:bdsm_app_client_view} client::view andando ad aggiungere le direttive di AngularJS utilizzate per i template HTML relativi ai requisiti obbligatori} & 
			\begin{tabular}[c]{c c}
				Tesser Paolo \\
				\emph{Progettista} \\
		\end{tabular} & 2015-04-10 & v0.0.3 \\

		\emph{Importato parte del contenuto dal documento \docNameVersionST} & 
			\begin{tabular}[c]{c c}
				Santacatterina Luca \\
				\emph{Progettista} \\
		\end{tabular} & 2015-04-02 & v0.0.2 \\

		\emph{Creato scheletro del documento} & 
			\begin{tabular}[c]{c c}
				Santacatterina Luca \\
				\emph{Progettista} \\
		\end{tabular} & 2015-04-01 & v0.0.1 \\
		\hline
	\end{longtable}

\end{small}
\end{center}