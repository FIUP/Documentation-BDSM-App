% =================================================================================================
% File:			nome_del_capitolo.tex
% Description:	Defiinisce la sezione relativa a ...
% Created:		2014/12/05
% Author:		Santacatterina Luca
% Email:		s88.luca@gmail.com
% =================================================================================================
% Modification History:
% Version		Modifier Date		Change											Author
% 0.0.1 		2014/12/05 			iniziata stesura documento di prova				Luca S.
% =================================================================================================
%

% CONTENUTO DEL CAPITOLO
\section{Informazioni generali}
\begin{itemize}
  \item \bfseries{Data:} \textnormal{2014-11-26}
  \item \bfseries{Ora:} \textnormal{14:30}
  \item \bfseries{Luogo:} \textnormal{Laboratorio P036, Plesso Paolotti, Via G.B. Belzoni, 7 Padova}
  \item \bfseries{Partecipanti:} \textnormal{Carnovalini Filippo, Ceccon Lorenzo, Cusinato Giacomo, Faccin Nicola, Roetta Marco, Santacatterina Luca, Tesser Paolo}
\end{itemize}

\section{Argomenti trattati}
\begin{itemize}
  \item \bfseries{Scelta del nome del gruppo:} \textnormal{Come prima attività si è cercato di trovare un nome per il gruppo, dopo diverse proposte la scelta è ricaduta sul nome pensato da Tesser Paolo: \bfseries{MashUp}\textnormal{. È stata inoltre data una prima impostazione grafica al logo da utilizzare.}}
  \item \bfseries{Scelta del capitolato d'appalto:} \textnormal{Dopo un'attenta analisi dei singoli membri sui capitolati proposti si è discusso  insieme dei pregi e dei difetti di ogni capitolato. Dopo aver scartato i capitolati C2, C4 e C5 si è ricorsi ad una votazione tra il capitolato C1 e C3. La maggioranza ha quindi deciso di sviluppare il capitolato \bfseries{C1} \textnormal{intitolato:} \bfseries{BDSMApp: Big Data Social Monitoring App\textnormal{.}}}
  \item \bfseries{Scelta degli strumenti da utilizzare:} \textnormal{Per le comunicazioni urgenti è stato scelto di creare una chat di gruppo su WhatsApp, mentre per assegnare i lavori da svolgere ai vari membri del team si è scelto di utilizzare un sistema di ticketing denominato Asana.
  È stato scelto anche di registrare un dominio su Netsons e di crearci delgli indirizzi email personali.
  Per quanto riguarda il repository si è deciso di utilizzare Git a discapito di SVN, mentre come servizio hosting per il repository si è optato per GitHub. Per la gestione di documenti che non necessitano versionamento si è scelto di creare una cartella condivisa su Google Drive.}
  \item \bfseries{Suddivisione dei ruoli:} \textnormal{Prima della fine dell'incontro si sono assegnati i ruoli ai vari membri del gruppo e si è consigliata la lettura di alcuni manuali per \LaTeX\ e Git.}
\end{itemize}


