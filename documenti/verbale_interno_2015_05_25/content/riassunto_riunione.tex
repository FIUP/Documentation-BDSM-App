% =================================================================================================
% File:			riassunto_riunione.tex
% Description:	Defiinisce la sezione relativa a ...
% Created:		2015-02-19
% Author:		Cusinato Giacomo
% Email:		cusinato.giacomo@mashup-unipd.it
% =================================================================================================
% Modification History:
% Version		Modifier Date		Change											Author
% 0.0.1 		2015-02-19 			stesura riassunto della riunione				Cusinato Giacomo
% =================================================================================================
%

% CONTENUTO DEL CAPITOLO

\section{Riassunto della riunione} % (fold)
\label{sec:riassunto_della_riunione}

\subsection{Risposte all'ordine del giorno}
Le risposte di seguito fornite non sono la trascrizione esatta di quanto detto al momento, ma una elaborazione finale in accordo con il proponente. \newline
Esse saranno fonte di raffinamento/cambio di alcuni requisiti presenti nel \docNameVersionAdR.

\begin{itemize}
  	\item una volta esposto il lavoro fino ad ora svolto, il proponente è rimasto soddisfatto di quanto implementato e a lui mostrato, non soffermarsi solamente a guardare l'applicazione visiva, ma anche sul codice prodotto in particolare modo per la componente \textbf{server};
	\item si è discusso inoltre su un problema riscontrato durante l'implementazione dei Geo/Map Chart previsti in alcuni requisiti obbligatori. Si è esposto che questi sono di difficile implementazione per come è stata prevista la generazione in maniera dinamica degli altri grafici in quanto viene usata una libreria diversa che non consente lo stesso tipo di utilizzo. Il proponente ha accettato quindi che questi non fossero implementati, purché fossero disponibili i dati ad essi relativi attraverso i servizi REST offerti. \newline
	I requisiti che da obbligatori sono diventati desiderabili sono: ROF5.5.1.9, ROF5.4.1.3, ROF5.3.1.5.
\end{itemize}
