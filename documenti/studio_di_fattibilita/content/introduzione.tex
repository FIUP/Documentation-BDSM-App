% =================================================================================================
% File:			introduzione.tex
% Description:	Definisce la sezione relativa all'introduzione
% Created:		2014/12/11
% Author:		Cusinato Giacomo
% Email:		cusinato.giacomo@mashup-unipd.it
% =================================================================================================
% Modification  History:
% Version		Modifier Date		Change											Author
% 0.0.1 		2014/12/11			Iniziata stesura introduzione					Giacomo C.
% =================================================================================================
%

% CONTENUTO DEL CAPITOLO

\section{Introduzione}

\subsection{Scopo del documento}
Il presente documento mira ad esporre le motivazioni che hanno portato il gruppo a scegliere il capitolato C1 ed un'analisi dei restanti capitolati d'appalto.

\subsection{Capitolato scelto}
\begin{itemize}
\item Capitolato: Big Data Social Monitoring App (BDSMApp)
\item Proponente: Zing s.r.l. (\url{http://www.zing-store.com/})
\item Committente: prof. Tullio Verdanega
\end{itemize}

\subsection{Glossario}
Al fine di evitare ogni ambiguità relativa al linguaggio e ai termini utilizzati nei documenti formali, viene allegato il “Glossario v1 ”. In questo documento vengono definiti e descritti tutti i termini con un significato non comune. Per rendere più facile la lettura, i termini saranno posti in corsivo e accanto a questi ci sarà una ‘g’, compresa tra
parentesi quadre, a pedice (esempio: Glossario \ped{[g]}).

\subsection{Riferimenti}

\subsubsection{Normativi}
\begin{itemize}
\item Norme di Progetto: “Norme di Progetto v1”
\item Capitolato d’appalto C1: BDSMApp: Big Data Social Monitoring App \url{http://www.math.unipd.it/~tullio/IS-1/2014/Progetto/C1.pdf}
\end{itemize}
\subsubsection{Capitolati}
\begin{itemize}
\item Capitolato C1: \url{"http://www.math.unipd.it/~tullio/IS-1/2014/Progetto/C1.pdf"}
\item Capitolato C2: \url{"http://www.math.unipd.it/~tullio/IS-1/2014/Progetto/C2.pdf"}
\item Capitolato C3: \url{"http://www.math.unipd.it/~tullio/IS-1/2014/Progetto/C3.pdf"}
\item Capitolato C4: \url{"http://www.math.unipd.it/~tullio/IS-1/2014/Progetto/C4.pdf"}
\item Capitolato C5: \url{"http://www.math.unipd.it/~tullio/IS-1/2014/Progetto/C5.pdf"}
\end{itemize}