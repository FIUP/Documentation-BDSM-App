% =================================================================================================
% File:			altri_capitolati.tex
% Description:	Definisce la sezione relativa ad Altri Capitolati
% Created:		2014/12/13
% Author:		Cusinato Giacomo
% Email:		cusinato.giacomo@mashup-unipd.it
% =================================================================================================
% Modification  History:
% Version		Modifier Date		Change											Author
% 0.0.1 		2014/12/13 			Iniziata stesura capitolo						Giacomo C.
% =================================================================================================
%

% CONTENUTO DEL CAPITOLO
\section{Altri Capitolati}

% CAPITOLATO 2

\subsection{Capitolato C2: GUS}

\subsubsection{Descrizione}
Il capitolato propone lo sviluppo di un software applicabile nell'industria del vetro per il controllo della presenza di impurità e imperfezioni nelle fasi finali del ciclo produttivo del materiale. Il software dovrà analizzare un'immagine del prodotto ottenuta tramite uno scanner momenti prima dell'utima fase di lavorazione e, tramite un'accurata analisi, dovrà stabilire la conformità o meno del vetro analizzato. L'applicativo finale dovrà produrre risultati efficaci al 100\% e fornire le seguenti caratteristiche:
\begin{itemize}
\item \textbf{Interfaccia user friendly}: il prodotto sarà infatti utilizzato da diverse tipologie di utenza.
\item \textbf{Gestione delle ricette}: la tipologia di analisi del vetro può subire variazioni a seconda di diversi fattori, come l'utilizzo finale o il cliente del prodotto ma anche la dimensione della lastra o la forma e posizione dell'impurità trovata. Ogni analisi può quindi essere assegnata ad una ricetta diversa secondo le normative e le richieste dell'ambito.
\item \textbf{Errori e reportistica}: definire due livelli di segnalazioni (errore e difetto).
\item \textbf{Statistiche}: per eseguire un controllo qualità continuo ed una chiara casistica dei possibili difetti presenti in fase di produzione.
\item \textbf{Interfaccia web}: per controllare lo stato della produzione in tempo reale da remoto
\end{itemize} 
Per quanto riguarda le tecnologie in uso, il software principale dovrà essere sviluppato in ambito C++ tramite le librerie Qt mentre il database dovrà essere di tipo relazionale utilizzando i DBMS MySQL o PostgreSQL. Per quanto riguarda l'interfaccia web, invece, è necessario progettare una UI responsive per ogni dispositivo, linguaggi consigliati: PHP e JavaScript (in ambito AngularJS). 

\subsubsection{Valutazione}
Il gruppo ha subito deciso di scartare il secondo capitolato. Sebbene tutti gli studenti del gruppo abbiano una buona familiarità con la maggior parte delle tecnologie in uso, è stato scelto di non scegliere il capitolato in questione per le seguenti motivazioni:
\begin{itemize}
\item \textbf{Poco interesse nel dominio tecnologico}: in particolare, il gruppo ha da subito dimostrato poco interesse sul lato algoritmico del progetto, in particolare sull'analisi di risoluzione del problema con alte prestazioni e percentuali di afficacia.
\item \textbf{Tecnologie già utilizzate}: il gruppo ha preferito evitare di lavorare con tecnologie già utilizzate per semplice interesse personale.
\end{itemize}

% CAPITOLATO 3

\subsection{Capitolato C3: Nor(r)is}

\subsubsection{Descrizione}
Il capitolato propone lo sviluppo di un framework, denominato Norris, per la produzione di grafici in tempo reale a partire da sorgenti arbitrarie. I grafici saranno sviluppati interamente tramite le API messe a disposizione da Norris, sia per la veste grafica che per i dati e dovranno essere disponibili quattro rappresentazioni:
\begin{itemize}
\item \textbf{Bar chart}
\item \textbf{Line chart}
\item \textbf{Map chart}
\item \textbf{Formato tabellare}
\end{itemize}
L'aggiornamento dei grafici, inoltre, potrà avvenire in tre modi: 
\begin{itemize}
\item \textbf{Place}: dove il valore aggiorato sostituisce il precedente.
\item \textbf{Stream}: dove i valori già presenti rimancono visibili, ma ne vengono aggiunti di nuovi.
\item \textbf{Movie}: combina i medoti precedenti, permettendo la sostituzione di dati, l'aggiunta di nuovi valori e la loro rimozione.
\end{itemize}
Il framework metterà dunque a disposizione una componente WebSocket che aggiornerà i grafici in tempo reale, che sarà realizzata tramite la libreria Socket.io. Tra le altre tecnologie, troviamo Node.js ed il relativo framework Express per lo sviluppo lato server, mentre AngularJS per il front-end. Opzionalmente, è inoltre richiesta la progettazione di un'applicazione Android per la visualizzazione dei grafici a partire da un server arbitrario in cui è presente un'istanza di Norris.

\subsubsection{Valutazione}
Il capitolato in questione è stato valutato attentamente dal gruppo ma nella fase finale di analisi, è stato scartato in favore del primo. Di seguito gli aspetti positivi identificati:
\begin{itemize}
\item \textbf{Interesse del dominio tecnologico}: il gruppo ha da subito dimostrato forte interesse per le tecnologie in uso, ritenute innovative e di grande utilità per la formazione personale e lavorativa di ognuno.
\end{itemize}
Di seguito, invece, gli aspetti negativi che hanno portato il gruppo a scartare il capitolato:
\begin{itemize}
\item \textbf{Scarsa conoscenza del dominio tecnologico}: sebbene l'interesse dimostrato per l'apprendimento delle tecnologie in uso, le scarse conoscenze di quest'ultime ha compromesso la decisione finale.
\item \textbf{Grossa mole di lavoro}: il gruppo ha concluso che la mole lavorativa necessaria per lo svolgimento del capitolato avrebbe potuto compromettere l'esito finale.
\end{itemize}

% CAPITOLATO 4

\subsection{Capitolato C4: Premi}

\subsubsection{Descrizione}

\subsubsection{Valutazione}


% CAPITOLATO 5

\subsection{Capitolato C5: sHike}

\subsubsection{Descrizione}

\subsubsection{Valutazione}