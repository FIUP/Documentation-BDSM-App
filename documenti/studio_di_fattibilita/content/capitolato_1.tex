% =================================================================================================
% File:			capitolato_1.tex
% Description:	Defiinisce la sezione relativa al Capitolato 1
% Created:		2014/12/11
% Author:		Cusinato Giacomo
% Email:		cusinato.giacomo@mashup-unipd.it
% =================================================================================================
% Modification History:
% Version		Modifier Date		Change											Author
% 0.0.1 		2014/12/11 			Iniziata stesura C1								Giacomo C.
% =================================================================================================
%

% CONTENUTO DEL CAPITOLO

\section{Capitolato C1: BDSMApp}

\subsection{Descrizione}
Il progetto consiste nel creare un'infrastruttura che interroghi e raccolga informazioni dai big data dei maggiori servizi social, quali Facebook, Twitter ed Instagram. L'acquisizione e la consultazione dei dati da parte dell'utente (sviluppatore) sarà resa disponibile tramite due principali applicativi: un'intefaccia web ed un'infrastuttura basata su servizi di tipo REST interrogabili. L'analisi sulla tipologia dei dati da interrogare è lasciata completamente al gruppo, che dovrà effettuare scelte pertinenti e  in particolar modo coerenti con il servizio che andrà ad offrire l'applicazione. La quantità di informazioni da analizzare e memorizzare, inoltre, è vastissima, sarà quindi necessario progettare un'infrastruttura quanto più scalabile ma anche automatizzata e continua, in modo da offrire un servizio il più possibile aggiornato.

\subsection{Studio del dominio}
\subsubsection{Dominio applicativo}
Il capitolato propone lo sviluppo di un'applicativo software che punti ad unificare informazioni in uscita dai principali social network ed esporle tramite un servizio. Le piattaforme social moderne generano ormai enormi quantitativi di dati e la possibilità di raccogliere un certo tipo di informazioni da un'unico flusso risulta certamente più accessibile. Una prima parte del progetto si basa infatti sull'analisi dei dati provenitenti dalle API dei social, in particolare la comprensione del contesto dei dati da studiare ed archiviare, la frequenza con cui aggiornali e la rilevanza delle informazioni stesse.
L'applicazione quindi andra a fornire un gran set di informazioni consumabili da sviluppatori tramite API esposte da un'architettura REST ed una relativa interfaccia web. 

\subsubsection{Dominio tecnologico}

\subsection{Valutazione}