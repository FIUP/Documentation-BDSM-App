% =================================================================================================
% File:			capitolato_1.tex
% Description:	Defiinisce la sezione relativa al Capitolato 1
% Created:		2014/12/11
% Author:		Cusinato Giacomo
% Email:		cusinato.giacomo@mashup-unipd.it
% =================================================================================================
% Modification History:
% Version		Modifier Date		Change											Author
% 0.0.1 		2014/12/11 			Iniziata stesura C1							    Cusinato Giacomo
% =================================================================================================
% 0.0.2 		2014/12/13 			Completata prima stesura C1						Cusinato Giacomo
% =================================================================================================
% 0.0.3 		2015/01/08 			Revisione capitolo 								Cusinato Giacomo
% =================================================================================================
%

% CONTENUTO DEL CAPITOLO

\section{Capitolato C1: BDSMApp}

\subsection{Descrizione}
Il progetto consiste nel creare un'infrastruttura che interroghi e raccolga informazioni dai big data dei maggiori servizi social, quali Facebook, Twitter ed Instagram. L'acquisizione e la consultazione dei dati da parte dell'utente (sviluppatore) sarà resa disponibile tramite due principali applicativi: un'intefaccia web ed un'infrastuttura basata su servizi di tipo REST interrogabili. L'analisi sulla tipologia dei dati da interrogare è lasciata completamente al gruppo, che dovrà effettuare scelte pertinenti e  in particolar modo coerenti con il servizio che andrà ad offrire l'applicazione. La quantità di informazioni da analizzare e memorizzare, inoltre, è vastissima, sarà quindi necessario progettare un'infrastruttura quanto più scalabile ma anche automatizzata e continua, in modo da offrire un servizio il più possibile aggiornato.

\subsection{Studio del dominio}
Il capitolato propone lo sviluppo di un'applicativo software che punti ad unificare informazioni in uscita dai principali social network ed esporle tramite un servizio. Prefissa approfondimenti di vario tipo, quali l'analisi sul contesto del prodotto, l'aggiornamento e la persistenza di grandi quantitativi di dati e le modalità di esposizione delle informazioni raccolte.
\subsubsection{Dominio applicativo}
Le piattaforme social moderne generano ormai enormi quantitativi di dati e la possibilità di raccogliere un certo tipo di informazioni da un'unico flusso risulta certamente più accessibile. Una prima parte del progetto si basa infatti sull'analisi dei dati provenitenti dalle API dei social, in particolare la comprensione del contesto dei dati da studiare ed archiviare, la frequenza con cui aggiornali e la rilevanza delle informazioni stesse.
L'applicazione quindi andra a fornire un gran set di informazioni consumabili da sviluppatori tramite API esposte da un'architettura REST ed una relativa interfaccia web. 

\subsubsection{Dominio tecnologico}
Per la realizzazione dell'infrastruttura richiesta, il proponente consiglia l'uso della Google Cloud Platform: un'insieme di tool e prodotti che facilitano lo sviluppo di applicativi software nel cloud grazie all'alta scalabilità ed affidabilità proprie dell'infrastruttura disposta da Google. App Engine (incluso appunto nella Google Cloud Platform), permette l'hosting, lo sviluppo, il testing e la gestione di applicazioni che si appoggiano al runtime offerto dalla piattaforma (che supporta diversi linguaggi, quali Java, Python, Go e PHP), senza costi di manutenzione e configurazione associati allo sviluppatore. Per gli applicativi che si appoggiano ad un database, inoltre, Google Cloud Platform mette a disposizione Cloud SQL e Cloud Datatore per basi di dati rispettivamente relazionali e non relazionali. In ambito big data analytics, la piattaforma di Google mette a disposizione BigQuery che consente l'analisi di grandi collezioni di dati ad elevate performance tramite l'utilizzo di servizi REST consumabili da tutti i linguaggi offerti tramite apposite librerie.
La raccolta dei dati viene effettuata grazie alle API esposte dai principali social network e gestite interamente tramite la Google Cloud Platform, tramite cui è possibile determinare le modalità di persistenza ed aggiornamento. 
Per quanto rigurda l'interfaccia web, le tecnologie da utilizzare sono a carico del gruppo, sebbene siano consigliati i classici linguaggi di programmazione web (HTML5, CSS3, JavaScript) e l'utilizzo di un framework resposive come Twitter Bootstrap.
\subsection{Valutazione}
Di seguito sono elencati gli aspetti positivi che hanno portato il gruppo alla scelta del primo capitolato.
\begin{itemize}
\item \textbf{Interesse verso il dominio applicativo}: la maggior parte del gruppo ha ritenuto d'interesse lo studio della Big Data Analisys e gli ambiti ad essa collegati.
\item \textbf{Interesse e conoscenze del dominio tecnologico}: il gruppo ha dimostrato forte interesse per quanto riguarda tutte le tecnologie disponibili per la creazione dell'infrastruttura software e forte propensione all'apprendimento di quelle sconosciute. Inoltre tutti i componenti del gruppo hanno già avuto esperienze con determinate tecnologie web front-end e linguaggi di programmazione proposti dalla Google Cloud Platform.
\item \textbf{Incontro col proponente (?)}
\end{itemize}
Di seguito, invece, gli aspetti negativi identificati:
\begin{itemize}
\item \textbf{Contesto dell'applicazione}: l'unico aspetto negativo rilevato dal gruppo è lo studio del contesto dell'applicazione, il quale è stato lasciato completamente agli studenti e richiederà un lunga analisi.
\end{itemize}
In conclusione, il gruppo ha scelto il primo capitolato con largo consenso da parte di tutti i componenti. L'interesse per le tecnologie in uso e la fattibilità del capitolato stesso hanno determinato l'esclusione dei rimanenti capitolati.





