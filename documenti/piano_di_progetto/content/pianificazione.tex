% =================================================================================================
% File:			pianificazione.tex
% Description:	Defiinisce la sezione relativa alla pianificazione delle attività nelle diverse fasi del ciclo di vita
% Created:		2014/12/30
% Author:		Tesser Paolo
% Email:		tesser.paolo@mashup-unipd.it
% =================================================================================================
% Modification History:
% Version		Modifier Date		Change											Author
% 0.0.1 		2014/12/30 			creata struttura base pianificazione			Tesser Paolo
% =================================================================================================
% 0.0.2			2015/01/02			Definizione periodi fasi e inizio stesura		Tesser Paolo
% =================================================================================================
% 0.0.3			2015/01/07			continuata stesura sulle attività da svolgere	Tesser Paolo
% =================================================================================================
%

% CONTENUTO DEL CAPITOLO

\section{Pianificazione} % (fold)
\label{sec:pianificazione}
I diagrammi delle attività presenti in questa sezione sono stati rappresentati tramite l'uso dei diagrammi di Gaant.
	\subsection{Ricerca e implementazione degli strumenti} % (fold)
	\label{sub:ricerca_e_implementazione_degli_strumenti}
	\textbf{Periodo}: 2014/11/26 - 2014/12/07 \\
	Questa fase comincia con la presentazione delle regole e delle scadenze da parte del committente e termina con un giorno scelto dal \roleProjectManager{} nel quale le scelte principali sugli strumenti e la loro implementazione sono state ritenute sufficienti per proseguire alla fase seguente. \\
	Le attività che verranno svolte sono:
		\begin{itemize}
			\item \textbf{Individualizzazione degli strumenti}: TO DO;
			\item \textbf{Apprendimento degli strumenti}: TO DO;
			\item \textbf{Implementazione degli strumenti}: TO DO.
		\end{itemize}
	\noindent
	Nonostante la terminazione di questa fase sia fissata in un giorno preciso, la ricerca di strumenti più efficienti e migliorativi per l'avanzamento deve essere sempre portata avanti da parte dell'\roleAdministrator.
	 
		\subsubsection{Diagramma delle attività} % (fold)
		\label{ssub:diagramma_delle_attivita}
		TO DO
		% subsubsection diagramma_delle_attività (end)
	% subsection ricerca_e_implementazione_degli_strumenti (end)
	
	\subsection{Analisi dei requisiti} % (fold)
	\label{sub:analisi_dei_requisiti}
	\textbf{Periodo}: 2014/12/03 - 2014/01/22 \\
	Questa fase comincia dopo la scelta e l'implementazione degli strumenti necessari a definire il repository utilizzato dal team e le strutture di base per redigere la documentazione. Terminerà con la scadenza di consegna dell'offerta, cioè con la consegna della \RR. \\
	Le attività che verranno svolte sono:
		\begin{itemize}
			\item \textbf{Norme di Progetto}: TO DO;
			\item \textbf{Studio di Fattibilità}: TO DO;
			\item \textbf{Analisi dei Requisiti}: TO DO;
			\item \textbf{Piano di Progetto}: TO DO;
			\item \textbf{Piano di Qualifica}: TO DO;
			\item \textbf{Glossario}: TO DO;
		\end{itemize}
	
		\subsubsection{Diagramma delle attività} % (fold)
		\label{ssub:diagramma_delle_attivita}
		TO DO
		% subsubsection diagramma_delle_attività (end)
	% subsection analisi_dei_requisiti (end)
	
	
	\subsection{Analisi di dettaglio} % (fold)
	\label{sub:analisi_di_dettaglio}
	\textbf{Periodo}: 2015/01/25 - 2015/02/15 \\
	Questa fase comincia qualche giorno dopo il termine della consegna della \RR{} e termina con la \RR{} stessa.
	Le attività che verranno svolte sono:
		\begin{itemize}
			\item \textbf{Analisi di dettaglio}: TO DO;
			\item \textbf{Incremento e Verifica}: TO CHANGE - i documenti già redatti verranno aggiornati e migliorati a seconda del risultato ottenuto nella \RR.
		\end{itemize}
	
		\subsubsection{Diagramma delle attività} % (fold)
		\label{ssub:diagramma_delle_attivita}
		TO DO
		% subsubsection diagramma_delle_attività (end)
	% subsection analisi_di_dettaglio (end)
	
	\subsection{Progettazione architetturale} % (fold)
	\label{sub:progettazione_architetturale}
	\textbf{Periodo}:  2015/02/10 - RP min (-1) \\
	Questa fase comincia qualche giorno prima della \RR{} e termina con la consegna della \RPmin.
	Le attività che verranno svolte sono:
		\begin{itemize}
			\item \textbf{Specifica Tecnica}: il \roleDesigner{} ha il compito di fornire delle scelte progettuali di alto livello del prodotto finale. Il tutto verrà redatto nel documento \docNameST{} \emph{v.1.0.0} TO DO;
			\item \textbf{Incontro con il proponente}: TO DO;
			\item \textbf{Incremento e Verifica}: i documenti già redatti verranno aggiornati e migliorati a seconda del risultato ottenuto nella \RR.
		\end{itemize}
		
		\subsubsection{Diagramma delle attività} % (fold)
		\label{ssub:diagramma_delle_attivita}
		TO DO
		% subsubsection diagramma_delle_attività (end)
	% subsection progettazione_architetturale (end)
	
	\subsection{Progettazione di dettaglio e codifica dei requisiti obbligatori} % (fold)
	\label{sub:progettazione_di_dettaglio_e_codifica_dei_requisiti_obbligatori}
	\textbf{Periodo}:  RP min (+1/2) - step 1 (da decidere) \\
	Questa fase comincia con la \RPmin{} e termina con un incontro con il proponente per ricevere un feedback sul lavoro svolto per soddisfare i requisiti obbligatori trovati.
	Le attività che verranno svolte sono:
		\begin{itemize}
			\item \textbf{Definizione di Prodotto}: TO DO;
			\item \textbf{Manuale utente e amministratore}: TO DO;
			\item \textbf{Incremento e Verifica}: i documenti già redatti verranno aggiornati e migliorati a seconda del riscontro ottenuto dall'incontro con il proponente.
		\end{itemize}
		
		\subsubsection{Diagramma delle attività} % (fold)
		\label{ssub:diagramma_delle_attivita}
		TO DO
		% subsubsection diagramma_delle_attività (end)
	% subsection progettazione_di_dettaglio_e_codifica_dei_requisiti_obbligatori (end)
	
	\subsection{Progettazione di dettaglio e codifica dei requisiti desiderabili} % (fold)
	\label{sub:progettazione_di_dettaglio_e_codifica_dei_requisiti_desiderabili}
	\textbf{Periodo}:  step 1 (da decidere) - step 2 (da decidere) \\
	Questa fase comincia dopo l'incontro con il proponente e termina con un altro incontro con il proponente per ricevere un feedback sul lavoro svolto per soddisfare i requisiti desiderabili trovati.
	Le attività che verranno svolte sono:
		\begin{itemize}
			\item \textbf{Incremento e Verifica}: i documenti già redatti verranno aggiornati e migliorati a seconda del riscontro ottenuto dall'incontro con il proponente.
		\end{itemize}
		
		\subsubsection{Diagramma delle attività} % (fold)
		\label{ssub:diagramma_delle_attivita}
		TO DO
		% subsubsection diagramma_delle_attività (end)
	% subsection progettazione_di_dettaglio_e_codifica_dei_requisiti_desiderabili (end)
	
	\subsection{Progettazione di dettaglio e codifica dei requisiti opzionali} % (fold)
	\label{sub:progettazione_di_dettaglio_e_codifica_dei_requisiti_opzionali}
	\textbf{Periodo}:  step 2 (da decidere) - 2015/05/29 (mettere data della consegna) \\
	Questa fase comincia dopo l'incontro con il proponente per la valutazione sui requisiti desiderabili e termina con la consegna della \RQ.
	Le attività che verranno svolte sono:
		\begin{itemize}
			\item \textbf{Incremento e Verifica}: TO DO.
		\end{itemize}
			
		\subsubsection{Diagramma delle attività} % (fold)
		\label{ssub:diagramma_delle_attivita}
		TO DO
		% subsubsection diagramma_delle_attività (end)
	% subsection progettazione_di_dettaglio_e_codifica_dei_requisiti_opzionali (end)
	
	\subsection{Validazione} % (fold)
	\label{sub:validazione}
	\textbf{Periodo}:  step 2 (da definire) - 2015/06/18 \\
	Questa fase comincia dopo la consegna della \RQ{} e termina con la consegna della \RA.
	Le attività che verranno svolte sono:
		\begin{itemize}
			\item \textbf{Incremento e Verifica}: TO DO;
			\item \textbf{Validazione}: TO DO;
		\end{itemize}
		
		\subsubsection{Diagramma delle attività} % (fold)
		\label{ssub:diagramma_delle_attivita}
		TO DO
		% subsubsection diagramma_delle_attività (end)
	% subsection validazione (end)
	
	
% section pianificazione (end)