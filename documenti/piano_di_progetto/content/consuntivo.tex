% =================================================================================================
% File:			consuntivo.tex
% Description:	Defiinisce la sezione relativa al consuntivo a finire dei costi
% Created:		2015-02-19
% Author:		Santacatterina Luca
% Email:		santacatterina.luca@mashup-unipd.it
% =================================================================================================
% Modification History:
% Version		Modifier Date		Change											Author
% 0.0.1 		2015-02-19 			creata struttura del consuntivo					Santacatterina Luca
% =================================================================================================
% 0.0.2			2015-02-20			aggiunta sotto sezione preventivo a finire		Santacatterina Luca
% =================================================================================================
% 2.0.0			2015-03-17			integrato sotto sezione preventivo a finire		Santacatterina Luca
% =================================================================================================

% CONTENUTO DEL CAPITOLO

\section{Consuntivo} % (fold)
\label{sec:consuntivo}
Verranno indicate di seguito le spese effettivamente sostenute, relative alle ore rendicontate. Quanto indicato è da attribuire come costi al proponente. \newline
Sarà infine presentato un bilancio riguardante che potrà essere:
	\begin{itemize}
		\item \textbf{Positivo}: se il preventivo supera il consuntivo;
		\item \textbf{Negativo}: se il consuntivo supera il preventivo;
		\item \textbf{In pari}: se consuntivo e preventivo sono uguali.
	\end{itemize}

	\subsection{Progettazione architetturale} % (fold)
	\label{sub:progettazione_architetturale}
		Di seguito di riporta il consuntivo per la fase di Progettazione Architetturale.\\
		Nella seguente tabella sono riportate le ore realmente impiegate e le spese effettivamente sostenute per ogni ruolo. Tra parentesi si evidenzia la differenza tra preventivo e consuntivo.
		\begin{table}[!h]
			\begin{center}
				\begin{tabularx}{0.90\textwidth}{|X|c|c|}
					\hline
					\textbf{Ruolo} & \textbf{Ore} & \textbf{Costo} \\
					\hline
					\roleProjectManager & -1 & \euro{} -30,00 \\
					\hline
					\roleAnalyst & +11 &  \euro{} +275,00 \\
					\hline
					\roleDesigner & +3 & \euro{} +66,00 \\
					\hline
					\roleAdministrator & -1  & \euro{} -20,00 \\
					\hline
					\roleProgrammer & 0 & \euro{} 0,00 \\
					\hline
					\roleVerifier & +1 & \euro{} +15,00 \\
					\hline
					\textbf{Differenza consuntivo/preventivo} & \textbf{+13} & \textbf{+\euro{} 306,00} \\
					\hline
				\end{tabularx}
			\end{center}
		\caption{Ore rendicontate - differenza consuntivo/preventivo}
		\end{table}
		% subsubsection consuntivo (end)


		\subsubsection{Riepilogo} % (fold)
		\label{ssub:riepilogo}
		In questa fase sono state impiegate 13 ore in più rispetto a quanto preventivato. \newline
		Questo è stato causato da una compensazione delle ore di lavoro che si sono dovute recuperare dalla precedente fase, riguardanti in particolare modo le attività di analisi. \newline
		Il consuntivo perciò è maggiore rispetto a quanto preventivato di \euro{} 306,00, portando quindi il bilancio in negativo.

		% subsubsection riepilogo (end)

		\subsubsection{Preventivo a finire} % (fold)
		\label{ssub:preventivo_a_finire}
		Da quanto detto precedentemente i costi sono aumentati rispetto a quanto preventivato. A tal proposito però, vengono utilizzati i soldi totali risparmiati nelle precedenti fasi (\euro{} + 345,00) per far si che il preventivo finale per il proponente rimanga invariato.


		% subsubsection preventivo_a_finire (end)
	% subsection progettazione_architetturale (end)

% section consuntivo (end)