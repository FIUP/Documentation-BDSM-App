% =================================================================================================
% File:			consuntivo.tex
% Description:	Defiinisce la sezione relativa al consuntivo a finire dei costi
% Created:		2015-02-19
% Author:		Santacatterina Luca
% Email:		santacatterina.luca@mashup-unipd.it
% =================================================================================================
% Modification History:
% Version		Modifier Date		Change											Author
% 0.0.1 		2015-02-19 			creata struttura del consuntivo					Santacatterina Luca
% =================================================================================================
% 0.0.2			2015-02-20			aggiunta sotto sezione preventivo a finire		Santacatterina Luca
% =================================================================================================
%

% CONTENUTO DEL CAPITOLO

\section{Consuntivo} % (fold)
\label{sec:consuntivo}
Verranno indicate di seguito le spese effettivamente sostenute, relative alle ore rendicontate. Quanto indicato è da attribuire come costi al proponente. \newline
Sarà infine presentato un bilancio riguardante che potrà essere:
	\begin{itemize}
		\item \textbf{Positivo}: se il preventivo supera il consuntivo;
		\item \textbf{Negativo}: se il consuntivo supera il preventivo;
		\item \textbf{In pari}: se consuntivo e preventivo sono uguali.
	\end{itemize}

	\subsection{Progettazione architetturale} % (fold)
	\label{sub:progettazione_architetturale}
	TO DO (prendere tabella e dicitura della parte relativa al consuntivo interno, apportandone le relative modifiche)
		\subsubsection{Consuntivo} % (fold)
		\label{ssub:consuntivo}
		TO DO
		% subsubsection consuntivo (end)

		\subsubsection{Riepilogo} % (fold)
		\label{ssub:riepilogo}
		TO DO
		% subsubsection riepilogo (end)

		\subsubsection{Preventivo a finire} % (fold)
		\label{ssub:preventivo_a_finire}
		TO DO
		% subsubsection preventivo_a_finire (end)
	% subsection progettazione_architetturale (end)

% section consuntivo (end)