% =================================================================================================
% File:			consuntivo.tex
% Description:	Defiinisce la sezione relativa al consuntivo a finire dei costi
% Created:		2015-02-19
% Author:		Santacatterina Luca
% Email:		santacatterina.luca@mashup-unipd.it
% =================================================================================================
% Modification History:
% Version		Modifier Date		Change											Author
% 0.0.1 		2015-02-19 			creata struttura del consuntivo					Santacatterina Luca
% =================================================================================================
% 0.0.2			2015-02-20			aggiunta sotto sezione preventivo a finire		Santacatterina Luca
% =================================================================================================
% 2.0.0			2015-03-17			integrato sotto sezione preventivo a finire		Santacatterina Luca
% =================================================================================================

% CONTENUTO DEL CAPITOLO

\section{Consuntivo} % (fold)
\label{sec:consuntivo}
Verranno indicate di seguito le spese effettivamente sostenute, relative alle ore rendicontate. Quanto indicato è da attribuire come costi al proponente. \newline
Sarà infine presentato un bilancio riguardante che potrà essere:
	\begin{itemize}
		\item \textbf{Positivo}: se il preventivo supera il consuntivo;
		\item \textbf{Negativo}: se il consuntivo supera il preventivo;
		\item \textbf{In pari}: se consuntivo e preventivo sono uguali.
	\end{itemize}

	\subsection{Progettazione architetturale} % (fold)
	\label{sub:progettazione_architetturale}
		Di seguito di riporta il consuntivo per la fase di Progettazione Architetturale.\\
		Nella seguente tabella sono riportate le ore realmente impiegate e le spese effettivamente sostenute per ogni ruolo. Tra parentesi si evidenzia la differenza tra preventivo e consuntivo.
		\begin{table}[!h]
			\begin{center}
				\begin{tabularx}{0.90\textwidth}{|X|c|c|}
					\hline
					\textbf{Ruolo} & \textbf{Ore} & \textbf{Costo} \\
					\hline
					\roleProjectManager & 9 (+1) & \euro{} 270,00 (+30,00) \\
					\hline
					\roleAnalyst & 43 (+5) &  \euro{} 1075,00 (+125,00) \\
					\hline
					\roleDesigner & 86 (+3) & \euro{} 1892,00 (+66,00) \\
					\hline
					\roleAdministrator & 7 (-1)  & \euro{} 140,00 (-20,00) \\
					\hline
					\roleProgrammer & 0 (0) & \euro{} 0,00 (0,00) \\
					\hline
					\roleVerifier & 45 (+1) & \euro{} 675,00 (+15,00) \\
					\hline
					\textbf{Differenza preventivo/consuntivo} & \textbf{+9} & \textbf{+\euro{} 216,00} \\
					\hline
				\end{tabularx}
			\end{center}
		\caption{Ore rendicontate - differenza preventivo/consuntivo}
		\end{table}
		% subsubsection consuntivo (end)


		\subsubsection{Riepilogo} % (fold)
		\label{ssub:riepilogo}
		Nelle fasi di analisi e progettazione sono state impiegate più ore rispetto a quanto preventivato.\\
		Questo rallentamento ha portato ad un aumento del bilancio, ma analizzando il problema ha permesso così al gruppo di implementare una \textbf{maggiore definizione del lavoro svolto} e riducendo la difficoltà successivamente.\\
		Si potrà così prevedere un incremento prestazionale nella fase di dettaglio, portando i componenti del gruppo ad essere più veloci e quindi meno costosi successivamente.\\
		L'aumento corrisponde alla cifra di \textbf{\euro{} 216,00}.\\
		Il bilancio attuale si chiude in negativo ma si prevede una maggiore efficienza nella fasi successive.
		% subsubsection riepilogo (end)

		\subsubsection{Preventivo a finire} % (fold)
		\label{ssub:preventivo_a_finire}
		Da quanto sopra descritto si dovrà \textbf{aumentare il ritmo di lavoro} per poter rientrare dei \textbf{\euro{} 216,00}.\\
		Non potendo aumentare le ore di lavoro si dovrà ottimizzare il lavoro da svolgere. Si richiede un maggiore sforzo a tutti i membri del gruppo per riallineare il preventivo con il consuntivo.
		% subsubsection preventivo_a_finire (end)
	% subsection progettazione_architetturale (end)

% section consuntivo (end)