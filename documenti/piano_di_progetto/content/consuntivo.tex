% =================================================================================================
% File:			consuntivo.tex
% Description:	Defiinisce la sezione relativa al consuntivo a finire dei costi
% Created:		2015-02-19
% Author:		Santacatterina Luca
% Email:		santacatterina.luca@mashup-unipd.it
% =================================================================================================
% Modification History:
% Version		Modifier Date		Change											Author
% 0.0.1 		2015-02-19 			creata struttura del consuntivo					Santacatterina Luca
% =================================================================================================
% 0.0.2			2015-02-20			aggiunta sotto sezione preventivo a finire		Santacatterina Luca
% =================================================================================================
% 2.0.0			2015-03-17			integrato sotto sezione preventivo a finire		Santacatterina Luca
% =================================================================================================
% 3.0.1			2015-05-11			migliorata sezione preventifo a finire di PA	Tesser Paolo
% =================================================================================================
% 3.0.2			2015-05-11			aggiunto scheletro consuntivo per PdDO			Tesser Paolo
% =================================================================================================
%

% CONTENUTO DEL CAPITOLO

\section{Consuntivo e preventivo a finire} % (fold)
\label{sec:consuntivo}
Verranno indicate di seguito le spese effettivamente sostenute, relative alle ore rendicontate. Quanto indicato è da attribuire come costi al proponente. \newline
Sarà infine presentato un bilancio riguardante che potrà essere:
	\begin{itemize}
		\item \textbf{Positivo}: se il preventivo supera il consuntivo;
		\item \textbf{Negativo}: se il consuntivo supera il preventivo;
		\item \textbf{In pari}: se consuntivo e preventivo sono uguali.
	\end{itemize}




	\subsection{Progettazione architetturale} % (fold)
	\label{sub:consuntivo_progettazione_architetturale}
		Di seguito di riporta il consuntivo di periodo per la fase di Progettazione Architetturale. \newline
		Nella seguente tabella sono riportate le ore realmente impiegate e le spese effettivamente sostenute per ogni ruolo. Tra parentesi si evidenzia la differenza tra preventivo e consuntivo.
		\begin{table}[!h]
			\begin{center}
				\begin{tabularx}{0.90\textwidth}{|X|c|c|}
					\hline
					\textbf{Ruolo} & \textbf{Ore} & \textbf{Costo} \\
					\hline
					\roleProjectManager & -1 & \euro{} -30,00 \\
					\hline
					\roleAnalyst & +11 &  \euro{} +275,00 \\
					\hline
					\roleDesigner & +3 & \euro{} +66,00 \\
					\hline
					\roleAdministrator & -1  & \euro{} -20,00 \\
					\hline
					\roleProgrammer & 0 & \euro{} 0,00 \\
					\hline
					\roleVerifier & +1 & \euro{} +15,00 \\
					\hline
					\textbf{Differenza consuntivo/preventivo} & \textbf{+13} & \textbf{+\euro{} 306,00} \\
					\hline
				\end{tabularx}
			\end{center}
		\caption{Ore rendicontate - differenza consuntivo/preventivo}
		\end{table}
		% subsubsection consuntivo (end)


		\subsubsection{Riepilogo} % (fold)
		In questa fase sono state impiegate 13 ore in più rispetto a quanto preventivato. \newline
		Questo è stato causato da una compensazione delle ore di lavoro che si sono dovute recuperare dalla precedente fase, riguardanti in particolare modo le attività di analisi. \newline
		Il consuntivo perciò è maggiore rispetto a quanto preventivato di \euro{} 306,00, portando quindi il bilancio in negativo.

		% subsubsection riepilogo (end)

		\subsubsection{Preventivo a finire} % (fold)
		Da quanto detto precedentemente i costi sono aumentati rispetto a quanto preventivato. A tal proposito però, vengono utilizzati i soldi totali risparmiati nelle precedenti fasi (\euro{} + 345,00) per far si che il preventivo finale per il proponente rimanga invariato. \newline
		Per compensare questo squilibrio e lo sforamento nei tempi verificatosi, verranno ripianificate le successive fasi all'inizio di quella di \textbf{Progettazione di dettaglio e codifica dei requisiti obbligatori}, andando a diminuire le ore predisposte inizialmente agli \emph{Analisti} e al \roleProjectManager.


		% subsubsection preventivo_a_finire (end)
	% subsection progettazione_architetturale (end)

	\subsection{Progettazione di dettaglio e codifica dei requisiti obbligatori} % (fold)
	\label{sub:consuntivo_progettazione_di_dettaglio_e_codifica_dei_requisiti_obbligatori}
		Di seguito di riporta il consuntivo di periodo per la fase di Progettazione di dettaglio e codifica dei requisiti obbligatori. \newline
		Nella seguente tabella sono riportate le ore realmente impiegate e le spese effettivamente sostenute per ogni ruolo. Tra parentesi si evidenzia la differenza tra preventivo e consuntivo.
		\begin{table}[!h]
			\begin{center}
				\begin{tabularx}{0.90\textwidth}{|X|c|c|}
					\hline
					\textbf{Ruolo} & \textbf{Ore} & \textbf{Costo} \\
					\hline
					\roleProjectManager & -1 & \euro{} -30,00 \\
					\hline
					\roleAnalyst & -10 &  \euro{} -250,00 \\
					\hline
					\roleDesigner & -2 & \euro{} -44,00 \\
					\hline
					\roleAdministrator & 0  & \euro{} 0,00 \\
					\hline
					\roleProgrammer & +3 & \euro{} +60,00 \\
					\hline
					\roleVerifier & +1 & \euro{} +15,00 \\
					\hline
					\textbf{Differenza consuntivo/preventivo} & \textbf{-9} & \textbf{-\euro{} 249,00} \\
					\hline
				\end{tabularx}
			\end{center}
		\caption{Ore rendicontate - differenza consuntivo/preventivo}
		\end{table}
		% subsubsection consuntivo (end)

		\subsubsection{Riepilogo} % (fold)
		In questa fase sono state impiegate 9 ore in meno a quanto preventivato. \newline
		Questo è stato dovuto da un risparmio sostanziale nel ruolo di \roleAnalyst{} dovuto a un forte impiego durante la fase precedente che ci ha consentito di arrivare ad un buon grado per quanto riguarda il documento di \docNameVersionAdR. \newline
		Il consuntivo perciò è minore rispetto a quanto preventivato di \euro{} 349,00.

		% subsubsection riepilogo (end)

		\subsubsection{Preventivo a finire} % (fold)
		Da quanto detto precedentemente i costi sono diminuiti rispetto a quanto preventivato. Questo ci consente di trattenere quelli avanzati per fasi successive in cui potrebbe esserci una necessità maggiore di fondi, non andando quindi a diminuire quanto stabilito inizialmente con il proponente. \newline
		Questo avanzo per il momento ci consente di non dovere ripianificare le attività già pianificate per la fase seguente.



		\subsection{Progettazione di dettaglio e codifica dei requisiti desiderabili} % (fold)
		\label{sub:consuntivo_progettazione_di_dettaglio_e_codifica_dei_requisiti_desiderabili}
			Di seguito di riporta il consuntivo di periodo per la fase di Progettazione di dettaglio e codifica dei requisiti desiderabili. \newline
			Nella seguente tabella sono riportate le ore realmente impiegate e le spese effettivamente sostenute per ogni ruolo. Tra parentesi si evidenzia la differenza tra preventivo e consuntivo.
			\begin{table}[!h]
				\begin{center}
					\begin{tabularx}{0.90\textwidth}{|X|c|c|}
						\hline
						\textbf{Ruolo} & \textbf{Ore} & \textbf{Costo} \\
						\hline
						\roleProjectManager & -4 & \euro{} -120,00 \\
						\hline
						\roleAnalyst & 0 &  \euro{} 0,00 \\
						\hline
						\roleDesigner & +2 & \euro{} +44,00 \\
						\hline
						\roleAdministrator & -2  & \euro{} -40,00 \\
						\hline
						\roleProgrammer & +8 & \euro{} +120,00 \\
						\hline
						\roleVerifier & +3 & \euro{} +45,00 \\
						\hline
						\textbf{Differenza consuntivo/preventivo} & \textbf{+7} & \textbf{+\euro{} 49,00} \\
						\hline
					\end{tabularx}
				\end{center}
			\caption{Ore rendicontate - differenza consuntivo/preventivo}
			\end{table}
			% subsubsection consuntivo (end)

			\subsubsection{Riepilogo} % (fold)
			In questa fase sono state impiegate 7 ore in più rispetto a quanto preventivato. \newline
			Questo è stato dovuto ad un maggiore impiego dei programmatori per sopperire a parti del programma non ancora pienamente complete e codificate. \newline
			Il consuntivo perciò è maggiore rispetto a quanto preventivato di \euro{} 49,00.

			% subsubsection riepilogo (end)

			\subsubsection{Preventivo a finire} % (fold)
			Da quanto detto precedentemente i costi sono aumentati rispetto a quanto preventivato. Il buon avanzo però derivante dalla fase precedente ci consente di far si che il preventivo finale per il proponente rimanga invariato.
			Questo scompenso ci obbliga a rivedere parte della pianificazione della successiva fase che avverrà durante il periodo iniziale.

% section consuntivo (end)