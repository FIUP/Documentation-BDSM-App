% =================================================================================================
% File:			introduzione.tex
% Description:	Defiinisce la sezione relativa all'introduzione del documento
% Created:		2014/12/17
% Author:		Tesser Paolo
% Email:		tesser.paolo@mashup-unipd.it
% =================================================================================================
% Modification History:
% Version		Modifier Date		Change											Author
% 0.0.1 		2014/12/18 			iniziata stesura capitolo introduzione			Tesser Paolo
% =================================================================================================
% 0.0.2			2014/12/19			stesura sezione sul ciclo di vita				Tesser Paolo
% =================================================================================================
% 0.0.3			2014/12/30			sistemato cap ciclo vita e indentazione			Tesser Paolo
% =================================================================================================
%

% CONTENUTO DEL CAPITOLO

\section{Introduzione}
	\subsection{Scopo del documento}
	Questo documento pianificherà il modo in cui verranno svolte le attività dai membri del gruppo \groupName.
	Gli scopi di questo documento sono:
		\begin{itemize}
			\item presentare l’organizzazione dei tempi e delle attività;
			\item analizzare i possibili rischi e gestire le contromisure necessarie per limitarli;
			\item preventivare l’impiego delle risorse;
			\item calcolare il consuntivo di utilizzo delle risorse durante lo svolgimento del progetto.
		\end{itemize}
		
	\subsection{Scopo del Prodotto}
		\productScope
		
	\subsection{Glossario}
		\glossarioDesc
		
	\subsection{Riferimenti}
		\subsubsection{Normativi}
			\begin{itemize}
				\item \textbf{Norme di Progetto}: \docNameVersionNdP
			\end{itemize}	
		\subsubsection{Informativi}
			\begin{itemize}
				\item \textbf{Analisi dei Requisiti}: \docNameVersionAdR;
				\item \textbf{Piano di Qualifica}: \docNameVersionPdQ;
				\item \textbf{Studio di Fattibilità}: \docNameVersionSdF;
				\item \textbf{Verbale Interno}: \emph{Verbale Interno del 2014/12/02};
				\item \textbf{\sommerville}:
					\begin{itemize}
						\item Part 4 - Software Management.					
					\end{itemize}
			\end{itemize}
			
	\subsection{Ciclo di vita}
	Il modello di ciclo di vita adottato per il prodotto è il modello incrementale. \\
	Questa scelta consente di sviluppare quanto richiesto in diversi periodi, che verranno segnati da una milestone. \\
	Grazie anche al rapporto che il proponente ha mostrato di volere instaurare è possibile suddividere il lavoro in fasi di breve durata, permettendo così un al termine di ognuna di ricevere un feedback sull'andamento del sistema.
		\begin{itemize}
			\item \textbf{Ricerca e implementazione degli strumenti}: in questa prima fase si ricercano gli strumenti necessari al lavoro collaborativo con maggiore attenzione per quelli che serviranno ad assegnare i diversi compiti ai componenti, a redigere la documentazione e a gestire lo spazio dove andare a versionare il lavoro che si svolgerà. \\
			Si inizieranno a sviluppare inoltre degli script che permettano di rendere il più possibile automatizzate le norme introdotte nel documento \docNameVersionNdP. \\
			In questo primo momento si andrà a scegliere quale capitolato il gruppo andrà a sviluppare;

			\item \textbf{Analisi dei requisiti}: dopo aver preso la decisione sul capitolato da eseguire si potranno iniziare le attività necessarie alla pianificazione e alla qualità che si prefigge di ottenere.
			Agli analisti spetterà il compito di fornire un'analisi dei requisiti adeguata.\\
			Questa fase terminerà con la Revisione dei Requisiti;

			\item \textbf{Analisi di dettaglio}: in seguito alla prima analisi, si procederà con una nuova nella quale verranno inseriti i nuovi requisiti trovati dagli analisti e migliorati o corretti quelli che non trovavano un pieno riscontro con le richieste nel proponente nella versione precedente;
			
			\item \textbf{Progettazione architetturale}: successivamente si andrà ad individuare una soluzione generale ad alto livello che soddisfi i requisiti richiesti. Questa fase terminerà con la Revisione di Progettazione minima e la presentazione del documento di Specifica Tecnica;

			\item \textbf{Progettazione di dettaglio e codifica dei requisiti obbligatori}: in questa fase si procede con la progettazione di dettaglio e la codifica dei requisiti obbligatori. Questa fase terminerà con un incontro con il proponente al quale verrà presentato un applicativo che soddisfi i requisiti obbligatori;
			
			\item \textbf{Progettazione di dettaglio e codifica dei requisiti desiderabili}: in questa fase si procede con la progettazione di dettaglio e la codifica dei requisiti desiderabili. Questa fase terminerà con un incontro con il proponente al quale verrà presentato un applicativo che soddisfi i requisiti desiderabili;
			
			\item \textbf{Progettazione di dettaglio e codifica dei requisiti opzionali}: in questa fase si procede con la progettazione di dettaglio e la codifica dei requisiti obbligatori. Questa fase terminerà con la Revisione di Qualifica;
			
			\item \textbf{Validazione}: il progetto raggiungerà il termine in tale fase, in corrispondenza della Revisione di Accettazione. Verrà effettuata l’attività di validazione del software e successivamente si procederà con il collaudo dello stesso.
		\end{itemize}
\noindent
	Una forte suddivisione delle fasi permette di avere un maggiore controllo sull'andamento del lavoro e di poter applicare più facilmente il PDCA. \\
	Se la fase di progettazione di dettaglio e codifica dei requisiti obbligatori dovesse subire dei ritardi e richiedere più tempo del previsto è possibile che le successive due fasi possano non avviarsi.

	\subsection{Scadenze}
	Di seguito vengono presentate le scadenze che il gruppo \groupName{} ha deciso di rispettare e sulle quali verranno pianificate le attività da svolgere.
		\begin{itemize}
			\item \textbf{Revisione dei Requisiti (RR)}: 2015/01/23;
			\item \textbf{Revisione di Progettazione (RP)}: RP min (TO DO - definire la data che non trovo);
			\item \textbf{Revisione di Qualifica (RQ)}: 2015/05/29 (da definire se andare alla RQ in data della RP max);
			\item \textbf{Revisione di Accettazione (RA)}: 2015/06/18 (da definire).
		\end{itemize}