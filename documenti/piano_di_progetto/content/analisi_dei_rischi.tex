% =================================================================================================
% File:			analisi_dei_rischi.tex
% Description:	Defiinisce la sezione relativa all'analisi dei rischi del progetto
% Created:		2014/12/30
% Author:		Tesser Paolo
% Email:		tesser.paolo@mashup-unipd.it
% =================================================================================================
% Modification History:
% Version		Modifier Date		Change											Author
% 0.0.1 		2014/12/30 			creata struttura analisi dei rischi				Tesser Paolo
% =================================================================================================
% 0.0.2			2015/01/03			iniziata prima stesura							Tesser Paolo
% =================================================================================================
%
% =================================================================================================
%

% CONTENUTO DEL CAPITOLO

\section{Analisi dei rischi} % (fold)
\label{sec:analisi_dei_rischi}
Per gestire al meglio l'avanzamento del progetto, incorrendo il meno possibile in problematica di natura varia, si è effettuata un attenta analisi dei rischi. La procedura seguente viene utilizzata per gestire i rischi.
	\begin{enumerate}
		\item \textbf{Identificazione}: identificare tutti i possibili rischi che potrebbero ostacolare il corretto avanzamento del progetto. \\
		I rischi possono essere: di progetto, di prodotto e business;
		\item \textbf{Analisi}: valutare la probabilità di occorrenza del rischio e valutare i suoi possibili effetti sul progetto;
		\item \textbf{Pianificazione}: redigere un piano su come affrontare i rischi, istituendo metodi per annullare o per mitigare il loro effetto;
		\item \textbf{Monitoraggio}: controllare regolarmente i rischi pianificati e rivisitare il piano qualora si abbiano più informazione sui rischi trovati o ne venissero identificati di nuovi.
	\end{enumerate}
\noindent
Ogni rischio individuato viene descritto con: 
	\begin{itemize}
		\item Nome
		\item Probabilità di occorrenza
		\item Grado di pericolosità
		\item Descrizione
		\item Metodi di rilevazione
		\item Contromisure
	\end{itemize}
	\subsection{Rischi tecnologici} % (fold)
	\label{sub:rischi_tecnologici}
		\subsubsection{Inesperienza sulle tecnologie scelte} % (fold)
		\label{ssub:inesperienza_sulle_tecnlogie_scelte}
			\begin{itemize}
				\item \textbf{Probabilità di occorrenza}: Alta;
				\item \textbf{Grado di pericolosità}: Alto;
				\item \textbf{Descrizione}: TO DO;
				\item \textbf{Metodi di rilevazione}: TO DO;
				\item \textbf{Contromisure}: TO DO.
			\end{itemize}
		% subsubsection inesperienza_sulle_tecnlogie_scelte (end)
		
		\subsubsection{Problemi Hardware} % (fold)
		\label{ssub:problemi_hardware}
			\begin{itemize}
				\item \textbf{Probabilità di occorrenza}: Bassa;
				\item \textbf{Grado di pericolosità}: Medio;
				\item \textbf{Descrizione}: TO DO;
				\item \textbf{Metodi di rilevazione}: TO DO;
				\item \textbf{Contromisure}: TO DO.
			\end{itemize}
		% subsubsection problemi_hardware (end)
		
		\subsubsection{Problemi Software} % (fold)
		\label{ssub:problemi_software}
			\begin{itemize}
				\item \textbf{Probabilità di occorrenza}: Bassa;
				\item \textbf{Grado di pericolosità}: Basso;
				\item \textbf{Descrizione}: TO DO;
				\item \textbf{Metodi di rilevazione}: TO DO;
				\item \textbf{Contromisure}: TO DO.
			\end{itemize}
		% subsubsection problemi_software (end)
		
	% subsection rischi_tecnologici (end)
	
	\subsection{Rischi dovuti al personale} % (fold)
	\label{sub:rischi_dovuti_al_personale}
		\subsubsection{Impegni personali dei componenti} % (fold)
		\label{ssub:impegni_personali_dei_componenti}
			\begin{itemize}
				\item \textbf{Probabilità di occorrenza}: Medio;
				\item \textbf{Grado di pericolosità}: Medio;
				\item \textbf{Descrizione}: TO DO;
				\item \textbf{Metodi di rilevazione}: TO DO;
				\item \textbf{Contromisure}: TO DO.
			\end{itemize}
		% subsubsection impegni_personali_dei_componenti (end)
		
		\subsubsection{Problemi tra i componenti} % (fold)
		\label{ssub:problemi_tra_i_componenti}
			\begin{itemize}
				\item \textbf{Probabilità di occorrenza}: Basso;
				\item \textbf{Grado di pericolosità}: Bassa;
				\item \textbf{Descrizione}: TO DO;
				\item \textbf{Metodi di rilevazione}: TO DO;
				\item \textbf{Contromisure}: TO DO.
			\end{itemize}
		% subsubsection problemi_tra_i_componenti (end)
		
		\subsubsection{Inesperienza sul lavoro di gruppo} % (fold)
		\label{ssub:inesperienza_su_lavoro_di_gruppo}
			\begin{itemize}
				\item \textbf{Probabilità di occorrenza}: Alta;
				\item \textbf{Grado di pericolosità}: Media;
				\item \textbf{Descrizione}: TO DO;
				\item \textbf{Metodi di rilevazione}: TO DO;
				\item \textbf{Contromisure}: TO DO.
			\end{itemize}
		% subsubsection inesperienza_su_lavoro_di_gruppo (end)
		
	% subsection rischi_dovuti_al_personale (end)
	
	\subsection{Rischi organizzativi} % (fold)
	\label{sub:rischi_organizzativi}
	TO DO
	% subsection rischi_organizzativi (end)
	
	
	
	\subsection{Rischi dovuti agli strumenti} % (fold)
	\label{sub:rischi_dovuti_agli_strumenti}
	TO DO
	% subsection rischi_dovuti_agli_strumenti (end)
	
	
	
	\subsection{Rischi dovuti ai requisiti} % (fold)
	\label{sub:rischi_dovuti_ai_requisiti}
		\begin{itemize}
			\item \textbf{Probabilità di occorrenza}: Alta;
			\item \textbf{Grado di pericolosità}: Alto;
			\item \textbf{Descrizione}: TO DO;
			\item \textbf{Metodi di rilevazione}: TO DO;
			\item \textbf{Contromisure}: TO DO.
		\end{itemize}
	% subsection rischi_dovuti_ai_requisiti (end)
	
	
	
	\subsection{Rischi dovuti a stime errate} % (fold)
	\label{sub:rischi_dovuti_a_stime_errate}
		\subsubsection{Tempi di sviluppo sottostimati} % (fold)
		\label{ssub:tempi_di_sviluppo_sottostimati}
			\begin{itemize}
				\item \textbf{Probabilità di occorrenza}: Alta;
				\item \textbf{Grado di pericolosità}: Alto;
				\item \textbf{Descrizione}: TO DO;
				\item \textbf{Metodi di rilevazione}: TO DO;
				\item \textbf{Contromisure}: TO DO.
			\end{itemize}
		% subsubsection tempi_di_sviluppo_sottostimati (end)
		
		\subsubsection{Tempi di sviluppo sovrastimati} % (fold)
		\label{ssub:tempi_di_sviluppo_sovrastimati}
			\begin{itemize}
				\item \textbf{Probabilità di occorrenza}: Alta;
				\item \textbf{Grado di pericolosità}: Basso;
				\item \textbf{Descrizione}: TO DO;
				\item \textbf{Metodi di rilevazione}: TO DO;
				\item \textbf{Contromisure}: TO DO.
			\end{itemize}
		% subsubsection tempi_di_sviluppo_sovrastimati (end)
		
	% subsection rischi_dovuti_a_stime_errate (end)
	
	\subsection{Riepilogo rischi} % (fold)
	\label{sub:riepilogo_rischi}
	TO DO
		\begin{table}[!h]
			\begin{center}
				\begin{tabularx}{0.7\textwidth}{|l|X|X|}
					\hline
					\textbf{Nome rischio} & \textbf{Probabilità} & \textbf{Grado} \\
					\hline
					Sezione + Nome rischio &
					Bassa &
					Bassa \\
					\hline
					Sezione + Nome rischio &
					Bassa &
					Bassa \\
					\hline
					Sezione + Nome rischio &
					Bassa &
					Bassa \\
					\hline
					Sezione + Nome rischio &
					Bassa &
					Bassa \\
					\hline
					Sezione + Nome rischio &
					Bassa &
					Bassa \\
					\hline
					Sezione + Nome rischio &
					Bassa &
					Bassa \\
					\hline
					Sezione + Nome rischio &
					Bassa &
					Bassa \\
					\hline			
				\end{tabularx}
			\end{center}
		\caption{Rischi esposti in ordine del grado di pericolosità}
		\end{table}
		
	% subsection riepilogo_rischi (end)
	
% section analisi_dei_rischi (end)