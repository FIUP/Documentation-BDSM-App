% =================================================================================================
% File:			preventivo_interno.tex
% Description:	Defiinisce la sezione relativa al preventivo interno del team
% Created:		2015-02-19
% Author:		Santacatterina Luca
% Email:		santacatterina.luca@mashup-unipd.it
% =================================================================================================
% Modification History:
% Version		Modifier Date		Change											Author
% 0.0.1 		2015-02-19 			importati i dati da precedente sezione generica	Santacatterina Luca
% =================================================================================================
%
% =================================================================================================
%

% CONTENUTO DEL CAPITOLO

\section{Preventivo interno al team} % (fold)
\label{sec:preventivo_interno_al_team}
Verranno indicate di seguito le spese pianificate e la suddivisione dei ruoli effettuata, relative alle ore non rendicontate. Quanto indicato è fornito a solo scopo informativo e \textbf{non ha in alcun modo da attribuirsi a un rapporto con il proponente}.

	\subsection{Ricerca e implementazione degli strumenti} % (fold)
	\label{sub:ricerca_e_implementazione_degli_strumenti}
		\subsubsection{Suddivisione del lavoro} % (fold)
		\label{ssub:suddivisione_del_lavoro}
		Nella fase di \textbf{Ricerca e implementazione degli strumenti} ciascun componente rivestirà i seguenti ruoli: \\
			\begin{table}[!ht]
				\begin{center}
					\begin{tabularx}{0.9\textwidth}{|l|l|l|l|l|l|l|X|}
						\hline
						\textbf{Nome membro} & \textbf{PM} & \textbf{An} & \textbf{Pt} & \textbf{Am} & \textbf{Pr} & \textbf{Ve} & \textbf{Ore totali} \\
						\hline
						Carnovalini Filippo & 0 & 7 & 0 & 0 & 0 & 0 & \textbf{7} \\
						\hline
						Ceccon Lorenzo & 0 & 5 & 0 & 0 & 0 & 5 & \textbf{10} \\
						\hline
						Cusinato Giacomo & 0 & 7 & 0 & 0 & 0 & 0 & \textbf{7} \\
						\hline
						Faccin Nicola & 0 & 5 & 0 & 0 & 0 & 5 & \textbf{10} \\
						\hline
						Roetta Marco & 0 & 5 & 0 & 5 & 0 & 0 & \textbf{10} \\
						\hline
						Santacatterina Luca & 0 & 0 & 0 & 10 & 0 & 0 & \textbf{10} \\
						\hline
						Tesser Paolo & 4 & 0 & 0 & 8 & 0 & 0 & \textbf{12} \\
						\hline
					\end{tabularx}
				\end{center}
			\caption{Ricerca e implementazione degli strumenti - Ore non rendicontate}
			\end{table}
		
			\begin{figure}[htbp]
		
				\begin{tikzpicture}
					\begin{axis}[
					    xbar stacked,
					    legend style={
					    legend columns=6,
					        at={(xticklabel cs:0.5)},
					        anchor=north,
					        draw=none
					    },
					    ytick=data,
					    axis y line*=none,
					    axis x line*=bottom,
					    tick label style={font=\footnotesize},
					    legend style={font=\footnotesize},
					    label style={font=\footnotesize},
					    xtick={0,5,10,15,20,25,30},
					    width=.9\textwidth,
					    bar width=4mm,
					    xlabel={Time in ms},
					    yticklabels={Tesser Paolo, Santacatterina Luca, Roetta Marco, Faccin Nicola, Cusinato Giacomo, Ceccon Lorenzo, Carnovalini Filippo},
					    xmin=0,
					    xmax=30,
					    area legend,
					    y=6mm,
					    enlarge y limits={abs=0.625},
					]
					\addplot[blu,fill=blu] coordinates
					{(4,0) (0,1) (0,2) (0,3) (0,4) (0,5) (0,6)};
					\addplot[rosso,fill=rosso] coordinates
					{(0,0) (0,1) (5,2) (5,3) (7,4) (5,5) (7,6)};
					\addplot[giallo,fill=giallo] coordinates
					{(0,0) (0,1) (0,2) (0,3) (0,4) (0,5) (0,6)};
					\addplot[viola,fill=viola] coordinates
					{(8,0) (10,1) (5,2) (0,3) (0,4) (0,5) (0,6)};
					\addplot[verde,fill=verde] coordinates
					{(0,0) (0,1) (0,2) (0,3) (0,4) (0,5) (0,6)};
					\addplot[rame,fill=rame] coordinates
					{(0,0) (0,1) (0,2) (5,3) (0,4) (5,5) (0,6)};
				
					\legend{Responsabile,Analista,Progettista,Amministratore,Programmatore,Verificatore}
					\end{axis}  
				\end{tikzpicture}
			\vspace{0.2cm}	
			\caption{Ricerca e implementazione degli strumenti - Riepilogo}
			\end{figure}
		% subsubsection suddivisione_del_lavoro (end)
	
		\subsubsection{Prospetto economico} % (fold)
		\label{ssub:prospetto_economico}
		In questa fase il costo per ogni ruolo, non a carico del proponente, è il seguente: \\
			\begin{table}[!ht]
				\begin{center}
					\begin{tabularx}{0.65\textwidth}{|l|l|X|}
						\hline
						\textbf{Ruolo} & \textbf{Ore} & \textbf{Costo} \\
						\hline
						\roleProjectManager & 4 & \euro{} 120,00 \\
						\hline
						\roleAnalyst & 29 & \euro{} 725,00 \\
						\hline
						\roleDesigner & 0 & \euro{} 0,00 \\
						\hline
						\roleAdministrator & 23 & \euro{} 460,00 \\
						\hline
						\roleProgrammer & 0 & \euro{} 0,00 \\
						\hline
						\roleVerifier & 10 & \euro{} 150,00 \\
						\hline
						\textbf{Totale} & \textbf{66} & \textbf{\euro{} 1455,00} \\
						\hline
					\end{tabularx}
				\end{center}
			\caption{Ricerca e implementazione degli strumenti - Costo per ruolo}
			\end{table}

		\noindent
		I valori delle ore sono riassunte nel seguente grafico a torta che espone l’incidenza delle ore per ruolo sul totale.
		\begin{center}
			\begin{figure}[htbp]
				\begin{tikzpicture}
				[
				    pie chart,
				    slice type={pm}{blu},
				    slice type={an}{rosso},
				    slice type={pt}{giallo},
				    slice type={am}{viola},
				    slice type={pr}{verde},
					slice type={ve}{rame},
				    pie values/.style={font={\normalsize}},
				    scale=2.5
				]

				    \pie{Ore per ruolo sul totale}{6/pm,44/an,35/am,15/ve}

				    \legend[shift={(2cm,1cm)}]{{Responsabile}/pm, {Analista}/an, {Progettista}/pt, {Amministratore}/am, {Programmatore}/pr, {Verificatore}/ve}

				\end{tikzpicture}
			\vspace{0.8cm}
			\caption{Ricerca e implementazione strumenti - Ore per ruolo sul totale}
			\end{figure}
		\end{center}
	
		\noindent
		I valori dei costi sono riassunti nel seguente grafico a torta che espone l’incidenza dei costi per ruolo sul totale.
		\begin{center}
			\begin{figure}[htbp]
				\begin{tikzpicture}
				[
				    pie chart,
				    slice type={pm}{blu},
				    slice type={an}{rosso},
				    slice type={pt}{giallo},
				    slice type={am}{viola},
				    slice type={pr}{verde},
					slice type={ve}{rame},
				    pie values/.style={font={\normalsize}},
				    scale=2.5
				]

				    \pie{Costi per ruolo sul totale}{11/pm,42/an,19/am,28/ve}

				    \legend[shift={(2cm,1cm)}]{{Responsabile}/pm, {Analista}/an, {Progettista}/pt, {Amministratore}/am, {Programmatore}/pr, {Verificatore}/ve}

				\end{tikzpicture}
			\vspace{0.8cm}
			\caption{Ricerca e implementazione strumenti - Costi per ruolo sul totale}
			\end{figure}
		\end{center}			
	
		% subsubsection prospetto_economico (end)

	\noindent
	\textbf{Note:} in questa fase le ore vengono abbastanza equamente suddivise tra i componenti. Solo per Carnovalini Filippo e Cusinato Giacomo ne sono state allocate meno in quanto il primo ha fatto presente alcuni impegni personali durante il periodo in questione, mentre il secondo aveva un appello straordinario di recupero in vista.	
	% subsection ricerca_e_implementazione_degli_strumenti (end)

	\newpage
	\subsection{Analisi dei requisiti} % (fold)
	\label{sub:analisi_dei_requisiti}
		\subsubsection{Suddivisione del lavoro} % (fold)
		\label{ssub:suddivisione_del_lavoro}
		Nella fase di \textbf{Analisi dei requisiti} ciascun componente rivestirà i seguenti ruoli: \\
			\begin{table}[!ht]
				\begin{center}
					\begin{tabularx}{0.9\textwidth}{|l|l|l|l|l|l|l|X|}
						\hline
						\textbf{Nome membro} & \textbf{PM} & \textbf{An} & \textbf{Pt} & \textbf{Am} & \textbf{Pr} & \textbf{Ve} & \textbf{Totale} \\
						\hline
						Carnovalini Filippo & 0 & 17 & 0 & 0 & 0 & 5 & \textbf{22} \\
						\hline
						Ceccon Lorenzo & 0 & 7 & 0 & 0 & 0 & 15 & \textbf{22} \\
						\hline
						Cusinato Giacomo & 2 & 18 & 0 & 0 & 0 & 0 & \textbf{20} \\
						\hline
						Faccin Nicola & 0 & 7 & 0 & 0 & 0 & 15 & \textbf{22} \\
						\hline
						Roetta Marco & 0 & 17 & 0 & 10 & 0 & 0 & \textbf{27} \\
						\hline
						Santacatterina Luca & 0 & 0 & 0 & 15 & 0 & 10 & \textbf{25} \\
						\hline
						Tesser Paolo & 16 & 0 & 0 & 5 & 0 & 0 & \textbf{21} \\
						\hline		
					\end{tabularx}
				\end{center}
			\caption{Analisi dei requisiti - Ore non rendicontate}
			\end{table}
		
			\begin{figure}[htbp]
		
				\begin{tikzpicture}
					\begin{axis}[
					    xbar stacked,
					    legend style={
					    legend columns=6,
					        at={(xticklabel cs:0.5)},
					        anchor=north,
					        draw=none
					    },
					    ytick=data,
					    axis y line*=none,
					    axis x line*=bottom,
					    tick label style={font=\footnotesize},
					    legend style={font=\footnotesize},
					    label style={font=\footnotesize},
					    xtick={0,5,10,15,20,25,30},
					    width=.9\textwidth,
					    bar width=4mm,
					    xlabel={Time in ms},
					    yticklabels={Tesser Paolo, Santacatterina Luca, Roetta Marco, Faccin Nicola, Cusinato Giacomo, Ceccon Lorenzo, Carnovalini Filippo},
					    xmin=0,
					    xmax=40,
					    area legend,
					    y=6mm,
					    enlarge y limits={abs=0.625},
					]
					\addplot[blu,fill=blu] coordinates
					{(16,0) (0,1) (0,2) (0,3) (2,4) (0,5) (0,6)};
					\addplot[rosso,fill=rosso] coordinates
					{(0,0) (0,1) (17,2) (7,3) (18,4) (7,5) (17,6)};
					\addplot[giallo,fill=giallo] coordinates
					{(0,0) (0,1) (0,2) (0,3) (0,4) (0,5) (0,6)};
					\addplot[viola,fill=viola] coordinates
					{(5,0) (10,1) (10,2) (0,3) (0,4) (0,5) (0,6)};
					\addplot[verde,fill=verde] coordinates
					{(0,0) (0,1) (0,2) (0,3) (0,4) (0,5) (0,6)};
					\addplot[rame,fill=rame] coordinates
					{(0,0) (15,1) (0,2) (15,3) (0,4) (15,5) (5,6)};
				
					\legend{Responsabile,Analista,Progettista,Amministratore,Programmatore,Verificatore}
					\end{axis}  
				\end{tikzpicture}
			\vspace{0.2cm}	
			\caption{Analisi dei requisiti - Riepilogo}
			\end{figure}
		% subsubsection suddivisione_del_lavoro (end)
	
		\subsubsection{Prospetto economico} % (fold)
		\label{ssub:prospetto_economico}
		In questa fase il costo per ogni ruolo, non a carico del proponente, è il seguente: \\
			\begin{table}[!ht]
				\begin{center}
					\begin{tabularx}{0.65\textwidth}{|l|l|X|}
						\hline
						\textbf{Ruolo} & \textbf{Ore} & \textbf{Costo} \\
						\hline
						\roleProjectManager & 18 & \euro{} 540,00 \\
						\hline
						\roleAnalyst & 66 & \euro{} 1650,00 \\
						\hline
						\roleDesigner & 0 & \euro{} 0,00 \\
						\hline
						\roleAdministrator & 30 & \euro{} 600,00 \\
						\hline
						\roleProgrammer & 0 & \euro{} 0,00 \\
						\hline
						\roleVerifier & 45 & \euro{} 675,00 \\
						\hline
						\textbf{Totale} & \textbf{159} & \textbf{\euro{} 3465,00} \\
						\hline
					\end{tabularx}
				\end{center}
			\caption{Analisi dei requisiti - Costo per ruolo}
			\end{table}
		
			\noindent
			I valori delle ore sono riassunte nel seguente grafico a torta che espone l’incidenza delle ore per ruolo sul totale.
			\begin{center}
				\begin{figure}[htbp]
					\begin{tikzpicture}
					[
					    pie chart,
					    slice type={pm}{blu},
					    slice type={an}{rosso},
					    slice type={pt}{giallo},
					    slice type={am}{viola},
					    slice type={pr}{verde},
						slice type={ve}{rame},
					    pie values/.style={font={\normalsize}},
					    scale=2.5
					]

					    \pie{Ore per ruolo sul totale}{11/pm,42/an,19/am,28/ve}

					    \legend[shift={(2cm,1cm)}]{{Responsabile}/pm, {Analista}/an, {Progettista}/pt, {Amministratore}/am, {Programmatore}/pr, {Verificatore}/ve}

					\end{tikzpicture}
				\vspace{0.8cm}
				\caption{Analisi dei requisiti - Ore per ruolo sul totale}
				\end{figure}
			\end{center}

			\noindent
			I valori dei costi sono riassunti nel seguente grafico a torta che espone l’incidenza dei costi per ruolo sul totale.
			\begin{center}
				\begin{figure}[htbp]
					\begin{tikzpicture}
					[
					    pie chart,
					    slice type={pm}{blu},
					    slice type={an}{rosso},
					    slice type={pt}{giallo},
					    slice type={am}{viola},
					    slice type={pr}{verde},
						slice type={ve}{rame},
					    pie values/.style={font={\normalsize}},
					    scale=2.5
					]

					    \pie{Costi per ruolo sul totale}{16/pm,48/an,17/am,19/ve}

					    \legend[shift={(2cm,1cm)}]{{Responsabile}/pm, {Analista}/an, {Progettista}/pt, {Amministratore}/am, {Programmatore}/pr, {Verificatore}/ve}

					\end{tikzpicture}
				\vspace{0.8cm}
				\caption{Analisi dei requisiti - Costi per ruolo sul totale}
				\end{figure}
			\end{center}
		% subsubsection prospetto_economico (end)
		\noindent
		\textbf{Note:} in questa fase le ore sono equamente suddivise. Solo per il componente Roetta Marco ne sono state allocate di più in quanto ha preferito averne in numero maggiore durante la pausa natalizia essendo a casa dal lavoro.
	% subsection analisi_dei_requisiti (end)
	\newpage
	\subsection{Analisi di dettaglio} % (fold)
	\label{sub:analisi_di_dettaglio}
		\subsubsection{Suddivisione del lavoro} % (fold)
		\label{ssub:suddivisione_del_lavoro}
		Nella fase di \textbf{Analisi di dettaglio} ciascun componente rivestirà i seguenti ruoli: \\
			\begin{table}[!ht]
				\begin{center}
					\begin{tabularx}{0.9\textwidth}{|l|l|l|l|l|l|l|X|}
						\hline
						\textbf{Nome membro} & \textbf{PM} & \textbf{An} & \textbf{Pt} & \textbf{Am} & \textbf{Pr} & \textbf{Ve} & \textbf{Totale} \\
						\hline
						Carnovalini Filippo & 0 & 7 & 0 & 0 & 0 & 10 & \textbf{17} \\
						\hline
						Ceccon Lorenzo & 6 & 7 & 0 & 6 & 0 & 0 & \textbf{19} \\
						\hline
						Cusinato Giacomo & 0 & 0 & 0 & 0 & 0 & 15 & \textbf{15} \\
						\hline
						Faccin Nicola & 0 & 0 & 0 & 4 & 0 & 10 & \textbf{14} \\
						\hline
						Roetta Marco & 0 & 10 & 0 & 0 & 0 & 7 & \textbf{17} \\
						\hline
						Santacatterina Luca & 2 & 12 & 0 & 0 & 0 & 0 & \textbf{14} \\
						\hline
						Tesser Paolo & 0 & 11 & 0 & 0 & 0 & 7 & \textbf{18} \\
						\hline	
					\end{tabularx}
				\end{center}
			\caption{Analisi di dettaglio - Ore non rendicontate}
			\end{table}
		
			\begin{figure}[htbp]
		
				\begin{tikzpicture}
					\begin{axis}[
					    xbar stacked,
					    legend style={
					    legend columns=6,
					        at={(xticklabel cs:0.5)},
					        anchor=north,
					        draw=none
					    },
					    ytick=data,
					    axis y line*=none,
					    axis x line*=bottom,
					    tick label style={font=\footnotesize},
					    legend style={font=\footnotesize},
					    label style={font=\footnotesize},
					    xtick={0,5,10,15,20,25,30},
					    width=.9\textwidth,
					    bar width=4mm,
					    xlabel={Time in ms},
					    yticklabels={Tesser Paolo, Santacatterina Luca, Roetta Marco, Faccin Nicola, Cusinato Giacomo, Ceccon Lorenzo, Carnovalini Filippo},
					    xmin=0,
					    xmax=35,
					    area legend,
					    y=6mm,
					    enlarge y limits={abs=0.625},
					]
					\addplot[blu,fill=blu] coordinates
					{(0,0) (2,1) (0,2) (0,3) (0,4) (6,5) (0,6)};
					\addplot[rosso,fill=rosso] coordinates
					{(11,0) (12,1) (10,2) (0,3) (0,4) (7,5) (7,6)};
					\addplot[giallo,fill=giallo] coordinates
					{(0,0) (0,1) (0,2) (0,3) (0,4) (0,5) (0,6)};
					\addplot[viola,fill=viola] coordinates
					{(0,0) (0,1) (0,2) (4,3) (0,4) (6,5) (0,6)};
					\addplot[verde,fill=verde] coordinates
					{(0,0) (0,1) (0,2) (0,3) (0,4) (0,5) (0,6)};
					\addplot[rame,fill=rame] coordinates
					{(7,0) (0,1) (7,2) (10,3) (15,4) (0,5) (10,6)};
				
					\legend{Responsabile,Analista,Progettista,Amministratore,Programmatore,Verificatore}
					\end{axis}  
				\end{tikzpicture}
			\vspace{0.2cm}	
			\caption{Analisi di dettaglio - Riepilogo}
			\end{figure}
		% subsubsection suddivisione_del_lavoro (end)
	
		\subsubsection{Prospetto economico} % (fold)
		\label{ssub:prospetto_economico}
		In questa fase il costo per ogni ruolo, non a carico del proponente, è il seguente: \\
			\begin{table}[!ht]
				\begin{center}
					\begin{tabularx}{0.65\textwidth}{|l|l|X|}
						\hline
						\textbf{Ruolo} & \textbf{Ore} & \textbf{Costo} \\
						\hline
						\roleProjectManager & 8 & \euro{} 240,00 \\
						\hline
						\roleAnalyst & 47 & \euro{} 1175,00 \\
						\hline
						\roleDesigner & 0 & \euro{} 0,00 \\
						\hline
						\roleAdministrator & 10 & \euro{} 200,00 \\
						\hline
						\roleProgrammer & 0 & \euro{} 0,00 \\
						\hline
						\roleVerifier & 49 & \euro{} 735,00 \\
						\hline
						\textbf{Totale} & \textbf{114} & \textbf{\euro{} 2350,00} \\
						\hline
					\end{tabularx}
				\end{center}
			\caption{Analisi di dettaglio - Costo per ruolo}
			\end{table}

			\noindent
			I valori delle ore sono riassunte nel seguente grafico a torta che espone l’incidenza delle ore per ruolo sul totale.
			\begin{center}
				\begin{figure}[htbp]
					\begin{tikzpicture}
					[
					    pie chart,
					    slice type={pm}{blu},
					    slice type={an}{rosso},
					    slice type={pt}{giallo},
					    slice type={am}{viola},
					    slice type={pr}{verde},
						slice type={ve}{rame},
					    pie values/.style={font={\normalsize}},
					    scale=2.5
					]

					    \pie{Ore per ruolo sul totale}{7/pm,41/an,9/am,43/ve}

					    \legend[shift={(2cm,1cm)}]{{Responsabile}/pm, {Analista}/an, {Progettista}/pt, {Amministratore}/am, {Programmatore}/pr, {Verificatore}/ve}

					\end{tikzpicture}
				\vspace{0.8cm}
				\caption{Analisi di dettaglio - Ore per ruolo sul totale}
				\end{figure}
			\end{center}

			\noindent
			I valori dei costi sono riassunti nel seguente grafico a torta che espone l’incidenza dei costi per ruolo sul totale.
			\begin{center}
				\begin{figure}[htbp]
					\begin{tikzpicture}
					[
					    pie chart,
					    slice type={pm}{blu},
					    slice type={an}{rosso},
					    slice type={pt}{giallo},
					    slice type={am}{viola},
					    slice type={pr}{verde},
						slice type={ve}{rame},
					    pie values/.style={font={\normalsize}},
					    scale=2.5
					]

					    \pie{Costi per ruolo sul totale}{10/pm,50/an,9/am,31/ve}

					    \legend[shift={(2cm,1cm)}]{{Responsabile}/pm, {Analista}/an, {Progettista}/pt, {Amministratore}/am, {Programmatore}/pr, {Verificatore}/ve}

					\end{tikzpicture}
				\vspace{0.8cm}
				\caption{Analisi di dettaglio - Costi per ruolo sul totale}
				\end{figure}
			\end{center}
		% subsubsection prospetto_economico (end)
		\noindent
		\textbf{Note:} in questa fase le ore vengono abbastanza equamente suddivise. Per alcuni dei membri che stanno svolgendo anche il progetto di Tecnologie Web c'è stata una minore allocazione.
	% subsection analisi_di_dettaglio (end)

	\newpage
	\subsection{Riepilogo ore e costi} % (fold)
	\label{sub:riepilogo_ore_e_costi_interno}
			
		\subsubsection{Ore totali non rendicontate} % (fold)
		\label{ssub:ore_totali_non_rendicontate}
			\paragraph{Suddivisione del lavoro} % (fold)
			\label{par:suddivisione_del_lavoro}
				\begin{table}[!ht]
					\begin{center}
						\begin{tabularx}{0.9\textwidth}{|l|l|l|l|l|l|l|X|}
							\hline
							\textbf{Nome membro} & \textbf{PM} & \textbf{An} & \textbf{Pt} & \textbf{Am} & \textbf{Pr} & \textbf{Ve} & \textbf{Totale} \\
							\hline
							Carnovalini Filippo & 0 & 31 & 0 & 0 & 0 & 15 & \textbf{46} \\
							\hline
							Ceccon Lorenzo & 6 & 19 & 0 & 6 & 0 & 20 & \textbf{51} \\
							\hline
							Cusinato Giacomo & 2 & 25 & 0 & 0 & 0 & 15 & \textbf{42} \\
							\hline
							Faccin Nicola & 0 & 12 & 0 & 4 & 0 & 30 & \textbf{46} \\
							\hline
							Roetta Marco & 0 & 32 & 0 & 15 & 0 & 7 & \textbf{54} \\
							\hline
							Santacatterina Luca & 2 & 12 & 0 & 25 & 0 & 10 & \textbf{49} \\
							\hline
							Tesser Paolo & 20 & 11 & 0 & 13 & 0 & 7 & \textbf{51} \\
							\hline	
						\end{tabularx}
					\end{center}
				\caption{Ore totali non rendicontate - Suddivisione delle ore di lavoro}
				\end{table}
				
				\begin{figure}[htbp]

					\begin{tikzpicture}
						\begin{axis}[
						    xbar stacked,
						    legend style={
						    legend columns=6,
						        at={(xticklabel cs:0.5)},
						        anchor=north,
						        draw=none
						    },
						    ytick=data,
						    axis y line*=none,
						    axis x line*=bottom,
						    tick label style={font=\footnotesize},
						    legend style={font=\footnotesize},
						    label style={font=\footnotesize},
						    xtick={0,5,10,15,20,25,30},
						    width=.9\textwidth,
						    bar width=4mm,
						    xlabel={Time in ms},
						    yticklabels={Tesser Paolo, Santacatterina Luca, Roetta Marco, Faccin Nicola, Cusinato Giacomo, Ceccon Lorenzo, Carnovalini Filippo},
						    xmin=0,
						    xmax=50,
						    area legend,
						    y=6mm,
						    enlarge y limits={abs=0.625},
						]
						\addplot[blu,fill=blu] coordinates
						{(20,0) (2,1) (0,2) (0,3) (2,4) (6,5) (0,6)};
						\addplot[rosso,fill=rosso] coordinates
						{(11,0) (12,1) (32,2) (12,3) (25,4) (19,5) (31,6)};
						\addplot[giallo,fill=giallo] coordinates
						{(0,0) (0,1) (0,2) (0,3) (0,4) (0,5) (0,6)};
						\addplot[viola,fill=viola] coordinates
						{(13,0) (25,1) (15,2) (4,3) (0,4) (6,5) (0,6)};
						\addplot[verde,fill=verde] coordinates
						{(0,0) (0,1) (0,2) (0,3) (0,4) (0,5) (0,6)};
						\addplot[rame,fill=rame] coordinates
						{(7,0) (10,1) (7,2) (30,3) (15,4) (20,5) (15,6)};

						\legend{Responsabile,Analista,Progettista,Amministratore,Programmatore,Verificatore}
						\end{axis}  
					\end{tikzpicture}
				\vspace{0.2cm}	
				\caption{Ore totali non rendicontate - Riepilogo}
				\end{figure}

			% paragraph suddivisione_del_lavoro (end)
			
			\paragraph{Prospetto economico} % (fold)
			\label{par:prospetto_economico}
				\begin{table}[!ht]
					\begin{center}
						\begin{tabularx}{0.65\textwidth}{|l|l|X|}
							\hline
							\textbf{Ruolo} & \textbf{Ore} & \textbf{Costo} \\
							\hline
							\roleProjectManager & 30 & \euro{} 900,00 \\
							\hline
							\roleAnalyst & 142 & \euro{} 3550,00 \\
							\hline
							\roleDesigner & 0 & \euro{} 0,00 \\
							\hline
							\roleAdministrator & 63 & \euro{} 1260,00 \\
							\hline
							\roleProgrammer & 0 & \euro{} 0,00 \\
							\hline
							\roleVerifier & 104 & \euro{} 1560,00 \\
							\hline
							\textbf{Totale} & \textbf{339} & \textbf{\euro{} 7270,00} \\
							\hline
						\end{tabularx}
					\end{center}
				\caption{Ore totali non rendicontate - Costo per ruolo}
				\end{table}
				
				\noindent
				I valori delle ore sono riassunte nel seguente grafico a torta che espone l’incidenza delle ore per ruolo sul totale.
				\begin{center}
					\begin{figure}[htbp]
						\begin{tikzpicture}
						[
						    pie chart,
						    slice type={pm}{blu},
						    slice type={an}{rosso},
						    slice type={pt}{giallo},
						    slice type={am}{viola},
						    slice type={pr}{verde},
							slice type={ve}{rame},
						    pie values/.style={font={\normalsize}},
						    scale=2.5
						]

						    \pie{Ore per ruolo sul totale}{9/pm,41/an,19/am,31/ve}

						    \legend[shift={(2cm,1cm)}]{{Responsabile}/pm, {Analista}/an, {Progettista}/pt, {Amministratore}/am, {Programmatore}/pr, {Verificatore}/ve}

						\end{tikzpicture}
					\vspace{0.8cm}
					\caption{Ore totali non rendicontate - Ore per ruolo sul totale}
					\end{figure}
				\end{center}

				\noindent
				I valori dei costi sono riassunti nel seguente grafico a torta che espone l’incidenza dei costi per ruolo sul totale.
				\begin{center}
					\begin{figure}[htbp]
						\begin{tikzpicture}
						[
						    pie chart,
						    slice type={pm}{blu},
						    slice type={an}{rosso},
						    slice type={pt}{giallo},
						    slice type={am}{viola},
						    slice type={pr}{verde},
							slice type={ve}{rame},
						    pie values/.style={font={\normalsize}},
						    scale=2.5
						]

						    \pie{Costi per ruolo sul totale}{12/pm,49/an,17/am,22/ve}

						    \legend[shift={(2cm,1cm)}]{{Responsabile}/pm, {Analista}/an, {Progettista}/pt, {Amministratore}/am, {Programmatore}/pr, {Verificatore}/ve}

						\end{tikzpicture}
					\vspace{0.8cm}
					\caption{Ore totali non rendicontate - Costi per ruolo sul totale}
					\end{figure}
				\end{center}
			% paragraph prospetto_economico (end)
			
		% subsubsection ore_totali_non_rendicontate (end)

	% subsection riepilogo_ore_e_costi (end)
% section preventivo_interno_al_team (end)