% =================================================================================================
% File:			consuntivo_interno.tex
% Description:	Defiinisce la sezione relativa al consuntivo interno al team
% Created:		2014-12-30
% Author:		Tesser Paolo
% Email:		tesser.paolo@mashup-unipd.it
% =================================================================================================
% Modification History:
% Version		Modifier Date		Change												Author
% 0.0.1 		2014-12-30 			creata struttura del consuntivo						Tesser Paolo
% =================================================================================================
% 0.0.2			2015-01-08			scelta sezioni da redarre							Tesser Paolo
% =================================================================================================
% 0.0.3			2015-01-12			aggiunta nota introduttiva							Tesser Paolo
% =================================================================================================
% 1.0.1			2015-02-12			aggiunto consuntivo riguardante l'An. di dettaglio	Ceccon Lorenzo
% =================================================================================================
% 2.0.1			2015-02-19			migliorata sezione per separare ore rendic. da non	Santacatterina Luca
% =================================================================================================
%

% CONTENUTO DEL CAPITOLO

\section{Consuntivo interno al team} % (fold)
\label{sec:consuntivo_interno_al_team}
Verranno indicate di seguito le spese effettivamente sostenute, relative alle ore non rendicontate. Quanto indicato è fornito a solo scopo informativo e \textbf{non ha in alcun modo da attribuirsi a un rapporto con il proponente}.

	\subsection{Ricerca e implementazione degli strumenti} % (fold)
	\label{sub:ricerca_e_implementazione_degli_strumenti}
		\subsubsection{Consuntivo interno} % (fold)
		\label{ssub:consuntivo}
		Verrà indicata la differenza di ore per ruolo e di spese effettivamente sostenute durante la fase di approfondimento personale. Questi dati sono quindi relativi alle ore non rendicontate.
		\begin{table}[!h]
			\begin{center}
				\begin{tabularx}{0.75\textwidth}{|X|c|c|}
					\hline
					\textbf{Ruolo} & \textbf{Ore} & \textbf{Costo} \\
					\hline
					\roleProjectManager & +1 & +\euro{} 30,00 \\
					\hline
					\roleAnalyst & -2 & -\euro{} 50,00 \\
					\hline
					\roleDesigner & 0 & \euro{} 0,00 \\
					\hline
					\roleAdministrator & +1 & +\euro{} 20,00 \\
					\hline
					\roleProgrammer & 0 & \euro{} 0,00 \\
					\hline
					\roleVerifier & -2 & -\euro{} 30,00 \\
					\hline
					\textbf{Differenza preventivo/consuntivo} & \textbf{-2} & \textbf{+\euro{} 30,00} \\
					\hline
				\end{tabularx}
			\end{center}
		\caption{Ore non rendicontate - differenza preventivo interno/consuntivo interno al team}
		\end{table}
		% subsubsection consuntivo (end)

		\subsubsection{Riepilogo} % (fold)
		\label{ssub:riepilogo}
		Il gruppo nel complessivo ha impiegato due ore in meno per svolgere le attività pianificate. \newline
		Questo ha consentito un risparmio interno al team di \textbf{\euro{} 30,00}.
		% subsubsection riepilogo (end)

	% subsection ricerca_e_implementazione_degli_strumenti (end)
	\newpage
	\subsection{Analisi dei requisiti} % (fold)
	\label{sub:analisi_dei_requisiti}
		\subsubsection{Consuntivo interno} % (fold)
		\label{ssub:consuntivo}
		Verrà indicata la differenza di ore per ruolo e di spese effettivamente sostenute durante la fase di approfondimento personale. Questi dati sono quindi relativi alle ore non rendicontate.
		\begin{table}[!h]
			\begin{center}
				\begin{tabularx}{0.75\textwidth}{|X|c|c|}
					\hline
					\textbf{Ruolo} & \textbf{Ore} & \textbf{Costo} \\
					\hline
					\roleProjectManager & -2 & -\euro{} 60,00 \\
					\hline
					\roleAnalyst & +2 & + \euro{} 50,00 \\
					\hline
					\roleDesigner & 0 & \euro{} 0,00 \\
					\hline
					\roleAdministrator & +1  & +\euro{} 20,00 \\
					\hline
					\roleProgrammer & 0 & \euro{} 0,00 \\
					\hline
					\roleVerifier & +1 & +\euro{} 15,00 \\
					\hline
					\textbf{Differenza consuntivo/preventivo} & \textbf{+2} & \textbf{+\euro{} 25,00} \\
					\hline
				\end{tabularx}
			\end{center}
		\caption{Ore non rendicontate - differenza consuntivo interno/preventivo interno al team}
		\end{table}
		% subsubsection consuntivo (end)

		\subsubsection{Riepilogo} % (fold)
		\label{ssub:riepilogo}
		Il gruppo nel complessivo ha impiegato due ore in più per svolgere le attività pianificate. \newline
		Questo ha causato un aumento dei costi interni al team di \textbf{\euro{} 25,00} che nonostante ciò hanno un avanzo interno grazie al risparmio ottenuto nella precedente fase.
		% subsubsection riepilogo (end)

	% subsection analisi_dei_requisiti (end)

	\newpage
	\subsection{Analisi di dettaglio} % (fold)
	\label{sub:analisi_di_dettaglio}
		\subsubsection{Consuntivo interno} % (fold)
		\label{ssub:consuntivo}
		Verrà indicata la differenza di ore per ruolo e di spese effettivamente sostenute durante la fase di approfondimento personale. Questi dati sono quindi relativi alle ore non rendicontate. \newline
		\begin{table}[!h]
			\begin{center}
				\begin{tabularx}{0.75\textwidth}{|X|c|c|}
					\hline
					\textbf{Ruolo} & \textbf{Ore} & \textbf{Costo} \\
					\hline
					\roleProjectManager & +1 & + \euro{} 30,00 \\
					\hline
					\roleAnalyst & -10 & - \euro{} 250,00 \\
					\hline
					\roleDesigner & 0 & \euro{} 0,00 \\
					\hline
					\roleAdministrator & 0  & +\euro{} 0,00 \\
					\hline
					\roleProgrammer & 0 & \euro{} 0,00 \\
					\hline
					\roleVerifier & -8 & -\euro{} 120,00 \\
					\hline
					\textbf{Differenza consuntivo/preventivo} & \textbf{-17} & \textbf{-\euro{} 340,00} \\
					\hline
				\end{tabularx}
			\end{center}
		\caption{Ore non rendicontate - differenza consuntivo interno/preventivo interno al team}
		\end{table}
		% subsubsection consuntivo (end)

		\subsubsection{Riepilogo} % (fold)
		\label{ssub:riepilogo}
		Il gruppo nel complessivo ha impiegato 17 ore in meno per svolgere le attività pianificate. Questo è dovuto dal fatto che si sono riscontrati i rischi presenti alla sezione \ref{ssub:impegni_personali_dei_componenti}. \newline
		Questo ha consentito un risparmio interno al team di \textbf{\euro{} 340,00} in questa fase e di \textbf{\euro{} 345,00} nel complessivo. L'avanzo però dovrà essere assolutamente utilizzato nelle fasi successive per compensare le ore che non sono state svolte a causa dei rischi precedentemente citati.
		% subsubsection riepilogo (end)
	% subsection analisi_di_dettaglio (end)
% section consuntivo_interno_al_team (end)