% =================================================================================================
% File:			preventivo_finire.tex
% Description:	Defiinisce la sezione relativa al consuntivo a finire dei costi
% Created:		2015-02-19
% Author:		Santacatterina Luca
% Email:		santacatterina.luca@mashup-unipd.it
% =================================================================================================
% Modification History:
% Version		Modifier Date		Change											Author
% 0.0.1 		2015-02-19 			creata struttura del consuntivo					Santacatterina Luca
% =================================================================================================
%

% CONTENUTO DEL CAPITOLO

\section{Preventivo a finire} % (fold)

	\subsection{Progettazione architetturale} % (fold)

		\subsubsection{Preventivo a finire} % (fold)
		Da quanto emerso dal consuntivo di periodo \ref{sub:consuntivo_progettazione_architetturale}, i costi sono aumentati rispetto a quanto preventivato. A tal proposito però, vengono utilizzati i soldi totali risparmiati nelle precedenti fasi (\euro{} + 345,00) per far si che il preventivo finale per il proponente rimanga invariato. \newline
		Per compensare questo squilibrio e lo sforamento nei tempi verificatosi, verranno ripianificate le successive fasi all'inizio di quella di \textbf{Progettazione di dettaglio e codifica dei requisiti obbligatori}, andando a diminuire le ore predisposte inizialmente agli \emph{Analisti} e al \roleProjectManager.


		% subsubsection preventivo_a_finire (end)
	% subsection progettazione_architetturale (end)

	\subsection{Progettazione di dettaglio e codifica dei requisiti obbligatori} % (fold)

		\subsubsection{Preventivo a finire} % (fold)
		Da quanto emerso dal consuntivo di periodo \ref{sub:consuntivo_progettazione_di_dettaglio_e_codifica_dei_requisiti_obbligatori}, i costi sono diminuiti rispetto a quanto preventivato. Questo ci consente di trattenere quelli avanzati per fasi successive in cui potrebbe esserci una necessità maggiore di fondi, non andando quindi a diminuire quanto stabilito inizialmente con il proponente. \newline
		Questo avanzo per il momento ci consente di non dovere ripianificare le attività già pianificate per la fase seguente.



		\subsection{Progettazione di dettaglio e codifica dei requisiti desiderabili} % (fold)

			\subsubsection{Preventivo a finire} % (fold)
			Da quanto emerso dal consuntivo di periodo \ref{sub:consuntivo_progettazione_di_dettaglio_e_codifica_dei_requisiti_desiderabili}, i costi sono aumentati rispetto a quanto preventivato. Il buon avanzo però derivante dalla fase precedente ci consente di far si che il preventivo finale per il proponente rimanga invariato.
			Questo scompenso ci obbliga a rivedere parte della pianificazione della successiva fase che avverrà durante il periodo iniziale. \newline
			Si prevede però che parte dei requisiti desiderabili e quasi tutti i requisiti opzionali non verranno codificati, per andare a concentrare le attività sulla codifica dei test che consentiranno di avere una migliore qualità per quanto riguarda la parte principale dell'applicazione.


		\subsection{Consolidamento della progettazione e codifica dei requisiti desiderabili} % (fold)

			\subsubsection{Preventivo a finire} % (fold)
						Da quanto emerso dal consuntivo di periodo \ref{sub:consuntivo_consolidamento_req_desiderabili}, i costi sono diminuiti. Questo ci consente di trattenere l'avanzo per far fronte ad eventuali esigenze delle fasi successi non andando quindi a diminuire quanto stabilito inizialmente con il proponente. \newline
				Questo avanzo per il momento ci consente di non dovere ripianificare le attività già pianificate per la fase seguente.

		\subsection{Validazione} % (fold)

			\subsubsection{Preventivo a finire} % (fold)
						Da quanto emerso dal consuntivo di periodo \ref{sub:consuntivo_validazione}, i costi sono diminuiti rispetto a quanto preventivato. Essendo questa l'ultima fase pianificata, possiamo andare a ridurre il preventivo a finire per il proponente in quanto siamo riusciti a risparmiare  \euro{} -259,00. Possiamo quindi fissare l'importo finale a \euro{} 13500,00.

% section consuntivo (end)