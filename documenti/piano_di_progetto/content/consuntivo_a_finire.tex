% =================================================================================================
% File:			consuntivo_a_finire.tex
% Description:	Defiinisce la sezione relativa al consuntivo a finire dei costi
% Created:		2014-12-30
% Author:		Tesser Paolo
% Email:		tesser.paolo@mashup-unipd.it
% =================================================================================================
% Modification History:
% Version		Modifier Date		Change											Author
% 0.0.1 		2014-12-30 			creata struttura del consuntivo					Tesser Paolo
% =================================================================================================
% 0.0.2			2015-01-08			scelta sezioni da redarre						Tesser Paolo			
% =================================================================================================
% 0.0.3			2015-01-12			aggiunta nota introduttiva						Tesser Paolo
% =================================================================================================
%

% CONTENUTO DEL CAPITOLO

\section{Consuntivo a finire} % (fold)
\label{sec:consuntivo_a_finire}
Verranno indicate di seguito le spese sostenute, relative alle ore rendicontate e non, sia per ruolo che per persona. \\
Sarà infine presentato un bilancio riguardante solo le ore rendicontate che potrà essere:
	\begin{itemize}
		\item \textbf{Positivo}: se il preventivo supera il consuntivo;
		\item \textbf{Negativo}: se il consuntivo supera il preventivo;
		\item \textbf{In pari}: se consuntivo e preventivo sono uguali.
	\end{itemize}
	\subsection{Ricerca e implementazione degli strumenti} % (fold)
	\label{sub:ricerca_e_implementazione_degli_strumenti}
		\subsubsection{Consuntivo} % (fold)
		\label{ssub:consuntivo}
		Verranno indicate le ore di ruolo e le spese effettivamente sostenute durante la fase di approfondimento personale. Questi dati sono quindi relativi alle ore non rendicontate.
		\begin{table}[!h]
			\begin{center}
				\begin{tabularx}{0.65\textwidth}{|l|l|X|}
					\hline
					\textbf{Ruolo} & \textbf{Ore} & \textbf{Costo} \\
					\hline
					\roleProjectManager & 0 & \euro{} 0,00 \\
					\hline
					\roleAnalyst & 0 & \euro{} 0,00 \\
					\hline
					\roleDesigner & 0 & \euro{} 0,00 \\
					\hline
					\roleAdministrator & 0 & \euro{} 0,00 \\
					\hline
					\roleProgrammer & 0 & \euro{} 0,00 \\
					\hline
					\roleVerifier & 0 & \euro{} 0,00 \\
					\hline
					\textbf{Totale consuntivo} & \textbf{0} & \textbf{\euro{} 0,00} \\
					\hline
					\textbf{Totale preventivo} & \textbf{0} & \textbf{\euro{} 0,00} \\
					\hline
					\textbf{Totale (differenza)} & \textbf{0} & \textbf{\euro{} 0,00} \\
					\hline
				\end{tabularx}
			\end{center}
		\caption{Ore non rendicontate - differenza preventivo/consuntivo}
		\end{table}
		% subsubsection consuntivo (end)
		
		\subsubsection{Riepilogo} % (fold)
		\label{ssub:riepilogo}
		TO DO
		% subsubsection riepilogo (end)
		
	% subsection ricerca_e_implementazione_degli_strumenti (end)
	\newpage
	\subsection{Analisi dei requisiti} % (fold)
	\label{sub:analisi_dei_requisiti}
		\subsubsection{Consuntivo} % (fold)
		\label{ssub:consuntivo}
		Verranno indicate le ore di ruolo e le spese effettivamente sostenute durante la fase di approfondimento personale. Questi dati sono quindi relativi alle ore non rendicontate.
		\begin{table}[!h]
			\begin{center}
				\begin{tabularx}{0.65\textwidth}{|l|l|X|}
					\hline
					\textbf{Ruolo} & \textbf{Ore} & \textbf{Costo} \\
					\hline
					\roleProjectManager & 0 & \euro{} 0,00 \\
					\hline
					\roleAnalyst & 0 & \euro{} 0,00 \\
					\hline
					\roleDesigner & 0 & \euro{} 0,00 \\
					\hline
					\roleAdministrator & 0 & \euro{} 0,00 \\
					\hline
					\roleProgrammer & 0 & \euro{} 0,00 \\
					\hline
					\roleVerifier & 0 & \euro{} 0,00 \\
					\hline
					\textbf{Totale consuntivo} & \textbf{0} & \textbf{\euro{} 0,00} \\
					\hline
					\textbf{Totale preventivo} & \textbf{0} & \textbf{\euro{} 0,00} \\
					\hline
					\textbf{Totale (differenza)} & \textbf{0} & \textbf{\euro{} 0,00} \\
					\hline
				\end{tabularx}
			\end{center}
		\caption{Ore non rendicontate - differenza preventivo/consuntivo}
		\end{table}
		% subsubsection consuntivo (end)
	
		\subsubsection{Riepilogo} % (fold)
		\label{ssub:riepilogo}
		TO DO
		% subsubsection riepilogo (end)
		
	% subsection analisi_dei_requisiti (end)

% section consuntivo_a_finire (end)