%% start of file `template.tex'.
%% Copyright 2006-2013 Xavier Danaux (xdanaux@gmail.com).
%
% This work may be distributed and/or modified under the
% conditions of the LaTeX Project Public License version 1.3c,
% available at http://www.latex-project.org/lppl/.

\documentclass[10pt,a4paper,sans]{moderncv}        % possible options include font size ('10pt', '11pt' and '12pt'), paper size ('a4paper', 'letterpaper', 'a5paper', 'legalpaper', 'executivepaper' and 'landscape') and font family ('sans' and 'roman')
% moderncv themes
\moderncvstyle{banking}                            % style options are 'casual' (default), 'classic', 'oldstyle' and 'banking'
\moderncvcolor{blue}                               % color options 'blue' (default), 'orange', 'green', 'red', 'purple', 'grey' and 'black'
%\renewcommand{\familydefault}{\sfdefault}         % to set the default font; use '\sfdefault' for the default sans serif font, '\rmdefault' for the default roman one, or any tex font name
%\nopagenumbers{}                                  % uncomment to suppress automatic page numbering for CVs longer than one page

% character encoding
\usepackage[utf8]{inputenc}                       % if you are not using xelatex ou lualatex, replace by the encoding you are using
%\usepackage{CJKutf8}                              % if you need to use CJK to typeset your resume in Chinese, Japanese or Korean
\usepackage[italian]{babel}
% adjust the page margins
\usepackage[scale=0.85]{geometry}
%\setlength{\hintscolumnwidth}{3cm}                % if you want to change the width of the column with the dates
%\setlength{\makecvtitlenamewidth}{10cm}           % for the 'classic' style, if you want to force the width allocated to your name and avoid line breaks. be careful though, the length is normally calculated to avoid any overlap with your personal info; use this at your own typographical risks...
\usepackage{tabularx}
\usepackage{array}
\usepackage{eurosym}
\setlength{\extrarowheight}{0.1cm}


%%%%%%%%%%%%%%%%%%%%%%%%%%%%%%%%%%%%

% personal data
% \input{lettera}
% DEFINIZIONE COMANDI BASE CHE POI ANDRANNO RIDEFINITI


% COMANDI DA RIDEFINIRE

% Nome del documento
\newcommand{\documentName}{\_\_DOC\_NAME\_\_}

% Versione del documento
\newcommand{\documentVersion}{\_\_DOC\_VERSION\_\_}

% Data del documento
\newcommand{\documentDate}{\_\_YYYY-MM-GG\_\_}

% Editori del documento
\newcommand{\documentEditors}{
	\begin{itemize}[leftmargin=0cm] % Elimina lo spazio tra il simbolo dell' item e il suo contenuto
		\item[] \_\_Cognome1\_Nome1\_\_ % [] Rimuove il simbolo associato all'item
		\item[] \_\_Cognome2\_Nome2\_\_
	\end{itemize}
}

% Verificatore del documento
\newcommand{\documentVerifiers}{
	\begin{itemize}[leftmargin=0cm]
		\item[] \_\_Cognome1\_Nome1\_\_
	\end{itemize}
}

% Approvazione del documento
\newcommand{\documentApprovers}{
	\begin{itemize}[leftmargin=0cm]
		\item[] \_\_Cognome1\_Nome1\_\_
	\end{itemize}
}

% Lista di distribuzione del documento
\newcommand{\documentDistributionList}{
	\begin{itemize}[leftmargin=0cm]
		\item[] \groupName
		\item[] \commitNameM
		\item[] \commitNameS
		\item[] \proposerName
	\end{itemize}
}

% Uso del documento
\newcommand{\documentUsage}{{\_\_Interno/Esterno\_\_}}

% ====================================================================

% COMANDI DA NON RIDEFINIRE

% Nome del progetto
\newcommand{\projectName}{BDSMApp}

% Email ufficiale del gruppo
\newcommand{\groupEmail}{
	\textit{\href{mailto:mashup.unipd@gmail.com}{mashup.unipd@gmail.com}}
}
% Nome del gruppo
\newcommand{\groupName}{\emph{MashUp}}

% Ruoli del progetto
\newcommand{\roleProjectManager}{\emph{Responsabile di Progetto}}
\newcommand{\roleAdministrator}{\emph{Amministratore di Progetto}}
\newcommand{\roleAnalyst}{\emph{Analista}}
\newcommand{\roleProgrammer}{\emph{Programmatore}}
\newcommand{\roleDesigner}{\emph{Progettista}}
\newcommand{\roleVerifier}{\emph{Verificatore}}

% logo del gruppo
\newcommand{\logoHor}{\includegraphics[width=4cm,height=4cm]{../template/immagini/logo/logo\_orizzontale.pdf}}
\newcommand{\logoVerFirstPage}{\includegraphics[width=4cm,height=4cm]{../template/immagini/logo/logo\_verticale.pdf}}

% Referenti e commitenti (M = master, S = slave)
\newcommand{\proposerName}{\emph{Dott. David Santucci} - Zing}
\newcommand{\commitNameM}{\emph{Prof. Tullio Vardanega}}
\newcommand{\commitNameS}{\emph{Prof. Riccardo Cardin}}

% Abbreviazioni per richiamare il nome esteso dei diversi documenti da redarre
\newcommand{\glossDoc}{\emph{Glossario}}
\newcommand{\studio}{\emph{Studio di Fattibilità \currentVersion}}
\newcommand{\analisi}{\emph{Analisi dei Requisiti \currentVersion}}
\newcommand{\PdP}{\emph{Piano di Progetto \currentVersion}}
\newcommand{\PdQ}{\emph{Piano di Qualifica \currentVersion}}
\newcommand{\NdP}{\emph{Norme di Progetto \currentVersion}}

% TO DO - parte sul glossario e su alcune specifiche dello scopo del documento


\name{\includegraphics[width=4cm]{../template/images/logo/logo_verticale.pdf}}{}

%\title{}{}                               % optional, remove / comment the line if not wanted
%\address{street and number}{postcode city}{country}% optional, remove / comment the line if not wanted; the "postcode city" and and "country" arguments can be omitted or provided empty
%\phone[mobile]{+1~(234)~567~890}                   % optional, remove / comment the line if not wanted
%\phone[fixed]{+2~(345)~678~901}                    % optional, remove / comment the line if not wanted
%\phone[fax]{+3~(456)~789~012}                      % optional, remove / comment the line if not wanted
\email{info@mashup-unipd.it}                               % optional, remove / comment the line if not wanted
\homepage{www.mashup-unipd.it}                         % optional, remove / comment the line if not wanted
%\extrainfo{Il Responsabile di Progetto}                 % optional, remove / comment the line if not wanted
%\photo[64pt][0.4pt]{picture}                       % optional, remove / comment the line if not wanted; '64pt' is the height the picture must be resized to, 0.4pt is the thickness of the frame around it (put it to 0pt for no frame) and 'picture' is the name of the picture file
%\quote{Some quote}                                 % optional, remove / comment the line if not wanted

% to show numerical labels in the bibliography (default is to show no labels); only useful if you make citations in your resume
%\makeatletter
%\renewcommand*{\bibliographyitemlabel}{\@biblabel{\arabic{enumiv}}}
%\makeatother
%\renewcommand*{\bibliographyitemlabel}{[\arabic{enumiv}]}% CONSIDER REPLACING THE ABOVE BY THIS

% bibliography with mutiple entries
%\usepackage{multibib}
%\newcites{book,misc}{{Books},{Others}}
%----------------------------------------------------------------------------------
%            content
%----------------------------------------------------------------------------------
\begin{document}
%-----       letter       ---------------------------------------------------------
% recipient data
\recipient{Alla cortese attenzione del Committente:}{\textit{Prof. Tullio Vardanega, Prof. Riccardo Cardin\\ Università degli Studi di Padova \\ Dipartimento di Matematica \\ Via Trieste 63 \\ 35121, Padova}}
\date{\emph{2015-05-27}}
\opening{Egregi Professori,}
%\enclosure[Attached]{curriculum vit\ae{}}          % use an optional argument to use a string other than "Enclosure", or redefine \enclname
\makelettertitle

Con la presente, intendo comunicarVi ufficialmente la partecipazione del gruppo \groupName{} alla \textbf{Revisione di Qualifica}, per la realizzazione del prodotto da Voi commissionato, denominato \textit{BDMSApp} e proposto dalla ditta Zing S.r.l.\\
Il team è composto dalle seguenti persone: 
	\begin{center}
		\begin{tabularx}{0.8\linewidth}{|X |X |l|}
		\hline
		\hspace*{0.2cm} \textbf{Nome} & \hspace*{0.2cm} \textbf{Matricola} & \hspace*{0.2cm} \textbf{Email} \\ \hline

		\hspace*{0.2cm} Carnovalini Filippo & \hspace*{0.2cm}1048335 & \hspace*{0.2cm}\href{mailto:carnovalini.filippo@mashup-unipd.it}{carnovalini.filippo@mashup-unipd.it} \\ \hline

		\hspace*{0.2cm} Ceccon Lorenzo & \hspace*{0.2cm}1026118 & \hspace*{0.2cm}\href{mailto:ceccon.lorenzo@mashup-unipd.it}{ceccon.lorenzo@mashup-unipd.it} \\ \hline

		\hspace*{0.2cm} Cusinato Giacomo &\hspace*{0.2cm}1054137 & \hspace*{0.2cm}\href{mailto:cusinato.giacomo@mashup-unipd.it}{cusinato.giacomo@mashup-unipd.it} \\ \hline
 
		\hspace*{0.2cm} Faccin Nicola &\hspace*{0.2cm}1005999  & \hspace*{0.2cm}\href{mailto:faccin.nicola@mashup-unipd.it}{faccin.nicola@mashup-unipd.it} \\ \hline

		\hspace*{0.2cm} Roetta Marco & \hspace*{0.2cm}1001887 & \hspace*{0.2cm}\href{mailto:roetta.marco@mashup-unipd.it}{roetta.marco@mashup-unipd.it} \\ \hline

		\hspace*{0.2cm} Santacatterina Luca & \hspace*{0.2cm}619555 & \hspace*{0.2cm}\href{mailto:santacatterina.luca@mashup-unipd.it}{santacatterina.luca@mashup-unipd.it} \\ \hline

		\hspace*{0.2cm} Tesser Paolo & \hspace*{0.2cm}1026527 & \hspace*{0.2cm}\href{mailto:tesser.paolo@mashup-unipd.it}{tesser.paolo@mashup-unipd.it} \\
		\hline
		\end{tabularx}
	\end{center}
\vfill
I documenti allegati alla lettera descrivono i dettagli di qualifica del prodotto, di pianificazione, i requisiti individuati, la progettazione architetturale, la progettazione di dettaglio e il resoconto delle attività di verifica effetuate fino ad oggi.\\ 
\vfill
In particolare, sono allegati:
	\begin{itemize}
		\item \textit{Specifica Tecnica v2.0.0} ({\verb!esterni/specifica_tecnica_v2.0.0.pdf!});
		\item \textit{Definizione di Prodotto v2.0.0} ({\verb!esterni/definizione_di_prodotto_v2.0.0.pdf!});
		\item \textit{Analisi dei Requisiti v4.0.0} ({\verb!esterni/analisi_dei_requisiti_v4.0.0.pdf!});
		\item \textit{Piano di Progetto v5.0.0} ({\verb!esterni/piano_di_progetto_v5.0.0.pdf!});
		\item \textit{Piano di Qualifica v4.0.0} ({\verb!esterni/piano_di_qualifica_v4.0.0.pdf!});
		\item \textit{Glossario v3.0.0} ({\verb!esterni/glossario_v3.0.0.pdf!});
		\item \textit{Norme di Progetto v5.0.0} ({\verb!interni/norme_di_progetto_v5.0.0.pdf!});
	\end{itemize}
\noindent
Viene inoltre allegato il codice sorgente del prodotto, per quanto non sia completo, a riprova dell'avanzamento dei lavori descritto nel \textit{Piano di Progetto v5.0.0}. Si segnala che all'indirizzo {\color{blue}\url{http://mashup-unipd.github.io}} è disponibile un prototipo del prodotto dove è possibile provare le funzionalità fin'ora implementate dell'applicativo, previa registrazione. Tuttavia, data la necessità di procedere con la codifica non possiamo garantire che il servizio sia sempre disponibile.\\ 
\vfill
Il costo preventivato per la realizzazione del progetto è tuttora rimasto invariato rispetto al primo preventivo presentato ed ammonta ad \textbf{\euro{} 13731,00}, come specificato nel \textit{Piano di Progetto v5.0.0}. \newline Io e l'intero gruppo rimaniamo a Vostra completa disposizione per ogni chiarimento.
	\begin{flushright}
		\textit{Il responsabile,}\\ 
		\textit{Carnovalini Filippo}
	\end{flushright}
\end{document}


%% end of file `template.tex'.