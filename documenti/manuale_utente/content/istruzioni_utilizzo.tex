% =================================================================================================
% File:			istruzioni_utilizzo.tex
% Description:	Definisce la sezione relativa ad un capitolo del documento
% Created:		2015-04-21
% Author:		Santacatterina Luca
% Email:		santacatterina.luca@mashup-unipd.it
% =================================================================================================
% Modification History:
% Version		Modifier Date		Change											Author
% 0.0.1 		2015-05-24 			inizio stesura sezione capitolo					Santacatterina Luca
% =================================================================================================
%

% CONTENUTO DEL CAPITOLO
\section{Istruzioni per l'utilizzo} % (fold)
\label{sec:istruzioni_per_l_utilizzo}


	\subsection{Requisiti necessari} % (fold)
	\label{sec:requisiti_necessari}
		Prima di iniziare l'utilizzo del software \projectName{} leggere attentamente la corretta procedura d'utilizzo nel \docNameVersionMU.\newline
		Di seguito vengono riportati tutti i requisiti hardware\gloss{} e software\gloss{} necessari per poter utilizzare tutte le funzionalità del prodotto.\newline
		Per poter accedere all'applicazione \projectName, è necessario verificare che sia attiva una connessione ad Internet a banda larga\gloss{} in grado di collegarsi via URL\gloss{} e possedere le credenziali\gloss{} di accesso.


		\subsubsection{Requisiti hardware} % (fold)
		\label{sec:requisiti_hardware}
			L'applicazione è sviluppata per funzionare sia su dispositivi fissi che mobili quali tablet e smartphone.\newline
			Per usufruire di tutte le funzionalità si raccomanda l'utilizzo dell'applicativo tramite computer desktop e notebook.\newline
			È necessario operare da una postazione di lavoro dotata dei requisiti tecnici elencati di seguito:
			\begin{itemize}
				\item \textbf{Risoluzione}:
				\begin{itemize}
					\item per postazioni desktop risoluzioni maggiori di 1024x768;
					\item per postazioni mobile risoluzioni maggiori di 1024x600;
				\end{itemize}
			\end{itemize}
		% section Requisiti hardware (end)


		\subsubsection{Requisiti software} % (fold)
		\label{sec:requisiti_software}
			Affinchè l'applicazione possa funzionare correttamente tramite computer desktop è necessario disporre di uno dei seguenti browser\gloss{}:
			\begin{itemize}
				\item \textbf{Google Chrome}: versione 42 o superiore;
				\item \textbf{Internet Explorer}: versione 11 o superiore;
				\item \textbf{Mozilla Firefox}: versione 38 o superiore;
				\item \textbf{Safari}: versione 8.0.6 o superiore.
			\end{itemize}

			Affinchè l'applicazione possa funzionare correttamente tramite dispositivi mobile è necessario disporre di uno dei seguenti browser\gloss{}:
			\begin{itemize}
				\item \textbf{Google Chrome}:
				\begin{itemize}
					\item per Android versione 42 o superiore;
					\item per iOS versione 42 o superiore;
				\end{itemize}
			\item \textbf{Safari}: per iOS versione 8.0.6 o superiore.
			\end{itemize}

			Per una corretta visualizzazione è necessario:
			\begin{itemize}
				\item abilitare l'utilizzo dei cookies\gloss{}. Per le istruzioni fare riferimento alle specifiche funzionali di ciascun browser\gloss{};
				\item blocco dei pop-up\gloss{} disattivato. Per le istruzioni fare riferimento alle specifiche funzionali di ciascun browser\gloss{}.
			\end{itemize}
		% section Requisiti software (end)


		\subsubsection{Dati richiesti} % (fold)
		\label{sec:dati_richiesti}
			Per poter utilizzare l'applicazione è necessario fornire i seguenti dati:
			\begin{itemize}
				\item \textbf{Dati personali}: si richiedono dati personali per poter contattare in caso di guasti o notifiche;
				\item \textbf{Email}: si richiede un indirizzo di posta elettronica attivo su cui ricevere messaggi email.
			\end{itemize}
		% section Dati richiesti (end)


		\subsubsection{Installazione} % (fold)
		\label{sec:installazione}
			L'applicazione non prevede l'installazione di ulteriori componenti aggiuntivi per il regolare funzionamento.
			L'indirizzo per il collegamento all'applicazione è:\newline
			\begin{center}
				\url{http://mashup-unipd.github.io/}
			\end{center}
			Non appena la pagina si sarà caricata, sarà possibile effettuare l'accesso all'applicativo.
		% section installazione (end)


		\subsubsection{Comunicazioni malfunzionamenti} % (fold)
		\label{sec:installazione}
			Se durante l'utilizzo dell'applicazione \projectName{} si dovessero riscontrare problemi o malfunzionamenti, si invitano gli utilizzatori a segnalare l'errore inviando una email all'indirizzo:
			\begin{center}
				\url{info@mashup-unipd.it}
			\end{center}
			Nella mail di segnalazione si dovranno riportare:
			\begin{itemize}
				\item username ed email forniti in fase di registrazione dell'applicazione (ATTENZIONE: non è richiesta la password di autenticazione\gloss{});
				\item versione di browser\gloss{} e sistema operativo\gloss{} utilizzato. Nel caso di dispositivi mobile il modello utilizzato;
				\item descrizione dettagliata del problema riscontrato e relativi errori visualizzati;
			\end{itemize}
			Prima di aprire una richiesta di assistenza si consiglia di effettuare operazioni di pulizia dei file cookies\gloss{}. Per le istruzioni fare riferimento alle specifiche funzionali di ciascun browser\gloss{}.
		% section comunicazioni malfunzionamenti (end)


	% section Requisiti necessari (end)

% section Istruzioni per l'utilizzo (end)
