% =================================================================================================
% File:			introduzione.tex
% Description:	Defiinisce la sezione relativa al capitolo introduttivo del documento
% Created:		2015-04-21
% Author:		Tesser Paolo
% Email:		tesser.paolo@mashup-unipd.it
% =================================================================================================
% Modification History:
% Version		Modifier Date		Change											Author
% 0.0.1 		2015-04-21 			creato scheletro doc e primo abbozzo			Tesser Paolo
% =================================================================================================
%

% CONTENUTO DEL CAPITOLO

\section{Introduzione} % (fold)
\label{sec:introduzione}


	\subsection{Scopo del documento} % (fold)
	\label{sub:scopo_del_documento}
		Questo documento ha come scopo quello di illustrare le procedure da seguire per svolgere le operazioni previste per l'utente normale relative al prodotto \projectName. All'utilizzatore non è chiesta nessuna particolare conoscenza informatica. Alcune operazioni richiedono però che esso abbia dimestichezza con i social network e con le possibilità che offrono.
	% subsection scopo_del_documento (end)


	\subsection{Scopo del prodotto} % (fold)
	\label{sub:scopo_del_prodotto}
		\productScope
	% subsection scopo_del_prodotto (end)


	\subsection{Glossario} % (fold)
	\label{sub:glossario}
		Al fine di evitare ogni ambiguità relativa al linguaggio usato nei documenti viene allegato il glossario in appendice A.\newline
		Esso ha lo scopo di definire ed analizzare tutti i termini tecnici del progetto e di fugare eventuali ambiguità fornendo un'accurata descrizione.\newline
		Tutte le occorrenze di tali termini nei documenti verranno contrassegnate con una ``G'' a pedice.
	% subsection glossario (end)


	\subsection{Riferimenti} % (fold)
	\label{sub:riferimenti}


		\subsubsection{Normativi} % (fold)
		\label{ssub:normativi}
			\begin{itemize}
				\item \textbf{Analisi dei Requisiti}: \docNameVersionAdR
				\item \textbf{Specifica Tecnica}: \docNameVersionST
			\end{itemize}
		% subsubsection normativi (end)


	% subsection riferimenti (end)

% section introduzione (end)