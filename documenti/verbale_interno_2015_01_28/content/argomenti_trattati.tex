% =================================================================================================
% File:			argomenti_trattati.tex
% Description:	Defiinisce la sezione relativa a ...
% Created:		2015-01-28
% Author:		Tesser Paolo
% Email:		tesser.paolo@mashup-unipd-it
% =================================================================================================
% Modification History:
% Version		Modifier Date		Change											Author
% 0.0.1 		2015-01-28 			iniziata stesura documento						Tesser Paolo
% =================================================================================================
% 0.0.2			2015-03-20			terminata stesura completanto parti mancanti	Tesser Paolo
% =================================================================================================
%

% CONTENUTO DEL CAPITOLO


\section{Argomenti trattati} % (fold)
\label{sec:argomenti_trattati}
Di seguito vengono riportati i risultati riscontrati dalla discussione tra i componenti in merito all'ordine del giorno. \newline
Alcuni di essi saranno fonte di requisiti e di casi d'uso che verranno descritti in maniera più dettagliata nel documento di \docNameVersionAdR.
	\begin{enumerate}
		\item Ad ogni membro del gruppo è stato assegnato il compito di analizzare il documento sul quale ha lavorato maggiormente ed estrapolarne i contenuti principali, trovando delle immagini adatte ad esprimere quei concetti;
		\item Si è scelto di utilizzare Python come linguaggio per il back-end in quanto è quello maggiormente supporto dalla piattaforma di Google, che mettete a disposizione un maggior numero di API. \'E inoltre il linguaggio utilizzato e consigliato dal proponente. \newline
		Per quanto riguarda il front-end, per rispettare uno dei requisiti pervenuti durante la precedente fase, si è deciso di adottare AngularJS, che incorpora inoltre la libreria Bootstrap, utile per una interfaccia responsive. \newline
		Entrambe queste scelte saranno fonte di requisiti di vincolo;
		\item
			\begin{itemize}
				\item Python: il codice e la sua documentazione dovranno seguire gli stili descritti nella Google Python Style Guide presente al seguente link: \url{https://google-styleguide.googlecode.com/svn/trunk/pyguide.html};
				\item AngularJS: il codice e la documentazione dovranno seguire gli stili descritti nella AngularJS Style Guide presente al seguente link: \url{https://google-styleguide.googlecode.com/svn/trunk/angularjs-google-style.html};
				\item Javascript: il codice e la sua documentazione dovranno seguire gli stili descritti nella Google JavaScript Style Guide presente al seguente link: \url{http://google-styleguide.googlecode.com/svn/trunk/javascriptguide.xml?showone=Method_and_property_definitions#Method_and_property_definitions};
			\end{itemize}
			\noindent
		Queste scelte saranno fonte di requisiti di vincolo;

		\item Si è scelto di utilizzare PyCharm come IDE sia per quanto riguarda lo sviluppo del back-end sia per il front-end in quanto offre già il supporto ad AngularJS e permette di configurare dei progetti provenienti dalla Google App Engine;
		\item Si sono analizzati i diversi avanzamenti che dovranno essere effettuati in questa fase, come la stesura sulle \docNameVersionNdP{} dei linguaggi e strumenti scelti, o l'avanzamento nella stesura dei requisiti che emergeranno;
		\item Si è scelto di effettuare l'incontro con il proponente in data 2015-02-18, in un periodo meno concentrato a livello di esami per la maggior parte dei componenti.
	\end{enumerate}
% section argomenti_trattati (end)
