% =================================================================================================
% File:			capitolo_3.tex
% Description:	Definisce il capitolo che descrive generalmente il prodotto per il commitente
% Created:		2014/12/10
% Author:		Roetta Marco
% Email:		roetta.marco@mashup-unipd.it
% =================================================================================================
% Modification History:
% Version		Modifier Date		Change											Author
% 0.0.1 		2014/12/10 			aggiunta sezione e iniziata stesura				Roetta Marco
% =================================================================================================
% Modification History:
% Version		Modifier Date		Change											Author
% 0.0.2 		2015/01/12 			Aggiunto elenco Casi d'uso ( 1.1->1.7 )			Roetta Marco
% =================================================================================================
% Modification History:
% Version		Modifier Date		Change											Author
% 0.0.3 		2015/01/12 			Aggiunta descrizione ai casi principali			Roetta Marco
% =================================================================================================
% Modification History:
% Version		Modifier Date		Change											Author
% 0.0.4 		2015/01/13 			Aggiunta descrizione a tutti i Casi d'uso		Roetta Marco
% =================================================================================================
%

% CONTENUTO DEL CAPITOLO

\section{Casi d'uso}

\subsection{UC 1: Caso d’uso pubblico}

\begin{itemize}

\item \textbf{Attori Coinvolti:}
Utente sconosciuto: utente non autenticato che accede al servizio.
Utente autenticato: utente autenticato che ha acceduto al servizio.
Utente amministratore: utente autenticato che ha acceduto al servizio e che dispone dei permessi per visualizzare tutte le aree.

\item \textbf{Descrizione:}
Questa è la schermata che viene mostrata ad un utente che non si è ancora autenticato al sistema.
Un utente sconosciuto può autenticarsi. 
Un utente autenticato può vedere le statistiche e modificare la propria password, può visualizzare, creare e modificare le proprie viste e le proprie ricette. 
Un utente amministratore può fare tutte le operazioni di un utente autenticato, più ha la possibilità di aggiungere un utente, eliminare un utente e tutte le sue viste e modificare i privilegi di un altro utente.

\item \textbf{Possibili Errori:}
Viene visualizzato un messaggio di errore nel caso le informazioni di accesso inserite siano errate.

\item \textbf{Post-condizioni:}
Il servizio ha erogato correttamente le funzionalità richieste dall'utente.

\end{itemize}

\subsection{UC 1.1: Autenticazione utente}

\begin{itemize}

\item \textbf{Attori Coinvolti:}
Utente: utente non autenticato che accede al servizio.

\item \textbf{Pre Condizione:}
L’utente che accede alla pagina di accesso è sconosciuto al sistema

\item \textbf{Descrizione:}
L’utente può autenticarsi per poter accedere ai servizi offerti dalla piattaforma

\item \textbf{Post Condizione:}
L’utente è riconosciuto dal sistema

\item \textbf{Possibili Errori:}
Messaggio di errore in caso di informazioni di accesso errate

\end{itemize}

\subsubsection{UC 1.1.1: Inserimento dati di accesso}

\subsubsection{UC1.1.1.1: inserimento utente}

\subsubsection{UC1.1.1.2: inserimento password}

\subsubsection{UC1.1.1.3: conferma login}

\subsubsection{UC1.1.2: Autenticazione fallita}



\subsection{UC 1.2: Visione informazioni utente e statistiche}

\textbf{Attori Coinvolti:}
Utente autenticato: utente autenticato che ha acceduto al servizio
Utente amministratore autenticato: utente autenticato in grado di accedere a tutte le aree del servizio

\textbf{Pre Condizione:}
L'utente deve aver fatto l'accesso nell'apposita pagina.

\textbf{Descrizione:}
Gli utenti autenticati possono vedere una serie di informazioni relative alla loro attività, come la data dell’ultimo accesso effettuato e il numero di ricette e viste attive.

\textbf{Post Condizione:}
Le informazioni richieste dall'utente sono state fornite

\textbf{Possibili Errori:}
In caso di autenticazione scaduta verrà mostrata nuovamente la pagina di accesso.


\subsubsection{UC 1.2.1: Apertura pannello informazioni}

\subsubsection{UC 1.2.1.1: Visualizza informazioni utente}

\subsubsection{UC 1.2.1.2: Visualizza Statistiche utente}

\subsubsection{UC 1.3: Visualizzazione, creazione e modifica viste}

\textbf{Attori Coinvolti:}
Utente autenticato: utente autenticato che ha acceduto al servizio
Utente amministratore autenticato: utente autenticato in grado di accedere a tutte le aree del servizio

\textbf{Pre Condizione:}
L’utente deve aver fatto l’accesso nell’apposita pagina.

\textbf{Descrizione:}
Gli utenti autenticati possono vedere in questa pagina le viste e i dati ad esse associati. 
Si rende disponibile per ciascuna vista un pannello per modificarne i parametri. 
Si può creare una nuova vista. 
Si può eliminare una vista esistente.

\textbf{Post Condizione:}
Le informazioni richieste dall’utente sono state fornite
Le impostazioni modificate sono state salvate nel sistema.
Le informazioni aggiunte sono state salvate nel sistema.

\textbf{Possibili Errori:}    
In caso di autenticazione scaduta verrà mostrata nuovamente la pagina di accesso.

\subsubsection{UC 1.3.1: Visualizza viste}

\subsubsection{UC 1.3.1.1: Apertura elenco viste}


\subsubsection{UC 1.3.2: Aggiungi una vista}

\subsubsection{UC 1.3.2.1: Apertura pannello inserimento nuova lista}

\subsubsection{UC 1.3.2.2: Inserimento delle informazioni}

\subsubsection{UC 1.3.2.3: Conferma delle informazioni inserite}

\subsubsection{UC 1.3.3: Inserimetno fallito - Eccezione}

\subsection{UC 1.3.4: Modifica una vista}

\subsubsection{UC 1.3.4.1: selezione di una vista}

\subsubsection{UC 1.3.4.2: apertura pannello modifica}

\subsubsection{UC 1.3.4.3: inserimento dei dati}

\subsubsection{UC 1.3.4.4: conferma dei dati inseriti}

<<extend>>

\subsubsection{UC 1.3.5: Errore nella modifica dei dati}

\subsubsection{UC 1.3.6: Eliminazione di una vista}

\subsubsection{UC 1.3.6.1: selezione di una vista}

\subsubsection{UC 1.3.6.2: conferma eliminazione di una vista}

\subsection{UC 1.4: Visualizzazione, creazione e modifica ricette e schedulazioni associate}

\textbf{Attori Coinvolti:}
Utente autenticato: utente autenticato che ha acceduto al servizio
Utente amministratore autenticato: utente autenticato in grado di accedere a tutte le aree del servizio

\textbf{Pre Condizione:}
        L’utente deve aver fatto l’accesso nell’apposita pagina.

\textbf{Descrizione:}
        Gli utenti autenticati possono vedere in questa pagina le ricette e i dati ad esse associati. 
Si rende disponibile per ciascuna vista un pannello per modificarne i parametri. 
Si può creare una nuova ricetta. 
Si può eliminare una ricetta esistente.
Si può modificare la cadenza con cui i dati di una ricetta vengono aggiornati automaticamente.

\textbf{Post Condizione:}
        Le informazioni richieste dall’utente sono state fornite
Le impostazioni modificate sono state salvate nel sistema.
Le informazioni aggiunte sono state salvate nel sistema.

\textbf{Possibili Errori:}
    In caso di autenticazione scaduta verrà mostrata nuovamente la pagina di accesso.

\subsubsection{UC1.4.1: Visualizza ricette}
\subsubsection{UC1.4.1.1: Apertura elenco ricette}
\subsubsection{UC1.4.2: Aggiungi una vista}
\subsubsection{UC1.4.2.1: Apertura pannello inserimento nuova ricetta}
\subsubsection{UC1.4.2.2: Inserimento delle informazioni }
\subsubsection{UC1.4.2.3: Conferma delle informazioni inserite}
<<extend>>
\subsubsection{UC1.4.3: Inserimetno fallito - Eccezione}
\subsubsection{UC1.4.4: Modifica una ricetta}
\subsubsection{UC1.4.4.1: selezione di una ricetta}
\subsubsection{UC1.4.4.2: apertura pannello modifica ricetta}
\subsubsection{UC1.4.4.3: inserimento dei dati}
\subsubsection{UC1.4.4.4: conferma dei dati inseriti}
<<extend>>
\subsubsection{UC1.4.5: Errore nella modifica dei dati}
\subsubsection{UC1.4.6: Modifica di una schedulazione associata}
\subsubsection{UC1.4.6.1: selezione di una ricetta}
\subsubsection{UC1.4.6.2: apertura pannello modifica schedulazione}
\subsubsection{UC1.4.6.3: inserimento dei dati}
\subsubsection{UC1.4.6.4: conferma dei dati inseriti}
<<extend>>
\subsubsection{UC1.4.7: Errore nella modifica dei dati}

\subsubsection{UC1.4.8: Eliminazione di una ricetta}

\textbf{Attori Coinvolti:}
Utente autenticato: utente autenticato che ha acceduto al servizio
Utente amministratore autenticato: utente autenticato in grado di accedere a tutte le aree del servizio

\textbf{Pre Condizione:}
        L’utente deve aver fatto l’accesso nell’apposita pagina.

Descrizione:
L’utente può eliminare una ricetta e tutti i dati ad essa associati.
Si possono eliminare solo le ricette non più utilizzate in una o più viste.    

\textbf{Post Condizione:}
        Le informazioni richieste dall’utente sono state fornite
La ricetta(e) desiderata(e) e i dati ad essa(e) associati è(sono) stati cancellati dal sistema.

\textbf{Possibili Errori:}
In caso di autenticazione scaduta verrà mostrata nuovamente la pagina di accesso.
In caso di ricetta in uso o in fase di aggiornamento a causa di una schedulazione viene visualizzato un messaggio di errore e nessun dato viene alterato da questa oeprazione.


\subsubsection{UC1.4.8.1: selezione di una ricetta}
\subsubsection{UC1.4.8.2: conferma eliminazione di una ricetta}

\subsubsection{UC1.4.9: Modifica Layout viste }
\subsubsection{UC1.4.9.1: apertura pannello modifica layout}
\subsubsection{UC1.4.9.2: selezione della visualizzazione}
\subsubsection{UC1.4.9.3: conferma o annulla modifiche}

\subsection{UC1.5: Modifica Password di accesso}

\textbf{Attori Coinvolti:}
Utente autenticato: utente autenticato che ha acceduto al servizio
Utente amministratore autenticato: utente autenticato in grado di accedere a tutte le aree del servizio

\textbf{Pre Condizione:}
    L’utente dispone di una password di accesso valida e risulta autenticato al servizio.

\textbf{Descrizione:}
    L’utente può cambiare la password attualmente in uso inserendo la vecchia password, la nuova password, la conferma della nuova password e confermando la modifica.

\textbf{Post Condizione:}
    La nuova password per l’utente è attiva, la vecchia password non è più valida.

\textbf{Possibili Errori:}
Messaggio di errore in caso di informazioni inserite errate oppure la nuova password non soddisfa i criteri di sistema indicati.

\subsubsection{UC1.5.1: Inserimento di una nuova password}
\subsubsection{UC1.5.1.1: inserimento password}
\subsubsection{UC1.5.1.2: conferma inserimento password }
\subsubsection{UC1.5.1.3: conferma}
extend
\subsubsection{UC1.5.2: Password inserita non conforme alle regole}

\subsection{UC1.6: Logout}
\subsubsection{UC1.6.1: pressione tasto logout}
\subsubsection{UC1.6.2: conferma della volonta di fare logout}

\subsection{UC1.7: Amministrazione utenti}
\subsubsection{UC1.7.1: visualizzazione elenco utenti}
\subsubsection{UC1.7.2: modifica permessi utente}
\subsubsection{UC1.7.2.1: apertura elenco utenti}
\subsubsection{UC1.7.2.2: selezione modifica su un utente}
\subsubsection{UC1.7.2.3: selezione nuovo ruolo}
\subsubsection{UC1.7.2.4: conferma delle modifiche}
\subsubsection{UC1.7.3: crea un nuovo utente}
\subsubsection{UC1.7.3.1: apertura pannello inserimento}
\subsubsection{UC1.7.3.2: Inserimento dei parametri}
\subsubsection{UC1.7.3.3: conferma dei dati inseriti}
<<extend>>
\subsubsection{UC1.7.4: dati inseriti errati}
( nome esistente, caratteri speciali )

\subsubsection{UC1.7.5: elimina un utente}
\subsubsection{UC1.7.5.1: selezione di un utente}
\subsubsection{UC1.7.5.2: pressione pulsante elimina}
\subsubsection{UC1.7.5.3: conferma eliminazione}
<<extend>> 
\subsubsection{UC1.7.6: errore nell’eliminazione}
( risorsa occupata, utente loggato, viste in aggiornamento )





