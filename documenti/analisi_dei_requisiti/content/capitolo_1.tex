% =================================================================================================
% File:			capitolo_1.tex
% Description:	Definisce il capitolo di introduzione
% Created:		2014/12/10
% Author:		Roetta Marco
% Email:		roetta.marco@mashup-unipd.it
% =================================================================================================
% Modification History:
% Version		Modifier Date		Change											Author
% 0.0.1 		2014/12/10 			aggiunto sezione introduzione					Roetta Marco
% =================================================================================================
%

% CONTENUTO DEL CAPITOLO

\section{Introduzione}

\subsection{Scopo del documento}
Il presente documento ha lo scopo di descrivere in modo esaustivo i requisiti individuati per il prodotto. Tali requisiti sono stati identificati dall’analisi del capitolato C1 e il successivo incontro con il proponente.

\subsection{Scopo del prodotto}
Lo scopo del prodotto è di creare infrastruttura che permetta di interrogare big data dai social Facebook, Twitter e Instagram. Tale infrastruttura dovrà offrire una parte dedicata alla consultazione e interrogazione dei dati raccolti tramite interfaccia web e una seconda parte che dovrà offrire dei servizi REST interrogabili. L'interfaccia dovrà fornire una personalizzazione e una scelta dei dati da visualizzare per utente.

\subsection{Glossario}
Al fine di evitare ogni ambiguità relativa al linguaggio e ai termini utilizzati nei documenti formali, viene allegato il “Glossario v1 ”. In questo documento vengono definiti e descritti tutti i termini con un significato non comune. Per rendere più facile la lettura, i termini saranno posti in corsivo e accanto a questi ci sarà una ‘g’, compresa tra
parentesi quadre, a pedice (esempio: Glossario \ped{[g]}).

\subsection{Riferimenti}
\subsubsection{Normativi}
• Norme di Progetto: “Norme di Progetto v1”;
• Capitolato d’appalto C1: BDSMApp: Big Data Social Monitoring App. Reperibile
all’indirizzo: http://www.math.unipd.it/~tullio/IS-1/2014/Progetto/C1.pdf

\subsubsection{Informativi}
• Studio di Fattibilità: “Studio di Fattibilità v1.00 ”;
• Software Engineering - Ian Sommerville - 9th Edition 2010
– Chapter 4: Requirements engineering.
• IEEE 830-1998:.