% =================================================================================================
% File:			capitolo_3.tex
% Description:	Definisce il capitolo che descrive generalmente il prodotto per il commitente
% Created:		2014/12/10
% Author:		Roetta Marco
% Email:		roetta.marco@mashup-unipd.it
% =================================================================================================
% Modification History:
% Version		Modifier Date		Change											Author
% 0.0.1 		2014/12/10 			aggiunta sezione e iniziata stesura				Roetta Marco
% =================================================================================================
%

% CONTENUTO DEL CAPITOLO

\section{Requisiti}

A seguire sono riportati tutti i requisiti individuati dal gruppo. Essi sono stati ricavati dal capitolato proposto, dai casi d'uso, da necessità interne o in seguito agli incontri col proponente. Ogni requisito è identificato da un codice univoco a seconda dell'importanza, del tipo e della gerarchia a cui appartiene.
I requisiti saranno divisi in quattro tabelle distinte a seconda della tipologia, ognuna composta da tre colonne contententi i seguenti attributi:
\begin{itemize}
	\item \textbf{requisito}: il nominativo identificativo del requisito espresso tramite il formalismo indicato a seguire.
	\item \textbf{descrizione}: breve descrizione del requisito.
	\item \textbf{fonti}: rappresentate da uno o più dei seguenti valori:
	\begin{itemize}
		\item \emph{Capitolato}: requisito ricavato direttamente del capitolato.
		\item \emph{Casi d'uso}: requisito ricavato da uno o più casi d'uso. In questo caso sarà indicato il codice di quest'ultimo/i.
		\item \emph{Interno}: requisito ricavato in seguito ad un'analisi o una necessità interna al gruppo.
	\end{itemize}
\end{itemize}

La classificazione dei requisiti sarà riportata tramite il seguente formalismo:
\begin{center}
	R[importanza][tipo][codice]
\end{center}
\begin{itemize}
	\item \textbf{importanza}: può assumere uno tra i seguenti valori:
	\begin{itemize}
		\item \emph{0}: requisito obbligatorio;
		\item \emph{1}: requisito desiderabile;
		\item \emph{2}: requisito opzionale;
	\end{itemize}
	\item \textbf{tipo}: può assumere uno tra i seguenti valori:
	\begin{itemize}
		\item \emph{F}: requisito funzionale;
		\item \emph{Q}: requisito di qualità;
		\item \emph{P}: requisito prestazionale;
		\item \emph{V}: requisito vincolo;
	\end{itemize}
	\item \textbf{codice}: identificativo del requisito attribuito in modo gerarchico.
\end{itemize}


\subsection{Requisiti funzionali}

\begin{center}

	\def\arraystretch{1.5}
	\bgroup
	\begin{longtable}{| p{2cm} | p{7cm} | p{2cm} |}

		\hline
		\textbf{Requisito} & \textbf{Descrizione} & \textbf{Fonti} \\
		\hline

		R0F1  &  L'utente non autenticato può effettuare la registrazione al servizio  &  Interno \newline UC 1.8 \\
		\hline
		R0F1.1  &  La registrazone richiede l'inserimento dello username  &  Interno \newline UC 1.8.1 \\
		\hline
		R0F1.1.1  &  Lo username è univoco in tutto il sistema  &  Interno \newline \\
		\hline
		R0F1.1.2  &  Lo username è una striga contetente esclusivamente caratteri alfanumerici e underscores (\_)  &  Interno \\
		\hline
		R0F1.1.3  &  La lunghezza dello username è contenuta tra i 4 ed i 30 caratteri &  Interno \\
		\hline
		R0F1.3  &  La registrazione richiede l'inserimento della password  &  Interno \newline UC 1.8.3 \\
		\hline
		R0F1.3.1  &  La password non può contenere lo username  &  Interno \\
		\hline
		R0F1.3.2  &  La password è una striga contetente esclusivamente caratteri alfanumerici e speciali ( ? ! \$ + - / . , \textless \textgreater $@$ \textasciitilde \^{} \_ )  & Interno \\
		\hline
		R0F1.3.3  &  La lunghezza minima della password è di 6 caratteri  &  Interno \\
		\hline
		R0F1.4  &  La registrazione richiede la conferma della password & Interno \\
		\hline
		R0F1.4.1  &  La password di conferma deve coincidere con la password inserita precedentemente  &  Interno \\
		\hline
		R0F1.5  &  Il sistema avverte l'untente in fase di compilazione della form nel caso i dati inseriti in quel momento non risultino conformi alle norme del sistema  &  Interno \\
		\hline
		R0F1.6  &  L'utente non autenticato può confermare la registrazione tramite l'apposito pulsante & Interno \newline UC 1.8.4 \\
		\hline
		R0F1.7  &  Il sistema visualizza un messaggio di errore nel caso i dati inseriti risultino non conformi alle norme imposte dal sistema  &  UC 1.8.5 \\
		\hline
		R0F1.8  &  Nel caso la registrazione sia andata a buon fine, il sistema aggiunge i dati dell'utente nel rispettivo database  &  Interno \\
		\hline
		R0F2  &  L'utente non autenticato può effettuare il login tramite l'apposto form di autenticazione  &  Interno \newline UC 1.1 \\
		\hline
		R0F2.1  &  L'autenticazione richiede l'inserimento dello username  &  Interno \newline UC 1.1.1 \\
		\hline
		R0F2.2  &  L'autenticazione richiede l'inserimento della password  &  Interno \newline UC 1.1.2 \\
		\hline
		R0F2.3  &  L'utente può completare l'autenticazione tramite l'apposito pulsante di conferma  &  Interno \newline UC 1.1.3 \\
		\hline
		R0F2.4  &  Il sistema visualizza un messaggio di errore nel caso i dati inserito dall'untente risultio errati  &  Interno \newline UC 1.1.4 \\
		\hline
		R0F2.5  &  Il sistema aggiorna la pagina principale permettendo all'utente autenticato di accedere alle diverse aree del del servizio  &  Interno  \\
		\hline



	\end{longtable}
	\egroup
\end{center}


\subsection{Requisiti prestazionali}

\subsection{Requisiti di qualità}

\subsection{Requisiti di vincolo}

\begin{center}

	\def\arraystretch{1.5}
	\bgroup
	\begin{longtable}{| p{2cm} | p{7cm} | p{2cm} |}

		\hline
		\textbf{Requisito} & \textbf{Descrizione} & \textbf{Fonti} \\
		\hline

		R1V1  &  Il sistema sfrutterà gli strumenti offerti da Google Cloud Platform  &  Capitolato \newline Verbale 2015/01/14 \\
		\hline
		R1V1.1  &  Il sistema sfrutterà Google App Engine  &  Capitolato \newline Verbale 2015/01/14 \\
		\hline
		R1V1.2  &  Il sistema sfrutterà Google Cloud Datastore  &  Capitolato \newline Verbale 2015/01/14 \\
		\hline
		R1V1.3  &  Il sistema sfrutterà gli strumenti offerti da Google Cloud Platform  &  Capitolato \newline Verbale 2015/01/14 \\
		\hline
		R1V2  &  Il linguaggio di programmazione principalmente usato sarà Python  &  Capitolato \newline Verbale 2015/01/14 \\
		\hline

	\end{longtable}
	\egroup
\end{center}

\subsection{Riepilogo}

\subsection{Requisiti accettati}
