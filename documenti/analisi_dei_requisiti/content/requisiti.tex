% =================================================================================================
% File:			capitolo_3.tex
% Description:	Definisce il capitolo che descrive generalmente il prodotto per il commitente
% Created:		2014/12/10
% Author:		Roetta Marco
% Email:		roetta.marco@mashup-unipd.it
% =================================================================================================
% Modification History:
% Version		Modifier Date		Change											Author
% 0.0.1 		2014/12/10 			aggiunta sezione e iniziata stesura				Roetta Marco
% =================================================================================================
%

% CONTENUTO DEL CAPITOLO

\section{Requisiti}

A seguire sono riportati tutti i requisiti individuati dal gruppo. Essi sono stati ricavati dal capitolato proposto, dai casi d'uso, da necessità interne o in seguito agli incontri col proponente. Ogni requisito è identificato da un codice univoco a seconda dell'importanza, del tipo e della gerarchia a cui appartiene. 
I requisiti saranno divisi in quattro tabelle distinte a seconda della tipologia, ognuna composta da tre colonne contententi i seguenti attributi:
\begin{itemize}
\item \textbf{requisito}: il nominativo identificativo del requisito espresso tramite il formalismo indicato a seguire.
\item \textbf{descrizione}: breve descrizione del requisito.
\item \textbf{fonti}: rappresentate da uno o più dei seguenti valori:
	\begin{itemize}
	\item \emph{Capitolato}: requisito ricavato direttamente del capitolato.
	\item \emph{Casi d'uso}: requisito ricavato da uno o più casi d'uso. In questo caso sarà indicato il codice di quest'ultimo/i.
	\item \emph{Interno}: requisito ricavato in seguito ad un'analisi o una necessità interna al gruppo.
	\end{itemize}
\end{itemize}

La classificazione dei requisiti sarà riportata tramite il seguente formalismo:
\begin{center} 
R[importanza][tipo][codice]
\end{center}
\begin{itemize}
\item \textbf{importanza}: può assumere uno tra i seguenti valori:
	\begin{itemize}
	\item \emph{0}: requisito obbligatorio;
	\item \emph{1}: requisito desiderabile;
	\item \emph{2}: requisito opzionale;
	\end{itemize}
\item \textbf{tipo}: può assumere uno tra i seguenti valori:
	\begin{itemize}
	\item \emph{F}: requisito funzionale;
	\item \emph{Q}: requisito di qualità;
	\item \emph{P}: requisito prestazionale;
	\item \emph{V}: requisito vincolo;
	\end{itemize}
\item \textbf{codice}: identificativo del requisito attribuito in modo gerarchico.
\end{itemize}

\subsection{Requisiti funzionali}

\subsection{Requisiti prestazionali}

\begin{center}
\begin{small}
	\begin{tabbing}
		TODO
	\end{tabbing}
\end{small}
\end{center}

\subsection{Requisiti di qualità}

\subsection{Requisiti di vincolo}

\subsection{Tracciamento Requisiti-Fonti}

\subsection{Tracciamento Fonti-Requisiti}

\subsection{Riepilogo}

\subsection{Requisiti accettati}
