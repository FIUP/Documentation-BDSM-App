% =================================================================================================
% File:			requisiti.tex
% Description:	Definisce il capitolo che contiene i requisiti in forma tabellare
% Created:		2014/12/10
% Author:		Roetta Marco
% Email:		roetta.marco@mashup-unipd.it
% =================================================================================================
% Modification History:
% Version		Modifier Date		Change											Author
% 0.0.1 		2014/12/10 			aggiunta sezione e iniziata stesura				Roetta Marco
% =================================================================================================
% Version		Modifier Date		Change											Author
% 0.0.2 		2015/01/12 			aggiunte tabelle								Carnovalini Filippo
% =================================================================================================
% Version		Modifier Date		Change											Author
% 0.0.3 		2015/01/16 			requisiti prestazionali di qualità e di vincolo	Carnovalini Filippo
% =================================================================================================
% Version		Modifier Date		Change											Author
% 0.0.4 		2015/01/17 			requisiti tratti dagli UC 1.2, 1.5 e 1.6		Carnovalini Filippo
% =================================================================================================
% Version		Modifier Date		Change											Author
% 0.0.5 		2015/01/18 			requisiti tratti dagli UC 1.3 e 1.7				Carnovalini Filippo
% =================================================================================================

% CONTENUTO DEL CAPITOLO

\section{Requisiti}

A seguire sono riportati tutti i requisiti individuati dal gruppo. Essi sono stati ricavati dal capitolato proposto, dai casi d'uso, da necessità interne o in seguito agli incontri col proponente. Ogni requisito è identificato da un codice univoco a seconda dell'importanza, del tipo e della gerarchia a cui appartiene.
I requisiti saranno divisi in quattro tabelle distinte a seconda della tipologia, ognuna composta da tre colonne contententi i seguenti attributi:
\begin{itemize}
	\item \textbf{requisito}: il nominativo identificativo del requisito espresso tramite il formalismo indicato a seguire.
	\item \textbf{descrizione}: breve descrizione del requisito.
	\item \textbf{fonti}: rappresentate da uno o più dei seguenti valori:
	\begin{itemize}
		\item \emph{Capitolato}: requisito ricavato direttamente del capitolato.
		\item \emph{Casi d'uso}: requisito ricavato da uno o più casi d'uso. In questo caso sarà indicato il codice di quest'ultimo/i.
		\item \emph{Interno}: requisito ricavato in seguito ad un'analisi o una necessità interna al gruppo.
	\end{itemize}
\end{itemize}

La classificazione dei requisiti sarà riportata tramite il seguente formalismo:
\begin{center}
	R[importanza][tipo][codice]
\end{center}
\begin{itemize}
	\item \textbf{importanza}: può assumere uno tra i seguenti valori:
	\begin{itemize}
		\item \emph{0}: requisito obbligatorio;
		\item \emph{1}: requisito desiderabile;
		\item \emph{2}: requisito opzionale;
	\end{itemize}
	\item \textbf{tipo}: può assumere uno tra i seguenti valori:
	\begin{itemize}
		\item \emph{F}: requisito funzionale;
		\item \emph{Q}: requisito di qualità;
		\item \emph{P}: requisito prestazionale;
		\item \emph{V}: requisito vincolo;
	\end{itemize}
	\item \textbf{codice}: identificativo del requisito attribuito in modo gerarchico.
\end{itemize}


\subsection{Requisiti funzionali}

\begin{center}

	\def\arraystretch{1.5}
	\bgroup
	\begin{longtable}{| p{2cm} | p{8cm} | p{2cm} |}

		\hline
		\textbf{Requisito} & \textbf{Descrizione} & \textbf{Fonti} \\
		\hline

		R0F1  &  L'utente non autenticato può effettuare la registrazione al servizio  & UC 1.8 \\
		\hline
		R0F1.1  &  La registrazone richiede l'inserimento dello username  & UC 1.8.1 \\
		\hline
		R0F1.1.1  &  Lo username è univoco in tutto il sistema  &  Interno \\
		\hline
		R0F1.1.2  &  Lo username è una striga contetente esclusivamente caratteri alfanumerici e underscores (\_)  &  Interno \\
		\hline
		R0F1.1.3  &  La lunghezza dello username è contenuta tra i 4 ed i 30 caratteri &  Interno \\
		\hline
		R0F1.2  &  La registrazione richiede l'inserimento di un indirizzo e-mail valido & UC 1.8.2 \\
		\hline
		R0F1.3  &  La registrazione richiede l'inserimento della password  & UC 1.8.3 \\
		\hline
		R0F1.3.1  &  La password non può contenere lo username  &  Interno \\
		\hline
		R0F1.3.2  &  La password è una striga contetente esclusivamente caratteri alfanumerici e speciali ( ? ! \$ + - / . , \textless \textgreater $@$ \textasciitilde \^{} \_ )  & Interno \\
		\hline
		R0F1.3.3  &  La lunghezza minima della password è di 6 caratteri  &  Interno \\
		\hline
		R0F1.4  &  La registrazione richiede la conferma della password & Interno \\
		\hline
		R0F1.4.1  &  La password di conferma deve coincidere con la password inserita precedentemente  &  Interno \\
		\hline
		R0F1.5  &  Il sistema avverte l'utente in fase di compilazione della form nel caso i dati inseriti in quel momento non risultino conformi alle norme del sistema  &  Interno \\
		\hline
		R0F1.6  &  L'utente non autenticato può confermare la registrazione tramite l'apposito pulsante & UC 1.8.4 \\
		\hline
		R0F1.7  &  Il sistema visualizza un messaggio di errore nel caso i dati inseriti risultino non conformi alle norme imposte dal sistema  &  UC 1.8.5 \\
		\hline
		R0F1.8  &  Nel caso la registrazione sia andata a buon fine, il sistema aggiunge i dati dell'utente nel rispettivo database  &  Interno \\
		\hline
		R1F1.9  &  Nel caso la registrazione sia andata a buon fine, viene inviata una e-mail di conferma all'indirizzo inserito dall'utente  &  Interno \\
		\hline
		R0F2  &  L'utente non autenticato può effettuare il login tramite l'apposto form di autenticazione  &  UC 1.1 \\
		\hline
		R0F2.1  &  L'autenticazione richiede l'inserimento dello username  &  UC 1.1.1 \\
		\hline
		R0F2.2  &  L'autenticazione richiede l'inserimento della password  &  UC 1.1.2 \\
		\hline
		R0F2.3  &  L'utente può completare l'autenticazione tramite l'apposito pulsante di conferma  &  UC 1.1.3 \\
		\hline
		R0F2.4  &  Il sistema visualizza un messaggio di errore nel caso i dati inserito dall'utente risultino errati  &  UC 1.1.4 \\
		\hline
		R0F2.5  &  Il sistema aggiorna la pagina principale permettendo all'utente autenticato di accedere alle diverse aree del del servizio  &  Interno  \\
		\hline
		R0F3  &  Nella pagina principale dell'utente autenticato è presente un pulsante per aprire il menù delle informazioni personali &  UC 1.2 \newline UC 1.5 \\
		\hline
		R0F3.1  &  Nel menù delle informazioni personali è presente un pulsante per visualizzare i propri dati &  UC 1.2.1 \newline UC 1.2.1.1 \\
		\hline
		R0F3.1.1  &  Se l'utente seleziona il pulsante di visualizzazione dei dati personali il sistema deve reperire il suo nome utente, il suo indirizzo e-mail e la data del suo ultimo accesso &  Interno \\
		\hline
		R0F3.1.2  &  Se l'utente seleziona il pulsante di visualizzazione dei dati personali deve aprirsi una finestra che mostra il nome utente, l'indirizzo e-mail ad esso associata e la data del suo ultimo accesso &  UC 1.2.1.1 \\
		\hline
		R0F3.2  &  Nel menù delle informazioni personali è presente un pulsante per visualizzare le proprie statistiche &  UC 1.2.1 \newline UC 1.2.1.2 \\
		\hline
		R0F3.2.1  &  Se l'utente seleziona il pulsante di visualizzazione delle statistiche, il sistema deve reperire le informazioni necessarie  &  Interno \\
		\hline
		R0F3.2.2  &  Se l'utente seleziona il pulsante di visualizzazione delle statistiche deve aprirsi una finestra che mostra il numero di View attive, il numero di Recipes disponibili e il numero di accessi negli ultimi 30 giorni &  UC 1.2.1.2 \\
		\hline
		R0F3.3  &  Nel menù delle informazioni personali è presente un pulsante per modificare la propria password &  UC 1.5 \\
		\hline
		R0F3.3.1  &  Se l'utente seleziona il pulsante di cambio della password, di deve aprire una finestra con il form per l'inserimento di una nuova password e della password di conferma. &  UC 1.5 \newline UC 1.5.1 \newline UC 1.5.2 \\
		\hline
		R0F3.3.2  &  La password e la password di conferma devono essere nella forma descritta nei requisiti R0F1.3.1, R0F1.3.2, R0F1.3.3 e R0F1.4.1  &  Interno \\
		\hline
		R0F3.3.2.1  &  Il sistema visualizza eventuali non conformità in fase di compilazione  &  Interno \\
		\hline
		R0F3.3.3  &  L'utente può cliccare il tasto di conferma se ha riempito il form  &  UC 1.5.4 \\
		\hline
		R0F3.3.3.1  &  Una volta premuto il tasto di conferma appare un messaggio di errore se la password non è conforme o non coincide con la conferma  & UC 1.5.5 \\
		\hline
		R0F3.3.3.2  &  Una volta premuto il tasto di conferma appare un messaggio di conferma se la password è stata modificata  &  Interno \\
		\hline
		R0F3.3.3.3  &  Una volta premuto il tasto di conferma il sistema aggiorna la password nel Database  &  Interno \\
		\hline
		R0F4  &  Nella pagina principale è presente un pulsante di Logout  &  UC 1.6 \\
		\hline
		R0F4.1  &  Una volta premuto il tasto di logout il sistema verifica se la sessione in corso è valida o scaduta  &  Interno \\
		\hline
		R0F4.1.1  &  Se la sessione è valida appare un messaggio che chiede conferma del logout  &  UC 1.6.1 \\
		\hline
		R0F4.1.1.1  &  Se il logout viene confermato la sessione viene segnata come scaduta e il sistema aggiorna la pagina per mostrare la schermata di login  &  UC 1.6.2 \\
		\hline
		R0F4.1.2  &  Se la sessione è scaduta appare un messaggio di errore e il sistema aggiorna la pagina per mostrare la schermata di login  &  Interno \\
		\hline
		R0F5  &  Nella pagina principale dell'utente autenticato sono mostrare tutte le View che l'utente ha creato aggiornate al momento del caricamento della pagina &  UC 1.3 \newline UC 1.3.1 \\
		\hline
		R0F5.1  &  Il sistema ad ogni caricamento della pagina deve recuperare i dati relativi alle View dell'utente e generare i grafici relativi  &  Interno \\
		\hline
		R1F5.2  &  Nella pagina principale è disponibile un elenco delle View attive dell'utente &  UC 1.3.1.1 \\
		\hline
		R1F5.2.1  &  Cliccando su una View dall'elenco di viene portati al grafico relativo &  Interno \\
		\hline
		R1F5.2.2  &  Accanto ad ogni entry dell'elenco è disponibile il tasto modifica &  UC 1.3.4.1 \newline UC 1.3.4.2 \\
		\hline
		R1F5.2.2.1  &  Cliccando sul tasto modifica si viene reinderizzati alla pagina di modifica della View &  Interno \\
		\hline
		R1F6  &  Nella pagina di modifica della View è presente il grafico relativo alla View selezionata e un form per modificare i dati &  Interno \\
		\hline
		R1F6.1  &  Il sistema deve recuperare i dati relativi alla View selezionata e generare il grafico e riempire la form con i dati preesistenti &  Interno \\
		\hline
		R1F6.2  &  L'utente deve poter inserire nuovi dati oppure lasciare quelli preesistenti &  UC 1.3.4.3 \\
		\hline
		R1F6.2.1  &  Eventuali errori sui dati inseriti vengono segnalati dal sistema in fase di compilazione del form &  Interno \\
		\hline
		R1F6.3  &  L'utente deve poter confermare i dati modificati premendo il tasto di conferma &  UC 1.3.4.4 \\
		\hline
		R1F6.3.1  &  Se la modifica è avvenuta con successo appare un messaggio di conferma e si viene reinderizzati alla pagina principale &  Interno \\
		\hline
		R1F6.3.1.1  &  Se la modifica è avvenuta con successo il sistema memorizza i nuovi dati nel database sovrascrivendo i vecchi &  Interno \\
		\hline
		R1F6.3.2  &  Se c'è un errore sui dati o c'è un problema di connessione al server viene visualizzato un messaggio di errore e si viene riportati alla form di modifica &  UC 1.3.5 \\
		\hline
		R1F6.4  &  Nella pagina di modifica è presente un tasto per eliminare la View  &  UC 1.3.6 \newline UC 1.3.6.2 \\
		\hline
		R1F6.4.1  &  Se l'utente preme il pulsante di eliminazione appare un messaggio che chiede conferma dell'eliminazione  &  UC 1.3.6.3 \\
		\hline
		R1F6.4.2  &  Se l'eliminazione viene confermata la View e i relativi dati vengono eliminati dal database  &  Interno \\
		\hline
		R1F6.4.3  &  Una volta eliminata una View si viene reindirizzati alla pagina principale  &  Interno \\
		\hline
		R0F7  &  Nella pagina principale è presente un tasto per creare una nuova View  &  UC 1.3.2 \newline UC 1.3.2.1 \\
		\hline		
		R0F7.1  &  Nella pagina di creazione della View è presente un form per inserire i dati della View da generare &  UC 1.3.2.2 \\
		\hline
		R0F7.1.1  &  Eventuali errori sui dati inseriti vengono segnalati dal sistema in fase di compilazione del form &  Interno \\
		\hline
		R0F7.2  &  L'utente deve poter confermare i dati modificati premendo il tasto di conferma &  UC 1.3.2.3 \\
		\hline
		R0F7.3  &  Se la modifica è avvenuta con successo appare un messaggio di conferma e si viene reinderizzati alla pagina principale &  Interno \\
		\hline
		R0F7.3.1  &  Se la modifica è avvenuta con successo il sistema memorizza i nuovi dati nel database  &  Interno \\
		\hline
		R0F7.3.2  &  Se c'è un errore sui dati o c'è un problema di connessione al server viene visualizzato un messaggio di errore e si viene riportati alla form di inserimento &  UC 1.3.3 \\
		\hline
		R1F8  & Nella pagina principale dell'utente amministratore è presente un pulsante per accedere all'elenco degli utenti  &  UC 1.7.1 \\
		\hline
		R1F8.1  &  Accanto al nome di ogni utente nella lista è presente un pulsante modifica  &  Interno \\
		\hline
		R1F8.1.1  &  Cliccando sul pulsante modifica si viene reinderizzati alla pagina di modifica degli utenti  &  UC 1.7.2.2 \\
		\hline
		R2F8.2  &  Nella pagina di modifica di un utente l'amministratore può cambiare il ruolo di un utente da amministratore a utente e viceversa  &  UC 1.7.2.3 \\
		\hline
		R2F8.2.1  &  Premendo conferma il ruolo e i permessi dell'utente vengono modificati sul database  &  UC 1.7.2.4 \\
		\hline
		R1F8.3  &  Nella pagina di modifica di un utente è presente il pulsante elimina utente  & UC 1.7.3.2 \newline UC 1.7.3.2 \\
		\hline
		R1F8.3.1  &  Se viene premuto il pulsante di eliminazione di un utente appare un messaggio che chiede conferma dell'eliminazione  & UC 1.7.3.3\\
		\hline
		R1F8.3.1.1  &  Se viene confermata il sistema verifica se l'utente è loggato e in tal caso segnala l'errore e interrompe l'eliminazione  & Interno \\
		\hline
		R1F8.3.1.2  &  Se l'eliminazione ha successo appare un messaggio di conferma e si viene reinderizzati alla pagina principale  & UC 1.7.3.3\\
		\hline
		R1F8.3.1.3  &  Se c'è stato un errore appare un messaggio d'errore e si è riportati alla pagina di modifica di un utente  & UC 1.7.3.3\\
		\hline

	\end{longtable}
	\egroup
\end{center}


\subsection{Requisiti prestazionali}

\begin{center}

	\def\arraystretch{1.5}
	\bgroup
	\begin{longtable}{| p{2cm} | p{8cm} | p{2cm} |}

		\hline
		\textbf{Requisito} & \textbf{Descrizione} & \textbf{Fonti} \\
		\hline

		R1P1  &  Una volta autenticato l'utente deve poter visualizzare i propri grafici entro 10 secondi salvo problemi di connesione &  Interno \\
		\hline
		R1P1.1  &  Il recupero dei dati dal Database deve impiegare al più 7 secondi  &  Interno \\
		\hline

	\end{longtable}
	\egroup
\end{center}

\subsection{Requisiti di qualità}

\begin{center}

	\def\arraystretch{1.5}
	\bgroup
	\begin{longtable}{| p{2cm} | p{8cm} | p{2cm} |}

		\hline
		\textbf{Requisito} & \textbf{Descrizione} & \textbf{Fonti} \\
		\hline

		R0Q1  &  Viene fornito un manuale per l'utente  &  Interno \\
		\hline
		R0Q1.1  &  Il manuale utente deve rispettare le norme e le metriche descritte nel \docNameVersionPdQ  &  Interno \\
		\hline
		R0Q2  &  Tutto il codice rispetta le norme e le metriche descritte nel \docNameVersionPdQ{} e nelle \docNameVersionNdP &  Interno \\
		\hline


	\end{longtable}
	\egroup
\end{center}

\subsection{Requisiti di vincolo}

\begin{center}

	\def\arraystretch{1.5}
	\bgroup
	\begin{longtable}{| p{2cm} | p{8cm} | p{2cm} |}

		\hline
		\textbf{Requisito} & \textbf{Descrizione} & \textbf{Fonti} \\
		\hline

		R1V1  &  Il sistema sfrutterà gli strumenti offerti da Google Cloud Platform  &  Capitolato \newline Verbale 2015/01/14 \\
		\hline
		R1V1.1  &  Il sistema sfrutterà Google App Engine  &  Capitolato \newline Verbale 2015/01/14 \\
		\hline
		R1V1.2  &  Il sistema sfrutterà Google Cloud Datastore  &  Capitolato \newline Verbale 2015/01/14 \\
		\hline
		R1V1.3  &  Il sistema sfrutterà gli strumenti offerti da Google Cloud Platform  &  Capitolato \newline Verbale 2015/01/14 \\
		\hline
		R1V2  &  Il linguaggio di programmazione principalmente usato sarà Python  &  Capitolato \newline Verbale 2015/01/14 \\
		\hline

	\end{longtable}
	\egroup
\end{center}

\subsection{Riepilogo}

\subsection{Requisiti accettati}
