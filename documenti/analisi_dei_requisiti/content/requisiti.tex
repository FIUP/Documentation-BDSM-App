% =================================================================================================
% File:			requisiti.tex
% Description:	Definisce il capitolo che contiene i requisiti in forma tabellare
% Created:		2014-12-10
% Author:		Roetta Marco
% Email:		roetta.marco@mashup-unipd.it
% =================================================================================================
% Modification History:
% Version		Modifier Date		Change													Author
% 0.0.1 		2014-12-10 			aggiunta sezione e iniziata stesura						Roetta Marco
% =================================================================================================
% 0.0.2 		2015-01-12 			aggiunte tabelle										Carnovalini Filippo
% =================================================================================================
% 0.0.3 		2015-01-16 			requisiti prestazionali di qualità e di vincolo			Carnovalini Filippo
% =================================================================================================
% 0.0.4 		2015-01-16 			aggiunti requisiti funzionali, fixing generale 			Cusinato Giacomo
% =================================================================================================
% 0.0.5 		2015-01-17 			requisiti tratti dagli UC1.2, 1.5 e 1.6					Carnovalini Filippo
% =================================================================================================
% 0.0.6 		2015-01-18 			requisiti tratti dagli UC1.3 e 1.7						Carnovalini Filippo
% =================================================================================================
% 0.0.7 		2015-01-18 			aggiunti requsiti funzionali e di vincolo				Cusinato Giacomo
% =================================================================================================
% 1.0.0 		2015-01-21 			inizio verifica documento								Faccin Nicola
% =================================================================================================
% 1.1.0 		2015-02-20 			correzione in base all'esito RR	 						Carnovalini Filippo
% =================================================================================================
% 2.0.1			2015-03-12			correzione piccoli errori ortografici					Tesser Paolo
% =================================================================================================
% 2.0.2			2015-03-12			rimossi requisiti: ROF3.2.2.3, RDF5.2.2, RDF5.2.2.1		Tesser Paolo
% =================================================================================================
% 2.0.3			2015-03-12			rimossi req: RDF6, ROF7 e derivati						Tesser Paolo
% =================================================================================================
% 2.0.4			2015-03-12			modificati req: RDF6, ROF7 e derivati					Tesser Paolo
% =================================================================================================
% 2.0.5			2015-03-15			modificate fonti conformi ai nuovi UC					Tesser Paolo
% =================================================================================================
% 2.0.6			2015-03-15			inseriti req. ROF5 e derivati riguardanti View FB		Tesser Paolo
% =================================================================================================
% 2.0.7			2015-03-16			inseriti req. ROF6/7 con derivati						Tesser Paolo
% =================================================================================================
%

% CONTENUTO DEL CAPITOLO

\section{Requisiti}

A seguire sono riportati tutti i requisiti individuati dal gruppo. Essi sono stati ricavati dal capitolato proposto, dai casi d'uso, da necessità interne o in seguito all' incontro col proponente. Ogni requisito è identificato da un codice univoco a seconda dell'importanza, del tipo e della gerarchia a cui appartiene.
I requisiti saranno divisi in quattro tabelle distinte a seconda della tipologia, ognuna composta da tre colonne contenenti i seguenti attributi:
\begin{itemize}
	\item \textbf{Requisito}: il nominativo identificativo del requisito espresso tramite il formalismo indicato a seguire.
	\item \textbf{Descrizione}: breve descrizione del requisito.
	\item \textbf{Fonti}: rappresentate da uno o più dei seguenti valori:
	\begin{itemize}
		\item \emph{Capitolato}: requisito ricavato direttamente del capitolato.
		\item \emph{Casi d'uso}: requisito ricavato da uno o più casi d'uso. In questo caso sarà indicato il codice di quest'ultimo/i.
		\item \emph{Interno}: requisito ricavato in seguito ad un'analisi o una necessità interna al gruppo.
	\end{itemize}
\end{itemize}

La classificazione dei requisiti sarà riportata tramite il seguente formalismo:
\begin{center}
	R[importanza][tipo][codice]
\end{center}
\begin{itemize}
	\item \emph{importanza}: può assumere uno tra i seguenti valori:
	\begin{itemize}
		\item \emph{O}: requisito obbligatorio;
		\item \emph{D}: requisito desiderabile;
		\item \emph{F}: requisito facoltativo (opzionale);
	\end{itemize}
	\item \emph{tipo}: può assumere uno tra i seguenti valori:
	\begin{itemize}
		\item \emph{F}: requisito funzionale;
		\item \emph{Q}: requisito di qualità;
		\item \emph{P}: requisito prestazionale;
		\item \emph{V}: requisito vincolo;
	\end{itemize}
	\item \emph{codice}: identificativo del requisito attribuito in modo gerarchico.
\end{itemize}


\subsection{Requisiti funzionali}

\begin{center}

	\def\arraystretch{1.5}
	\bgroup
	\begin{longtable}{| p{2.5cm} | p{8cm} | p{2cm} |}

		\hline
		\textbf{Requisito} & \textbf{Descrizione} & \textbf{Fonti} \\
		\hline

		ROF1  &  L'utente non autenticato può effettuare la registrazione al servizio  &  UC1.8 \\
		\hline
		ROF1.1  &  La registrazione richiede l'inserimento dello username  &  UC1.8.1 \\
		\hline
		ROF1.1.1  &  Lo username è univoco in tutto il sistema  &  Interno \\
		\hline
		ROF1.1.2  &  Lo username è una stringa contenente esclusivamente caratteri alfanumerici e underscore (\_)  &  Interno \\
		\hline
		ROF1.1.3  &  La lunghezza dello username è contenuta tra i 4 ed i 30 caratteri  &  Interno \\
		\hline
		ROF1.2  &  La registrazione richiede l'inserimento di un indirizzo e-mail valido &  UC1.8.2 \\
		\hline
		ROF1.3  &  La registrazione richiede l'inserimento della password  &  UC1.8.3 \\
		\hline
		ROF1.3.1  &  La password non può contenere lo username  &  Interno \\
		\hline
		ROF1.3.2  &  La password è una stringa contenente esclusivamente caratteri alfanumerici e speciali ( ? ! \$ + - / . , \textless \textgreater $@$ \textasciitilde \^{} \_ )  &  Interno \\
		\hline
		ROF1.3.3  &  La lunghezza minima della password è di 6 caratteri  &  Interno \\
		\hline
		ROF1.4  &  La registrazione richiede la conferma della password  &  Interno \\
		\hline
		ROF1.4.1  &  La password di conferma deve coincidere con la password inserita precedentemente  &  Interno \\
		\hline
		ROF1.5  &  Il sistema avverte l'utente in fase di compilazione della form nel caso i dati inseriti in quel momento non risultino conformi alle norme del sistema  &  Interno \\
		\hline
		ROF1.6  &  L'utente non autenticato può confermare la registrazione tramite l'apposito pulsante  &  UC1.8.4 \\
		\hline
		ROF1.7  &  Il sistema visualizza un messaggio di errore nel caso i dati inseriti risultino non conformi alle norme imposte dal sistema  &  UC1.8.5 \\
		\hline
		RDF1.8  &  Nel caso la registrazione sia andata a buon fine, viene inviata una e-mail di conferma all'indirizzo inserito dall'utente  &  Interno \\
		\hline


		ROF2  &  L'utente non autenticato può effettuare il login tramite l'apposto form di login  &  UC1.1 \\
		\hline
		ROF2.1  &  Il login richiede l'inserimento dello username  &  UC1.1.1 \\
		\hline
		ROF2.2  &  Il login richiede l'inserimento della password  &  UC1.1.2 \\
		\hline
		ROF2.3  &  L'utente può completare il login tramite l'apposito pulsante di conferma  &  UC1.1.3 \\
		\hline
		ROF2.4  &  Il sistema visualizza un messaggio di errore nel caso i dati inseriti dall'utente risultino errati  &  UC1.1.4 \\
		\hline
		ROF2.4.1  &  Il sistema visualizza un messaggio di errore nel caso in cui non esista un utente registrato con l'username inserito  &  UC1.1.4 \\
		\hline
		ROF2.4.2  &  Il sistema visualizza un messaggio di errore nel caso in cui la password inserita non è quella corretta per l'username inserito  &  UC1.1.4 \\
		\hline
		ROF2.5  &  Il sistema aggiorna la home screen permettendo all'utente autenticato di accedere alle diverse aree del del servizio  &  Interno  \\
		\hline


		ROF3  &  Nella home screen dell'utente autenticato è presente un pulsante per aprire il menù delle informazioni personali  &  UC1.2 \newline UC1.5 \\
		\hline
		ROF3.1  &  Nel menù delle informazioni personali è presente un pulsante per visualizzare i propri dati  &  UC1.2.1 \\
		\hline
		ROF3.1.1  &  Se l'utente seleziona il pulsante di visualizzazione dei dati personali il sistema deve reperire il suo nome utente, il suo indirizzo e-mail e la data del suo ultimo accesso &  Interno \\
		\hline
		ROF3.1.2  &  Se l'utente seleziona il pulsante di visualizzazione dei dati personali deve aprirsi una finestra che mostra i suoi dati &  UC1.2.1 \\
		\hline
		ROF3.1.2.1  &  Nella finestra dei dati personali deve essere presente il nome utente dell'utente autenticato. l'indirizzo e-mail ad esso associata e la data del suo ultimo accesso &  UC1.2.1.1 \\
		\hline
		ROF3.1.2.2  &  Se nella finestra dei dati personali deve essere presente l'indirizzo e-mail associato all'account dell'utente autenticato &  UC1.2.1.2 \\
		\hline		
		ROF3.1.2.3  &  Nella finestra dei dati personali deve essere presente la data dell'ultimo accesso al sistema dell'utente autenticato precedente a quello in corso &  UC1.2.1.3 \\
		\hline
		ROF3.2  &  Nel menù delle informazioni personali è presente un pulsante per visualizzare le proprie statistiche &  UC1.2.2 \\
		\hline
		ROF3.2.1  &  Se l'utente seleziona il pulsante di visualizzazione delle statistiche, il sistema deve reperire le informazioni necessarie  &  Interno \\
		\hline
		ROF3.2.2  & Se l'utente seleziona il pulsante di visualizzazione delle statistiche, si deve aprire una finestra che mostra le statistiche &  UC1.2.2 \\
		\hline
		ROF3.2.2.1  &  Nella finestra delle statistiche viene mostrato il numero di View attive &  UC1.2.2.1 \\
		\hline
		ROF3.2.2.2  &  Nella finestra delle statistiche viene mostrato il numero di Recipe disponibili &  UC1.2.2.2 \\
		\hline
		ROF3.3  &  Nel menù delle informazioni personali è presente un pulsante per modificare la propria password &  UC1.5 \\
		\hline
		ROF3.3.1  &  Se l'utente seleziona il pulsante di cambio della password, si deve aprire una finestra con il form per l'inserimento di una nuova password e della password di conferma. Viene richiesto anche l'inserimento della vecchia password. &  UC1.5 \newline UC1.5.1 \newline UC1.5.2 \newline UC1.5.3 \\
		\hline
		ROF3.3.2  &  La password e la password di conferma devono essere nella forma descritta nei requisiti ROF1.3.1, ROF1.3.2, ROF1.3.3 e ROF1.4.1  &  Interno \\
		\hline
		ROF3.3.2.1  &  Il sistema visualizza eventuali non conformità in fase di compilazione  &  Interno \\
		\hline
		ROF3.3.3  &  L'utente può cliccare il tasto di conferma se ha riempito il form  &  UC1.5.4 \\
		\hline
		ROF3.3.3.1  &  Una volta premuto il tasto di conferma appare un messaggio di errore se la password non coincide con la conferma o se la vecchia password inserita non è corretta.  &  UC1.5.5 \\
		\hline
		ROF3.3.3.2  &  Una volta premuto il tasto di conferma appare un messaggio di conferma se la password è stata modificata  &  Interno \\


		\hline
		ROF4  &  Nella home screen è presente un pulsante di logout  &  UC1.6 \\
		\hline
		ROF4.1  &  Una volta premuto il tasto di logout il sistema verifica se la sessione in corso è valida o scaduta  &  Interno \\
		\hline
		ROF4.1.1  &  Se la sessione è valida appare un messaggio che chiede conferma del logout  &  UC1.6.1 \\
		\hline
		ROF4.1.1.1  &  Se il logout viene confermato la sessione viene segnata come scaduta e il sistema aggiorna la pagina per mostrare la schermata di login  &  UC1.6.2 \\
		\hline
		ROF4.1.2  &  Se la sessione è scaduta appare un messaggio di errore e il sistema aggiorna la pagina per mostrare la schermata di login  &  Interno \\



		\hline
		ROF5  &  Nella home screen dell'utente autenticato sono mostrare tutte le Recipe presenti nel sistema &  UC1.3 \\
		\hline
		ROF5.1  & Il sistema ad ogni caricamento della pagina deve recuperare l'elenco delle Recipe  &  Interno \\
		\hline
		ROF5.1.1  & Il sistema, per ogni Recipe dell'elenco, deve fornire il titolo e qualora presente la descrizione  &  Interno \\
		\hline
		ROF5.1.2  & Il sistema, per ogni Recipe dell'elenco, deve fornire due pulsanti per scegliere se visualizzare le metriche della Recipe o effettuare un confronto tra le metriche &  UC1.3 \\

		\hline
		ROF5.2  &  L'utente deve potere visualizzare tutte le metriche di una Recipe una volta premuto l'apposito pulsante  &  UC1.3.1 \newline Interno \\
		\hline
		ROF5.2.1  &  La visualizzazione delle metriche sarà suddivisa per categoria  &  UC1.3.1 \newline Interno \\
		\hline
		ROF5.2.1.1  &  Le categorie sono: Facebook, Twitter, Instagram &  UC1.3.1 \newline Capitolato \\		
		\hline
		ROF5.2.1.2  &  Il sistema, per ogni metrica dell'elenco, fornirà il nome, la descrizione se presente e la tipologia della metrica (vedi ROF7.1.1.5, ROF7.1.1.6, ROF7.1.1.7). &  UC1.3.1 \newline Interno \\
		\hline
		ROF5.2.1.3  &  Il sistema, per ogni metrica dell'elenco, deve fornire il pulsante per accedere alle visualizzazioni delle View presenti per quella metrica. &  UC1.3.1 \newline Interno \\
		\hline

		ROF5.3  & Premendo sul pulsante del ROF5.2.1.3, da dentro la categoria Facebook, vengono fornite tutte le View predisposte dal sistema per avere delle statistiche. & UC1.3.1.1 \newline Interno \\
		\hline
		ROF5.3.1  & Le View offerte per la categoria di Facebook possono mostrare grafici di diversi tipi: Line Chart, Bar Chart, Pie Chart e Map Chart. &  UC1.3.1.1 \newline Interno \\
		\hline
		ROF5.3.1.1  & Tipo di View: viene visualizzato un Line Chart con due linee per vedere l'andamento dei ``likes'' rispetto ai ``talking about'' di una pagina. Per questo grafico viene fornita una visualizzazione settimanale, mensile e annuale.  & Interno \\
		\hline
		ROF5.3.1.2  &  Tipo di View: viene visualizzato un Pie Chart che prende tutti gli eventi creati da una pagina e mostra, la percentuale di persone che hanno messo partecipa, forse o rifiuta. I dati vengono forniti anche non in percentuale. & Interno \\
		\hline
		ROF5.3.1.3  &  Tipo di View: viene visualizzato un Pie Chart che mostra la percentuale di commenti di terzi a tutti i post di una pagina rispetto la percentuale di commenti della pagina stessa. & Interno \\
		\hline
		ROF5.3.1.4  &  Tipo di View: viene visualizzato un Pie Chart che mostra la percentuale di commenti della pagina e di terzi rispetto a tutti i post di terzi sulla pagina in questione. & Interno \\
		\hline
		ROF5.3.1.5  &  Tipo di View: viene visualizzato un Map Chart che illustra le zone nella quale si sono creati degli eventi di una pagina e mostra dei cerchi di diverse dimensioni a seconda della media di partecipanti agli eventi di quella zona. & Interno \\
		\hline
		ROF5.3.1.6  &  Tipo di View: viene visualizzato un Bar Chart che mostra la media di commenti delle pagina per ogni post effettuato dalla stessa. Per questo grafico viene fornita una visualizzazione settimanale, mensile e annuale. & Interno \\
		\hline
		ROF5.3.1.7  &  Tipo di View: viene visualizzato un Line Chart che mostra l’andamento del numero di post giornalieri di una pagina [TO DO]. & Interno \\
		\hline


		ROF5.4  &  Premendo sul pulsante del ROF5.2.1.3, da dentro la categoria Twitter, vengono fornite tutte le View predisposte dal sistema per avere delle statistiche. &  UC1.3.1.2 \newline Interno \\
		\hline
		ROF5.4.1  & Le View offerte per la categoria di Twitter possono mostrare grafici di diversi tipi: Line Chart, Bar Chart, Pie Chart, Radar Chart e Map Chart.  &  UC1.3.1.2 \newline Interno \\
		\hline
		ROF5.4.1.1  &  Tipo di View: viene visualizzato un Bar Chart orizzontale che mostra un riepilogo istantaneo dei Tweet e dei Follower di un singolo utente. & Interno \\
		\hline	
		ROF5.4.1.2  &  Tipo di View: viene visualizzato un Line Chart che indica il numero di Tweet e il numero di Follower dell'utente, fornendo una visualizzazione settimanale, mensile ed annuale. & Interno \\
		\hline
		ROF5.4.1.3  &  Tipo di View: viene visualizzato un Map Chart che illustra l'area geografica di appartenenza dei Tweet con un particolare hash tag. & Interno \\
		\hline
		ROF5.4.1.4  &  Tipo di View: viene visualizzato un Line Chart che mostra il numero di Tweet con un hash tag.  & Interno \\
		\hline
		ROF5.4.1.5  &  Tipo di View: viene visualizzato un Pie Chart che illustra quanti Tweet, quanti Preferiti e quanti Retweet ha un utente. & Interno \\
		\hline
		ROF5.4.1.6  &  Tipo di View: viene visualizzato un Radar Chart che illustra in che momento della giornata vengono fatti i Tweet. & Interno \\
		\hline
		ROF5.4.1.7  &  Tipo di View: viene visualizzato un Pie Chart che indica su che tipo di piattaforma gli utenti hanno twittato. & Interno \\
		\hline


		ROF5.5  &  Premendo sul pulsante del ROF5.2.1.3, da dentro la categoria Instagram, vengono fornite tutte le View predisposte dal sistema per avere delle statistiche &  UC1.3.1.3 \newline Interno \\
		\hline
		ROF5.5.1  & Le View offerte per la categoria di Instagram possono mostrare grafici di diversi tipi: Line Chart, Bar Chart, Pie Chart e Map Chart  &  UC1.3.1.3 \newline Interno \\
		\hline
		ROF5.5.1.1  &  Tipo di View: vengono visualizzati per ogni post di un utente il numero di like e il numero di commenti in un Bar Chart & Interno \\
		\hline	
		ROF5.5.1.2  &  Tipo di View: vengono visualizzati i nuovi followers di un utente per giorno, in un Line Chart. Nel grafico vengono anche segnati come linee verticali i momenti in cui l'utente ha pubblicato un post. Per questo grafico viene fornita una visualizzazione settimanale, mensile e annuale. & Interno \\
		\hline
		ROF5.5.1.2.1  &  Per questo tipo di View è possibile confrontare due utenti, che appariranno sul grafico con due colori diversi. & Interno \\
		\hline
		ROF5.5.1.3  &  Tipo di View: vengono visualizzati i nuovi like e i nuovi commenti ricevuti da un utente per giorno, in un Line Chart. Nel grafico vengono anche segnati come linee verticali i momenti in cui l'utente ha pubblicato un post. Per questo grafico viene fornita una visualizzazione settimanale, mensile e annuale. & Interno \\
		\hline
		ROF5.5.1.3.1  &  Per questo tipo di View è possibile confrontare due utenti, che appariranno sul grafico con due colori diversi. & Interno \\
		\hline
		ROF5.5.1.4  & Tipo di View: viene visualizzato in un Line Chart l'andamento dei like e quello dei commenti ricevuti da un utente fratto il numero di followers che possiede. Per questo grafico viene fornita una visualizzazione settimanale, mensile e annuale. & Interno \\
		\hline
		ROF5.5.1.5  &  Tipo di View: viene visualizzato in un Line Chart il numero di like ricevuti al giorno da un utente fratto la frequenza di post dell'utente (ovvero il numero di post al giorno). Per questo grafico viene fornita una visualizzazione settimanale, mensile e annuale. & Interno \\
		\hline
		ROF5.5.1.5.1  &  Per questo tipo di View è possibile confrontare due utenti, che appariranno sul grafico con due colori diversi. & Interno \\
		\hline
		ROF5.5.1.6  &  Tipo di View: viene visualizzato in un Line Chart il numero di post e il numero utenti che hanno usato un dato hashtag. Per questo grafico viene fornita una visualizzazione settimanale, mensile e annuale. & Interno \\
		\hline
		ROF5.5.1.6.1  &  Per questo tipo di View è possibile confrontare due hashtag, che appariranno sul grafico con due colori diversi. & Interno \\
		\hline

		ROF5.5.1.7  &  Tipo di View: viene visualizzato in un Line Chart il numero di like e il numero di commenti ricevuti dai post che contengono un dato hashtag. Per questo grafico viene fornita una visualizzazione settimanale, mensile e annuale. & Interno \\
		\hline
		ROF5.5.1.7.1  &  Per questo tipo di View è possibile confrontare due hashtag, che appariranno sul grafico con due colori diversi. & Interno \\
		\hline
		ROF5.5.1.8  &  Tipo di View: vengono visualizzate in un Map Chart le zone geografiche dove sono stati fatti i post contenenti un dato hashtag (di cui è nota la localizzazione), colorandole con una colorazione di intensità proporzionale al numero di post contenti l'hashtag nella zona. & Interno \\
		\hline
		ROF5.5.1.8.1  &  Per questo tipo di View è possibile confrontare due hashtag, che appariranno sul grafico con due colori diversi. & Interno \\
		\hline
		ROF5.5.1.9  &  Tipo di View: viene visualizzato un Pie Chart che mostra per un dato hashtag la percentuale di post fatti in un dato stato sul totale di post contenenti l'hashtag dei quali è noto lo stato in cui sono stati fatti. I dati vengono forniti anche non in percentuale. & Interno \\
		\hline

		ROF5.5.2  & Le View offerte per la categoria di Instagram possono fornire dati in forma tabellare.  &  UC1.3.1.2 \newline Interno \\
		\hline
		ROF5.5.2.1  &  [TO DO] & Interno \\
		\hline

		ROF5.6  &  L'utente deve potere visualizzare tutte le metriche di una Recipe una volta premuto l'apposito pulsante  &  UC1.3.2 \newline Interno \\
		\hline
		ROF5.6.1  &  [TO DO]  &  TO DO \newline Interno \\
		\hline
		ROF5.6.1  &  [TO DO]  &  TO DO \newline Interno \\
		\hline
		ROF5.6.1  &  [TO DO]  &  TO DO \newline Interno \\
		\hline
		ROF5.6.1  &  [TO DO]  &  TO DO \newline Interno \\
		\hline
		ROF5.6.1  &  [TO DO]  &  TO DO \newline Interno \\
		\hline
		ROF5.6.1  &  [TO DO]  &  TO DO \newline Interno \\
		\hline




		ROF6  & L'utente ha la possibilità di gestire le View che più preferisce in maniera da averle a disposizione più velocemente. & UC1.11 \\
		\hline
		ROF6.1  & L'utente visualizza tutte le View che sono state marcate da lui come preferite entrando dal menù nell'apposita pagina. & UC1.11.1 \newline Interno \\
		\hline
		ROF6.1.1  & Il sistema mostra un messaggio qualora non ci fossero View salvate come preferite. &  Interno \newline UC1.11.4 \\
		\hline
		ROF6.2  & L'utente ha la possibilità di aggiungere ogni View tra quelle preferite premendo sull'apposito pulsante vicino la View che desidera marcare.  & UC1.11.2 \newline Interno \\
		\hline
		ROF6.2.1  & Il sistema visualizza un messaggio se la View non è stata aggiunta correttamente tra quelle preferite.  & Interno \\
		\hline
		ROF6.3  & L'utente ha la possibilità di rimuovere le View che ha aggiunto tra quelle preferite premendo sull'apposito pulsante vicino la View che desidera rimuovere.  &  UC1.11.4 \newline Interno \\
		\hline
		ROF6.3.1  & Il sistema visualizza un messaggio se la View non è stata rimossa correttamente tra quelle preferite.  &  Interno \\
		\hline
		ROF6.4 & L'utente può marcare come preferite un numero illimitato di View. &  Interno \\
		\hline



		ROF7  & L'utente ha la possibilità di richiedere l'inserimento di una nuova Recipe. & UC1.9 \\
		\hline
		ROF7.1  & L'utente può generare una richiesta dall'apposito form presente in un'apposita sezione della Web GUI. & UC1.9 \newline UC1.9.1 \\
		\hline
		ROF7.1.1 & L'utente deve inserire i parametri richiesti dal form perché la richiesta sia considerata valida dal sistema. & UC1.9.2 \newline Interno \\
		\hline
		ROF7.1.1.1 & L'utente deve inserire il titolo della Recipe che desidera richiedere. Questo campo è obbligatorio. & Interno \\
		\hline
		ROF7.1.1.2 & L'utente può inserire una breve descrizione della Recipe che desidera richiedere. Questo campo è facoltativo. & Interno \\
		\hline
		ROF7.1.1.3 & L'utente deve inserire almeno una metrica nella richiesta della Recipe che desidera. & Interno \\
		\hline
		ROF7.1.1.4 & L'utente deve scegliere la categoria della metrica tra: Facebook, Twitter o Instagram. Questa selezione è obbligatoria e vincolata dal sistema. & Interno \\
		\hline
		ROF7.1.1.5 & L'utente deve scegliere la tipologia della metrica tra: Evento o Pagina, se ha scelto la categoria Facebook. Questa selezione è obbligatoria e vincolata dal sistema. & Interno \\
		\hline
		ROF7.1.1.6 & L'utente deve scegliere la tipologia della metrica tra: Pagina o Hashtag, se ha scelto la categoria Twitter. Questa selezione è obbligatoria e vincolata dal sistema. & Interno \\
		\hline
		ROF7.1.1.7 & L'utente deve scegliere la tipologia della metrica tra: Pagina o Hashtag, se ha scelto la categoria Instagram. Questa selezione è obbligatoria e vincolata dal sistema. & Interno \\
		\hline
		ROF7.1.1.8 & Se l'utente sceglie la categoria Facebook deve inserire il ``nome'' o ``l'identificativo'' della Pagina o dell'Evento del quale vuole raccogliere i dati. Questo campo è obbligatorio. & Interno \\
		\hline
		ROF7.1.1.9 & Se l'utente sceglie la categoria Twitter deve inserire il ``nome'' della Pagina o dell'Hashtag del quale vuole raccogliere i dati. Questo campo è obbligatorio. & Interno \\
		\hline
		ROF7.1.1.10 & Se l'utente sceglie la categoria Instagram deve inserire il ``nome'' della Pagina o dell'Hashtag del quale vuole raccogliere i dati. Questo campo è obbligatorio. & Interno \\
		\hline
		ROF7.1.1.11 & Il sistema avviserà l'utente in maniera istantanea se un campo è compilato correttamente. & Interno \\
		\hline
		ROF7.1.1.12 & Il sistema avviserà l'utente in maniera istantanea se è stato lasciato vuoto un campo obbligatorio. & Interno \\
		\hline
		ROF7.1.2 & L'utente deve premere sul pulsante di invio della richiesta per inviarla agli amministratori. & UC1.9.3 \newline Interno \\
		\hline
		ROF7.1.3 & La richiesta può fallire se i dati inseriti non sono corretti o ci sono dei campi obbligatori vuoti. & UC1.9.4 \newline Interno \\
		\hline
		ROF7.2  & Il sistema invierà una notifica agli amministratori che è stata inserita una nuova richiesta per l'inserimento di una Recipe. & Interno \\
		\hline




		ROF8  &  Un utente che possiede i privilegi di amministratore può accedere all'area riservata del sistema una volta effettuata l'autenticazione tramite il login.  &  UC1.4 \newline UC1.7 \\
		\hline
		ROF8.1  &  Un amministratore può visualizzare tutte le Recipe memorizzate nel sistema selezionando l'apposito pulsante.  &  UC1.4.1 \\
		\hline
		ROF8.2  &  Un amministratore può accedere al pannello per inserire una nuova Recipe  &  UC1.4.2 \newline UC1.4.2.1 \newline UC1.4.2.2  \\
		\hline
		ROF8.2.1  &  Il sistema richiede l'inserimento di determinati parametri per individuare il contesto dei dati richiesti dalla Recipe.  &  Interno \newline UC1.4.2.3 \\
		\hline
		ROF8.2.2  &  Il sistema richiede la conferma dei dati inseriti prima di creare la nuova Recipe.  &  UC1.4.2.4 \\
		\hline
		ROF8.3  &  Il sistema visualizza un messaggio di errore contenente i parametri inseriti non corretti oppure il messaggio di una eccezione lasciato in fase di creazione della Recipe.  &  UC1.4.3 \\
		\hline
		ROF8.3.1  &  Nel caso venga confermata  una nuova Recipe corretta, il sistema salva i suoi parametri nel database  e riporta l'amministratore all'elenco delle Recipe  &  Interno \\
		\hline
		ROF8.4  &  L'utente amministratore può eliminare una Recipe memorizzata nel sistema tramite il relativo pulsante nella schermata di elenco delle Recipe  &  UC1.4.4 \newline UC1.4.4.1 \newline UC1.4.4.2 \\
		\hline
		ROF8.4.1  &  L'utente amministratore deve confermare l'eliminazione della Recipe una volta premuto l'apposito pulsate  &  UC1.4.4.3 \\
		\hline
		ROF8.4.2  &  Una volta confermata l'eliminazione di una Recipe, il sistema provvede all'eliminazione di tutte le informazioni legate ad essa presenti nel database  &  Interno \\
		\hline
		ROF8.4.3  &  Una volta confermata l'eliminazione di una Recipe, il sistema provvede ad eliminare tutti i dati grezzi presenti nel database legati a tale Recipe  &  Interno \\
		\hline
		RFF8.5  &  Dal pannello di amministrazione è possibile visualizzare feedback e statistiche sugli utenti in relazione ai dati delle Recipe utilizzati  &  Interno \\
		\hline


		RDF9 [TO CHANGE] & [TO CHANGE] Nella home screen dell'utente amministratore è presente un pulsante per accedere all'elenco degli utenti  &  UC1.7.1 \newline UC1.7.2.1 \newline UC1.7.3.1 \\
		\hline
		RDF9.1  &  Accanto al nome di ogni utente nella lista è presente un pulsante modifica  &  Interno \\
		\hline
		RDF9.1.1  &  Cliccando sul pulsante modifica si viene reindirizzati alla pagina di modifica degli utenti  &  UC1.7.2.2 \\
		\hline
		RFF9.2  &  Nella pagina di modifica di un utente l'amministratore può cambiare i permessi di un utente  &  UC1.7.2 \newline UC1.7.2.3 \\
		\hline
		RFF9.2.1  &  Premendo conferma il ruolo e i permessi dell'utente vengono modificati sul database  &  UC1.7.2.4 \\
		\hline
		RDF9.3  &  Nella pagina di modifica di un utente è presente il pulsante elimina utente  &  UC1.7.3 \newline UC1.7.3.2 \\
		\hline
		RDF9.3.1  &  Se viene premuto il pulsante di eliminazione di un utente appare un messaggio che chiede conferma dell'eliminazione  & UC1.7.3.3\\
		\hline
		RDF9.3.1.1  &  Se viene confermata il sistema verifica se l'utente è loggato e in tal caso segnala l'errore e interrompe l'eliminazione  & Interno \\
		\hline
		RDF9.3.1.2  &  Se l'eliminazione ha successo appare un messaggio di conferma e si viene reindirizzati alla home screen  & UC1.7.3.3\\
		\hline
		RDF9.3.1.3  &  Se c'è stato un errore appare un messaggio d'errore e si è riportati alla pagina di modifica di un utente  & UC1.7.4\\
		\hline



		ROF10  &  L'amministratore può gestire tutte le richieste di nuove Recipe, ricevute dagli utenti, da una apposita sezione.  & UC1.10 \\
		\hline
		ROF10.1  &  L'amministratore può vedere l'elenco delle richieste di nuove Recipe.  & UC1.10.1 \\
		\hline
		ROF10.1.1  &  Per ogni voce dell'elenco deve essere fornito il titolo della Recipe richiesta, la sua descrizione qualora presente e lo username dell'utente che ha fatto richiesta.  & Interno \\
		\hline
		ROF10.1.2  &  Per ogni voce dell'elenco deve essere fornito un pulsante per potere accedere alla visualizzazione dei dettagli della richiesta.  & Interno \\
		\hline
		ROF10.2  &  Il sistema visualizza un messaggio qualora non fossero presenti richieste di nuove Recipe.  & UC1.10.6 \\
		\hline
		ROF10.3  &  Premendo sul pulsante della visualizzazione dettagli si accede a un pannello contenente il form per la creazione di una nuova Recipe già compilato con i dati richiesti.  &  UC1.10.2 \newline Interno \\
		\hline
		ROF10.3.1  &  L'amministratore può decidere se modificare il titolo e la descrizione generale della Recipe richiesta.  & UC1.10.3.1 \newline UC1.10.3.2 \\
		\hline
		ROF10.3.2  &  L'amministratore non può modificare altri campi della richiesta oltre a quelli descritti in ROF10.3.1.  & Interno \\
		\hline
		RDF10.3.2.1  &  I campi non modificabili saranno disabilitati.  & Interno \\
		\hline
		ROF10.4  &  L'amministratore premendo sul pulsante di accettazione andrà a inserire la Recipe richiesta nel sistema.  & UC1.10.4 \newline Interno \\
		\hline
		ROF10.5  &  L'amministratore, premendo sul pulsante di respinta andrà a rifiutare la Recipe richiesta.  & UC1.10.5 \\
		\hline
		RDF10.5.1  &  L'amministratore potrà fornire un messaggio di risposta che motivi il rifiuto della richiesta.  & Interno \\
		\hline
		RDF10.5.1.1  &  Il sistema invierà una mail all'utente per notificare il messaggio di rifiuto.  & Interno \\
		\hline




		ROF11  &  Il sistema fornisce una serie di servizi REST  & Capitolato \\
		\hline
		ROF11.1  &  Il sistema è un grado di gestire una richiesta di un token di accesso per utilizzare i servizi REST se l'utente è autenticato  & UC2.1 \newline UC2.1.1 \\
		\hline
		ROF11.1.1  &  [TO CHANGE] Il sistema restituisce un errore se il sistema ha avuto problemi nel generare il token & UC2.1.2 \\
		\hline
		ROF11.1.2  &  Il sistema genera e restituisce un token di accesso se l'utente è autenticato  & UC2.1.3 \\
		\hline
		ROF11.2  &  I token di accesso sono stringhe alfanumeriche di [TO DO] caratteri  & Interno \\
		\hline
		ROF11.3  &  L'utente può richiedere che un token valido venga considerato non valido  & UC2.3 \\
		\hline
		ROF11.3.1  &  Il sistema è in grado di gestire una richiesta di chiusura sessione contenente un token di accesso valido rendendolo non valido & UC2.3 \newline UC2.3.1 \\
		\hline
		ROF11.3.2  &  [TO CHANGE] Il sistema restituisce un errore se non è riuscito a invalidare il token & UC2.3.2 \\
		\hline
		ROF11.3.3  &  Il sistema restituisce una conferma dell'avvenuta invalidazione del token & UC2.3.3 \\
		\hline
		ROF11.4  &  Il sistema è in grado di gestire una richiesta che richiede i dati di una View contenente un token di accesso  &  UC2.2 \newline UC2.2.1 \\
		\hline
		ROF11.4.1  &  Il sistema restituisce i dati richiesti se il token utilizzato è valido  &  UC2.2.3 \\
		\hline
		ROF11.4.2  &  Il sistema restituisce un errore se il token utilizzato è non valido  &  UC2.2.2 \\
		\hline

	\end{longtable}
	\egroup
\end{center}


\subsection{Requisiti prestazionali}

\begin{center}

	\def\arraystretch{1.5}
	\bgroup
	\begin{longtable}{| p{2.5cm} | p{8cm} | p{2cm} |}

		\hline
		\textbf{Requisito} & \textbf{Descrizione} & \textbf{Fonti} \\
		\hline

		RDP1  &  Una volta autenticato l'utente deve poter visualizzare le proprie View entro 10 secondi salvo problemi di connessione &  Interno \\
		\hline
		RDP1.1  &  Il recupero dei dati dal database deve impiegare al più 7 secondi  &  Interno \\
		\hline
		RFP2  &  L'interfaccia web utilizzerà un design di tipo responsive  &  Capitolato \\
		\hline

	\end{longtable}
	\egroup
\end{center}

\subsection{Requisiti di qualità}

\begin{center}

	\def\arraystretch{1.5}
	\bgroup
	\begin{longtable}{| p{2.5cm} | p{8cm} | p{2cm} |}

		\hline
		\textbf{Requisito} & \textbf{Descrizione} & \textbf{Fonti} \\
		\hline

		ROQ1  &  Viene fornito un manuale per l'utente  &  Interno \\
		\hline
		ROQ1.1  &  Il manuale utente deve rispettare le norme e le metriche descritte nel \docNameVersionPdQ  &  Interno \\
		\hline
		ROQ2  &  Tutto il codice rispetta le norme e le metriche descritte nel \docNameVersionPdQ{} e nelle \docNameVersionNdP  &  Interno \\
		\hline
		RDQ1  &  Viene fornito un manuale per l'uso dei servizi REST offerti dal sistema  &  Interno \\
		\hline
		RDQ1.1  &  Il manuale dei servizi REST deve rispettare le norme e le metriche descritte nel \docNameVersionPdQ  &  Interno \\
		\hline


	\end{longtable}
	\egroup
\end{center}

\subsection{Requisiti di vincolo}

\begin{center}

	\def\arraystretch{1.5}
	\bgroup
	\begin{longtable}{| p{2.5cm} | p{8cm} | p{2cm} |}

		\hline
		\textbf{Requisito} & \textbf{Descrizione} & \textbf{Fonti} \\
		\hline

		RDV1  &  Il sistema sfrutterà gli strumenti offerti da Google Cloud Platform  &  Capitolato \newline Verbale 2015/01/14 \\
		\hline
		RDV1.1  &  Il sistema sfrutterà Google App Engine  &  Capitolato \newline Verbale 2015/01/14 \\
		\hline
		RDV1.2  &  Il sistema sfrutterà Google Cloud Datastore  &  Capitolato \newline Verbale 2015/01/14 \\
		\hline
		RDV1.2.1  &  Il database sarà di tipo non relazione composto da documenti in stile JSON, stesso formato dei dati in entrata  &  Interno \newline Verbale 2015/01/14 \\
		\hline
		RDV1.3  &  Il sistema sfrutterà gli strumenti offerti da Google Cloud Platform  &  Capitolato \newline Verbale 2015/01/14 \\
		\hline
		RDV2  &  Il linguaggio di programmazione principalmente usato per il server sarà Python  &  Capitolato \newline Verbale 2015/01/14 \\
		\hline
		RDV3  &  Il codice sorgente sarà soggetto a versionamento tramite il modello di branching descritto nelle \docNameVersionNdP  &  Capitolato \\
		\hline
		RDV4  &  L'interfaccia web sarà di tipo single-page  &  Interno \\
		\hline
		ROV5  &  Il corretto funzionamento dell'interfaccia web deve essere garantito per l'utilizzo con i principali browser attualmente sul mercato   &  Interno \\
		\hline
		ROV5.1  &  Ogni funzionalità dell'interfaccia web descritta nei requisiti funzionali deve essere compatibile con il browser Google Chrome v. 39.0 e successive   &  Interno \\
		\hline
		ROV5.2  &  Ogni funzionalità dell'interfaccia web descritta nei requisiti funzionali deve essere compatibile con il browser Firefox v. 35.0 e successive   &  Interno \\
		\hline
		ROV5.3  &  Ogni funzionalità dell'interfaccia web descritta nei requisiti funzionali deve essere compatibile con il browser Safari v. 8.0 e successive   &  Interno \\
		\hline

		ROV6  &  Il sistema utilizzerà delle librerie esterne per effettuare le chiamate alle API dei diversi social.  &  Interno \\
		\hline
		ROV6.1  &  Il sistema utilizzerà facebook-sdk per effettuare le chiamate alle API di Facebook.  &  Interno \\
		\hline
		ROV6.2  &  Il sistema utilizzerà tweepy per effettuare le chiamate alle API di Twitter.  &  Interno \\
		\hline
		ROV6.3  &  Il sistema utilizzerà python-instagram per effettuare le chiamate alle API di Instagram.  &  Interno \\
		\hline

		ROV7  &  Il sistema utilizzerà delle librerie esterne per generare i grafici necessari alle View.  &  Interno \\
		\hline
		ROV7.1  &  Il sistema utilizzerà Chart.js per generare i Line Chart, Bar Chart e Pie Chart.  &  Interno \\
		\hline
		ROV7.2  &  Il sistema utilizzerà [TO DO] per generare i Map Chart.  &  Interno \\
		\hline

	\end{longtable}
	\egroup
\end{center}

