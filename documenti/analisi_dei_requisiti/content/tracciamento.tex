% =================================================================================================
% File:			capitolo_3.tex
% Description:	Definisce il capitolo che descrive generalmente il prodotto per il commitente
% Created:		2014-12-10
% Author:		Roetta Marco
% Email:		roetta.marco@mashup-unipd.it
% =================================================================================================
% Modification History:
% Version		Modifier Date		Change											Author
% 0.0.1 		2014-12-10 			aggiunta sezione e iniziata stesura				Roetta Marco
% =================================================================================================
% 0.0.2 		2015-01-13 			aggiunte tabelle								Carnovalini Filippo
% =================================================================================================
% 0.0.3 		2015-01-18 			riempite tabelle fino a requisiti.tex v0.0.6 	Carnovalini Filippo
% =================================================================================================
% 0.0.4 		2015-01-19 			stesura rimaneti voci tracciamento			 	Cusinato Giacomo
% =================================================================================================
% 1.0.0			2015-01-21 			inizio verifica documento			 			Nicola Faccin
% =================================================================================================
% 2.0.1			2015-02-18			eleminate sezioni e gestite tramite inclusione	Tesser Paolo
% =================================================================================================
%

% CONTENUTO DEL CAPITOLO

\section{Tracciamento} % (fold)
\label{sec:tracciamento}
	\subsection{Requisiti-Fonti} % (fold)
\label{ssub:requisiti_fonti}
\begin{center}
\def\arraystretch{1.5}
\bgroup
\begin{longtable}{| p{4cm} | p{4cm} |}
\hline
\textbf{Requisito} & \textbf{Fonti} \\
\hline
ROF1   &  UC1 \newline UC1.8 \\
\hline
ROF1.1   &  UC1.8.1 \\
\hline
ROF1.1.1   &  Interno \\
\hline
ROF1.1.2   &  Interno \\
\hline
ROF1.1.3   &  Interno \\
\hline
ROF1.2   &  UC1.8.2 \\
\hline
ROF1.3   &  UC1.8.3 \\
\hline
ROF1.3.1   &  Interno \\
\hline
ROF1.3.2   &  Interno \\
\hline
ROF1.3.3   &  Interno \\
\hline
ROF1.4   &  Interno \\
\hline
ROF1.4.1   &  Interno \\
\hline
RDF1.5   &  Interno \\
\hline
ROF1.6   &  UC1.8.4 \\
\hline
ROF1.7   &  UC1.8.5 \\
\hline
RDF1.8   &  Interno \\
\hline
ROF2   &  UC1 \newline UC1.1 \\
\hline
ROF2.1   &  UC1.1.1 \\
\hline
ROF2.2   &  UC1.1.2 \\
\hline
ROF2.3   &  UC1.1.3 \\
\hline
ROF2.4   &  UC1.1.4 \\
\hline
ROF2.4.1   &  UC1.1.4 \\
\hline
ROF2.4.2   &  UC1.1.4 \\
\hline
ROF2.5   &  Interno  \\
\hline
ROF3   &  UC1 \newline UC1.2 \\
\hline
ROF3.1   &  UC1.2.1 \\
\hline
ROF3.1.1   &  Interno \\
\hline
ROF3.1.2   &  UC1.2.1.1 \\
\hline
ROF3.1.3   &  UC1.2.1.2 \\
\hline
RFF3.1.4   &  UC1.2.1.3 \\
\hline
RFF3.2   &  UC1.2.2 \\
\hline
RFF3.2.1   &  Interno \\
\hline
RFF3.2.2   &  UC1.2.2.1 \\
\hline
RFF3.2.3   &  UC1.2.2.2 \\
\hline
ROF3.3   &  UC1 \newline UC1.5 \\
\hline
ROF3.3.1   &  UC1.5 \newline UC1.5.1 \newline UC1.5.2 \newline UC1.5.3 \newline UC1.5.4 \\
\hline
ROF3.3.2   &  UC1.5.2 \\
\hline
ROF3.3.2.1   &  Interno \\
\hline
ROF3.3.2.2   &  Interno \\
\hline
ROF3.3.3   &  UC1.5.3 \newline UC1.5.3.1  \\
\hline
ROF3.3.3.1   &  UC1.5.3 \newline UC1.5.3.2  \\
\hline
ROF3.3.3.2   &  Interno \\
\hline
ROF3.3.3.3   &  Interno \\
\hline
ROF3.3.4   &  UC1.5.4 \\
\hline
ROF3.3.4.1   &  UC1.5.4.1 \\
\hline
ROF3.3.4.2   &  UC1.5.4.2 \\
\hline
ROF3.3.4.2.1   &  Interno \\
\hline
ROF3.3.4.3   &  UC1.5.4.3 \\
\hline
ROF3.3.4.3.1   &  Interno \\
\hline
RDF3.3.5   &  Interno \\
\hline
ROF3.3.6   &  UC1.5.6 \\
\hline
ROF3.3.6.1   &  Interno \\
\hline
ROF3.3.6.2   &  UC1.5.7 \\
\hline
ROF3.3.6.3   &  UC1.5.5 \\
\hline
ROF3.3.7   &  UC1.5.6 \\
\hline
ROF3.3.7.1   &  UC1.5.5.1 \\
\hline
ROF3.3.7.2   &  UC1.5.5.2 \\
\hline
ROF3.3.7.2.1   &  Interno \\
\hline
ROF3.3.7.2.2   &  Interno \\
\hline
ROF4   &  UC1 \newline UC1.6 \\
\hline
ROF4.1   &  Interno \\
\hline
ROF4.1.1   &  UC1.6.1 \\
\hline
ROF4.1.1.1   &  UC1.6.2 \\
\hline
ROF4.1.2   &  Interno \\
\hline
ROF5   &  UC1 \newline UC1.3 \\
\hline
ROF5.1   &  Interno \\
\hline
ROF5.1.1   &  Interno \\
\hline
ROF5.1.2   &  UC1.3 \\
\hline
RFF5.1.3   &  UC1.3.3 \\
\hline
RFF5.1.3.1   &  UC1.3.3 \\
\hline
RFF5.1.3.2   &  Intero \\
\hline
ROF5.2   &  UC1.3.1 \\
\hline
ROF5.2.1   &  UC1.3.1 \\
\hline
ROF5.2.1.1   &  UC1.3.1 \newline Capitolato \\
\hline
ROF5.2.1.2   &  UC1.3.1 \\
\hline
ROF5.2.1.3   &  UC1.3.1 \\
\hline
ROF5.2.2   &  UC1.3.1.4  \\
\hline
ROF5.3   &  UC1.3.1.1 \\
\hline
ROF5.3.1   &  UC1.3.1.1  \\
\hline
ROF5.3.1.1   &  Interno \\
\hline
ROF5.3.1.2   &  Interno \\
\hline
ROF5.3.1.3   &  Interno \\
\hline
ROF5.3.1.4   &  Interno \\
\hline
ROF5.3.1.5   &  Interno \\
\hline
ROF5.3.1.6   &  Interno \\
\hline
ROF5.4   &  UC1.3.1.2  \\
\hline
ROF5.4.1   &  UC1.3.1.2  \\
\hline
ROF5.4.1.1   &  Interno \\
\hline
ROF5.4.1.2   &  Interno \\
\hline
ROF5.4.1.3   &  Interno \\
\hline
ROF5.4.1.4   &  Interno \\
\hline
ROF5.4.1.5   &  Interno \\
\hline
ROF5.4.1.6   &  Interno \\
\hline
ROF5.4.1.7   &  Interno \\
\hline
ROF5.5   &  UC1.3.1.3  \\
\hline
ROF5.5.1   &  UC1.3.1.3  \\
\hline
ROF5.5.1.1   &  Interno \\
\hline
ROF5.5.1.2   &  Interno \\
\hline
ROF5.5.1.3   &  Interno \\
\hline
ROF5.5.1.4   &  Interno \\
\hline
ROF5.5.1.5   &  Interno \\
\hline
ROF5.5.1.6   &  Interno \\
\hline
ROF5.5.1.7   &  Interno \\
\hline
ROF5.5.1.8   &  Interno \\
\hline
ROF5.5.1.9   &  Interno \\
\hline
ROF5.5.1.10   &  Interno \\
\hline
ROF5.6   &  UC1.3.2 \newline UC1.3.2.1 \\
\hline
ROF5.6.1   &  UC1.3.2.1.1 \\
\hline
ROF5.6.2   &  UC1.3.2.1.2 \\
\hline
ROF5.6.2.1   &  Interno \\
\hline
ROF5.6.2.2   &  Interno \\
\hline
ROF5.6.2.3   &  Interno \\
\hline
ROF5.6.3   &  UC1.3.2.1.3 \\
\hline
ROF5.6.3.1   &  UC1.3.2.1.3 \\
\hline
ROF5.6.4   &  UC1.3.2.2 \\
\hline
ROF5.6.4.1   &  UC1.3.2.3 \\
\hline
ROF5.6.4.1.1   &  Interno \\
\hline
ROF5.6.4.1.2   &  Interno \\
\hline
ROF5.6.4.1.3   &  Interno \\
\hline
ROF5.6.4.1.4   &  Interno \\
\hline
ROF5.6.4.1.5   &  Interno \\
\hline
ROF5.6.4.1.6   &  Interno \\
\hline
ROF5.6.4.2   &  Interno \\
\hline
ROF5.6.4.3   &  UC1.3.2.4 \\
\hline
RDF6   &  UC1 \newline UC1.11 \\
\hline
RDF6.1   &  UC1.11.1 \\
\hline
RDF6.1.1   &  UC1.11.4 \\
\hline
RDF6.2   &  UC1.11.2 \\
\hline
RDF6.2.1   &  Interno \\
\hline
RDF6.2.2   &  Interno \\
\hline
RDF6.3   &  UC1.11.3 \\
\hline
RDF6.3.1   &  Interno \\
\hline
RDF6.4   &  Interno \\
\hline
RFF7   &  UC1 \newline UC1.9 \\
\hline
RFF7.1   &  UC1.9 \newline UC1.9.1 \\
\hline
RFF7.1.1   &  UC1.9.2  \\
\hline
RFF7.1.1.1   &  UC1.9.2.1 \\
\hline
RFF7.1.1.2   &  UC1.9.2.2 \\
\hline
RFF7.1.1.3   &  UC1.9.2.3 \\
\hline
RFF7.1.1.4   &  Interno \\
\hline
RFF7.1.1.5   &  Interno \\
\hline
RFF7.1.1.6   &  Interno \\
\hline
RFF7.1.1.7   &  Interno \\
\hline
RFF7.1.1.8   &  Interno \\
\hline
RFF7.1.1.9   &  Interno \\
\hline
RFF7.1.1.10   &  Interno \\
\hline
RFF7.1.1.11   &  Interno \\
\hline
RFF7.1.1.12   &  Interno \\
\hline
RFF7.1.2   &  UC1.9.3  \\
\hline
RFF7.1.3   &  UC1.9.4  \\
\hline
RFF7.2   &  Interno \\
\hline
ROF8   &  UC1 \newline UC1.4 \newline UC1.7 \\
\hline
ROF8.1   &  UC1 \newline UC1.4.1 \newline UC1.4.1.1 \\
\hline
ROF8.1.1   &  UC1.4.1.2 \\
\hline
ROF8.1.1.1   &  UC1.4.1.2.1 \\
\hline
ROF8.1.1.2   &  UC1.4.1.2.2 \\
\hline
ROF8.1.1.3   &  UC1.4.1.2.3 \\
\hline
ROF8.1.1.4   &  Interno \\
\hline
ROF8.2   &  UC1.4.1.3 \\
\hline
ROF8.2.1   &  UC1.4.2 \\
\hline
ROF8.2.1.1   &  Interno \\
\hline
ROF8.2.1.2   &  Interno \\
\hline
ROF8.2.2   &  Interno \\
\hline
ROF8.3   &  UC1.4.3 \newline UC1.4.3.1  \\
\hline
ROF8.3.1   &  UC1.4.3.2 \\
\hline
ROF8.3.2   &  Interno \\
\hline
ROF8.3.3   &  Interno \\
\hline
RFF8.4   &  UC1.4.4  \\
\hline
RFF8.4.1   &  UC1.4.4  \\
\hline
RFF8.4.1.1   &  UC1.4.4  \\
\hline
RFF8.4.1.2   &  Interno  \\
\hline
RFF8.4.2   &  UC1.4.5  \\
\hline
ROF9   &  UC1.7.1 \\
\hline
ROF9.1   &  UC1.7.2 \\
\hline
RFF9.1.1   &  Interno \\
\hline
ROF9.1.2   &  Interno \\
\hline
ROF9.2   &  UC1.7.3 \\
\hline
RFF9.2.1   &  Interno \\
\hline
ROF9.2.2   &  Interno \\
\hline
ROF9.3   &  UC1.7.4 \\
\hline
ROF9.3.1   &  UC1.7.4 \\
\hline
ROF9.3.2   &  UC1.7.5 \\
\hline
RFF10   &  UC1 \newline UC1.10 \\
\hline
RFF10.1   &  UC1.10.1 \\
\hline
RFF10.1.1   &  Interno \\
\hline
RFF10.1.2   &  Interno \\
\hline
RFF10.2   &  UC1.10.6 \\
\hline
RFF10.3   &  UC1.10.2 \newline Interno \\
\hline
RFF10.3.1   &  UC1.10.3 \newline UC1.10.3.1 \newline UC1.10.3.2 \\
\hline
RFF10.3.2   &  Interno \\
\hline
RFF10.3.2.1   &  Interno \\
\hline
RFF10.4   &  UC1.10.4 \\
\hline
RFF10.5   &  UC1.10.5 \newline UC1.10.5.1 \newline UC1.10.5.3 \\
\hline
RFF10.5.1   &  UC1.10.5.2 \\
\hline
RFF10.5.1.1   &  Interno \\
\hline
ROF11   &  Capitolato \newline UC2 \\
\hline
ROF11.1   &  UC2.1 \newline UC2.1.1 \\
\hline
ROF11.1.1   &  UC2.1.2 \\
\hline
ROF11.1.2   &  Interno \\
\hline
ROF11.2   &  UC2.3  \\
\hline
ROF11.2.1   &  UC2.3 \newline UC2.3.1 \\
\hline
ROF11.2.2   &  UC2.3.2 \\
\hline
ROF11.2.3   &  UC2.3.3 \\
\hline
ROF11.3   &  UC2.2 \newline UC2.2.1 \\
\hline
ROF11.3.1   &  UC2.2.3 \\
\hline
ROF11.3.1.1   &  Interno \\
\hline
ROF11.3.1.2   &  Interno \\
\hline
ROF11.3.1.3   &  Interno \\
\hline
ROF11.3.2   &  UC2.2.2 \\
\hline
ROF12   &  UC2.4  \\
\hline
ROF12.1   &  UC2.4  \\
\hline
ROF12.1.1   &  Interno \\
\hline
ROF12.1.2   &  Interno \\
\hline
ROF12.1.3   &  Interno \\
\hline
ROF12.2   &  Interno \\
\hline
RDP1   &  Interno \\
\hline
RDP1.1   &  Interno \\
\hline
RFP2   &  Capitolato \\
\hline
ROQ1   &  Interno \\
\hline
ROQ1.1   &  Interno \\
\hline
ROQ2   &  Interno \\
\hline
ROQ3   &  Interno \\
\hline
ROQ3.1   &  Interno \\
\hline
RDV1   &  Capitolato \newline Verbale 2015/01/14 \\
\hline
RDV1.1   &  Capitolato \newline Verbale 2015/01/14 \\
\hline
RDV1.2   &  Capitolato \newline Verbale 2015/01/14 \\
\hline
RDV1.2.1   &  Interno \newline Verbale 2015/01/14 \\
\hline
RDV1.3   &  Capitolato \newline Verbale 2015/01/14 \\
\hline
RDV1.4   &  Capitolato \newline Verbale 2015/01/14 \\
\hline
RDV2   &  Capitolato \newline Verbale 2015/01/28 \\
\hline
RDV2.1   &  Verbale 2015/01/28 \\
\hline
RDV3   &  Capitolato \\
\hline
RDV4   &  Interno \\
\hline
ROV5   &  Interno \\
\hline
ROV5.1   &  Interno \\
\hline
ROV5.2   &  Interno \\
\hline
ROV5.3   &  Interno \\
\hline
ROV6   &  Interno \\
\hline
ROV6.1   &  Interno \\
\hline
ROV6.2   &  Interno \\
\hline
ROV6.3   &  Interno \\
\hline
ROV7   &  Interno \\
\hline
ROV7.1   &  Interno \\
\hline
ROV7.2   &  Interno \\
\hline
ROV8   &  Verbale 2015/01/28 \\
\hline
ROV8.1   &  Verbale 2015/01/28 \\
\hline
ROV8.2   &  Verbale 2015/01/28 \\
\hline
ROV8.2.1   &  Verbale 2015/01/28 \\
\hline
ROV9   &  Interno \\
\hline
\end{longtable}
\egroup
\end{center}
% subsubsection requisiti_fonti (end)

	\subsection{Fonti-Requisiti} % (fold)
\label{ssub:fonti_requisiti}
\begin{center}
\def\arraystretch{1.5}
\bgroup
\begin{longtable}{| p{4cm} | p{4cm} |}
\hline
\textbf{Fonte} & \textbf{Requisiti derivati} \\
\hline
Interno & ROF1.1.1 \newline ROF1.1.2 \newline ROF1.1.3 \newline ROF1.3.1 \newline ROF1.3.2 \newline ROF1.3.3 \newline ROF1.4 \newline ROF1.4.1 \newline ROF1.5 \newline RDF1.8 \newline ROF2.5 \newline ROF3.1.1 \newline ROF3.2.1 \newline ROF3.3.2 \newline ROF3.3.2.1 \newline ROF3.3.3.2 \newline ROF4.1 \newline ROF4.1.2 \newline ROF5.1 \newline ROF5.1.1 \newline ROF5.2 \newline ROF5.2.1 \newline ROF5.2.1.2 \newline ROF5.2.1.3 \newline ROF5.3 \newline ROF5.3.1 \newline ROF5.3.1.1 \newline ROF5.3.1.2 \newline ROF5.3.1.3 \newline ROF5.3.1.4 \newline ROF5.3.1.5 \newline ROF5.3.1.6 \newline ROF5.3.1.7 \newline ROF5.4 \newline ROF5.4.1 \newline ROF5.4.1.1 \newline ROF5.4.1.2 \newline ROF5.4.1.3 \newline ROF5.4.1.4 \newline ROF5.4.1.5 \newline ROF5.4.1.6 \newline ROF5.4.1.7 \newline ROF5.5 \newline ROF5.5.1 \newline ROF5.5.1.1 \newline ROF5.5.1.2 \newline ROF5.5.1.3 \newline ROF5.5.1.4 \newline ROF5.5.1.5 \newline ROF5.5.1.6 \newline ROF5.5.1.7 \newline ROF5.5.1.8 \newline ROF5.5.1.9 \newline ROF5.5.1.10 \newline ROF5.6.2.1 \newline ROF5.6.2.2 \newline ROF5.6.2.3 \newline ROF5.6.4.1.1 \newline ROF5.6.4.1.2 \newline ROF5.6.4.1.3 \newline ROF5.6.4.1.4 \newline ROF5.6.4.1.5 \newline ROF5.6.4.1.6 \newline ROF5.6.4.2 \newline ROF6.1 \newline ROF6.1.1 \newline ROF6.2 \newline ROF6.2.1 \newline ROF6.3 \newline ROF6.3.1 \newline ROF6.4 \newline ROF7.1.1 \newline ROF7.1.1.4 \newline ROF7.1.1.5 \newline ROF7.1.1.6 \newline ROF7.1.1.7 \newline ROF7.1.1.8 \newline ROF7.1.1.9 \newline ROF7.1.1.10 \newline ROF7.1.1.11 \newline ROF7.1.1.12 \newline ROF7.1.2 \newline ROF7.1.3 \newline ROF7.2 \newline ROF8.1.1.4 \newline ROF8.2.1.1 \newline ROF8.2.2 \newline ROF8.3.2 \newline ROF8.3.3 \newline RFF9.1.1 \newline ROF9.2.1 \newline ROF10.1.1 \newline ROF10.1.2 \newline ROF10.3 \newline ROF10.3.2 \newline RDF10.3.2.1 \newline ROF10.4 \newline RDF10.5.1.1 \newline ROF11.1.2 \newline ROF11.2 \newline ROF11.2.2 \newline ROF11.3.1.1 \newline ROF11.3.1.X \newline ROF11.3.1.X \newline ROF11.3.1.X \newline ROF11.3.1.X \newline RDP1 \newline RDP1.1 \newline ROQ1 \newline ROQ1.1 \newline ROQ2 \newline ROQ3 \newline ROQ3.1 \newline RDV1.2.1 \newline RDV4 \newline ROV5 \newline ROV5.1 \newline ROV5.2 \newline ROV5.3 \newline ROV6 \newline ROV6.1 \newline ROV6.2 \newline ROV6.3 \newline ROV7 \newline ROV7.1 \newline ROV7.2 \newline ROV9 \newline \\
\end{longtable}
\egroup
\end{center}
% subsubsection fonti_requisiti (end)


	\subsection{Riepilogo}

	\begin{center}

		\def\arraystretch{1.5}
		\bgroup
		\begin{longtable}{| p{2.7cm} | p{2.4cm} | p{2.4cm} | p{2.4cm} | p{1.7cm} |}

			\hline
			\textbf{Categoria} & \textbf{Obbligatori} & \textbf{Desiderabili} & \textbf{Facoltativi} & \textbf{Totali} \\
			\hline

			\textbf{Funzionali}  & 157 & 12 & 46 & 215 \\
			\hline
			\textbf{Prestazionali} & 0 & 2 & 1 & 3 \\
			\hline
			\textbf{Di qualità} & 5 & 0 & 0 & 5 \\
			\hline
			\textbf{Di vincolo} & 17 & 8 & 0 & 25 \\
			\hline
			\textbf{Totali}  & 179 & 22 & 47 & 248 \\
			\hline
		\end{longtable}
		\egroup
	\end{center}

	\subsection{Requisiti-Test} % (fold)
	\label{sub:requisiti_test}
	In questa sezione vengono riportati, per ogni requisito, il corrispondente test di sistema e test di validazione. I requisiti di qualità non sono tracciati in quanto sono verificati costantemente durante tutto lo sviluppo del progetto. La descrizione dei singoli test è riportata nel \docNameVersionPdQ.
		\begin{center}

\def\arraystretch{1.5}
\bgroup
\begin{longtable}{| p{4cm} | p{4cm} | p{4cm} |}

	\hline
	\textbf{Requisito} & \textbf{Test di validazione} &  \textbf{Test di sistema} \\
	\hline

	% Requisiti funzionali

	ROF1  & TVF1 & TSF1 \\
	\hline
	\hspace{2 mm} ROF1.1  & TVF1.1 &  \\
	\hline
	\hspace{2 mm} ROF1.2  & TVF1.2 &  \\
	\hline
	\hspace{2 mm} ROF1.3  & TVF1.3 &  \\
	\hline
	\hspace{2 mm} ROF1.4  & TVF1.4 &  \\
	\hline
	\hspace{2 mm} ROF1.8  & TVF1.8 &  \\
	\hline
	ROF2  & TVF2 & TSF2 \\
	\hline
	\hspace{2 mm} ROF2.4  & TVF2.4 &  \\
	\hline
	ROF3  & TVF3 & TSF3 \\
	\hline
	\hspace{2 mm} ROF3.1  & TVF3.1 &  \\
	\hline
	\hspace{2 mm} RFF3.2  & TVF3.2 &  \\
	\hline
	\hspace{2 mm} ROF3.3  & TVF3.3 &  \\
	\hline
	\hspace{4 mm} ROF3.3.2  & TVF3.3.2 &  \\
	\hline
	\hspace{4 mm} ROF3.3.3  & TVF3.3.3 &  \\
	\hline
	\hspace{4 mm} ROF3.3.4  & TVF3.3.4 &  \\
	\hline
	\hspace{4 mm} ROF3.3.7  & TVF3.3.7 &  \\
	\hline
	ROF4  & TVF4 & TSF4 \\
	\hline
	\hspace{2 mm} ROF4.1  & TVF4.1 &  \\
	\hline
	ROF5  & TVF5 & TSF5 \\
	\hline
	\hspace{2 mm} ROF5.1  & TVF5.1 &  \\
	\hline
	\hspace{4 mm} RFF5.1.3  & TVF5.1.3 &  \\
	\hline
	\hspace{2 mm} ROF5.2  & TVF5.2 &  \\
	\hline
	\hspace{2 mm} ROF5.3  & TVF5.3 &  \\
	\hline
	\hspace{4 mm} ROF5.3.1  & TVF5.3.1 &  \\
	\hline
	\hspace{2 mm} ROF5.4  & TVF5.4 &  \\
	\hline
	\hspace{4 mm} ROF5.4.1  & TVF5.4.1 &  \\
	\hline
	\hspace{2 mm} ROF5.5  & TVF5.5 &  \\
	\hline
	\hspace{4 mm} ROF5.5.1  & TVF5.5.1 &  \\
	\hline
	\hspace{2 mm} ROF5.6  & TVF5.6 &  \\
	\hline
	\hspace{4 mm} ROF5.6.4  & TVF5.6.4 &  \\
	\hline
	RDF6  & TVF6 & TSF6 \\
	\hline
	\hspace{2 mm} RDF6.1  & TVF6.1 &  \\
	\hline
	\hspace{2 mm} RDF6.2  & TVF6.2 &  \\
	\hline
	\hspace{2 mm} RDF6.3  & TVF6.3 &  \\
	\hline
	RFF7  & TVF7 & TSF7 \\
	\hline
	\hspace{2 mm} RFF7.1  & TVF7.1 &  \\
	\hline
	\hspace{4 mm} RFF7.1.1  & TVF7.1.1 &  \\
	\hline
	ROF8  & TVF8 & TSF8 \\
	\hline
	\hspace{2 mm} ROF8.1  & TVF8.1 &  \\
	\hline
	\hspace{4 mm} ROF8.1.1  & TVF8.1.1 &  \\
	\hline
	\hspace{2 mm} ROF8.2  & TVF8.2 &  \\
	\hline
	\hspace{2 mm} ROF8.3  & TVF8.3 &  \\
	\hline
	\hspace{2 mm} RFF8.4  & TVF8.4 &  \\
	\hline
	ROF9  & TVF9 & TSF9 \\
	\hline
	\hspace{2 mm} ROF9.1  & TVF9.1 &  \\
	\hline
	\hspace{2 mm} ROF9.2  & TVF9.2 &  \\
	\hline
	\hspace{2 mm} ROF9.3  & TVF9.3 &  \\
	\hline
	RFF10  & TVF10 & TSF10 \\
	\hline
	\hspace{2 mm} RFF10.1  & TVF10.1 &  \\
	\hline
	\hspace{2 mm} RFF10.3  & TVF10.3 &  \\
	\hline
	ROF11  & TVF11 & TSF11 \\
	\hline
	\hspace{2 mm} ROF11.1  & TVF11.1 &  \\
	\hline
	\hspace{2 mm} ROF11.2  & TVF11.2 &  \\
	\hline
	\hspace{2 mm} ROF11.3  & TVF11.3 &  \\
	\hline
	ROF12  & TVF12 & TSF12 \\
	\hline

	% Requisiti di qualità
	
	RDP1  &  & TSP1 \\
	\hline
	RFP2  &  & TSP2 \\
	\hline
	
	% Requisiti di qualità
	
	ROQ1  & TVQ1 & TSQ1 \\
	\hline
	ROQ2  &  & TSQ2 \\
	\hline
	ROQ3  & TVQ3 & TSQ3 \\
	\hline
	
	% Requisiti di vincolo
	
	RDV1  & TVV1 & TSV1 \\
	\hline
	\hspace{2 mm} RDV1.1  & TVV1.1 &  \\
	\hline
	\hspace{2 mm} RDV1.2  & TVV1.2 &  \\
	\hline
	RDV2  & TVV2 & TSV2 \\
	\hline
	RDV3  & TVV3 & TSV3 \\
	\hline
	RDV4  & TVV4 & TSV4 \\
	\hline
	ROV5  & TVV5 & TSV5 \\
	\hline
	ROV6  & TVV6 & TSV6 \\
	\hline
	\hspace{2 mm} ROV6.1  & TVV6.1 &  \\
	\hline
	\hspace{2 mm} ROV6.2  & TVV6.2 &  \\
	\hline
	\hspace{2 mm} ROV6.3  & TVV6.3 &  \\
	\hline
	ROV7  & TVV7 & TSV7 \\
	\hline
	\hspace{2 mm} ROV7.1  & TVV7.1 &  \\
	\hline
	\hspace{2 mm} ROV7.2  & TVV7.2 &  \\
	\hline
	ROV8  & TVV8 & TSV8 \\
	\hline
	\hspace{2 mm} ROV8.1  & TVV8.1 & \\
	\hline
	ROV9  & TVV9 & TSV9 \\
	\hline
	

\end{longtable}
\egroup
\end{center}
% subsection requisiti_test (end)

% section tracciamento (end)
