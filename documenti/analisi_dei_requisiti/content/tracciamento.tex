% =================================================================================================
% File:			capitolo_3.tex
% Description:	Definisce il capitolo che descrive generalmente il prodotto per il commitente
% Created:		2014-12-10
% Author:		Roetta Marco
% Email:		roetta.marco@mashup-unipd.it
% =================================================================================================
% Modification History:
% Version		Modifier Date		Change											Author
% 0.0.1 		2014-12-10 			aggiunta sezione e iniziata stesura				Roetta Marco
% =================================================================================================
% 0.0.2 		2015-01-13 			aggiunte tabelle								Carnovalini Filippo
% =================================================================================================
% 0.0.3 		2015-01-18 			riempite tabelle fino a requisiti.tex v0.0.6 	Carnovalini Filippo
% =================================================================================================
% 0.0.4 		2015-01-19 			stesura rimaneti voci tracciamento			 	Cusinato Giacomo
% =================================================================================================
% 1.0.0			2015-01-21 			inizio verifica documento			 			Nicola Faccin
% =================================================================================================
% 2.0.1			2015-02-18			eleminate sezioni e gestite tramite inclusione	Tesser Paolo
% =================================================================================================
%

% CONTENUTO DEL CAPITOLO

\section{Tracciamento} % (fold)
\label{sec:tracciamento}
	\subsection{Requisiti-Fonti} % (fold)
\label{ssub:requisiti_fonti}

\begin{center}

\def\arraystretch{1.5}
\bgroup
\begin{longtable}{| p{4cm} | p{4cm} |}

	\hline
	\textbf{Requisito} & \textbf{Fonti} \\
	\hline

	% Requisiti funzionali

	ROF1  & UC1.8 \\
	\hline
	ROF1.1  &  UC1.8.1 \\
	\hline
	ROF1.1.1  &  Interno \\
	\hline
	ROF1.1.2  &  Interno \\
	\hline
	ROF1.1.3  &  Interno \\
	\hline
	ROF1.2  &  UC1.8.2 \\
	\hline
	ROF1.3  &  UC1.8.3 \\
	\hline
	ROF1.3.1  &  Interno \\
	\hline
	ROF1.3.2  &  Interno \\
	\hline
	ROF1.3.3  &  Interno \\
	\hline
	ROF1.4  &  Interno \\
	\hline
	ROF1.4.1  &  Interno \\
	\hline
	ROF1.5  &  Interno \\
	\hline
	ROF1.6  &  UC1.8.4 \\
	\hline
	ROF1.7  &  UC1.8.5 \\
	\hline
	RDF1.8  &  Interno \\
	\hline
	ROF2  &  UC1.1 \\
	\hline
	ROF2.1  &  UC1.1.1 \\
	\hline
	ROF2.2  &  UC1.1.2 \\
	\hline
	ROF2.3  &  UC1.1.3 \\
	\hline
	ROF2.4  &  UC1.1.4 \\
	\hline
	ROF2.4.1  &  UC1.1.4 \\
	\hline
	ROF2.4.2  &  UC1.1.4 \\
	\hline
	ROF2.5  &  Interno  \\
	\hline
	ROF3  &  UC1.2 \newline UC1.5 \\
	\hline
	ROF3.1  &  UC1.2.1 \\
	\hline
	ROF3.1.1  &  Interno \\
	\hline
	ROF3.1.2  &  UC1.2.1 \\
	\hline
	ROF3.1.2.1  &  UC1.2.1.1 \\
	\hline
	ROF3.1.2.2  &  UC1.2.1.2 \\
	\hline
	ROF3.1.2.3  &  UC1.2.1.3 \\
	\hline
	ROF3.2  &  UC1.2.2  \\
	\hline
	ROF3.2.1  &  Interno \\
	\hline
	ROF3.2.2  &  UC1.2.2 \\
	\hline
	ROF3.2.2.1  &  UC1.2.2.1 \\
	\hline
	ROF3.2.2.2  &  UC1.2.2.2 \\
	\hline
	ROF3.2.2.3  &  UC1.2.2.3 \\
	\hline				
	ROF3.3  &  UC1.5 \\
	\hline
	ROF3.3.1  &  UC1.5 \newline UC1.5.1 \newline UC1.5.2 \newline UC1.5.3 \\
	\hline
	ROF3.3.2  &  Interno \\
	\hline
	ROF3.3.2.1  &  Interno \\
	\hline
	ROF3.3.3  &  UC1.5.4 \\
	\hline
	ROF3.3.3.1  &  UC1.5.5 \\
	\hline
	ROF3.3.3.2  &  Interno \\
	\hline
	ROF4  &  UC1.6 \\
	\hline
	ROF4.1  &  Interno \\
	\hline
	ROF4.1.1  &  UC1.6.1 \\
	\hline
	ROF4.1.1.1  &  UC1.6.2 \\
	\hline
	ROF4.1.2  &  Interno \\
	\hline
	ROF5  &  UC1.3 \newline UC1.3.1 \\
	\hline
	ROF5.1  &  Interno \\
	\hline
	RDF5.2  &  UC1.3.1.1 \\
	\hline
	RDF5.2.1  &  Interno \\
	\hline
	RDF5.2.2  &  UC1.3.4 \newline UC1.3.4.1 \newline UC1.3.4.2 \newline UC1.3.6.1 \\
	\hline
	RDF5.2.2.1  &  Interno \\
	\hline
	RDF6  &  Interno \\
	\hline
	RDF6.1  &  Interno \\
	\hline
	RDF6.2  &  UC1.3.4.3 \\
	\hline
	RDF6.2.1  &  Interno \\
	\hline
	RDF6.3  &  UC1.3.4.4 \\
	\hline
	RDF6.3.1  &  Interno \\
	\hline
	RDF6.3.1.1  &  Interno \\
	\hline
	RDF6.3.2  &  UC1.3.5 \\
	\hline
	RDF6.4  &  UC1.3.6 \newline UC1.3.6.2 \\
	\hline
	RDF6.4.1  &  UC1.3.6.3 \\
	\hline
	RDF6.4.2  &  Interno \\
	\hline
	RDF6.4.3  &  Interno \\
	\hline
	ROF7  &  UC1.3.2 \newline UC1.3.2.1 \\
	\hline
	ROF7.1  &  UC1.3.2.2 \\
	\hline
	ROF7.1.1  &  Interno \\
	\hline
	ROF7.2  &  UC1.3.2.3 \\
	\hline
	ROF7.3  &  Interno \\
	\hline
	ROF7.3.1  &  Interno \\
	\hline
	ROF7.3.2  &  UC1.3.3 \\
	\hline
	ROF8  &  UC1.4 \newline UC1.7 \\
	\hline
	ROF8.1  &  UC1.4.1 \\
	\hline
	ROF8.2  &  UC1.4.2 \newline UC1.4.2.1 \newline UC1.4.2.2 \\
	\hline
	ROF8.2.1  &  Interno \newline UC1.4.2.3 \\
	\hline
	ROF8.2.2  &  UC1.4.2.4 \\
	\hline
	ROF8.3  &  UC1.4.3 \\
	\hline
	ROF8.3.1  &  Interno \\
	\hline
	ROF8.4  &  UC1.4.4 \newline 1.4.4.1 \newline UC1.4.4.2 \\
	\hline
	ROF8.4.1  &  UC1.4.4.3 \\
	\hline
	ROF8.4.2  &  Interno \\
	\hline
	ROF8.4.3  &  Interno \\
	\hline
	RFF8.5	&	Interno\\
	\hline
	RDF9  &  UC1.7.1 \newline UC1.7.2.1 \newline UC1.7.3.1 \\
	\hline
	RDF9.1  &  Interno \\
	\hline
	RDF9.1.1  &  UC1.7.2.2 \\
	\hline
	RFF9.2  &  UC1.7.2 \newline UC1.7.2.3  \\
	\hline
	RFF9.2.1  &  UC1.7.2.4 \\
	\hline
	RDF9.3  & UC1.7.3  \newline UC1.7.3.2 \\
	\hline
	RDF9.3.1  &  UC1.7.3.3\\
	\hline
	RDF9.3.1.1  &  Interno \\
	\hline
	RDF9.3.1.2  &  UC1.7.3.3\\
	\hline
	RDF9.3.1.3  &  UC1.7.4\\
	\hline


	% Requisiti prestazionali

	RDP1  &  Interno \\
	\hline
	RDP1.1  &  Interno \\
	\hline
	RFP2  &  Capitolato \\
	\hline


	% Requisiti di qualità

	ROQ1  &  Interno \\
	\hline
	ROQ1.1  &  Interno \\
	\hline
	ROQ2  &  Interno \\
	\hline



	% Requisiti di vincolo

	RDV1 &  Capitolato \newline Verbale 2015/01/14 \\
	\hline
	RDV1.1  &  Capitolato \newline Verbale 2015/01/14 \\
	\hline
	RDV1.2  &  Capitolato \newline Verbale 2015/01/14 \\
	\hline
	RDV1.2.1  &  Interno \newline Verbale 2015/01/14 \\
	\hline
	RDV1.3  &  Capitolato \newline Verbale 2015/01/14 \\
	\hline
	RDV2  &  Capitolato \newline Verbale 2015/01/14 \\
	\hline
	RDV3  &  Capitolato \\
	\hline
	RDV4  &  Interno \\
	\hline
	ROV5  &  Interno \\
	\hline
	ROV5.1  &  Interno \\
	\hline
	ROV5.2  &  Interno \\
	\hline
	ROV5.3  &  Interno \\
	\hline


\end{longtable}
\egroup
\end{center}

% subsubsection requisiti_fonti (end)
	\subsection{Fonti-Requisiti} % (fold)
\label{ssub:fonti_requisiti}
\begin{center}
\def\arraystretch{1.5}
\bgroup
\begin{longtable}{| p{4cm} | p{4cm} |}
\hline
\textbf{Fonte} & \textbf{Requisiti derivati} \\
\hline
Interno & ROF1.1.1 \newline ROF1.1.2 \newline ROF1.1.3 \newline ROF1.3.1 \newline ROF1.3.2 \newline ROF1.3.3 \newline ROF1.4 \newline ROF1.4.1 \newline ROF1.5 \newline RDF1.8 \newline ROF2.5 \newline ROF3.1.1 \newline ROF3.2.1 \newline ROF3.3.2 \newline ROF3.3.2.1 \newline ROF3.3.3.2 \newline ROF4.1 \newline ROF4.1.2 \newline ROF5.1 \newline ROF5.1.1 \newline ROF5.2 \newline ROF5.2.1 \newline ROF5.2.1.2 \newline ROF5.2.1.3 \newline ROF5.3 \newline ROF5.3.1 \newline ROF5.3.1.1 \newline ROF5.3.1.2 \newline ROF5.3.1.3 \newline ROF5.3.1.4 \newline ROF5.3.1.5 \newline ROF5.3.1.6 \newline ROF5.3.1.7 \newline ROF5.4 \newline ROF5.4.1 \newline ROF5.4.1.1 \newline ROF5.4.1.2 \newline ROF5.4.1.3 \newline ROF5.4.1.4 \newline ROF5.4.1.5 \newline ROF5.4.1.6 \newline ROF5.4.1.7 \newline ROF5.5 \newline ROF5.5.1 \newline ROF5.5.1.1 \newline ROF5.5.1.2 \newline ROF5.5.1.3 \newline ROF5.5.1.4 \newline ROF5.5.1.5 \newline ROF5.5.1.6 \newline ROF5.5.1.7 \newline ROF5.5.1.8 \newline ROF5.5.1.9 \newline ROF5.5.1.10 \newline ROF5.6.2.1 \newline ROF5.6.2.2 \newline ROF5.6.2.3 \newline ROF5.6.4.1.1 \newline ROF5.6.4.1.2 \newline ROF5.6.4.1.3 \newline ROF5.6.4.1.4 \newline ROF5.6.4.1.5 \newline ROF5.6.4.1.6 \newline ROF5.6.4.2 \newline ROF6.1 \newline ROF6.1.1 \newline ROF6.2 \newline ROF6.2.1 \newline ROF6.3 \newline ROF6.3.1 \newline ROF6.4 \newline ROF7.1.1 \newline ROF7.1.1.4 \newline ROF7.1.1.5 \newline ROF7.1.1.6 \newline ROF7.1.1.7 \newline ROF7.1.1.8 \newline ROF7.1.1.9 \newline ROF7.1.1.10 \newline ROF7.1.1.11 \newline ROF7.1.1.12 \newline ROF7.1.2 \newline ROF7.1.3 \newline ROF7.2 \newline ROF8.1.1.4 \newline ROF8.2.1.1 \newline ROF8.2.2 \newline ROF8.3.2 \newline ROF8.3.3 \newline RFF9.1.1 \newline ROF9.2.1 \newline ROF10.1.1 \newline ROF10.1.2 \newline ROF10.3 \newline ROF10.3.2 \newline RDF10.3.2.1 \newline ROF10.4 \newline RDF10.5.1.1 \newline ROF11.1.2 \newline ROF11.2 \newline ROF11.2.2 \newline ROF11.3.1.1 \newline ROF11.3.1.X \newline ROF11.3.1.X \newline ROF11.3.1.X \newline ROF11.3.1.X \newline RDP1 \newline RDP1.1 \newline ROQ1 \newline ROQ1.1 \newline ROQ2 \newline ROQ3 \newline ROQ3.1 \newline RDV1.2.1 \newline RDV4 \newline ROV5 \newline ROV5.1 \newline ROV5.2 \newline ROV5.3 \newline ROV6 \newline ROV6.1 \newline ROV6.2 \newline ROV6.3 \newline ROV7 \newline ROV7.1 \newline ROV7.2 \newline ROV9 \newline \\
\end{longtable}
\egroup
\end{center}
% subsubsection fonti_requisiti (end)


	\subsection{Riepilogo}

	\begin{center}

		\def\arraystretch{1.5}
		\bgroup
		\begin{longtable}{| p{2.7cm} | p{2.4cm} | p{2.4cm} | p{2.4cm} | p{1.7cm} |}

			\hline
			\textbf{Categoria} & \textbf{Obbligatori} & \textbf{Desiderabili} & \textbf{Facoltativi} & \textbf{Totali} \\
			\hline

			\textbf{Funzionali}  & 157 & 12 & 46 & 215 \\
			\hline
			\textbf{Prestazionali} & 0 & 2 & 1 & 3 \\
			\hline
			\textbf{Di qualità} & 5 & 0 & 0 & 5 \\
			\hline
			\textbf{Di vincolo} & 17 & 8 & 0 & 25 \\
			\hline
			\textbf{Totali}  & 179 & 22 & 47 & 248 \\
			\hline
		\end{longtable}
		\egroup
	\end{center}

	\subsection{Requisiti-Test} % (fold)
	\label{sub:requisiti_test}
	In questa sezione vengono riportati, per ogni requisito, il corrispondente test di sistema e test di validazione. I requisiti di qualità non sono tracciati in quanto sono verificati costantemente durante tutto lo sviluppo del progetto. La descrizione dei singoli test è riportata nel \docNameVersionPdQ.
		\begin{center}

\def\arraystretch{1.5}
\bgroup
\begin{longtable}{| p{4cm} | p{4cm} | p{4cm} |}

	\hline
	\textbf{Requisito} & \textbf{Test di validazione} &  \textbf{Test di sistema} \\
	\hline

	% Requisiti funzionali

	ROF1  & [TO DO] & [TO DO] \\
	\hline


\end{longtable}
\egroup
\end{center}
% subsection requisiti_test (end)

% section tracciamento (end)
