% =================================================================================================
% File:			requisiti_accettati.tex
% Description:	Definisce il capitolo che contiene i requisiti accettati in forma tabellare
% Created:		2015-06-15
% Author:		Roetta Marco
% Email:		roetta.marco@mashup-unipd.it
% =================================================================================================
% Modification History:
% Version		Modifier Date		Change											Author
% 0.0.1 		2015-06-15 			aggiunta sezione e iniziata stesura				Roetta Marco
% =================================================================================================

\section{Requisiti Accettati}
\label{sec:requisiti_accettati}
A seguire sono riportati tutti i requisiti accettati dal proponente. I requisiti saranno divisi in tre gruppi a seconda della tipologia: obbligatori, desiderabili e facoltativi.

% =================================================================================================
\subsection{Requisiti Obbligatori}
Tutti i requisiti obbligatori, cosi come esposti nella sezione \ref{sec:requisiti}, sono stati soddisfatti.

% =================================================================================================
\subsection{Requisiti Desiderabili}
\begin{center}

	\def\arraystretch{1.5}
	\bgroup
	\begin{longtable}{| p{2.5cm} | p{8cm} | p{3.5cm} |}
		\hline
		\textbf{Requisito} & \textbf{Descrizione} & \textbf{Stato} \\
		\hline
		RDF1.8  &  Nel caso la registrazione sia andata a buon fine, viene inviata una e-mail di conferma all'indirizzo inserito dall'utente  &  \textbf{\textbf{\textcolor{red}{Non soddisfatto}}} \\
		\hline
		RDF3.3.5  &  Il sistema visualizza eventuali non conformità in fase di compilazione  &  \textbf{\textcolor{forestgreen}{Soddisfatto}} \\
		\hline
		RDF5.3.1.5  &  Tipo di View: viene visualizzato un Map Chart che illustra le zone nella quale si sono creati degli eventi di una pagina e mostra dei cerchi di diverse dimensioni a seconda della media di partecipanti agli eventi di quella zona. & \textbf{\textcolor{red}{Non soddisfatto}} \\
		\hline
		RDF5.5.1.9  &  Tipo di View: vengono visualizzate in un Map Chart le zone geografiche dove sono stati fatti i post contenenti un dato hashtag (di cui è nota la localizzazione). & \textbf{\textcolor{red}{Non soddisfatto}} \\
		\hline
		RDF6  & L'utente ha la possibilità di gestire le View che più preferisce in maniera da averle a disposizione più velocemente. &  \textbf{\textcolor{forestgreen}{Soddisfatto}} \\
		\hline
		RDF6.1  & L'utente visualizza tutte le View che sono state marcate da lui come preferite entrando dal menù nell'apposita pagina. & \textbf{\textcolor{forestgreen}{Soddisfatto}} \\
		\hline
		RDF6.1.1  & Il sistema mostra un messaggio qualora non ci fossero View salvate come preferite. &  \textbf{\textcolor{red}{Non soddisfatto}} \\
		\hline
		RDF6.2  & L'utente ha la possibilità di aggiungere qualsiasi View a quelle preferite premendo sull'apposito pulsante vicino alla View che desidera aggiungere. & \textbf{\textcolor{forestgreen}{Soddisfatto}} \\
		\hline
		RDF6.2.1  & Il pulsante per aggiungere una View alle preferite è presente vicino a ogni View in qualsiasi parte del sistema, eccetto nella pagina delle View preferite.  & \textbf{\textcolor{forestgreen}{Soddisfatto}} \\
		\hline
		RDF6.2.2  & Il sistema visualizza un messaggio se la View non è stata aggiunta correttamente tra quelle preferite.  & \textbf{\textcolor{red}{Non soddisfatto}} \\
		\hline
		RDF6.3  & L'utente ha la possibilità di rimuovere le View che ha aggiunto tra quelle preferite premendo sull'apposito pulsante vicino la View che desidera rimuovere.  &  \textbf{\textcolor{red}{Non soddisfatto}} \\
		\hline
		RDF6.3.1  & Il sistema visualizza un messaggio se la View non è stata rimossa correttamente tra quelle preferite.  &  \textbf{\textcolor{red}{Non soddisfatto}} \\
		\hline
		RDF6.4 & L'utente può marcare come preferite un numero illimitato di View. &  \textbf{\textcolor{forestgreen}{Soddisfatto}} \\
		\hline
		RDP1  &  Una volta autenticato l'utente deve poter visualizzare le proprie View entro 10 secondi salvo problemi di connessione &  \textbf{\textcolor{forestgreen}{Soddisfatto}} \\
		\hline
		RDP1.1  &  Il recupero dei dati dal database deve impiegare al più 7 secondi  &  \textbf{\textcolor{forestgreen}{Soddisfatto}} \\
		\hline
		RDV1  &  Il sistema sfrutterà gli strumenti offerti da Google Cloud Platform.  &  \textbf{\textcolor{forestgreen}{Soddisfatto}} \\
		\hline
		RDV1.1  &  Il sistema sfrutterà la Google App Engine.  &  \textbf{\textcolor{forestgreen}{Soddisfatto}} \\
		\hline
		RDV1.2  &  Il sistema sfrutterà il Google Cloud Datastore.  &  \textbf{\textcolor{forestgreen}{Soddisfatto}} \\
		\hline
		RDV1.2.1  &  Il database sarà di tipo non relazione composto da documenti in stile JSON, stesso formato dei dati in entrata.  &  \textbf{\textcolor{forestgreen}{Soddisfatto}} \\
		\hline
		RDV1.3  &  Il sistema sfrutterà i Google Cloud Endpoints.  &  \textbf{\textcolor{forestgreen}{Soddisfatto}} \\
		\hline
		RDV2  &  Il linguaggio di programmazione principalmente usato per il back-end sarà Python.  &  \textbf{\textcolor{forestgreen}{Soddisfatto}} \\
		\hline
		RDV2.1 & Il codice e la documentazione relativi al linguaggio Python dovranno seguire la Google Python Style Guide. & \textbf{\textcolor{forestgreen}{Soddisfatto}} \\
		\hline
		RDV3  &  Il codice sorgente sarà soggetto a versionamento tramite il modello di branching descritto nelle \docNameVersionNdP  &  \textbf{\textcolor{forestgreen}{Soddisfatto}} \\
		\hline
		RDV4  &  L'interfaccia web sarà di tipo single-page.  &  \textbf{\textcolor{forestgreen}{Soddisfatto}} \\
		\hline
		
	\end{longtable}
	\egroup
\end{center}

% =================================================================================================
\subsection{Requisiti Facoltativi}
\begin{center}

	\def\arraystretch{1.5}
	\bgroup
	\begin{longtable}{| p{2.5cm} | p{8cm} | p{3.5cm} |}

		\hline
		\textbf{Requisito} & \textbf{Descrizione} & \textbf{Stato} \\
		\hline
		RFF3.1.4  &  Nel pannello dei dati personali l'utente può visualizzare l'ultimo accesso effettuato  &  \textbf{\textcolor{red}{Non soddisfatto}} \\
		\hline
		RFF3.2  &  Il sistema offre un pannello per la visualizzazione delle statistiche utente e di sistema  &  \textbf{\textcolor{red}{Non soddisfatto}} \\
		\hline
		RFF3.2.1  &  Se l'utente seleziona il pulsante di visualizzazione delle statistiche, il sistema deve reperire le informazioni necessarie  &  \textbf{\textcolor{red}{Non soddisfatto}} \\
		\hline
		RFF3.2.2 &  Nel pannello delle statistiche viene mostrato il numero di View attive &  \textbf{\textcolor{red}{Non soddisfatto}} \\
		\hline
		RFF3.2.3  &  Nel pannello delle statistiche viene mostrato il numero di Recipe disponibili &  \textbf{\textcolor{red}{Non soddisfatto}} \\
		\hline
		RFF5.1.3  & Il sistema, per ogni Recipe dell'elenco, deve fornire un selettore che permetta di dare un voto alla recipe &  \textbf{\textcolor{forestgreen}{Soddisfatto}} \\
		\hline
		RFF5.1.3.1  & Il voto dato a una recipe è un intero compreso tra 1 e 5 &  \textbf{\textcolor{forestgreen}{Soddisfatto}} \\
		\hline
		RFF5.1.3.2  & Il voto sarà espresso selezionando da una a cinque stelle da un apposito selettore &  \textbf{\textcolor{red}{Non soddisfatto}} \\
		\hline
		RFF7  & L'utente ha la possibilità di richiedere l'inserimento di una nuova Recipe. & \textbf{\textcolor{forestgreen}{Soddisfatto}} \\
		\hline
		RFF7.1  & L'utente può generare una richiesta dall'apposito form presente in un'apposita sezione della Web GUI. & \textbf{\textcolor{forestgreen}{Soddisfatto}} \\
		\hline
		RFF7.1.1 & L'utente deve inserire i parametri richiesti dal form perché la richiesta sia considerata valida dal sistema. & \textbf{\textcolor{forestgreen}{Soddisfatto}}  \\
		\hline
		RFF7.1.1.1 & L'utente deve inserire il titolo della Recipe che desidera richiedere. Questo campo è obbligatorio. & \textbf{\textcolor{forestgreen}{Soddisfatto}} \\
		\hline
		RFF7.1.1.2 & L'utente può inserire una breve descrizione della Recipe che desidera richiedere. Questo campo è facoltativo. & \textbf{\textcolor{forestgreen}{Soddisfatto}} \\
		\hline
		RFF7.1.1.3 & L'utente deve inserire almeno una metrica nella richiesta della Recipe che desidera. & \textbf{\textcolor{forestgreen}{Soddisfatto}} \\
		\hline
		RFF7.1.1.4 & L'utente deve scegliere la categoria della metrica tra: Facebook, Twitter o Instagram. Questa selezione è obbligatoria e vincolata dal sistema. & \textbf{\textcolor{forestgreen}{Soddisfatto}} \\
		\hline
		RFF7.1.1.5 & L'utente deve scegliere la tipologia della metrica tra: Evento o Pagina, se ha scelto la categoria Facebook. Questa selezione è obbligatoria e vincolata dal sistema. & \textbf{\textcolor{forestgreen}{Soddisfatto}} \\
		\hline
		RFF7.1.1.6 & L'utente deve scegliere la tipologia della metrica tra: Utente o hashtag, se ha scelto la categoria Twitter. Questa selezione è obbligatoria e vincolata dal sistema. & \textbf{\textcolor{forestgreen}{Soddisfatto}} \\
		\hline
		RFF7.1.1.7 & L'utente deve scegliere la tipologia della metrica tra: Utente o hashtag, se ha scelto la categoria Instagram. Questa selezione è obbligatoria e vincolata dal sistema. & \textbf{\textcolor{forestgreen}{Soddisfatto}} \\
		\hline
		RFF7.1.1.8 & Se l'utente sceglie la categoria Facebook deve inserire il ``nome'' o ``l'identificativo'' della Pagina o dell'Evento del quale vuole raccogliere i dati. Questo campo è obbligatorio. & \textbf{\textcolor{forestgreen}{Soddisfatto}} \\
		\hline
		RFF7.1.1.9 & Se l'utente sceglie la categoria Twitter deve inserire il ``nome'' dell'Utente o dell'hashtag del quale vuole raccogliere i dati. Questo campo è obbligatorio. & \textbf{\textcolor{forestgreen}{Soddisfatto}} \\
		\hline
		RFF7.1.1.10 & Se l'utente sceglie la categoria Instagram deve inserire il ``nome'' dell'Utente o dell'hashtag del quale vuole raccogliere i dati. Questo campo è obbligatorio. & \textbf{\textcolor{forestgreen}{Soddisfatto}} \\
		\hline
		RFF7.1.1.11 & Il sistema avviserà l'utente in maniera istantanea se un campo è compilato correttamente. & \textbf{\textcolor{red}{Non soddisfatto}} \\
		\hline
		RFF7.1.1.12 & Il sistema avviserà l'utente in maniera istantanea se è stato lasciato vuoto un campo obbligatorio. & \textbf{\textcolor{red}{Non soddisfatto}} \\
		\hline
		RFF7.1.2 & L'utente deve premere sul pulsante di invio della richiesta per inviarla agli amministratori. & \textbf{\textcolor{forestgreen}{Soddisfatto}}  \\
		\hline
		RFF7.1.3 & La richiesta può fallire se i dati inseriti non sono corretti o ci sono dei campi obbligatori vuoti. & \textbf{\textcolor{forestgreen}{Soddisfatto}}  \\
		\hline
		RFF7.2  & Il sistema invierà una notifica agli amministratori che è stata inserita una nuova richiesta per l'inserimento di una Recipe. & \textbf{\textcolor{red}{Non soddisfatto}} \\
		\hline
		RFF8.4 & L'utente amministratore può visualizzare la classifica delle Recipe valutate dagli utenti  & \textbf{\textcolor{forestgreen}{Soddisfatto}}  \\
		\hline
		RFF8.4.1 & Cliccando il pulsante apposito viene aperto l'elenco delle Recipe che hanno ricevuto una valutazione, in ordine da quella che ha il voto più alto a quella che ha il voto più basso  & \textbf{\textcolor{red}{Non soddisfatto}}  \\
		\hline
		RFF8.4.1.1 & Il voto di una Recipe è dato dalla media aritmetica dei voti assegnati dagli utenti a quella Recipe  & \textbf{\textcolor{forestgreen}{Soddisfatto}}  \\
		\hline
		RFF8.4.1.2 & Il voto di una Recipe è un numero compreso tra 1 e 5 espresso con al più due cifre decimali  &  \textbf{\textcolor{forestgreen}{Soddisfatto}}  \\
		\hline
		RFF8.4.2 & Se nessuna Recipe ha ricevuto una valutazione al posto della lista viene visualizzato un messaggio che riporta l'assenza di valutazioni  & \textbf{\textcolor{forestgreen}{Soddisfatto}}  \\
		\hline
		RFF9.2.1  &  Una volta premuto l'apposito pulsate per l'eliminazione di un utente, le modifiche effettuate vengono applicate localmente in attesa che le modifiche generali agli utenti vengano confermate  &  \textbf{\textcolor{forestgreen}{Soddisfatto}} \\
		\hline
		RFF10  &  L'amministratore può gestire tutte le richieste di nuove Recipe, ricevute dagli utenti, da una apposita sezione.  & \textbf{\textcolor{forestgreen}{Soddisfatto}} \\
		\hline
		RFF10.1  &  L'amministratore può vedere l'elenco delle richieste di nuove Recipe.  & \textbf{\textcolor{forestgreen}{Soddisfatto}} \\
		\hline
		RFF10.1.1  &  Per ogni voce dell'elenco deve essere fornito il titolo della Recipe richiesta, la sua descrizione qualora presente e lo username dell'utente che ha fatto richiesta.  & \textbf{\textcolor{forestgreen}{Soddisfatto}} \\
		\hline
		RFF10.1.2  &  Per ogni voce dell'elenco deve essere fornito un pulsante per potere accedere alla visualizzazione dei dettagli della richiesta.  & \textbf{\textcolor{forestgreen}{Soddisfatto}} \\
		\hline
		RFF10.2  &  Il sistema visualizza un messaggio qualora non fossero presenti richieste di nuove Recipe.  & \textbf{\textcolor{forestgreen}{Soddisfatto}} \\
		\hline
		RFF10.3  &  Premendo sul pulsante della visualizzazione dettagli si accede a un pannello contenente il form per la creazione di una nuova Recipe già compilato con i dati richiesti.  &  \textbf{\textcolor{forestgreen}{Soddisfatto}} \\
		\hline
		RFF10.3.1  &  L'amministratore può decidere se modificare il titolo e la descrizione generale della Recipe richiesta.  & \textbf{\textcolor{red}{Non soddisfatto}} \\
		\hline
		RFF10.3.2  &  L'amministratore non può modificare altri campi della richiesta oltre a quelli descritti in ROF10.3.1.  & \textbf{\textcolor{red}{Non soddisfatto}} \\
		\hline
		RFF10.3.2.1  &  I campi non modificabili saranno disabilitati.  & \textbf{\textcolor{forestgreen}{Soddisfatto}} \\
		\hline
		RFF10.4  &  L'amministratore premendo sul pulsante di accettazione andrà a inserire la Recipe richiesta nel sistema.  & \textbf{\textcolor{forestgreen}{Soddisfatto}} \\
		\hline
		RFF10.5  &  L'amministratore, premendo sul pulsante di respinta andrà a rifiutare la Recipe richiesta.  & \textbf{\textcolor{red}{Non soddisfatto}} \\
		\hline
		RFF10.5.1  &  L'amministratore potrà fornire un messaggio di risposta che motivi il rifiuto della richiesta.  & \textbf{\textcolor{red}{Non soddisfatto}} \\
		\hline
		RFF10.5.1.1  &  Il sistema invierà una mail all'utente per notificare il messaggio di rifiuto.  &  \textbf{\textcolor{red}{Non soddisfatto}} \\
		\hline
		RFP2  &  L'interfaccia web utilizzerà un design di tipo responsive  &  \textbf{\textcolor{forestgreen}{Soddisfatto}} \\
		\hline
	\end{longtable}
	\egroup
\end{center}

% =================================================================================================
\begin{center}
		\def\arraystretch{1.5}
		\bgroup
		\begin{longtable}{| p{8cm} | p{2cm} | p{2cm} |}
			\hline
			\textbf{Categoria} & \textbf{Totali} & \textbf{Soddisfatti} \\
			\hline
			\textbf{Obbligatori}  & 173 & 173 \\
			\hline
			\textbf{Desiderabili} & 25 & 11 \\
			\hline
			\textbf{Facoltativi} & 47 & 31 \\
			\hline
		\end{longtable}
		\egroup
	\end{center}
