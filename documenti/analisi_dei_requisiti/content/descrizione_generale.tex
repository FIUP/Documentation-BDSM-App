% =================================================================================================
% File:			capitolo_2.tex
% Description:	Definisce il capitolo che descrive generalmente il prodotto per il commitente
% Created:		2014/12/10
% Author:		Roetta Marco
% Email:		roetta.marco@mashup-unipd.it
% =================================================================================================
% Modification History:
% Version		Modifier Date		Change											Author
% 0.0.1 		2014/12/30 			aggiunta sezione e iniziata stesura				Roetta Marco
% =================================================================================================
% Version		Modifier Date		Change											Author
% 0.0.2 		2015/01/12 			Aggiunto contenuto								Roetta Marco
% =================================================================================================
% Version		Modifier Date		Change											Author
% 0.1.0			2015/01/15 			Completata sezione						 		Roetta Marco
% =================================================================================================
% Version		Modifier Date		Change											Author
% 0.1.1			2015/01/15 			Correzioni ortografiche					 		Roetta Marco
% =================================================================================================
%

% CONTENUTO DEL CAPITOLO

\section{Descrizione generale}

\subsection{Contesto d'uso}
L'obbiettivo primario che il prodotto si pone è la creazione di una infrastruttura che
possa gestire diverse funzioni di monitoraggio, aggregazione e analisi statistica di parole chiave utilizzate nei social network precedentemente citati. Ogni utente che usa la piattaforma può creare, modificare ed eliminare View relativi a Tag suddivisi in Recipes predefinite e fornite dal sistema.
La gestione delle Recipes e degli utenti è disponibile solo per un sottoinsieme di utenti identificati come utenti amministratori.

\subsection{Funzioni del prodotto}
L'applicativo vuole fornire un'interfaccia web per poter consultare, manipolare e creare grafici che soddisfino gli obbietti sopra citati. L'utente a prodotto finito potrà:

\begin{itemize}
\item Visualizzare tutti i grafici creati;
\item Modificare i dati su cui sono basati i grafici;
\item Eliminare un grafico;
\end{itemize}

\subsubsection{Sezioni del prodotto}
Il prodotto sarà diviso in due macro parti distinte:

\begin{itemize}
\item Un applicativo web per consultazione e personalizzazione dei grafici e dei dati associati;
\item Un applicativo server-side che interroga i social network in base ai parametri predefiniti rappresentati dalle Recipes. I risultati di tali elaborazioni verranno utilizzati per realizzare i grafici visualizzati dagli utenti tramite l'interfaccia grafica web.

L'applicativo server-side sarà inoltre suddiviso in diverse parti funzionali distinte:
\begin{itemize}
\item \textbf{Miner:} Processo che interroga i social network tramite API dedicate;
\item \textbf{Cron:} Processo che genera eventi di sincronizzazione dei dati per il Processor;
\item \textbf{Processor:} Processo principale di elaborazione delle richieste utente, comunicazione tra basi di dati e interfaccia grafica e gestione del funzionamento del Miner.
\end{itemize}

\end{itemize}

\subsection{Caratteristiche degli utenti}
Il prodotto è rivolto a due classi distinte di persone:

\begin{itemize}
\item persone esterne all'azienda che desiderano utilizzare i servizi offerti;
\item dipendenti di Zing s.r.l. che assumono il ruolo di amministratori in grado di gestire il prodotto in tutte le sue parti.
\end{itemize}

Da questi ultimi sono richieste conoscenze di tipo sistemistico per poter comprendere tutte le funzionalità offerte.

\subsection{Vincoli generali}
Per usufruire delle funzionalità offerte dal prodotto si richiede all'utente di disporre
di una connessione internet e delle credenziali di accesso fornite durante la registrazione al servizio nell'apposita pagina.

\subsubsection{Piattaforme di esecuzione}
Il prodotto finale è fruibile da qualsiasi piattaforma che disponga di un browser per la navigazione web, tuttavia sarà garantito il funzionamento corretto solo con alcuni browser: Chrome v. 39.0 e successive, Firefox v. 35.0 e successive, Safari v. 8.0 e successive.
Il back-end del sistema verrà invece eseguito e gestito sulla piattaforma Google App Engine.