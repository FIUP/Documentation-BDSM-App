% =================================================================================================
% File:			capitolo_2.tex
% Description:	Definisce il capitolo che descrive generalmente il prodotto per il commitente
% Created:		2014/12/10
% Author:		Roetta Marco
% Email:		roetta.marco@mashup-unipd.it
% =================================================================================================
% Modification History:
% Version		Modifier Date		Change											Author
% 0.0.1 		2014/12/10 			aggiunta sezione e iniziata stesura				Roetta Marco
% =================================================================================================
% Modification History:
% Version		Modifier Date		Change											Author
% 0.0.2 		2015/01/12 			aggiunto contenuto								Roetta Marco
% =================================================================================================
%

% CONTENUTO DEL CAPITOLO

\section{Descrizione generale}

\subsection{Obiettivi del prodotto}
L'obbiettivo primario che il prodotto si pone è la creazione di una infrastruttura che
possa gestire diverse funzioni di monitoraggio, aggregazione e analisi statistica di parole chiave utilizzate nei social network precedentemente citati. Ogni utente che usa la piattaforma deve poter personalizzare e manipolare a piacimento le statistiche e la loro visualizzazione.

\subsection{Funzioni del prodotto}
L'applicativo vuole fornire un'interfaccia web per poter consultare, manipolare e creare grafici che soddisfano gli obbietti sopra citati. L'utente a prodotto finito potrà:

\begin{itemize}
\item Visualizzare tutti i propri grafici;
\item Modificare i dati su cui sono basati i grafici;
\item Modificare il tempo di aggiornamento dei dati nei grafici;
\item Eliminare un grafico;
\end{itemize}

\subsection{Sezioni del prodotto}
Il prodotto sarà diviso in due macro parti distinte:

\begin{itemize}
\item Un applicativo web per la consultazione e personalizzazione dei grafici e dei dati associati;
\item Un applicativo server-side che interroga i social network in base ai criteri determinati dagli utenti tramite l'interfaccia grafica. I risultati di tali elaborazioni costituiranno i grafici visualizzati dagli utenti tramite l'interfacci grafica Web.

L'applicativo server-side sarà inoltre suddiviso in diverse parti funzionali distinte:
\begin{itemize}
\item  TODO
\end{itemize}

\end{itemize}

\subsection{Caratteristiche degli utenti}
La classe di utenti a cui il prodotto si rivolge è identificata dai dipendenti di TODO. 
Il prodotto si rivolge anche ad utenti amministratori che potranno gestire il sistema|g|.
Da questi utenti sono richieste conoscenze di tipo sistemistico per poter comprendere tutte le funzionalità offerte.

\subsection{Piattaforme di esecuzione}
Il prodotto finale è fruibile da qualsiasi piattaforma che disponga di un browser per la navigazione web. Sarà garantito il funzionamento corretto solo con particolari browser.
Il back-end server del sistema funzionerà invece su piattaforma Unix TODO.

\subsection{Vincoli generali}
Per usufruire delle funzionalità offerte dal prodotto si richiede all'utente di disporre
di una connessione al server interno dell'azienda ( via rete locale oppure via Internet ) e delle credenziali di accesso fornite durante la registrazione al servizio nell'apposita pagina.
