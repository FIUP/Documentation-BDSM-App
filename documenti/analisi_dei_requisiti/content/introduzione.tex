% =================================================================================================
% File:			capitolo_1.tex
% Description:	Definisce il capitolo di introduzione
% Created:		2014/12/10
% Author:		Roetta Marco
% Email:		roetta.marco@mashup-unipd.it
% =================================================================================================
% Modification History:
% Version		Modifier Date		Change											Author
% 0.0.1 		2014/12/10 			aggiunto sezione introduzione					Roetta Marco
% =================================================================================================
% 0.0.2			2014/12/17			sistemata alcune parti utilizzando i comandi	Tesser Paolo
% =================================================================================================
%

% CONTENUTO DEL CAPITOLO

\section{Introduzione}

\subsection{Scopo del documento}
Il presente documento ha lo scopo di descrivere in modo esaustivo i requisiti individuati per il prodotto. Tali requisiti sono stati identificati dall’analisi del capitolato C1 e il successivo incontro con il proponente.

\subsection{Scopo del prodotto}
	\productScope

\subsection{Glossario}
	\glossarioDesc

\subsection{Riferimenti}
	\subsubsection{Normativi}
		\begin{itemize}
			\item Norme di Progetto: \docNameVersionNdP;
			\item Capitolato d’appalto C1: BDSMApp: Big Data Social Monitoring App. Reperibile all’indirizzo: \url{http://www.math.unipd.it/~tullio/IS-1/2014/Progetto/C1.pdf}
		\end{itemize}
	
	\subsubsection{Informativi}
		\begin{itemize}
			\item Studio di Fattibilità: \docNameVersionSdF;
			\item Software Engineering - Ian Sommerville - 9th Edition 2010 – Chapter 4: Requirements engineering;
			\item IEEE 830-1998: \url{http://en.wikipedia.org/wiki/Software_requirements_specification}.
		\end{itemize}
	