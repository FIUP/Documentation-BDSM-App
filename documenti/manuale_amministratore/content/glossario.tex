% =================================================================================================
% File:			glossario.tex
% Description:	Defiinisce la sezione relativa ad un capitolo del documento
% Created:		2015-05-25
% Author:		Santacatterina Luca
% Email:		santacatterina.luca@mashup-unipd.it
% =================================================================================================
% Modification History:
% Version		Modifier Date		Change											Author
% 0.0.1 		2015-05-25 			prima abbozza glossario							Santacatterina Luca
% =================================================================================================
%

% CONTENUTO DEL CAPITOLO
\section{Glossario} % (fold)
\label{sec:glossario}

	\section*{\Huge A} % (fold)
		\begin{itemize}
			\item \textbf{Android:} sistema operativo per sistemi mobile sviluppato da Google e basato su kernel Linux. Sviluppato principalmente per dispositivi touchscreen quali smartphone e tablet, lo si può trovare con interfaccie utenti specializzate per televisori (Android TV), automobili (Android Auto), orologi da polso (Android Wear) e occhiali (Google Glass);
			\item \textbf{API:} acronimo di Application Programming Interface, è un insieme di routine, protocolli e strumenti che consentono ai software di interagire tra di loro. Funziona come un'interfaccia tra software differenti facilitandone la loro interazione, allo stesso modo in cui un'interfaccia utente facilita l'interazione tra uomo e computer;
			\item \textbf{Autenticazione:} verifica attraverso un sistema automatico di interrogazione e risposta in grado di stabilire un collegamento autorizzato e valido.;
		\end{itemize}
	% section a (end)

	\section*{\Huge B} % (fold)
		\begin{itemize}
			\item \textbf{Browser:} programma che consente di visualizzare i contenuti delle pagine dei siti web e di interagire con essi, permettendo così all’utente di navigare in internet. Il browser è in grado di interpretare l’HTML (il codice con il quale sono scritte la maggior parte delle pagine web) e di visualizzarlo. I browser attualmente più noti e diffusi sono Internet Explorer, Mozilla Firefox, Google Chrome, Safari e Opera;
		\end{itemize}
	% section b (end)

	\section*{\Huge C} % (fold)
		\begin{itemize}
			\item \textbf{Cookies:} files di piccole dimensioni all'interno del dispositivo che contengono delle informazioni registrate e fungono da identificatori automatici per riscontrare i visitatori e le loro sessioni di visita nel sito;
			\item \textbf{Credenziali:} dati in possesso dall'utente, da questi conosciuti o ad essa univocamente correlati, utilizzati per l'autenticazione informatica;
		\end{itemize}
	% section c (end)

	\section*{\Huge D} % (fold)
		\begin{itemize}
			\item \textbf{Database:} collezione di dati organizzata. Le informazioni contenute in esso sono strutturate e collegate tra loro mediante un particolare modello logico in modo tale da consentire l'organizzazione efficiente dei dati stessi e l'interfacciamento con le richieste dell'utente attraverso le query di interrogazione;
			\item \textbf{Dashboard:} interfaccia grafica che organizza e presenta le informazioni in modo semplice, intuitivo, immediato ed ordinato. Consente all'utente di consulatere il pannello in modo ordinato;
		\end{itemize}
	% section d (end)

	\section*{\Huge F} % (fold)
		\begin{itemize}
			\item \textbf{Form:} nota anche come modulo web, all'interno di una pagina web permette ad un utente di inserire dei dati da inviare al server per essere poi processati. Una form può assumere l'aspetto di un modulo cartaceo poiché è tipicamente composto da checkbox, radio button e campi di testo. Sono utilizzate, per esempio, per inserire i dati di spedizione o della carta di credito per l'acquisto di un prodotto via internet, o possono essere utilizzate per recuperare i risultati di una ricerca da un motore di ricerca;
		\end{itemize}
	% section f (end)

	\section*{\Huge H} % (fold)
		\begin{itemize}
			\item \textbf{Hardware:} Tradotto letteralmente dall'inglese significa ferramenta. Termine generico per indicare le componenti fisiche (device, dispositivi) di un elaboratore quali i circuiti elettronici, chip, schede, disk drive, stampanti, mouse, lettori CD-ROM, monitor, tastiera ecc;
		\end{itemize}
	% section h (end)

	\section*{\Huge I} % (fold)
		\begin{itemize}
			\item \textbf{Internet a banda larga:} in lingua inglese indicata con il termine broadband. Nel campo delle telecomunicazioni e informatica, indica generalmente la trasmissione e ricezione di dati informativi, inviati e ricevuti simultaneamente in maggiore quantità, sullo stesso cavo o mezzo radio grazie all'uso di mezzi trasmissivi e tecniche di trasmissione che supportino e sfruttino un'ampiezza di banda superiore ai precedenti sistemi di telecomunicazioni detti invece a banda stretta;
		\end{itemize}
	% section h (end)

	\section*{\Huge L} % (fold)
		\begin{itemize}
			\item \textbf{Login:} procedura di accesso al sistema informatico. Prevede l'inserimento di un codice identificativo e di una parola d'ordine da parte dell'utente;
			\item \textbf{Logout:} procedura di uscita dal sistema informatico. Non prevede l'inserimento di ulteriori informazioni;
		\end{itemize}
	% section f (end)

	\section*{\Huge P} % (fold)
	\begin{itemize}
			\item \textbf{Pop-up:} ingrandimenti di parti di applicazione su finestra dedicata. Presentano nel contesto funzione di attrazione dell'attenzione dei visitatori;
		\end{itemize}
	% section p (end)

	\section*{\Huge R} % (fold)
		\begin{itemize}
			\item \textbf{Recipe:} collezioni di informazioni raccolte dai social network periodicamente e generate dall'applicativo server-side. Vengono utilizzate per generare e aggiornare i dati e i grafici dagli utenti;
			\item \textbf{Responsive:} tipo di design utilizzato nei siti web che permette ai siti di adattarsi graficamente in modo automatico al dispositivo coi quali vengono visualizzati, riducendo al minimo la necessità per l'utente di ridimensionamento e scorrimento dei contenuti;
			\item \textbf{REST:} acronimo di REpresentational State Transfer, è un tipo di architettura software per il Word Wide Web molto usato nell'HTTP. REST infatti si riferisce ad un insieme di principi di architetture di rete, i quali delineano come le risorse sono definite e indirizzate;
			\end{itemize}
	% section r (end)

	\section*{\Huge S} % (fold)
		\begin{itemize}
			\item \textbf{Screenshot:} processo di acquisizione istantanea che consente di salvare sotto forma di immagini ciò che viene visualizzato sullo schermo di un computer;
			\item \textbf{Software:} termine generico che intende un programma, quindi un insieme di istruzioni (algoritmi e dati) che possono essere eseguite dalla CPU. Se le istruzioni devono essere compilate, ossia sono istruzioni ad alto livello si parla di codice sorgente del programma. Se il software viene utilizzato direttamente dagli utenti si parla di applicazione. In genere se il software viene utilizzato dal sistema operativo lo si indica con il termine software di sistema;
		\end{itemize}
	% section s (end)

	\section*{\Huge T} % (fold)
		\begin{itemize}
			\item \textbf{Token:} sequenza di caratteri alfanumerici generata dal server e fornita all'utente singolarmente. Le informazioni necessarie risiedono direttamente nel computer dell'utente, e non in un oggetto esterno;
		\end{itemize}
	% section t (end)

	\section*{\Huge U} % (fold)
		\begin{itemize}
			\item \textbf{URL:} con il termine URL si identifica l’acronimo Uniform Resource Locator. Definito da una sequenza di caratteri che identifica univocamente l’indirizzo di una risorsa in Internet, come un documento o un’immagine;
		\end{itemize}
	% section u (end)

	\section*{\Huge V} % (fold)
		\begin{itemize}
			\item \textbf{View:} dati e grafici creati dagli utenti autenticati utilizzando le recipe;
		\end{itemize}
	% section v (end)


% section Glossario (end)