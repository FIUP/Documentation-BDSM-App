% =================================================================================================
% File:			gest_richieste.tex
% Description:	Defiinisce la sezione relativa ad un capitolo del documento
% Created:		2015-04-21
% Author:		Tesser Paolo
% Email:		tesser.paolo@mashup-unipd.it
% =================================================================================================
% Modification History:
% Version		Modifier Date		Change											Author
% 0.0.1 		2015-04-21 			creato scheletro doc							Tesser Paolo
% =================================================================================================
%

% CONTENUTO DEL CAPITOLO
\section{Gestione richieste recipe} % (fold)
\label{sec:gest_richieste}
	\subsection{Contenuti Sezione} % (fold)
	\label{sub:contenuti_sezione}
	\begin{itemize}
		\item Visualizza elenco richieste;
		\item Visualizza dettagli richiesta;
		\item Modifica richiesta Recipe;
		\item Accettazione richiesta;
		\item Respinta richiesta;
	\end{itemize}

	\subsection{Visualizza elenco richieste}
	L'amministratore può visualizzare un elenco delle richieste per delle nuove Recipe che sono state inviate dagli utenti normali e non ancore marcate come ''chiuse'' da un utente amministratore.
	Per accedere all'elenco delle richieste, selezionare l'apposito pulsante dal Menu principale nella Home Page del sistema.
	% END
	
	\subsection{Visualizza dettagli richiesta}
	L'amministratore può visualizzare tutti i dettagli di una richiesta, dopo averla selezionata dall'elenco delle richieste.
	I dettagli di una richiesta includono:
	\begin{itemize}
		\item Il nome dell'utente che l'ha inoltrata;
		\item Il titolo della richiesta;
		\item La descrizione della richiesta;
	\end{itemize}
	Un utente amministratore può accettare o respingere una richiesta.
	Può inoltre modificare il titolo o la descrizione qualora li ritenesse inopportuni.
	% END
	
	
	\subsection{Modifica richiesta Recipe}
	L'amministratore può modificare titolo e descrizione della richiesta per renderla più efficace da comprendere per gli altri amministratori.
	Per eseguire questa operazione accedere all'elenco delle richieste Recipe, selezionare una richiesta per visualizzarne i dettagli e selezionare il pulsante di Modifica richiesta nella schermata corrente.
	I parametri che si possono modificare sono i seguenti:
	\begin{itemize}
	 	\item Il titolo della richiesta;
	 	\item La descrizione della richiesta;
	\end{itemize}
	Una volta terminate le modifiche è necessario selezionare il pulsante Conferma modifiche affinché queste siano attive nel sistema.
	\subsubsection{Note all'operazione}
	Prestare attenzione, questa operazione non è reversibile.
	% END
	
	\subsection{Accettazione richiesta}
	L'amministratore, una volta controllata la bontà della Recipe richiesta, potrà accettarla in modo che venga inserita nel sistema come Recipe
	Per svolgere questa operazione, accedere innanzitutto all'elenco delle richieste Recipe dal Menu principale situato nella Home Page.
	Una volta visualizzato l'elenco, selezionare la richiesta desiderata per visualizzarne i dettagli.
	A questo punto è possibile verificare che i dati richiesti siano accettabili e confermare la richiesta selezionando il pulsante Accetta richiesta nella schermata corrente. 
	E' necessario confermare l'operazione selezionando il pulsante Conferma quando richiesto.
	Il sistema a questo punto crea una Recipe utilizzando i dati forniti dall'utente e confermati con l'operazione appena descritta.
	\subsubsection{Note all'operazione}
	Una richiesta marcata come accettata viene rimossa dall'elenco delle richieste e non più visualizzata dagli altri amministratori.
	% END
	
	\subsection{Respinta richiesta}
	L'amministratore, una volta controllata la bontà della Recipe richiesta, potrà rifiutarla, qualora non la ritenesse conforme o appropriata come contenuti.
	Per svolgere questa operazione, accedere innanzitutto all'elenco delle richieste Recipe dal Menu principale situato nella Home Page.
	Una volta visualizzato l'elenco, selezionare la richiesta desiderata per visualizzarne i dettagli.
	A questo punto è possibile respingere la richiesta inopportuna o non conforme selezionando il pulsante Respingi richiesta, visualizzato nella schermata corrente.
	E' necessario confermare l'operazione selezionando il pulsante Conferma quando richiesto.
	\subsubsection{Note all'operazione}
	Una richiesta marcata come respinta viene rimossa dall'elenco delle richieste e non più visualizzata dagli altri amministratori.
	% END
	
% section Gestione degli utenti (end)