% =================================================================================================
% File:			argomenti_trattati.tex
% Description:	Defiinisce la sezione relativa al riassunto dell'incontro
% Created:		2014/12/08
% Author:		Ceccon Lorenzo
% Email:		ceccon.lorenzo@mashup-unipd.it
% =================================================================================================
% Modification History:
% Version		Modifier Date		Change											Author
% 0.0.1 		2014/12/08 			iniziata stesura riassunto dell'incontro		Lorenzo C.
% =================================================================================================
%

% CONTENUTO DEL CAPITOLO
\section{Argomenti trattati}
	Per prima cosa si è discusso in merito alla suddivisione dei ruoli tra i membri del gruppo.\\
	La decisione presa riguarda solo il primo periodo. In seguito il \roleProjectManager{} andrà a definire la rotazione dei ruoli nel \docNameVersionPdP.\\
	Si è deciso quindi per la seguente assegnazione:
		\begin{itemize}
			\item \textbf{\roleProjectManager :} \textnormal{Tesser Paolo;}
			\item \textbf{\roleAdministrator :} \textnormal{Santacatterina Luca;}
			\item \textbf{\roleAnalyst :} \textnormal{Carnovalini Filippo, Cusinato Giacomo, Roetta Marco;}
			\item \textbf{\roleVerifier :} \textnormal{Ceccon Lorenzo, Faccin Nicola.}
		\end{itemize}
	Come modello di ciclo di vita da adottare si è optato per un modello di tipo incrementale. Questo modello prevede rilasci multipli dove nei primi rilasci vengono soddisfatti i requisiti più importanti mentre nei successivi vengono sviluppati quelli meno importanti. Si incrementa fino a quando il prodotto sviluppato soddisferà tutti i requisiti richiesti.
