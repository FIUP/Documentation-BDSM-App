% =================================================================================================
% File:			gest_recipe.tex
% Description:	Defiinisce la sezione relativa ad un capitolo del documento
% Created:		2015-04-21
% Author:		Tesser Paolo
% Email:		tesser.paolo@mashup-unipd.it
% =================================================================================================
% Modification History:
% Version		Modifier Date		Change											Author
% 0.0.1 		2015-04-21 			creato scheletro doc							Tesser Paolo
% =================================================================================================
%

% CONTENUTO DEL CAPITOLO
\section{Gestione delle Recipe} % (fold)
\label{sec:gestione_delle_recipe}
	\subsection{Contenuti Sezione} % (fold)
	\label{sub:contenuti_sezione}
	\begin{itemize}
		\item Aggiungi nuova Recipe;
		\item Eliminazione Recipe;
		\item Visualizzazione classifica Recipe;
	\end{itemize}

	\subsection{Aggiungi nuova Recipe}
	Dal Menu principale situato nella Home Page del sistema è possibile visualizzare l'elenco delle Recipe premendo sull'apposito pulsante.
	E' possibile premere Poi sul pulsante di Inserimento nuova Recipe per avviare la procedura guidata.
		\subsubsection{Inserimento dei parametri}
		E' necessario compilare i seguenti campi per poter portare a compimento la creazione della nuova Recipe:
		\begin{itemize}
			\item Nome della Recipe;
			\item Descrizione della Recipe;
			\item Parametri relativi alla Recipe dipendenti dal/dai social network di interesse;
		\end{itemize}
		E' possibile selezionare uno o più social network e uno o più parametri per ciascuna selezione.
		\newline
		Al termine della procedura guidata è necessario selezionare il pulsante di Conferma Creazione Recipe prima che le modifiche siano apportate nel sistema.
	% END	
	
	
	\subsection{Eliminazione Recipe}
	L'amministratore può eliminare una Recipe e tutti i dati ad essa associati dal sistema.
	Per eseguire questa operazione è richiesto di premere il pulsante Elimina Recipe in prossimità della Recipe da eliminare dall'elenco delle Recipe. 
	E' possibile visualizzare l'elenco delle Recipe selezionando l'apposito pulsante dal Menu principale nella Home Page, dopo aver effettuato l'accesso al sistema.
	% END
	
	
	\subsection{Visualizzazione classifica Recipe}
	L'amministratore può vedere la classifica delle Recipe in ordine dalla più apprezzata alla meno apprezzata.
	Per accedere al pannello richiesto, selezionare l'apposito pulsante dal menu principale nella Home Page.
	\subsubsection{Note all'operazione}
	Nel caso in cui nessun utente abbia ancora dato un giudizio sulle Recipe presenti nel sistema, la classifica risulterà vuota.
	Un messaggio avviserà l'amministratore se questa situazione dovesse verificarsi.
	% END
	
% section Gestione delle Recipe (end)