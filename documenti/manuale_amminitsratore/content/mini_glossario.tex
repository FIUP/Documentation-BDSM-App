% =================================================================================================
% File:			nome_del_capitolo.tex
% Description:	Defiinisce la sezione relativa ad un capitolo del documento
% Created:		2015-04-21
% Author:		Tesser Paolo
% Email:		tesser.paolo@mashup-unipd.it
% =================================================================================================
% Modification History:
% Version		Modifier Date		Change											Author
% 0.0.1 		2015-04-21 			creato scheletro doc							Tesser Paolo
% =================================================================================================
%

% CONTENUTO DEL CAPITOLO
\section{Glossario} % (fold)
\label{sec:glossario}

	\letteraGlossario{F}
	\textbf{Form:} nota anche come modulo web, all'interno di una pagina web permette
ad un utente di inserire dei dati da inviare al server per essere poi processati.
Una form può assumere l'aspetto di un modulo cartaceo poiché è tipicamente
composto da checkbox, radio button e campi di testo. Sono utilizzate, per
esempio, per inserire i dati di spedizione o della carta di credito per l'acquisto
di un prodotto via internet, o possono essere utilizzate per recuperare i risultati
di una ricerca da un motore di ricerca;

	\letteraGlossario{R}
	\textbf{Recipe:} collezioni di informazioni raccolte dai social network periodicamente e generate dall'applicativo server-side. Vengono utilizzate per generare e aggiornare i dati e i gra?ci dagli utenti;

% section Glossario (end)