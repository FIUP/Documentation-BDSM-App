% =================================================================================================
% File:			amm_utenti.tex
% Description:	Defiinisce la sezione relativa ad un capitolo del documento
% Created:		2015-04-21
% Author:		Tesser Paolo
% Email:		tesser.paolo@mashup-unipd.it
% =================================================================================================
% Modification History:
% Version		Modifier Date		Change											Author
% 0.0.1 		2015-04-21 			creato scheletro doc							Tesser Paolo
% =================================================================================================
%

% CONTENUTO DEL CAPITOLO
\section{Gestione degli utenti} % (fold)
\label{sec:gestione_utenti}
	\subsection{Contenuti Sezione} % (fold)
	\label{sub:contenuti_sezione}
	\begin{itemize}
		\item Visualizza elenco utenti;
		\item Modifica permessi utente;
		\item Eliminazione utente;
	\end{itemize}

	\subsection{Visualizza elenco utenti}
	L'amministratore può visualizzare l'elenco di tutti gli utenti salvati nel sistema selezionando l'apposito pulsante dal Menu principale situato nella Home Page.
	In questo elenco è possibile visualizzare tutti i dati di ciascun utente, modificare  i suoi permessi oppure eliminare un utente, purché esso non sia un amministratore a sua volta.
	% END	
	
	\subsection{Modifica permessi utente}
	L'amministratore può abilitare i permessi di amministratore ad un utente diverso da se stesso.
	Per effettuare questa operazione, accedere all'elenco degli utenti premendo l'apposito pulsante dal Menu principale dell'applicazione situato nella Home Page.
	Selezionare poi il pulsante di modifica dei permessi in corrispondenza del nome dell'utente desiderato nell'elenco.
	Una volta effettuata la selezione è richiesto di premere il pulsante di conferma dell'operazione per rendere effettive le modifiche.
	% END
	
	\subsection{Eliminazione utente}
	L'amministratore può eliminare un utente diverso da se stesso dal sistema.
	Per effettuare questa operazione, accedere all'elenco degli utenti premendo l'apposito pulsante dal Menu principale dell'applicazione situato nella Home Page.
	Selezionare poi il pulsante di eliminazione dell'utente in corrispondenza del nome dell'utente desiderato nell'elenco.
	Una volta effettuata la selezione è richiesto di premere il pulsante di conferma dell'operazione di eliminazione per rendere effettive le modifiche.
	\subsubsection{Note all'operazione}	
	Prestare attenzione, questa operazione non è reversibile.
	% END
		

% section Gestione degli utenti (end)